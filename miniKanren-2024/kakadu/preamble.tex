% !TEX TS-program = pdflatex
% !TeX spellcheck = en_US
% !TEX root = main.tex

% ======== Custom definitions =============
\usepackage[british]{datetime2}
\usepackage[normalem]{ulem} % \sout
\usepackage{graphicx}
\usepackage{tikz}
\usetikzlibrary{shapes.geometric,backgrounds,fit,
  %positioning-plus,
  %node-families,
  calc}
\usepackage{xcolor}
\usepackage{mathtools}
%\usepackage[draft]{hyperref}
\usepackage{marginnote}
%\usepackage{verbatim}
\usepackage{listings}
%\usepackage{multirow}
%\usepackage{array}
\usepackage{url}
%\usepackage{caption}
\usepackage{subcaption}
\usepackage{xspace}
\usepackage{flushend} % Equalizes columns on the last page
\usepackage{placeins} % Defines a \FloatBarrier command, beyond which floats may not pass; useful, for example, to ensure all floats for a section appear before the next \section command.
%\usepackage{verbatimbox}
\usepackage[title]{appendix}

\newcommand{\todo}[1]{{\bf\color{red}#1}}
\renewcommand{\leq}{\ensuremath{\leqslant}}
\renewcommand{\geq}{\ensuremath{\geqslant}}

\def\faster{\textsc{faster-miniKanren}}
\newcommand{\miniKanren}{\textsc{miniKanren}}
\newcommand{\OCanren}{\textsc{OCanren}}
\newcommand{\OCaml}{\textsc{OCaml}}
\newcommand{\Scheme}{\textsc{Scheme}}
\newcommand{\Racket}{\textsc{Racket}}
\newcommand{\Chez}{\textsc{Chez}}
\newcommand{\Kotlin}{\textsc{Kotlin}}
\newcommand{\Klogic}{\textsc{Klogic}}
\newcommand{\Java}{\textsc{Java}}
\newcommand{\CoreLogic}{\textsc{CoreLogic}}

\def\HaskellTypeclassColor\PYG{k+kt}
\def\HaskellCommentColor\PYG{c+c1}

\lstdefinelanguage{Kotlin}{
comment=[l]{//},
basicstyle=\ttfamily,
identifierstyle=\ttfamily,
commentstyle=\HaskellCommentColor\itshape\HaskellCommentColor,
%commentstyle={\color{gray}\ttfamily},
emph={filter, first, firstOrNull, forEach, lazy, map, mapNotNull, println},
emphstyle={\color{orange}},
%identifierstyle=\color{black},
keywords={!in, !is, abstract, actual, annotation, as, as?, break, by, catch, class, companion, const, constructor, context, continue, crossinline, data, delegate, do, dynamic, else, enum, expect, external, false, field, file, final, finally, for, fun, get, if, import, in, infix, init, inline, inner, interface, internal, is, lateinit, noinline, null, object, open, operator, out, override, package, param, private, property, protected, public, receiveris, reified, return, return@,sealed, set, setparam, super, suspend, tailrec, this, throw, true, try, typealias, typeof, val, var, vararg, when, where, while
, value
},
keywordstyle={\color{blue}\bfseries},
morecomment=[s]{/*}{*/},
morestring=[b]",
morestring=[s]{"""*}{*"""},
ndkeywords={@Deprecated, @JvmField, @JvmName, @JvmOverloads, @JvmStatic, @JvmSynthetic, @JvmInline, Array, Byte, Double, Float, Int, Integer, Iterable, Long, Runnable, Short, String, Any, Unit, Nothing},
ndkeywordstyle={\color{orange}\bfseries},
sensitive=true,
stringstyle={\color{ForestGreen}\ttfamily},
}


% TODO: https://tex.stackexchange.com/questions/4198/emphasize-word-beginning-with-uppercase-letters-in-code-with-lstlisting-package
\lstdefinelanguage{ocaml}{
%basicstyle=\large\ttfamily,   % Вот тут надо стиль ставить, а не у идентификаторов
basicstyle=\ttfamily,
identifierstyle=\ttfamily,
commentstyle=\HaskellCommentColor\itshape\HaskellCommentColor,
sensitive=true,
%
classoffset=0
, keywords={type, rec, in, when, of, as, val, let, fun,
module, sig, deriving, if, then, else, assert, true, false, end}
, keywordstyle=\ttfamily\bfseries\color{blue} %\underbar
, classoffset=1
, morekeywords={pure,empty,select,branch,oneOf}
, keywordstyle=\color{RawSienna},
classoffset=2
, morekeywords={Monad,Applicative,Selective,String,Either,Left,Right,Maybe,Some,None}
%, keywordstyle=\PYG{k+kt}
, keywordstyle={\color{blue}\bfseries}
, classoffset=0,
%keywordstyle=[2]{\color{orange}},
otherkeywords={::,<$>, >?>},
%identifierstyle=\fontfamily{cmtt}\selectfont\ttfamily,
%basewidth={0.5em,0.5em},
columns=fixed,
%fontadjust=true,
%literate={->}{{$\to$}}3 {===}{{$\equiv$}}1 {=/=}{{$\not\equiv$}}1 {|>}{{$\triangleright$}}3 {\\/}{{$\vee$}}2 {/\\}{{$\wedge$}}2 {>=}{{$\ge$}}1 {<=}{{$\le$}} 1,
%morecomment=[s]{(*}{*)}
%, literate={\$}{{\textcolor{blue}{\$}}}1
%, literate={<\$>}{{\textcolor{RawSienna}{\ <\$>\ } }}1
%           {>?>}{{\textcolor{RawSienna}{\ >?>\ } }}1
}

\lstdefinelanguage{miniKanren}{
comment=[l]{//},
basicstyle=\ttfamily,
identifierstyle=\ttfamily,
commentstyle=\HaskellCommentColor\itshape\HaskellCommentColor,
%emph={filter, first, firstOrNull, forEach, lazy, map, mapNotNull, println},
%emphstyle={\color{orange}},
keywords={conde, defrel, fresh},
keywordstyle={\color{blue}\bfseries},
%morecomment=[s]{/*}{*/},
%morestring=[b]",
%morestring=[s]{"""*}{*"""},
%ndkeywords={@Deprecated, @JvmField, @JvmName, @JvmOverloads, @JvmStatic, @JvmSynthetic, @JvmInline, Array, Byte, Double, Float, Int, Integer, Iterable, Long, Runnable, Short, String, Any, Unit, Nothing},
ndkeywordstyle={\color{black}\bfseries\underbar},
sensitive=true,
}

\lstset{
mathescape=true,
%basicstyle=\small,
identifierstyle=\ttfamily,
keywordstyle=\bfseries,
commentstyle=\scriptsize\rmfamily,
basewidth={0.5em,0.5em},
fontadjust=true,
language=ocaml
}
