% !TEX TS-program = pdflatex
% !TeX spellcheck = en_US
% !TEX root = main.tex

\section{Relational Programming, Verifiers, and Solvers}
\label{sec:rel}

The implementation of our layout synthesizer employs the techniques of relational programming. Here
we briefly recollect what relational programming is and how the approaches and tools we use work.

Relational programming~\cite{TRS} is an approach based on the idea of describing programs as
relations. It can be considered as a branch of logic programming in which the use of
all non-relational constructs (side-effects, extra-logic features) is discouraged. In a
narrow sense relational programming amounts to writing programs in \textsc{miniKanren}~--- a
specifically designed for this purpose embedded DSL. Initially developed for \textsc{Scheme}/\textsc{Racket}
\textsc{miniKanren} later was ported for dozens of host
languages\footnote{\url{http://minikanren.org/\#implementations}}.
We, specifically, use a strongly typed \textsc{miniKanren} implementation for \textsc{OCaml}~\cite{ocaml}, called \textsc{OCanren}~\cite{OCanren}.
\textsc{miniKanren} uses the same theory of Horn clauses as \textsc{Prolog} but with a different
concrete syntax with explicit unification, conjunction, disjunction and fresh variable introduction, and
employs a different \emph{interleaving} search strategy~\cite{interleaving}, which is known to be complete~\cite{certified}.
Besides unification with occurs-check, enabled by default, \textsc{miniKanren} can be equipped with other
basic constraints like disequality constraint~\cite{disuni}, finite-domain constraints~\cite{cKanren}, or
constructs of nominal logic~\cite{aKanren}.

In the context of our work the most valuable property of \textsc{miniKanren} is its capability of expressing \emph{reverse computations}.
It is well-known~\cite{SemanticsModifiers,SemanticsModifiers1} that some complicated programs can be constructed as
the results of inversion of some other, much simpler, programs. In particular, a \emph{solver} for a
certain search problem can be considered as an inversion of its \emph{verifier}; it is rather a matter of common knowledge that verifying a
solution is, as a rule, much easier than finding one. The relational nature of \textsc{miniKanren} makes
inverse computations particularly easy~\cite{searchproblems}, which opens a way for program
synthesis~\cite{Untagged,WBirdSeven,PatternMatching}.

Another component of our approach is \emph{relational conversion}. In many cases (but not always!) it is easier to obtain a relational specification
from functional program than writing the one by hands. We use a tool, called \textsc{noCanren}, which converts programs written is a reasonable
subset of \textsc{OCaml} into correct-by-construction\cite{conversion} \textsc{OCanren} specifications.

We demonstrate the roadmap of our approach by the following observable example. Let us have a program
shown in Fig.~\ref{fun_vs_rel}\subref{funadd} which adds two natural numbers in Peano form. Its relational counterpart,
acquired via relational conversion (not literally, but in equivalent form), is shown in Fig.~\ref{fun_vs_rel}\subref{reladd}.
Comparing both of them we can notice that pattern-matching was replaced by disjunction (\lstinline[language=ocanren,basicstyle=\small]|\/|)
and unification (\lstinline[language=ocanren,basicstyle=\small]|===|), data-flow dependent computations with conjunction (\lstinline[language=ocanren,basicstyle=\small]|/\|),
and fresh variables can be allocated when needed; the postfix ``$^o$'' is traditionally used to distinguish relational definitions. Unlike its functional
counterpart the relational specification has \emph{three} arguments \lstinline[language=ocanren,basicstyle=\small]|x|, \lstinline[language=ocanren,basicstyle=\small]|y|,
and \lstinline[language=ocanren,basicstyle=\small]|z|, each of which can contain fresh (initially undefined) variables.
The evaluation of construction \lstinline[language=ocanren,basicstyle=\small]|run$^*$ {add$^o$ x y z}| returns all substitutions for these
fresh variables which make the relation $\{(\mbox{\texttt x}, \mbox{\texttt y}, \mbox{\texttt z})\in\mathbb{N}^3\, |\, \mbox{\texttt x+y=z}\}$ hold, represented as a \emph{lazy stream of answers}. This stream can
then be inspected in a host functional application to retrieve individual answers. Thus the same specification can be equally used for plain addition,
subtraction or decomposition of a number into two summands.

\begin{figure}[t]
  \begin{subfigure}[t]{0.5\textwidth}
  \begin{lstlisting}[language=ocanren,basicstyle=\small]
   let rec add x y =
     match x with
     | O    -> y
     | S x' -> S (add x' y)
  \end{lstlisting}
  %\vskip12mm
  \caption{Functional addition}
  \label{funadd}
  \end{subfigure}
  \begin{subfigure}[t]{0.5\textwidth}
    \begin{lstlisting}[language=ocanren,basicstyle=\small]
  let rec add$^o$ x y z = ocanren {
    x === O /\ z === y \/
    fresh x', z' in
      x === S x' /\
      z === S z' /\
      add$^o$ x' y z'
  }
    \end{lstlisting}
    \caption{Relational addition}
    \label{reladd}
  \end{subfigure}
  \caption{Functional vs. relational addition of Peano numbers}
  \label{fun_vs_rel}
\end{figure}
