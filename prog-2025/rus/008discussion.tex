% !TeX spellcheck = ru_RU
% !TEX root = mainrus.tex

\section{Discussion and Future Work}

Данная работа посвящена синтезу  GUI на основе гайдлайна.
Наш подход позволяет автоматически располагать элементы управления GUI так, что результат по построению соответствует набору правил, сформулированных дизайнером.
Прототип реализации позволяет синтезировать реалистичные примеры промышленных интерфейсов, с учетом промышленных гайдлайнов за разумное время.

%We presented the results of our work on guideline-based synthesis of GUI layouts. The approach we take makes it possible to
%automatically build GUI layouts which by construction comply with a designer-specified set of rules. The prototype implementation
%we developed allows to synthesize the layouts for the real-word industrial GUI components w.r.t. the real-world guidelines in
%appropriate time.

Одним из интересных вопросов, является необходимость использования реляционного программирования для решения нашей задачи.
Да, набор правил, регламентирующих систему переписывания, в принципе, может быть реализован без использования реляционного подхода.
Однако мы считаем, что в этом случае большая часть работы по обоснованию корректности решения должна быть повторена заново.
В нашем же случае обоснование тривиально следует из полноты поиска \textsc{miniKanren} и полноты по опровержению (англ. refutational completeness) нашего решения.
Мы также предсказываем, что решение потребует изобретения заново недетерминизма и поиска с возвратами, которые уже есть в реляционном программировании.
Также дуальность между экземплярами в структуре и реляционными отношениями, изначально не ожидаемая нами, по нашему мнению, демонстрирует, что реляционное программирование --- это подходящий подход для решения поставленной задачи.

%One interesting question which may arise is if the application of relational programming is essential for this problem to be solved. Indeed, the set of guidelines
%describes a matching (or rewriting) system which, in principle, can be directly implemented without any use of relational techniques. We argue, however,
%that in this case a whole piece of work on justification of the correctness of the solution would have been repeated anew. In our
%case, the justification trivially follows from the completeness of the \textsc{miniKanren} search and refutational completeness of our solution. We also speculate
%that such a solution would require reinventing of some implementation techniques to support nondeterminism and backtracking, which are already native to relational
%programming. Finally, the duality between patterns over structure and relational goals (initially unexpected for us), to our opinion, witnesses, that relational
%programming is a truly relevant technique for this problem.
%
%
%
