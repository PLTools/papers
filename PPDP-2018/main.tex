\documentclass[sigconf]{acmart}

\usepackage{booktabs} % For formal tables
\usepackage{amssymb}
\usepackage{amsmath}
\usepackage{mathrsfs}
\usepackage{mathtools}
\usepackage{multirow}
\usepackage{listings}
\usepackage{indentfirst}
\usepackage{verbatim}
\usepackage{amsmath, amssymb}
\usepackage{graphicx}
\usepackage{xcolor}
\usepackage{url}
\usepackage{stmaryrd}
\usepackage{xspace}
\usepackage{comment}
\usepackage{wrapfig}
\usepackage[caption=false]{subfig}
\usepackage{placeins}
\usepackage{tabularx}
\usepackage{ragged2e}

\newtheorem{theorem}{Theorem}
\newtheorem{lemma}{Lemma}

\def\transarrow{\xrightarrow}
\newcommand{\setarrow}[1]{\def\transarrow{#1}}

\def\padding{\phantom{X}}
\newcommand{\setpadding}[1]{\def\padding{#1}}

\def\subarrow{}
\newcommand{\setsubarrow}[1]{\def\subarrow{#1}}

\newcommand{\trule}[2]{\frac{#1}{#2}}
\newcommand{\crule}[3]{\frac{#1}{#2},\;{#3}}
\newcommand{\withenv}[2]{{#1}\vdash{#2}}
\newcommand{\trans}[3]{{#1}\transarrow{\padding#2\padding}\subarrow{#3}}
\newcommand{\ctrans}[4]{{#1}\transarrow{\padding#2\padding}\subarrow{#3},\;{#4}}
\newcommand{\llang}[1]{\mbox{\lstinline[mathescape]|#1|}}
\newcommand{\pair}[2]{\inbr{{#1}\mid{#2}}}
\newcommand{\inbr}[1]{\left<{#1}\right>}
\newcommand{\highlight}[1]{\color{red}{#1}}
\newcommand{\ruleno}[1]{\eqno[\scriptsize\textsc{#1}]}
\newcommand{\rulename}[1]{\textsc{#1}}
\newcommand{\inmath}[1]{\mbox{$#1$}}
\newcommand{\lfp}[1]{fix_{#1}}
\newcommand{\gfp}[1]{Fix_{#1}}
\newcommand{\vsep}{\vspace{-2mm}}
\newcommand{\supp}[1]{\scriptsize{#1}}
\renewcommand{\G}{\mathfrak G}
\newcommand{\sembr}[1]{\llbracket{#1}\rrbracket}
\newcommand{\cd}[1]{\texttt{#1}}
\newcommand{\miniKanren}{miniKanren\xspace}
\newcommand{\ocanren}{OCanren\xspace}
\newcommand{\free}[1]{\boxed{#1}}
\newcommand{\binds}{\;\mapsto\;}
\newcommand{\dbi}[1]{\mbox{\bf{#1}}}
\newcommand{\sv}[1]{\mbox{\textbf{#1}}}
\newcommand{\bnd}[2]{{#1}\mkern-9mu\binds\mkern-9mu{#2}}

\newcommand{\meta}[1]{{\mathcal{#1}}}
\renewcommand{\emptyset}{\varnothing}

\lstdefinelanguage{ocanren}{
keywords={fresh, let, in, match, with, when, class, type,
object, method, of, rec, repeat, until, while, not, do, done, as, val, inherit,
new, module, sig, deriving, datatype, struct, if, then, else, open, private, virtual, include, success, failure,
true, false},
sensitive=true,
commentstyle=\small\itshape\ttfamily,
keywordstyle=\ttfamily\underbar,
identifierstyle=\ttfamily,
basewidth={0.5em,0.5em},
columns=fixed,
fontadjust=true,
literate={fun}{{$\lambda$}}1 {->}{{$\to$}}3 {===}{{$\equiv$}}1 {=/=}{{$\not\equiv$}}1 {|>}{{$\triangleright$}}3 {\\/}{{$\vee$}}2 {/\\}{{$\wedge$}}2 {^}{{$\uparrow$}}1,
morecomment=[s]{(*}{*)}
}

\lstset{
mathescape=true,
%basicstyle=\small,
identifierstyle=\ttfamily,
keywordstyle=\bfseries,
commentstyle=\scriptsize\rmfamily,
basewidth={0.5em,0.5em},
fontadjust=true,
language=ocanren
}

\usepackage{letltxmacro}
\newcommand*{\SavedLstInline}{}
\LetLtxMacro\SavedLstInline\lstinline
\DeclareRobustCommand*{\lstinline}{%
  \ifmmode
    \let\SavedBGroup\bgroup
    \def\bgroup{%
      \let\bgroup\SavedBGroup
      \hbox\bgroup
    }%
  \fi
  \SavedLstInline
}

\sloppy

\begin{document}

\title{Improving Refutational Completeness\\
of Relational Search via Divergence Test$^*$}

\thanks{$^*\;$This work is supported by RFBR grant No 18-01-00380.}

%\titlenote{Produces the permission block, and
%  copyright information}

\author{Dmitri Rozplokhas}
\affiliation{%
  \institution{St. Petersburg Academic University}
  \streetaddress{Khlopina st., 8-3-А}
  \city{St. Petersburg}
  \state{Russia}
  \postcode{194021}
}
\email{rozplokhas@gmail.com}

\author{Dmitri Boulytchev}
\affiliation{%
  \institution{St. Petersburg State University}
  \streetaddress{Universitetskaya emb., 7-9}
  \city{St. Petersburg}
  \state{Russia}
  \postcode{199034}
}
\email{dboulytchev@math.spbu.ru}

\begin{abstract}
We describe a search optimization technique for implementation of relational programming language
miniKanren which makes more queries converge. Specifically, we address the problem of conjunction
non-commutativity. Our technique is based on a certain divergence criterion that we use to trigger a
dynamic reordering of conjuncts. We present a formal semantics of a miniKanren-like language and prove
that our optimization does not compromise already converging programs, thus, being a proper improvement.
We also present the prototype implementation of the improved search and demonstrate its application for a
number of realistic specifications.
\end{abstract}

%
% The code below should be generated by the tool at
% http://dl.acm.org/ccs.cfm
% Please copy and paste the code instead of the example below.
%
\begin{CCSXML}
<ccs2012>
<concept>
<concept_id>10003752.10003790.10003795</concept_id>
<concept_desc>Theory of computation~Constraint and logic programming</concept_desc>
<concept_significance>500</concept_significance>
</concept>
<concept>
<concept_id>10003752.10010124.10010131.10010134</concept_id>
<concept_desc>Theory of computation~Operational semantics</concept_desc>
<concept_significance>100</concept_significance>
</concept>
<concept>
<concept_id>10011007.10011006.10011008.10011009.10011015</concept_id>
<concept_desc>Software and its engineering~Constraint and logic languages</concept_desc>
<concept_significance>500</concept_significance>
</concept>
</ccs2012>
\end{CCSXML}

\ccsdesc[500]{Theory of computation~Constraint and logic programming}
\ccsdesc[100]{Theory of computation~Operational semantics}
\ccsdesc[500]{Software and its engineering~Constraint and logic languages}

\keywords{relational programming, refutational completeness, divergence test}

\copyrightyear{2018}
\acmYear{2018}
\setcopyright{acmcopyright}
\acmConference[PPDP '18]{The 20th International Symposium on Principles and Practice of Declarative Programming}{September 3--5, 2018}{Frankfurt am Main, Germany}
\acmBooktitle{The 20th International Symposium on Principles and Practice of Declarative Programming (PPDP '18), September 3--5, 2018, Frankfurt am Main, Germany}
\acmPrice{15.00}
\acmDOI{10.1145/3236950.3236958}
\acmISBN{978-1-4503-6441-6/18/09}

\maketitle

\section{Introduction}
\label{intro}

Relational programming is an attractive technique, based on the idea of constructing programs as relations.
While in general some relational effects can be reproduced with a number of languages for logic programming, such as
Prolog, Mercury\footnote{\url{https://mercurylang.org}}, or Curry\footnote{\url{http://www-ps.informatik.uni-kiel.de/currywiki}}, in
a narrow sense relational programming amounts to writing relational specifications in \miniKanren~\cite{TRS}. \miniKanren\footnote{\url{http://minikanren.org}},
initially designed as a small relational DSL, embedded in Scheme/Racket, was later implemented for a number of general-purpose host languages,
including Scala, Haskell, Standard ML and OCaml.

With relational approach, it becomes possible to give simple and elegant solutions for the problems, otherwise
considered as tricky, tough, tedious, or boring~\cite{unified}. For example, relational interpreters can be used to derive
\emph{quines}~--- programs, which reduce to themselves, as well as \emph{twines} or \emph{thrines} (pairs or triples of
programs, reducing to each other)~\cite{Untagged}; a straightforward relational description of
simply typed lambda calculus~\cite{Lambda} inference rules works both as type inferencer and inhabitation problem solver~\cite{WillThesis};
relational list sorting can be used to generate all permutations~\cite{ocanren}, etc. 

On the other hand, writing relational specifications can sometimes be a tricky and error-prone task. Fortunately, many 
specifications can be written systematically by ``generalizing'' a certain functional program. From the very beginning, 
the conversion from functional to relational form was considered as an element of relational programming thesaurus~\cite{TRS}. However,
the traditional approach~--- \emph{unnesting}~--- was formulated for an untyped case, worked only for specifically written
programs and was never implemented.

We present a generalized form of relational conversion, which can be applied to typed terms in general form. We study the relational conversion 
for a small ML-like language (essentially, a certain subset of OCaml), equipped with Hindley-Milner type system with let-polymorphism~\cite{Types}. 
We start from retelling the syntax, typing rules, and operational semantics, and then extend the source language with a conventional set of 
relational constructs. This set corresponds to existing typed embedding of \miniKanren into OCaml~\cite{ocanren}. We then present typing rules and 
develop operational semantics for this relational extension; to our knowledge, this is the first attempt to specify formal semantics for
\miniKanren. Next, we develop formal rules for relational conversion and prove, that these rules respect both typing and
semantics. Finally, we describe the implementation of a relational converter and demonstrate its application for a number of problems, for some
of which we present a relational solution for the first time.

We would like to express our gratitude to William Byrd and the anonymous reviewers for their constructive remarks, which, we believe, led to the
improvement of the presentation. 

\section{The Source Language and Relational Extension}

Our development of relational conversion is based on the idea of transforming a program in a functional
language into a program in \emph{relational extension} of that language. In the context of 
miniKanren, this approach looks quite natural, since miniKanren itself, as a DSL, reuses 
many important features (for example, function definitions) from a host language.

In this section, we present a formal description of a small functional language, taken as a source
for relational conversion. We describe its syntax, typing rules, and semantics, and then extend it
with relational constructs. We specify the typing rules and semantics for the extension as well.


\subsection{The Source Language}
\label{source_language}

The syntax of our source functional language is shown on Figure~\ref{functional_syntax}. It consists of a lambda calculus, 
enriched with constructors with fixed arities $C^n$, patterns $p$ and pattern-matching constructs, and  
expressions for recursive/non-recursive let-bindings.
Among the constructors we distinguish two nullary interpreted constructors \lstinline|true| and \lstinline|false|, and add a boolean equality
operator ``$=$''. 

\begin{wrapfigure}{r}{0.5\textwidth}
\centering
%\scalebox{0.9}{
$$
\begin{array}{rcl}
 \mathcal E &=&x\\
     & &\lambda x.e\\
     & &e_1\;e_2\\
     & &C^n(e_1,\dots, e_n)\\
     & &\lstinline|true|\\
     & &\lstinline|false|\\
     & &\lstinline|let $x$ = $e_1$ in $e_2$|\\
     & &\lstinline|let rec $f$ = $\lambda x.e_1$ in $e_2$|\\
     & &e_1\,=\,e_2\\
     & &\lstinline|match $e$ with $\{p_i$ -> $e_i\}$|\\
     & &\\
 \mathcal P &=&C^n(x_1,\dots,x_n)\\
\end{array}
$$
%}
\caption{The syntax of source language}
\label{functional_syntax}
\end{wrapfigure}

In a pattern matching, we only allow shallow patterns (which is not an essential limitation) and do not allow wildcards (which is 
important~--- converting wildcard pattern matching into relational form would require essentially different projections).

\begin{comment}
This choice of a language may 
look quite a restrictive. However, in terms of relational programming, the language contains virtually everything one would need. Indeed, from
a relational conversion standpoint the standard built-in integer arithmetics, for example, is of no use~--- 
there is simply no way to convert integer expression into relational form, using integer expressions. In order to use relational 
arithmetics, one needs to reimplement everything from scratch, using, for example, Peano encoding; but Peano arithmetics can be
easily expressed in the language we present.
\end{comment}

Our language is equipped with Hindley-Milner type system, and we present the typing rules in a conventional syntax-directed form 
on Fig.~\ref{functional_typing}. Besides type variables and function types, our system contains a number of implicitly defined 
algebraic datatypes $T^k$, and we stipulate, that each constructor $C^n$ belongs to the exactly one
datatype. In the rule \textsc{Constr$_T$}, we assume that type $t^C$ has the form $T^k(t_1,\dots,t_k)$, where each of the types
$t_i$ is recovered from the types $t_i^C$ of arguments of constructor $C^n$ and, moreover, these types agree in the sense of
constructor application. Similarly, in the rule \textsc{Match$_T$}, the types of all $C_i^{k_i}(x^i_1,\dots,x^i_{k_i})$ are expected
to be equal $t^C$, and $t^{C_i}_j$ is a type of $j$-th argument of constructor $C_i$, used in the pattern. The rule \textsc{Eq$_T$}
specifies that both operands of equality operator must have the same (but arbitrary) type. Thus, we can call this operator
``polymorphic equality''.

\setarrow{:}
\newcommand{\typed}[3]{\withenv{#1}{\trans{#2}{}{#3}}}

\begin{figure}
\centering
{\bf Types:}
$$
\begin{array}{rcll}
  \mathcal X &=&\alpha, \beta, \dots                            &\mbox{\supp{(type variables)}}\\
  \mathcal D &=&\lstinline|bool|,\,T^n,...                      &\mbox{\supp{(datatype constructors)}}\\
  \mathcal T &=&\alpha\mid T^k(t_1,\dots,t_k)\mid t_1\to t_2 &\mbox{\supp{(types)}}\\
  \mathcal S &=&\forall\bar{\alpha}.t                           &\mbox{\supp{(type schemas)}}
\end{array}
$$
{\bf Typing rules:}
\def\arraystretch{0}
\begin{tabular}{p{7cm}p{7cm}}
$$
\typed{\Gamma}{\lstinline|true|,\;\lstinline|false|}{\lstinline|bool|}
\ruleno{Bool$_T$}
$$ 
&
$$
\trule{\typed{\Gamma}{e_1}{t}\;\;\;\;\typed{\Gamma}{e_2}{t}}
      {\typed{\Gamma}{e_1=e_2}{\lstinline|bool|}}
\ruleno{Eq$_T$}
$$
\\
$$
\trule{\typed{\Gamma}{e_i}{t^C_i}}
      {\typed{\Gamma}{C^n(e_1,\dots,e_n)}{t^C}}
\ruleno{Constr$_T$}
$$
&
$$
\typed{\Gamma,x:\forall\bar{\alpha}.t}{x}{t[\bar{\alpha}\gets\bar{t^\prime}]}
\ruleno{Var$_T$}
$$
\\[-2mm]
$$
\trule{\typed{\Gamma}{f}{t_1\to t_2}\;\;\;\;\typed{\Gamma}{e}{t_1}}
      {\typed{\Gamma}{f\;e}{t_2}}
\ruleno{App$_T$}
$$
&
$$
\trule{\typed{\Gamma,\,x:t_1}{f}{t_2}}
      {\typed{\Gamma}{\lambda x.f}{t_1\to t_2}}
\ruleno{Abs$_T$}
$$
\\[-2mm]
\multicolumn{2}{p{14cm}}{
$$
\trule{\typed{\Gamma}{e_1}{t_1}\;\;\;\;\typed{\Gamma,x:\forall\bar{\alpha}.t_1}{e_2}{t}}
      {\typed{\Gamma}{\lstinline|let $\;x\;$ = $\;e_1\;$ in $\;e_2$|}{t}},\;\bar{\alpha}=FV(t_1)\setminus FV(\Gamma)
\ruleno{Let$_T$}
$$}\\[-2mm]
\multicolumn{2}{p{14cm}}{
$$
\trule{\typed{\Gamma,f:t_1}{\lambda x.e_1}{t_1}\;\;\;\;\typed{\Gamma,f:\forall\bar{\alpha}.t_1}{e_2}{t}}
      {\typed{\Gamma}{\lstinline|let rec $\;f\;$ = $\;\lambda x.e_1\;$ in $\;e_2$|}{t}},\;\bar{\alpha}=FV(t_1)\setminus FV(\Gamma)
\ruleno{LetRec$_T$}
$$}\\[-2mm]
\multicolumn{2}{p{14cm}}{
$$
\trule{\typed{\Gamma}{e}{t^C}\;\;\;\;\typed{\Gamma,x^i_1:t^{C_i}_1,\dots,x^i_{k_i}:t^{C_i}_{k_i}}{e_i}{t}}
      {\typed{\Gamma}{\lstinline|match $\;e\;$ with $\;\{C_i^{k_i}(x^i_1,\dots,x^i_{k_i})$ -> $e_i\}$|}{t}}
\ruleno{Match$_T$}
$$}
\end{tabular}
\caption{Typing rules for the source language}
\label{functional_typing}
\end{figure}


\setarrow{\to}
\newcommand{\step}[2]{\trans{\inbr{#1}}{}{\inbr{#2}}}

\begin{figure}[t]
\centering
{\bf Values:}
$$
\mathcal V = C^n(v_1,\dots,v_n)\mid\lambda x.e\mid\mu f\lambda x.e\mid\lstinline|true|\mid\lstinline|false|
$$
{\bf Contexts:}
$$
\mathcal C = \Box\;e\mid v\;\Box\mid\lstinline|let $x$ = $\Box$ in $e$|\mid\lstinline|match $\;\Box\;$ with $\{p_i$->$e_i\}$|\mid C^n(\bar{v},\Box,\bar{e})\mid\Box\lstinline|=e|\mid\lstinline|v=|\Box
$$
{\bf Stack of contexts:}
$$
\mathcal S=\epsilon\mid\mathcal C : \mathcal S
$$
{\bf States:}
$$
\inbr{\mathcal S, e}\mbox{\supp{(stack of contexts, expression)}};\;\inbr{\epsilon,e}\mbox{\supp{(initial state)}};\;\inbr{\epsilon,v}\mbox{\supp{(final state)}}
$$
{\bf Transitions:}
\vskip2mm
\bgroup
\def\arraystretch{0}
\begin{tabular}{p{7cm}p{7cm}}
\multicolumn{2}{p{14cm}}{
$$
\step{C:\mathcal S,\, v}{\mathcal S,\, C[v]}\ruleno{Value}
$$}\\[-4mm]
$$
\step{\mathcal S,\, f\;e}{\Box\;e:\mathcal S,\, f}\ruleno{AppL}
$$&
$$
\step{\mathcal S,\, v\;e_2}{v\;\Box:\mathcal S,\, e_2}\ruleno{AppR}
$$\\[-4mm]
$$
\step{\mathcal S,\,e_1=e_2}{\Box=e_2:\mathcal S,\,e_1}\ruleno{EqL}
$$&
$$
\step{\mathcal S,\,v=e}{v=\Box:\mathcal S,\,e}\ruleno{EqR}
$$\\[-4mm]
\multicolumn{2}{p{14cm}}{
$$
\step{\mathcal S,\,v=v}{\mathcal S,\,\lstinline|true|}\ruleno{EqTrue}
$$}\\[-4mm]
\multicolumn{2}{p{14cm}}{
$$
\step{\mathcal S,\,v_1=v_2}{\mathcal S,\,\lstinline|false|},\;v_1\ne v_2\ruleno{EqFalse}
$$}\\[-4mm]
\multicolumn{2}{p{14cm}}{
$$
\step{\mathcal S,\, (\lambda x.e)\;v}{\mathcal S,\, e[x\gets v]}\ruleno{Beta}
$$}\\[-4mm]
\multicolumn{2}{p{14cm}}{
$$
\step{\mathcal S,\, (\mu f\lambda x.e)\;v}{\mathcal S,\, e[f\gets\mu f\lambda x.e,\, x\gets v]}\ruleno{Mu}
$$}\\[-4mm]
\multicolumn{2}{p{14cm}}{
$$
\step{\mathcal S,\, C^n(v_1,\dots,v_{k-1},e_k,\dots,e_n)}{C^n(v_1,\dots,v_{k-1},\Box,\dots,e_n):\mathcal S,\, e_k}\ruleno{Constr}
$$}\\[-4mm]
\multicolumn{2}{p{14cm}}{
$$
\step{\mathcal S,\, \lstinline|let $\;x\;$ = $\;e_1\;$ in $\;e_2$|}{\lstinline|let $\;x\;$ = $\;\Box\;$ in $\;e_2$|:\mathcal S,\, e_1}\ruleno{Let}
$$}\\[-4mm]
\multicolumn{2}{p{14cm}}{
$$
\step{\mathcal S,\, \lstinline|let $\;x\;$ = $\;v\;$ in $\;e$|}{\mathcal S,\,e[x\gets v]}\ruleno{LetVal}
$$}\\[-4mm]
\multicolumn{2}{p{14cm}}{
$$
\step{\mathcal S,\, \lstinline|let rec $\;f\;$ = $\;\lambda x.e_1\;$ in $\;e_2$|}{\mathcal S,\, e_2[f\gets\mu f\lambda x.e_1]}\ruleno{LetRec}
$$}\\[-4mm]
\multicolumn{2}{p{14cm}}{
$$
\step{\mathcal S,\,\lstinline|match $\;e\;$ with $\;\{p_i$->$e_i\}$|}{\lstinline|match $\;\Box\;$ with $\;\{p_i$->$e_i\}$|:\mathcal S,\, e}\ruleno{Match}
$$}\\[-4mm]
\multicolumn{2}{p{14cm}}{
$$
\step{\mathcal S,\,\lstinline|match $\;C_k^{n_k}(v_1,\dots,v_{n_k})\;$ with $\;\{C_i^{n_i}(x^i_1,\dots,x^i_{n_i})\to e_i\}$|}{\mathcal S,\,e_k[x^k_j\gets v_j]}\ruleno{MatchVal}
$$}
\end{tabular}
\egroup
\caption{Semantics for the source language}
\label{functional_semantics}
\end{figure}

We describe the semantics of our language in the form of transition system. The transition relation

$$
\step{\mathcal S,\,e}{\mathcal S^\prime,\,e^\prime}
$$

\noindent describes a one step of evaluation of expression $e$ with a stack of contexts $\mathcal S$, which results in
a new stack $\mathcal S^\prime$ and a new expression $e^\prime$. A context is an expression with a unique hole; informally speaking, 
a stack of contexts describes a path in the expression being evaluated from the topmost construct to the point, where the evaluation 
currently is taking place. For a context $C$ and an expression $e$, we denote by $C[e]$ a complete expression with no holes, which is 
obtained by plugging $e$ into the unique hole of $C$. From each state $\inbr{C_1:C_2:\dots:C_k,e}$ we can build an 
expression $C_k[\dots[C_2[C_1[e]]]\dots]$, which represents an intermediate result of evaluation according to a small-step semantics. 
This form of semantic description originates from Felleisen-style~\cite{Felleisen} approach for small-step semantics, and we've
chosen it since it can be naturally extended for a relational case.

Our semantics describes call-by-value left-to-right evaluation; in the rules $\textsc{Beta}$, $\textsc{Mu}$, $\textsc{LetVal}$,
$\textsc{LetRec}$ and $\textsc{MatchVal}$, we perform capture-avoiding substitutions, which respect the names in abstractions and let-bindings.
In the rule $\textsc{MatchVal}$ we assume, that at most one pattern matches the scrutinee~--- this is an important difference from the usual 
semantics of pattern matching, when the patterns are examined in a top-down manner until the matching succeeds. In the rules $\textsc{EqTrue}$
and $\textsc{EqFalse}$ we assume, that the values $v$, $v_1$, $v_2$ do not have the forms $\lambda x\dots$ or $\mu f\dots$.

Finally, for a closed expression $e$ and a value $v$, we write $e \leadsto^f v$, iff 

$$\inbr{\epsilon,e}\to^*\inbr{\epsilon,v}$$

\noindent where $\epsilon$~--- an empty stack, and ``$\to^*$'' is a reflexive-transitive closure of ``$\to$''. 


\subsection{Relational Extension}
\label{relational_extension}

The relational extension adds five conventional miniKanren expressions for constructing goals; the syntax is shown on Fig.~\ref{relational_syntax}.
Since relational constructs are added to regular functional ones, it becomes possible to construct expressions like \lstinline|fun x.(x /\ fun y.y)|, etc.
In order to rule such pathological expressions out, we devised an extension for the type system of the source language. In fact, this approach follows the
actual implementation for OCaml, where a careful choice of types for representing terms and goals made it possible to reject the majority of non-well-formed
programs at compile-time.

Our extension for the type system introduces one interpreted datatype constructor $\Box^o$ with one data constructor $\uparrow$~--- a polymorphic type and
a constructor for logical terms. In addition, we introduce an interpreted type of goals $\G$, which is distinct from all other types. The typing rules for the relational 
extension are shown on Fig.~\ref{relational_typing}. These rules describe rather expected typing: in unification and disequality constraints only
terms of the same logical type can be used, and conjunction and disjunction can only be taken for goals. Note, in our extension a term can be calculated as
a result of arbitrary expression in initial functional language (as long as this expression has expected logical type), but such ``higher-order'' terms will
never appear as a result of relational conversion, so, in fact, relational extension we describe here defines a richer language, than we actually need.

The semantics of extended language is shown on Fig.~\ref{relational_semantics}. First, the state is extended: besides the stack of contexts and
current expression it now contains a set of used \emph{semantic variables} $\Sigma$ and a \emph{logical state} $\sigma$. 
Semantic variables are allocated and substituted for syntactic logic variable occurrences, when \lstinline|fresh| expression is evaluated 
(see rule \textsc{Fresh}). Logical states are affected, when unification or disequality constraint is evaluated; we explain them
in details below. All existing rules for the initial language are considered rewritten to propagate newly added components of states unchanged.
Then, we modify the substitution to respect names, bound in \lstinline|fresh| as well. 
Next, we consider two new kinds of values: a semantic variable and a special value \lstinline|success|. The former is a result of evaluation for
a free logic variable, the latter~--- the result of evaluation for a succeeded goal.

\begin{wrapfigure}[11]{r}{0.5\textwidth}  
  \centering
  \vspace{-11pt}
  $$
  \begin{array}{rl}
    \mathcal E\mathrel{{+}{=}}&\lstinline|fresh ($x$) $\;e$|\\
    &e_1\equiv e_2\\
    &e_1\not\equiv e_2\\
    &e_1\vee e_2\\
    &e_1\wedge e_2
  \end{array}
  $$
  \caption{The syntax of relational extension}
  \label{relational_syntax}
\end{wrapfigure}

We also extend the definition of context to handle the new kinds of expressions. In unification and disequality constraint, the terms are evaluated left-to-right.
Conjunction and disjunction, however, evaluate nondeterministically: in disjunction only one subgoal is chosen (see rules \textsc{DisjL} and \textsc{DisjR}),
a conjunction can evaluate either left, or right subgoal first (see rules \textsc{ConjStartL} and \textsc{ConjStartR}). When chosen subgoal is evaluated
to the value \lstinline|success|, the other subgoal starts its evaluation (rules \textsc{ConjL} and \textsc{ConjR}).
We have chosen a nondeterministic variant for the semantics, since different existing miniKanren implementations use (a little bit) different search, and we do 
not want to depend on the implementation details. An opposite side of this solution is that for a concrete program and a concrete miniKanren implementation,
the result of the evaluation might not coincide with that, prescribed by the semantics: in some concrete implementation a program can diverge, while
nondeterministic semantics may still define a certain scenario to complete with a result. We argue, that in this case, it will always be possible to
rewrite a program or/and interpreter to converge according to that scenario.
\FloatBarrier

\setarrow{:}
\begin{figure}[t]
\centering
{\bf Types:}
$$
\begin{array}{rcl}
 \mathcal L &=               &\alpha^o \mid (T^n(l_1,\dots,l_n))^o\;\;\mbox{\supp{(logical types)}}\\
 \mathcal T &\mathrel{{+}{=}}&\G
\end{array}
$$
{\bf Typing rules:}
\def\arraystretch{0}
\begin{tabular}{p{7cm}p{7cm}}
\multicolumn{2}{p{14cm}}{
$$
\trule{\typed{\Gamma,x:l}{e}{\G}}
      {\typed{\Gamma}{\lstinline|fresh ($x$) $\;e$|}{\G}}
\ruleno{Fresh$_T$}
$$}\\[-2mm]
$$
\trule{\typed{\Gamma}{e_1}{l}\;\;\;\;\typed{\Gamma}{e_2}{l}}
      {\typed{\Gamma}{e_1\equiv e_2}{\G}}
\ruleno{Unify$_T$}
$$&
$$
\trule{\typed{\Gamma}{e_1}{l}\;\;\;\;\typed{\Gamma}{e_2}{l}}
      {\typed{\Gamma}{e_1\not\equiv e_2}{\G}}
\ruleno{Disequality$_T$}
$$\\[-2mm]
$$
\trule{\typed{\Gamma}{e_1}{\G}\;\;\;\;\typed{\Gamma}{e_2}{\G}}
      {\typed{\Gamma}{e_1\wedge e_2}{\G}}
\ruleno{Conjunction$_T$}
$$&
$$
\trule{\typed{\Gamma}{e_1}{\G}\;\;\;\;\typed{\Gamma}{e_2}{\G}}
      {\typed{\Gamma}{e_1\vee e_2}{\G}}
\ruleno{Disjunction$_T$}
$$
\end{tabular}
\caption{Typing rules for the relational extension}
\label{relational_typing}
\end{figure}

\setarrow{\leadsto}
\def\arraystretch{0}
\begin{figure}[t]
\centering
{\bf Semantic variables:}\vspace{-2mm}
\begin{gather*}
\mathfrak S = \mathfrak s_1, \mathfrak s_2, \dots\\[-2mm]
\Sigma, \Sigma^\prime\dots \subset 2^{\mathcal S}\;\mbox{\supp{(sets of allocated semantics variables)}}\\[-1mm]
\inbr{\Sigma^\prime, \mathfrak s}\gets\lstinline|new|\;\Sigma,\;\Sigma^\prime=\Sigma\cup\{\mathfrak s\},\;{\mathfrak s}\notin\Sigma\;\mbox{\supp{(allocation of a new semantic variable)}}\vspace{-2mm}
\end{gather*}
{\bf Values:}\vspace{-2mm}
$$
\mathcal V \mathrel{{+}{=}} \lstinline|success|\mid\mathfrak s
$$\vspace{-2mm}
{\bf Contexts:}\vspace{-2mm}
$$
\mathcal C \mathrel{{+}{=}}\Box\equiv e\mid v\equiv\Box\mid\Box\not\equiv e\mid v\not\equiv\Box\mid\Box\wedge e\mid e\wedge\Box
$$\vspace{-2mm}
{\bf States:}\vspace{-2mm}
\begin{gather*}
\inbr{\Sigma,\mathcal S,e,\sigma}\mbox{\supp{(set of allocated semantic variables, stack of contexts, expression, logical state)}}\\[-1mm]
\inbr{\emptyset,\epsilon,e,\iota}\mbox{\supp{(initial state)}}
\end{gather*}\vspace{-2mm}
{\bf Transitions:}\vspace{1mm}
{\def\arraystretch{0}
\begin{tabular}{p{14cm}}
$$
\step{\Sigma,\,\mathcal S,\,\lstinline|fresh($x$) $\;e$|,\,\sigma}{\Sigma^\prime,\,\mathcal S,\,e[x\gets\mathfrak s],\,\sigma},\,\inbr{\Sigma^\prime,\mathfrak s}\gets\lstinline|new|\;\Sigma\ruleno{Fresh}
$$\\[-4mm]
$$
\step{\Sigma,\,\mathcal S,\,e_1\equiv e_2,\,\sigma}{\Sigma,\,\Box\equiv e_2:\mathcal S,\,e_1,\,\sigma}\ruleno{UnifyL}
$$\\[-4mm]
$$
\step{\Sigma,\,\mathcal S,\,v\equiv e,\,\sigma}{\Sigma,\,v\equiv\Box:\mathcal S,\,e,\,\sigma}\ruleno{UnifyR}
$$\\[-4mm]
$$
\step{\Sigma,\,\mathcal S,\,v_1\equiv v_2,\,\sigma}{\Sigma,\,\mathcal S,\,\lstinline|success|,\,\sigma^\prime},\,{\bf unify}\,(\sigma,\,v_1,\,v_2)=\sigma^\prime\ruleno{Unify}
$$\\[-4mm]
$$
\step{\Sigma,\,\mathcal S,\,e_1\not\equiv e_2,\,\sigma}{\Sigma,\,\Box\not\equiv e_2:\mathcal S,\,e_1,\,\sigma}\ruleno{DisEqL}
$$\\[-4mm]
$$
\step{\Sigma,\,\mathcal S,\,v\not\equiv e,\,\sigma}{\Sigma,\,v\not\equiv\Box:\mathcal S,\,e,\,\sigma}\ruleno{DisEqR}
$$\\[-4mm]
$$
\step{\Sigma,\,\mathcal S,\,v_1\not\equiv v_2,\,\sigma}{\Sigma,\,\mathcal S,\,\lstinline|success|,\,\sigma^\prime},\,{\bf diseq}\,(\sigma,\,v_1,\,v_2)=\sigma^\prime\ruleno{DisEq}
$$\\[-4mm]
$$
\step{\Sigma,\,\mathcal S,\,e_1\vee e_2,\,\sigma}{\Sigma,\,\mathcal S,\,e_1,\,\sigma}\ruleno{DisjL}
$$\\[-4mm]
$$
\step{\Sigma,\,\mathcal S,\,e_1\vee e_2,\,\sigma}{\Sigma,\,\mathcal S,\,e_2,\,\sigma}\ruleno{DisjR}
$$\\[-4mm]
$$
\step{\Sigma,\,\mathcal S,\,e_1\wedge e_2,\,\sigma}{\Sigma,\,\Box\wedge e_2:\mathcal S,\,e_1,\,\sigma}\ruleno{ConjStartL}
$$\\[-4mm]
$$
\step{\Sigma,\,\mathcal S,\,e_1\wedge e_2,\,\sigma}{\Sigma,\,e_1\wedge\Box:\mathcal S,\,e_2,\,\sigma}\ruleno{ConjStartR}
$$\\[-4mm]
$$
\step{\Sigma,\,\mathcal S,\,\lstinline|success|\wedge e,\,\sigma}{\Sigma,\,\mathcal S,\,e,\,\sigma}\ruleno{ConjL}
$$\\[-4mm]
$$
\step{\Sigma,\,\mathcal S,\,e\wedge\lstinline|success|,\,\sigma}{\Sigma,\,\mathcal S,\,e,\,\sigma}\ruleno{ConjR}
$$
\end{tabular}}
\caption{Semantics for the relational extension}
\label{relational_semantics}
\end{figure}

Finally, we describe the structure of a logical state and the implementation of unification and disequality constraint. The development is mainly based on the existing implementation~\cite{CKanren} and standard approaches for implementing unification~\cite{Unification,UnificationRevisited}. We, therefore, assume the familiarity of the reader with the following notions:

\begin{itemize}
  \item substitution ($\theta$);
  \item application of substitution $\theta$ to a term $t$ ($t\,\theta$);
  \item composition of substitutions ($\theta\theta^\prime$);
  \item most general unifier of two terms ($mgu\,(t_1, t_2)$).
\end{itemize}

\begin{comment}

As it can be seen from the semantics and typing rules, a unification (or disequality constraint) can only
be applied to equally-typed logical values, and we consider substitutions to be partial functions from
semantic variables ($\mathfrak S$) to logical values.

Before giving the detailed description, we consider the following example, which is called to reveal the essence of unification and
disequality constraint coexistence. Note, only these two kinds of goals deliver new information~--- all other kinds (conjunction, disjunction, etc.) are rather needed to
put them in a right context or order. Let us have the following sequence of goals, which we have to evaluate one after another (we assume all logical 
variables are already properly allocated):

$$
\def\arraystretch{1}
\begin{array}{rcl}
x&\not\equiv&y\\
z&\not\equiv&y\\
y&\equiv&\lstinline|A|\\
x&\equiv&\lstinline|B|\\
z&\equiv&\lstinline|A|
\end{array}
$$

Since in the beginning we know nothing yet, we are unable to determine, if the first goal~--- \mbox{$x\not\equiv y$}~--- succeeds or fails right now. All we can do is to
remember the substitution \mbox{$[x\binds y]$} with the note, that we \emph{do not want} to allow it. We call this kind of substitutions \emph{negative}. Similarly,
after the next goal we have another negative substitution \mbox{$[z\binds y]$}. 

The next goal is the unification \mbox{$y\equiv\lstinline|A|$}. Obviously, it provides us with the substitution \mbox{$[y\binds\lstinline|A|]$}. What effect (if any)
should it have on the set of negative substitutions? Since the unification completely eliminates the variable $y$, replacing negative substitutions with
\mbox{$[x\binds\lstinline|A|]$}, \mbox{$[z\binds\lstinline|A|]$} looks reasonable.

The next goal is \mbox{$x\equiv\lstinline|B|$}. It extends the current substitution, making it \mbox{$[y\binds\lstinline|A|,\,x\binds\lstinline|B|]$}. Now we may
note, that since $x$ already became \lstinline|B|, it can never be \lstinline|A| anymore. Thus, we can drop \mbox{$[x\binds\lstinline|A|]$} from the set of
negative substitutions.

Finally, we unify $z$ with \lstinline|A|. This, however, gets us a substitution \mbox{$[z\binds\lstinline|A|]$}, which we have to disallow since it
matches with the negative one.

In other words, disequality constraints may not succeed or fail right away, and the results of unification have to be additionally matched against the set of 
previously encountered disequality constraints, which, in turn, can be altered by a unification or another disequality constraint. Now we can continue.
\end{comment}

A logical state contains two components

$$
\sigma=(\theta,\Theta^-)
$$

\noindent where $\theta$ is a substitution, $\Theta^-$~--- a set of negative substitutions, describing disequality constraints, 
which can potentially be violated. The initial state contains undefined substitution and empty set:

$$
\iota=(\bot,\emptyset)
$$

The effect of unification is described by the following primitive:

$$
{\bf unify}\,(\sigma,\,t_1,\,t_2)={\bf unify}\,((\theta,\Theta^-),\,t_1,\,t_2)
$$

First, it calculates the most general unifier for the terms under consideration w.r.t. current substitution:

$$
\rho=mgu\,(t_1\,\theta,t_2\,\theta)
$$

If there is no such $\rho$, the unification fails, and the evaluation terminates unsuccessfully. Otherwise,
$\rho$ has to be checked against the disequality constraints, represented by $\Theta^-$ (if $\Theta^-$ is empty, the
check succeeds immediately).

Being a substitution, $\rho$ at the same time can be considered as the following unification problem: we can try to unify a pair of terms 

$$
\begin{array}{rcl}
t_l&=&(\mathfrak s_1,\dots,\mathfrak s_k)\\
t_r&=&(\rho(\mathfrak s_1),\dots,\rho(\mathfrak s_k))
\end{array}
$$

\noindent where $\{\mathfrak s_i\}=dom\,(\rho)$. We pick every substitution $\theta^-\in\Theta^-$ and calculate 
the $mgu\,(t_l\,\theta^-,t_r\,\theta^-)$. There are three possible outcomes:

\begin{enumerate}
\item The unification fails. This means, that disequality constraint, represented by $\theta^-$, can no
longer be violated. We remove $\theta^-$ from $\Theta^-$ and continue with the next disequality constraint.
\item The unification succeeds with the empty substitution. This means, that 
disequality constraint, represented by $\theta^-$, is violated. The check stops, and the whole top-level 
unification fails.
\item The unification succeeds with a non-empty substitution $\theta^{\prime-}$. This means, that in order not to 
violate disequality constraint, represented by $\theta^-$, $\theta^{\prime-}$ has to be respected. We replace
$\theta^-$ with $\theta^{\prime-}$ in $\Theta^-$ and continue with the next disequality constraint.
\end{enumerate}

If the disequality check succeeds, by the end we have a modified set $\Theta^{\prime-}$, and we assume

$$
{\bf unify}\,((\theta,\Theta^-),\,t_1,\,t_2)=(\theta\rho,\Theta^{\prime-})
$$

The evaluation of disequality constraint is performed in a similar manner using the primitive

$$
{\bf diseq}\,(\sigma,\,t_1,\,t_2)={\bf diseq}\,((\theta,\Theta^-),\,t_1,\,t_2)
$$

First, the $mgu\,(t_1\,\theta,t_2\,\theta)$ is calculated. Again, there are three
possible cases:

\begin{enumerate}
\item The unification fails. This means, that disequality constraint is satisfied.
\item The unification succeeds with the empty substitution. This means, that disequality
constraint is violated.
\item The unification succeeds with a non-empty substitution $\theta^{\prime-}$. This means, that 
this substitution describes the disequality constraint, which has to be respected in
the future, so we add it to $\Theta^-$. 
\end{enumerate}

If disequality constraint succeeds, we obtain a (potentially) modified set $\Theta^{\prime-}$, and we
assume

$$
{\bf diseq}\,((\theta,\Theta^-),\,t_1,\,t_2)=(\theta,\Theta^{\prime-})
$$

Finally, for a closed goal $g$ and a logical state $\sigma$, we define $g \leadsto^r \sigma$, iff

$$
\inbr{\emptyset,\epsilon,g,\iota}\leadsto^*\inbr{\Sigma,\epsilon,\lstinline|success|,\sigma}\mbox{ for some $\Sigma$}
$$
 
\noindent where ``$\leadsto^*$'' is a reflexive-transitive closure of ``$\leadsto$''. 

One may notice, that the typing rules for the relational extension add nothing more than some
interpreted types and symbols w.r.t. the type system of the substrate language. Thus, it 
is rather expected, that the relational extension inherits all its useful properties (like progress and
type preservation). Surprisingly, this is not completely so. Indeed, the only value for goals is
\lstinline|success|, but, obviously, not every goal succeeds (for example, \lstinline|A === B| always
fails). Thus, our relational extension lacks the progress property~--- a decently typed non-value
goal sometimes cannot make a step. This makes no harm in the context of the paper; in any case,
a failure value for goals can be added to the language together with the failure propagation rules. 




\section{Refutational Incompleteness and Conjunction Non-Commutativity}
\label{incompleteness}

The language, defined in the previous section, is expected to allow defining computable relations in a 
very concise and declarative form. In particular, it is expected from a relational 
specification to preserve its behavior regardless the order of conjunction/disjunction 
constituents. Regretfully, in general this is not true, and one of the most important
manifestations of this deficiency is \emph{refutational incompleteness}.  

In the context of relational programming, refutational completeness~\cite{WillThesis} is understood as 
a capability of a program to discover the absence of solutions and stop. At the first glance,
the divergence in the case of solution absence does not seem to be a severe problem. However, as
we see shortly, refutational incompleteness leads to many observable negative effects in numerous
practically important cases. 

We demonstrate the effect of refutational incompleteness with a very simple example. Let us take the
definition of \lstinline{append$^o$} from the previous section and try to evaluate the following query:

\begin{lstlisting}
   fresh ($p\;q$) (append$^o$ $p$ $q$ Nil)
\end{lstlisting}

We would expect this query to converge to the single answer \mbox{$p=\lstinline|Nil|$}, \mbox{$q=\lstinline|Nil|$};
however, in the reality the query diverges. We sketch here the explanation, omitting some non-essential technical
details, such as semantic variables allocation, etc.:

\begin{itemize}
\item First we evaluate the first disjunct of \lstinline|append$^o$|'s body and unify $p$ with \lstinline|Nil| (successfully)
and \lstinline|Nil| with $q$ (successfully), which gives us the first (expected) answer.

\item Then we proceed to the second disjunct, which is a conjunction of three simpler goals:

  \begin{itemize} 
     \item in the first one we unify $p$ with \lstinline|Cons ($h$, $t$)| (successfully);
     \item in the second we encounter a recursive call \lstinline|append$^o$ $t$ $q$ Nil|; since its arguments are merely the renamings of the enclosing one, we repeat from the top and never stop.
  \end{itemize} 
\end{itemize}

The problem is that the semantics of conjunction, in fact, is not commutative: when the first conjunct diverges and the second fails, the whole
conjunction diverges. We stress that this is not a deviation of our semantics, but a well-known phenomenon, manifesting itself in all known
miniKanren implementations. In our example, switching two last conjuncts in the definition of \lstinline|append$^o$| solves the problem~---
now the whole search stops after the unsuccessful attempt to unify \lstinline|Nil| and \lstinline|Cons ($h$, $ty$)| with no recursive call.
This, improved version of \lstinline|append$^o$|, is known to be refutationally complete. In fact, there is a conventional ``rule of thumb''
for miniKanren programming to place the recursive call as far right as possible in a list of conjuncts. 

This convention, however, does not always help; to tell the truth, it often makes the things worse. Consider 
as an example yet another relation on lists:

\begin{lstlisting}
   revers$^o$ $\binds$ $\lambda\;x\;x_r$ . 
     (($x$ === $\;\;$Nil) /\ ($x_r$ === $\;\;$Nil)) \/
     (fresh ($h$ $t$ $t_r$)
        ($x$  === $\;$Cons ($h$, $t$)) /\
        (append$^o$ $t_r$ (Cons ($h$, Nil)) $x_r$) /\
        (revers$^o$ $t$ $t_r$)
     )
\end{lstlisting}

This relation corresponds to a relational list reversing; as we see, the recursive call is placed to
the end. However, the following query

\begin{lstlisting}
   fresh ($q$) (revers$^o$ (Cons (A, Nil)) $q$)
\end{lstlisting}

\noindent diverges, while

\begin{lstlisting}
   fresh ($q$) (revers$^o$ $q$ (Cons (A, Nil)))
\end{lstlisting}

\noindent converges to the expected results. If we switch the two last conjuncts in the definition of
\lstinline|revers$^o$|, the situation reverses: the first query converges, while the second diverges. 
This example demonstrates that the desired position of a recursive call (and, in general, the order of
conjuncts) depends on the direction, in which the relation of interest is evaluated.

There are, however, some cases, when the same relation is evaluated in both directions, regardless
the query. We can take as an example relational permutations, which can be implemented by running
relational list sorting in both directions:

\begin{lstlisting}
   sort$^o$ $\binds\lambda\;x\;x_s\;.\; \dots$
   perm$^o$ $\binds\lambda\;x\;x_p\;.$
     fresh ($x_s$) 
       (sort$^o$ $x$ $x_s$) $\wedge$ (sort$^o$ $x_p$ $x_s$) 
\end{lstlisting}

The idea of this implementation is very simple. Let us want to calculate all permutations of some list $l$.
We first sort $l$, obtaining the sorted version $l^\prime$; then we ask for all lists which, being sorted,
become equal to $l^\prime$. Obviously, all such lists are merely permutations of the original list $l$. The
important observation is that the existence of a single list sorting relation is sufficient to implement this
idea.

The concrete definition of the relational list sorting \lstinline|sort$^o$| is not important, so we
omit it due to the space considerations (an interested reader can refer to~\cite{OCanren}). The important part 
is that it is obviously recursive and not refutationally complete, and it is being evaluated 
in \emph{both} directions within the body of \lstinline|perm$^o$|. So, \lstinline|perm$^o$| is expected 
to perform poorly regardless the order of recursive calls in \lstinline|sort$^o$| implementation; it, 
indeed, does. First, if we request all solutions, both \lstinline|fresh ($q$) (perm$^o$ l $q$)| and \lstinline|fresh ($q$) (perm$^o$ $q$ l)| diverge for arbitrary non-empty list \lstinline|l| regardless the implementation of \lstinline|sort$^o$|; second, even if we request only a first few existing solutions, it does not provide any results in a reasonable time even for very small list lengths (4, 5, etc.).

Interesting, that if we interested in all solutions,
we have to accurately precompute their number in order not to request more, than exists. For some problems,
it may be not so simple, as it looks at a first glance (for example, the number of all permutations is
not a factorial, but a number of permutations with repetitions). Finally, getting the number of solutions can 
itself be an objective for writing a relational specification (we provide some examples in Section~\ref{evaluation}),
and without refutational completeness requesting all solutions to calculate their number is out of
reach.

\section{Search Improvement}
\label{improvement}

As we've seen in the previous section, the non-commutativity of conjunction in the presence of recursion
is one of the reasons for refutational incompleteness. Switching arguments of a certain conjunction
can sometimes improve the results; there is, however, no certain static order, beneficial in all cases.
Thus, we can make the following observations:

\begin{itemize}
\item the conjunction to change has to be properly identified;
\item the order of conjunct evaluation has to be a subject of a \emph{dynamic} choice.
\end{itemize}

Our improvement of the search is based on the idea of switching the order of conjuncts only when
the divergence of the first one is detected. More specifically: 

\begin{itemize}
\item during the search, we keep track of all conjunctions being performed;
\item when we detect the divergence, we roll back to the nearest conjunction, for which 
we did not try all orders of constituents yet, switch its constituents, and rerun 
the search from that conjunction.
\end{itemize}

The important detail is the divergence test. Of course, due to the fundamental results in computability
theory, there is no hope to find a \emph{precise} computable test that constitutes the necessary and 
sufficient condition of divergence. However, in our case a sufficient condition is sufficient. Indeed,  
a sufficient condition identifies a case, when the search, being continued, will lead to an incompleteness 
(since a divergence in our semantics always means incompleteness). Thus, it is no harm to try some other way. 

Another important question is the discipline of conjuncts reordering. Indeed, simply switching any two operands
of, for example, \mbox{$(g_1\wedge g_2)\wedge g_3$}, would not allow us to try \mbox{$(g_1\wedge g_3)\wedge g_2$}.
Thus, we have to flatten each ``cluster'' of nested conjunctions into a list of conjuncts\mbox{$\bigwedge g_i$}, 
where none of the goals $g_i$ is a conjunction. Then, it may seem at the first glance that the number of orderings to try 
is exponential on the number of conjuncts; we are going to show that, fortunately, this is not the case, and
a quadratic number of orders is sufficient.

In the rest of the section we address all these issues in details: first, we formally present the divergence
criterion and prove the necessity property; then, we describe an efficient reordering discipline. Finally, we present a
modified version of the semantics with incorporated divergence test and reordering. This semantics can be
considered as a modified version of the search, and we prove that this modification is a proper improvement in terms
of convergence.

\subsection{The Divergence Test}

Our divergence test is based on the following notion:

\begin{definition}
\normalfont 
We say that a vector of terms $\overline{a^{\phantom{x}}_i}$ is more general, than a vector of terms $\overline{b^{\phantom{x}}_i}$ (notation 
$\overline{a^{\phantom{x}}_i}\succeq\overline{b^{\phantom{x}}_i}$), if there is a substitution $\tau$, such that $\forall i\;b_i = a_i \tau$.
\end{definition}

The idea of the divergence test is rather simple: it identifies a recursive call with more general arguments 
than (some) enclosing one. To state it formally and prove it using the semantics from section~\ref{language}, we need several definitions and lemmas.

\begin{definition}
\normalfont
A semantic variable $v$ is \emph{observable} w.r.t. the interpretation $\iota$ and substitution $\sigma$, if there exists 
a syntactic variable $x$, such that \mbox{$v \in FV(\iota(x) \sigma)$}.
\end{definition}

\begin{definition}
A triplet of interpretation, substitution and a set of allocated semantic variables \mbox{$(\iota,\sigma,\delta)$} is
called \emph{coherent}, if \mbox{$dom(\sigma) \subseteq \delta$}, and any semantic variable, observable w.r.t. $\iota$ and $\sigma$,
belongs to $\delta$.  
\end{definition}

\begin{definition}
\normalfont
A semantic statement 

$$
\otrans{\Gamma,\iota}{(\sigma,\,\delta)}{g}{S}
$$ 

\noindent is \emph{well-formed}, if \mbox{$(\iota,\sigma,\delta)$} is a coherent triplet.
\end{definition}

Note, the root semantic statement \mbox{$\otrans{\Gamma,\bot}{(\epsilon,\,\emptyset)}{g}{S}$} is always well-formed.

\begin{lemma}
\label{one}
\normalfont
 For a well-formed semantic statement, every statement in its derivation tree is also well-formed.
\end{lemma}

The proof is by induction on the derivation tree. Note, we need to generalize the statement of the lemma, adding the condition that
\mbox{$(\iota,\sigma_r,\delta_r)$} is a coherent triplet for any \mbox{$(\sigma_r,\,\delta_r) \in S$}.

The next lemma ensures that any substitution in the RHS of a semantic statement is a correct refinement of that in the LHS:

\begin{lemma}
\label{two}
\normalfont
For a well-formed semantic statement 

$$
\otrans{\Gamma,\iota}{(\sigma,\,\delta)}{g}{S}
$$ 

\noindent and any result \mbox{$(\sigma_r,\,\delta_r) \in S$}, there exists a substitution $\Delta$, such that:
  \begin{enumerate}
    \item \mbox{$\sigma_r = \sigma\circ\Delta$};
    \item any semantic variable \mbox{$v\in dom(\Delta)\cup ran(\Delta)$} either is observable w.r.t. $\iota$ and $\sigma$,
 or does not belong to $\delta$ (where \mbox{$ran(\Delta)=\bigcup_{v\in dom(\Delta)}FV(\Delta(v))$}).
  \end{enumerate}   
\end{lemma}

The proof is by induction on the derivation tree; we as well need to generalize the statement of the lemma, adding the condition that the 
set of all allocated semantic variables $\delta$ can only grow during the evaluation.

The final lemma formalizes the intuitive considerations that the evaluation for a certain state $(\sigma^\prime,\delta^\prime)$ cannot
diverge, if the evaluation for a more general state $(\sigma,\delta)$ doesn't diverge:

\begin{lemma}
\label{three}
\normalfont
Let 

$$
\otrans{\Gamma,\iota}{(\sigma,\,\delta)}{g}{S}
$$ 

\noindent be a well-formed semantic statement, \mbox{$(\iota^\prime,\sigma^\prime,\delta^\prime)$} be a coherent triplet,
and let $\tau$ be a substitution, such that \mbox{$\iota^\prime(x) \sigma^\prime = \iota(x) \sigma \tau$} for any syntactic
variable $x$. Then

$$
\otrans{\Gamma,\iota^\prime}{(\sigma^\prime,\,\delta^\prime)}{g}{S^\prime}
$$

\noindent is well-formed and its derivation height is not greater than that for the first statement.
\end{lemma}

The proof is by induction on the derivation tree for the first statement. We need to generalize the statement of the lemma, adding the requirement that 
for any substitution $s^\prime_r$ in the RHS of the second statement, there has to be a substitution $s_r$ in the RHS of the first statement,
such that there exists a substitution $\tau_r$, such that \mbox{$\iota^\prime(x) \sigma^\prime_r = \iota(x) \sigma_r \tau_r$} for any syntactic variable $x$. 
In the cases of $\textsc{Fresh}$ and $\textsc{Invoke}$ rules, some semantic variables can become non-observable, and we need to define a substitution $\tau_r$ 
separately for these ``forgotten'' variables and those, which remain observable, using Lemma~\ref{two}.

Now we are ready to claim and prove the divergence criterion.

\setcounter{theorem}{0}
\begin{theorem}[Divergence criterion]
\label{criterion}
\normalfont
For any well-formed semantic statement 

$$
\otrans{\Gamma,\iota}{(\sigma,\,\delta)}{r^k\,t_1\dots t_k}{S}
$$ 

if its proper derivation subtree has a semantic statement 

$$
\otrans{\Gamma,\iota^\prime}{(\sigma^\prime,\,\delta^\prime)}{r^k\,t^\prime_1\dots t^\prime_k}{S^\prime}
$$

then \mbox{$\overline{t^\prime_i \iota^\prime \sigma^\prime} \not \succeq \overline{t^{\phantom{\prime}}_i \iota \sigma}$}. 
\end{theorem}
\begin{proof}
Assume that \mbox{$\overline{t^\prime_i \iota^\prime \sigma^\prime}\succeq \overline{t^{\phantom{\prime}}_i \iota \sigma}$}. 

By Lemma~\ref{one}, the semantic statement

$$
\otrans{\Gamma,\iota^\prime}{(\sigma^\prime,\,\delta^\prime)}{r^k\,t^\prime_1\dots t^\prime_k}{S^\prime}
$$

\noindent is well-formed.

By Lemma~\ref{three}, the derivation tree for

$$
\otrans{\Gamma,\iota^\prime}{(\sigma^\prime,\,\delta^\prime)}{r^k\,t^\prime_1\dots t^\prime_k}{S^\prime}
$$

\noindent has greater or equal height than that for

$$
\otrans{\Gamma,\iota}{(\sigma,\,\delta)}{r^k\,t_1\dots t_k}{S}
$$ 

\noindent which contradicts the theorem condition.

\end{proof}

The theorem justifies that, indeed, our test constitutes a sufficient condition for a divergence: if the execution
reaches a relation call with more general arguments, than those of some enclosing one, then it has no derivation
in our semantics, and, thus, it is not terminating.

\setarrow{\xRightarrow}
\setsubarrow{_e}
\begin{figure*}
\begin{minipage}[t]{\textwidth}
\small
\[
\cotrans{\Gamma,\,\iota,\,h}{(\sigma,\,\delta)}{t_1\equiv t_2}{\emptyset}{mgu\,(t_1\iota\sigma,\,t_2\iota\sigma) = \bot}\ruleno{UnifyFail$^+$}
\]
\[
\cotrans{\Gamma,\,\iota,\,h}{(\sigma,\,\delta)}{t_1\equiv t_2}{(\sigma\circ\Delta,\,\delta)}{mgu\,(t_1\iota\sigma,\,t_2\iota\sigma) = \Delta\ne\bot}\ruleno{UnifySuccess$^+$}
\]
\[
\trule{\otrans{\Gamma,\,\iota,\,h}{(\sigma,\,\delta)}{g_1}{S_1};\quad
       \otrans{\Gamma,\,\iota,\,h}{(\sigma,\,\delta)}{g_2}{S_2}
      }
      {\otrans{\Gamma,\,\iota,\,h}{(\sigma,\,\delta)}{g_1\vee g_2}{S_1\cup S_2}}\ruleno{Disj$^+$}
\]
\[
\crule{\otrans{\Gamma,\,\iota[x\gets\alpha],\,h}{(\sigma,\,\delta\cup\{\alpha\})}{g}{S^\dagger}}
      {\otrans{\Gamma,\,\iota,\,h}{(\sigma,\,\delta)}{\lstinline|fresh($x$) $\;g$|}{S^\dagger}}
      {\alpha\in\meta{W}\setminus\delta}\ruleno{Fresh$^+$}
\]
\end{minipage}      
\caption{Improved search: inherited rules}
\label{improved-semantics-normal}
\end{figure*}

\begin{figure*}
\begin{minipage}[t]{\textwidth}
\small
\[
   \cotrans{\Gamma,\,\iota,\,h}{(\sigma,\,\delta)}{r^k t_1 \dots t_k}{\dagger}{v_i = t_i \iota \sigma, \; (v_1, \dots, v_k) \succeq h\,r^k}
   \ruleno{InvokeDiv$^+$}
\]

\[
\crule{\otrans{\Gamma,\,\epsilon[x_i\gets v_i],\,h[r^k\gets(v_1, \dots, v_k)]}{(\epsilon,\,\delta)}{g}{\bigcup_j\{(\sigma_j,\,\delta_j)\}}}
      {\otrans{\Gamma,\,\iota,\,h}{(\sigma,\,\delta)}{r^k t_1 \dots t_k}{\bigcup_j\{(\sigma\circ\sigma_j, \delta_j)\}}}
      {v_i=t_i\iota\sigma,\;\Gamma\,r^k=\lambda x_1 \dots x_k. g,\; (v_1, \dots, v_k) \nsucceq h\,r^k}
      \ruleno{Invoke$^+$}
\]
\end{minipage}      
\caption{Improved search: invocation and divergence detection}
\label{improved-semantics-invoke}
\end{figure*}

\begin{figure*}
\begin{minipage}[t]{\textwidth}
\small
\[
\trule{\otrans{\Gamma,\,\iota,\,h}{(\sigma,\,\delta)}{g_1}{\dagger}}
      {\otrans{\Gamma,\,\iota,\,h}{(\sigma,\,\delta)}{g_1\vee g_2}{\dagger}}\ruleno{DivDisjLeft$^+$}
\]
\[
\trule{\otrans{\Gamma,\,\iota,\,h}{(\sigma,\,\delta)}{g_2}{\dagger}}
      {\otrans{\Gamma,\,\iota,\,h}{(\sigma,\,\delta)}{g_1\vee g_2}{\dagger}}\ruleno{DivDisjRight$^+$}
\]
\[
\crule{\otrans{\Gamma,\,\epsilon[x_i\gets v_i],\,h[r^k\gets(v_1, \dots, v_k)]}{(\epsilon,\,\delta)}{g}{\dagger}}
      {\otrans{\Gamma,\,\iota,\,h}{(\sigma,\,\delta)}{r^k t_1 \dots t_k}{\dagger}}
      {v_i=t_i\iota\sigma,\;\Gamma\,r^k=\lambda x_1 \dots x_k. g,\; (v_1, \dots, v_k) \nsucceq h\,r^k}
      \ruleno{DivInvoke$^+$}
\]      
\end{minipage}      
\caption{Improved search: divergence propagation}
\label{improved-semantics-divergence-prop}
\end{figure*}

\subsection{Conjuncts Reordering}
\label{sec:reordering}

In this section we consider the discipline of conjuncts reordering. Recall, we flatten all nested conjunctions in 
clusters $\wedge g_i$, where none of $g_i$ is a conjunction. To evaluate a cluster, we have to evaluate
its conjuncts one after another, threading the results, starting from the initial substitution. Each time we
evaluate a conjunct, we can have three possible outcomes:

\begin{itemize}
\item The evaluation converges with some result. In this case, we can proceed with the next conjunct.
\item The evaluation diverges undetected. In this case, nothing can be done.
\item A divergence is detected by the test. This is the case when the reordering takes place.
\end{itemize}

In a general case, for each cluster there can be some converging prefix $\omega$ we've managed to evaluate so far (initially empty),
and the rest of the conjuncts $g_i$. Since $\omega$ converges, we have some set of substitutions $S_\omega$ that corresponds to the
result of $\omega$ evaluation.

Suppose none of $g_i$ converges on $S_\omega$ (i.e. for each $g_i$ there is at least one substitution in $S_\omega$, on which
$g_i$ diverges). We claim that reordering conjuncts inside $\omega$ would not help. Indeed, with any other order
of conjuncts, $\omega$ either diverges or converges with the same result (up to the renaming of semantic variables). Thus,
making any permutations inside $\omega$ is superfluous.

Next, suppose we have two different goals $g_1$ and $g_2$, which both converge on $S_\omega$ (i.e. both converge on each
substitution in $S_\omega$). Do we need to try both cases ($g_1$ and $g_2$) to extend the converging prefix?
It is rather easy to see that if, say, $g_2$ converges on $S_\omega$, then it will as well converge on the result of evaluation
of $g_1$ on $S_\omega$. Indeed, for arbitrary \mbox{$(\sigma, \delta)\in S_\omega$} we have

\[
\otrans{\dots}{(\sigma, \delta)}{g_1}{S^\prime_\omega}
\]

where each $\sigma^\prime$ (such that \mbox{$(\sigma^\prime, \delta^\prime)\in S^\prime_\omega$}) is a ``more specific'',
than $\sigma$, by Lemma~\ref{two}. By Lemma~\ref{three}, since $g_2$ converges on \mbox{$(\sigma, \delta)\in S_\omega$},
it converges on each \mbox{$(\sigma^\prime, \delta^\prime)\in S^\prime_\omega$} as well.

In other words, to extend a converging prefix we can choose arbitrary conjunct, which converges immediately
after this prefix, and this choice will never have to be undone.

Now we can specify the reordering discipline. Since we never re-evaluate a converging prefix, we do not
represent it. Thus, each cluster we consider from now on is a suffix of some initial cluster after
evaluation of some converging prefix (and, perhaps, after some reorderings performed so far).

Let us have a cluster \mbox{$\bigwedge_{i=1}^k g_i$}. We evaluate it on some substitution $\sigma$ in the context of some integer
value $p$ (initially $p=1$), which describes, which conjunct we have to try next. We operate as follows:

\begin{enumerate}
\item\label{reorder:top} We try to evaluate $g_p$ on $\sigma$. If the evaluation succeeds with a result $S^\prime$, we 
remove $g_p$ from the cluster and evaluate the rest for each substitution in $S^\prime$ and $p=1$.
  
\item If a divergence is detected, and $p\le k$, then increment $p$, and repeat from step~\ref{reorder:top} (which will try the next goal).
  
\item Otherwise, we give up and rollback to the enclosing cluster (if any).
\end{enumerate}

Thus, we apply a greedy approach: each time we have a converging prefix of conjuncts (possibly empty), and some tail.
We try to put each conjunct from the tail immediately after the prefix. If we find a converging conjunct, we attach
it to the prefix and continue; if no, then the list of conjuncts diverges. Thus, we can find a converging order
(if any) in a quadratic time. Note, for different substitutions in the result of a converging prefix evaluation
the order of remaining conjuncts can be different.

\begin{figure*}
\begin{minipage}[t]{\textwidth}
\small
\[
\trule{\setsubarrow{_r}\otrans{\Gamma,\,\iota,\,h,\,1}{(\sigma,\,\delta)}{\bigwedge\limits_{i=1}^n g_i}{S^\dagger}}
      {\otrans{\Gamma,\,\iota,\,h}{(\sigma,\,\delta)}{\bigwedge\limits_{i=1}^n g_i}{S^\dagger}}
      \ruleno{ClusterStart$^+$}
\]
\vskip3mm
\[
\crule{\otrans{\Gamma,\,\iota,\,h}{(\sigma,\,\delta)}{g_p}{\bigcup_j\{(\sigma_j,\,\delta_j)\}};\quad
       \forall j\;:\;\otrans{\Gamma,\,\iota,\,h}{(\sigma_j,\,\delta_j)}{\bigwedge\limits_{i\ne p}g_i}{S_j}
      }
      {\setsubarrow{_r}\otrans{\Gamma,\,\iota,\,h,\,p}{(\sigma,\,\delta)}{\bigwedge\limits_{i=1}^n g_i}{\bigcup S_j}}
      {1 \le p \le n}
\ruleno{ClusterStep$^+$}
\]
\vskip3mm
\[
\crule{\otrans{\Gamma,\,\iota,\,h}{(\sigma,\,\delta)}{g_p}{\bigcup_j\{(\sigma_j,\,\delta_j)\}};\quad
       \exists j\;:\;\otrans{\Gamma,\,\iota,\,h}{(\sigma_j,\,\delta_j)}{\bigwedge\limits_{i\ne p}g_i}{\dagger}
      }
      {\setsubarrow{_r}\otrans{\Gamma,\,\iota,\,h,\,p}{(\sigma,\,\delta)}{\bigwedge\limits_{i=1}^n g_i}{\dagger}}
      {1 \le p \le n}
\ruleno{ClusterDiv$^+$}
\]
\vskip3mm
\[
\crule{\otrans{\Gamma,\,\iota,\,h}{(\sigma,\,\delta)}{g_p}{\dagger};\quad
       {\setsubarrow{_r}\otrans{\Gamma,\,\iota,\,h,\,p+1}{(\sigma,\,\delta)}{\bigwedge\limits_{i=1}^n g_i}{S^\dagger}}
      }
      {\setsubarrow{_r}\otrans{\Gamma,\,\iota,\,h,\,p}{(\sigma,\,\delta)}{\bigwedge\limits_{i=1}^n g_i}{S^\dagger}}
      {1 \le p \le n}
\ruleno{ClusterNext$^+$}
\]
\vskip3mm
\[
{\setsubarrow{_r}\cotrans{\Gamma,\,\iota,\,h,\,p}{(\sigma,\,\delta)}{\bigwedge\limits_{i=1}^n g_i}{\dagger}{p>n}}
\ruleno{ClusterStop$^+$}
\]
\end{minipage}      
\caption{Improved search: conjuncts reordering}
\label{improved-semantics-reordering}
\end{figure*}

\subsection{Improved Search Semantics}

Here we combine all observations, presented in the preceding subsections~--- the divergence test, conjunct clustering
and reordering,~--- and express the improved search in terms of a big-step operational semantics that is an extension
of the initial one, presented in Section~\ref{language}.

We denote ``$\xRightarrow{}_e$'' the semantic relation for the improved search, and we add another component to the
environment~--- a history $h$,~--- which maps a relational symbol to a list of fully interpreted terms as its arguments.
As we are (sometimes) capable of detecting the divergence, besides a regular set of answers $S$ as a result of evaluation
we can have a divergence signal, which we denote $\dagger$; $S^\dagger$ ranges over both the set of answers $S$ and the divergence
signal $\dagger$.

For the convenience of presentation we split the set of semantic rules into a few groups. The first one is the inherited
rules (see Fig.~\ref{improved-semantics-normal})~--- those, which did not change (except for the extension in the
environment and evaluation result). Note, the rule \rulename{Disj$^+$} does not handle the divergence detection
in either of disjuncts.

The next group describes the invocation and divergence detection (see Fig.~\ref{improved-semantics-invoke}). On
relation invocation, we first consult with the history. If the history indicates that the invocation is performed in the
context of the same relation evaluation with more specific arguments, then we raise the divergence signal; otherwise
we perform normally. Note, the rule \rulename{Invoke$^+$} does not handle the divergence in the \emph{body} of
invoked relation.

The next group describes the divergence signal propagation (see Fig.~\ref{improved-semantics-divergence-prop}). Here
the divergence signal, raised in one of the disjuncts or in the body of relational definition, is propagated to the upper
levels of the derivation tree.

The final group handles the conjunct reordering (see Fig.~\ref{improved-semantics-reordering}). As we need a reordering
parameter $p$ (see Section~\ref{sec:reordering}), we introduce another relation ``$\xRightarrow{}_r$'' with environment,
enriched by $p$.

The rule \rulename{ClusterStart$^+$} describes the case, when we make an attempt to evaluate a cluster. It can happen, when
we either first encounter an original cluster or try to evaluate a suffix of some initial cluster past some converging
prefix. As the reordering starts now, we recurse to the reordering relation with the parameter \mbox{$p=1$} (which means,
that the first conjunct will be tried to evaluate next).

Two next rules describe the case, when the $p$-th conjunct, being tried to evaluate, succeeds with some result. In the rule
\rulename{ClusterStep$^+$} we handle the case, when all other conjuncts can be evaluated in the context of that result: we
combine the outcomes, which completes the evaluation of the whole cluster. In the rule \rulename{ClusterDiv$^+$} we consider
the opposite case: now there is some conjunct $g_j$, which raises a divergence signal, being evaluated in the context of
the results, delivered by the evaluation of $g_p$. As we argued in Section~\ref{sec:reordering}, nothing can be done, and we
propagate the divergence signal.

The rule \rulename{ClusterNext$^+$} describes the case, when the $p$-th conjunct raises the divergence signal, and there are
some other conjuncts to try. We increment $p$ and proceed.

Finally, in the rule \rulename{ClusterStop$^+$} we handle the situation, when all available conjuncts in a cluster were tried to
evaluate first and raised the divergence signal. We propagate the signal in this case.

The following theorem is rather easy to prove:

\begin{theorem} For arbitrary $\Gamma$ and $g$ if

  \[{\setsubarrow{}\otrans{\Gamma,\,\bot}{(\epsilon,\,\emptyset)}{g}{S}}\]

  then
  
  \[{\setsubarrow{_e}\otrans{\Gamma,\,\bot,\,\bot}{(\epsilon,\,\emptyset)}{g}{S}}\]

\end{theorem}

Indeed, due to Theorem~\ref{criterion}, from the condition we can conclude that the divergence signal is
never raised during the evaluation, according to ``$\xRightarrow{}_e$''; but in this case the evaluation
steps coincide with those, according to ``$\xRightarrow{}$''. Thus, the improved search preserves the convergence.

\section{Evaluation}

\label{sec:evaluation}

In this section, we present an evaluation of 
implemented constructive negation on a series of examples.

\subsection{If-then-else}

Using relational if-then-else operator, 
presented in section~\ref{sec:ifte},
we have implemented several 
higher-order relations over lists, namely 
\lstinline{find} (Listing~\ref{lst:eval-find}), 
\lstinline{remove}\footnote{Note, this implementation 
differs from the one in Section~\ref{sec:intro}, but 
it is easy to see that these two are semantically equivalent.} (Listing~\ref{lst:eval-remove}) 
and \lstinline{filter} (Listing~\ref{lst:eval-filter}).
These relations are almost identical (syntactically) to their
functional implementations.
We have tested that these relations can be run
in various directions and produce the expected results.
For example, the goal \lstinline{filter p q q}
with the predicate \lstinline{p} equal to

\begin{lstlisting}
  fun l -> fresh (x) (l === [x])
\end{lstlisting}

stating that the given list should be a singleton list,
starts to generate all singleton lists.
Vice versa, the goal \lstinline{filter p q []} 
with that same \lstinline{p} generates 
all lists, constrained to be not a singleton list.

Listings~\ref{lst:eval-p}-\ref{lst:eval-filter-queries} give 
more concrete examples of queries to these relations.
In the listing the syntax \lstinline{run n q g}
means running a goal \lstinline{g} with 
the free variable \lstinline{q}
taking the first \lstinline{n} answers (``\lstinline{*}'' denotes all answers).
After the sign $\leadsto$ the result of the query is given.
The result \lstinline{fail} means that the query has failed.
The result \lstinline[mathescape]|succ {{a$_1$}; ... {a$_n$}} |
means that the query has succeeded delivering $n$ answers.
Each answer represents a set of constraint on free variables.
Constraints are of two forms: equality constraints, e.g. \lstinline{q = (1, _.$_0$)}, 
or disequality constraints, e.g. \lstinline{q $\neq$ (1, _.$_0$)}.
The terms of the form \lstinline{_.$_i$} in the answer
denote some universally quantified variables.

\begin{minipage}[thb]{.3\textwidth}
\begin{lstlisting}[
  caption={A definition of \code{find} relation},
  label={lst:eval-find}
]
let find p e xs =
  fresh (x xs' ys') (
    xs === x::xs' /\
    ifte (p x)
      (e === x)
      (find p e xs')
  )
\end{lstlisting}
\end{minipage}\hfill
\begin{minipage}[thb]{.3\textwidth}
\begin{lstlisting}[
  caption={A definition of \code{remove} relation},
  label={lst:eval-remove}
]
let remove p xs ys =
  (xs === [] /\ ys === [])
  \/
  fresh (x xs' ys') (
    xs === x::xs' /\
    ifte (p x)
      (ys === xs')
      (ys === x::ys' /\ 
       remove p xs' ys')
  )
\end{lstlisting}
\end{minipage}\hfill
\begin{minipage}[thb]{.3\textwidth}
\begin{lstlisting}[
  caption={A definition of \code{filter} relation},
  label={lst:eval-filter}
]
let filter p xs ys =
  (xs === [] /\ ys === [])
  \/
  fresh (x xs' ys') (
    xs === x::xs' /\
    (ifte (p x)
      (ys === x :: ys')
      (ys === ys')) /\
    filter p xs' ys'
  )
\end{lstlisting}
\end{minipage}

% \vspace{3cm}

\begin{minipage}[thb]{0.4\textwidth}
\begin{lstlisting}[
  caption={Definition of the predicate \lstinline{p}},
  label={lst:eval-p}
]
let p l = fresh (x) (l === [x])
\end{lstlisting}
\begin{lstlisting}[
  caption={Example of queries to \lstinline{find}},
  label={lst:eval-find-queries}
]
run 3 q (fresh (e) find p e q) 
$\leadsto$ succ {
     { q = [_.$_0$] :: _.$_1$ }
     { q = _.$_0$ :: [_.$_1$] :: _.$_2$; 
         _.$_0$ $\neq$ [_.$_3$] }
     { q = _.$_0$ :: _.$_1$ :: [_.$_2$] :: _.$_3$; 
         _.$_0$ $\neq$ [_.$_4$]; _.$_1$ $\neq$ [_.$_5$] }
   }
\end{lstlisting}
\end{minipage}\hfill
\begin{minipage}[thb]{0.4\textwidth}
\begin{lstlisting}[
  caption={Example of queries to \lstinline{remove}},
  label={lst:eval-remove-queries}
]
run * q (fresh (e) remove p q [[ ]]) 
$\leadsto$ succ {
     { q = [[_.$_0$]; [ ]] }
     { q = [[ ]] }
     { q = [[ ]; [_.$_0$]] }
   }

run 3 q (fresh (e) remove p q q) 
$\leadsto$ succ {
     { q = [] }
     { q = [_.$_0$], _.$_0$ $\neq$ [_.$_1$] }
     { q = [_.$_0$; _.$_1$]; 
         _.$_0$ $\neq$ [_.$_2$]; _.$_1$ $\neq$ [_.$_3$] }
   }
\end{lstlisting}
\end{minipage}

\begin{minipage}[thb]{0.4\textwidth}
\begin{lstlisting}[
  caption={Example of queries to \lstinline{filter}},
  label={lst:eval-filter-queries}
]
run 3 q (filter p q q) 
$\leadsto$ succ {
     { q = [ ] }
     { q = [_.$_0$] }
     { q = [_.$_0$; _.$_1$] }
   }

run 3 q (filter p q [1]) 
$\leadsto$ succ {
     { q = [[1]] }
     { q = [_.$_0$; [1]]; _.$_0$ $\neq$ [_.$_1$] }
     { q = [[1]; _.$_0$]; _.$_0$ $\neq$ [_.$_1$] }
   }

run 3 q (filter p q [ ]) 
$\leadsto$ succ {
     { q = [] }
     { q = [_.$_0$]; _.$_0$ $\neq$ [_.$_1$] }
     { q = [_.$_0$; _.$_1$]; 
            _.$_0$ $\neq$ [_.$_2$]; _.$_1$ $\neq$ [_.$_3$] }
   }
\end{lstlisting}
\end{minipage}

\subsection{Universal quantification}

In the Section~\ref{sec:impl-univ} we presented 
the \lstinline{forall} goal constructor 
which is implemented through the double negation.
We have observed, that although \lstinline{forall g}
does not terminate when the goal \lstinline{g x} 
has an infinite number of answers 
(assuming \lstinline{x} is a fresh variable),
it does terminate in the case when \lstinline{g x} has 
a finite number of answers.
The behavior of \lstinline{forall} in this case is sound
even in the presence of disequality constraints or nested quantifiers. 

The Table~\ref{tab:univ} gives some concrete examples.
The left column contains the tested goals\footnote{
We typeset the goals in terms of first-order logic syntax 
instead of \textsc{OCanren} syntax for brevity and clarity.} 
and the right column gives the obtained results.
For the results we use the same notation 
as in the previous section.

\begin{table}[th]
  \centering
  \def\arraystretch{1.5}
  \begin{tabularx}{\textwidth}{|X|X|}
    \hline

    $\forall x\ldotp x = q$ & 
      \texttt{fail} \\
    \hline

    $\forall x\ldotp \exists y\ldotp x = y$ & 
      \texttt{succ \{[q = \_.$_0$]\}} \\
    \hline

    $\forall x\ldotp \exists y\ldotp x = y \wedge y = q$ &
      \texttt{fail} \\
    \hline

    $\forall x\ldotp q = (1, x)$ & 
      \texttt{fail} \\
    \hline

    $\forall x\ldotp \exists y\ldotp y = (1, x)$ & 
      \texttt{succ \{[q = \_.$_0$]\}} \\
    \hline

    $\forall x\ldotp \exists y\ldotp x = (1, y)$ &
      \texttt{fail} \\
    \hline

    $\forall x\ldotp x \neq q$ & \texttt{fail} \\
    \hline

    $\forall x\ldotp \exists y\ldotp x \neq y$ & 
      \texttt{succ \{[q = \_.$_0$]\}} \\
    \hline

    $\forall x\ldotp \exists y\ldotp x \neq y \wedge y = q$ & 
      \texttt{fail} \\
    \hline

    $\forall x\ldotp q \neq (1, x)$ & 
      \texttt{succ \{[q $\neq$ (1, \_.$_0$)]\}} \\
    \hline

    $(\exists x\ldotp q = (1, x)) \wedge (\forall x\ldotp q \neq (1, x))$ & 
      \texttt{fail} \\
    \hline

    $\forall x\ldotp (x, x) \neq (0, 1)$ & 
      \texttt{succ \{[q = \_.$_0$]\}} \\
    \hline

    $\forall x\ldotp (x, x) \neq (1, 1)$ & 
      \texttt{fail} \\
    \hline

    $\forall x\ldotp (x, x) \neq (q, 1)$ & 
      \texttt{succ \{[q $\neq$ 1]\}} \\
    \hline

    $\exists a~ b\ldotp q = (a, b) \wedge \forall x\ldotp (x, x) \neq (a, b)$ & 
      \texttt{succ \{[q = (\_.$_0$, \_.$_1$); \_.$_0$ $\neq$ \_.$_1$]\}} \\
    \hline

  \end{tabularx}
  \caption{\lstinline{forall} evaluation}
  \label{tab:univ}
\end{table}

% !TEX TS-program = pdflatex
% !TeX spellcheck = en_US
% !TEX root = main.tex

\section{Related Work}
\label{related}

GUI design and implementation has been a hot topic for decades. Thus, to no surprise there is a lot of frameworks, approaches, papers
and reports on the subject. A fair share of them (if not all) present declarative and automatic solutions. A careful study, however,
discovers that this ``declarativeness'' and ``automation'' is understood differently then in our case.

First of all, we need to mention some software frameworks and tools for design and implementation of GUI and visualization of data, for example, \textsc{React}~\cite{react},
\textsc{Jetpack Compose}~\cite{Jetpack}, \textsc{SwiftUI}~\cite{SwiftUI}, \textsc{Streamlit}~\cite{Streamlit}, \textsc{D3}~\cite{D3} and others.
These frameworks provide a number of layout primitives which end-users can employ in order to render their data or UI. For example, \textsc{Streamlit}
provides a number of builtin layout primitives like ``columns'', ``container'', ``modal dialog'', etc.~\cite{StreamlitLayout} and an endless
variety of third-party external components. These primitives allow end-users to abstract away of concrete controls coordinate calculation and their
relative alignment; they also prescribe a reasonable behavior on enclosing pane resizing. However, which layout primitives to use is decided by
end-users, not the system. If due to any reason the layout needs to be changed these changes have to be implemented manually. In our case
end-users do not specify concrete layouts, only the logical structure of the UI. The guideline takes care of concrete layout, depending on
external constraints such as enclosing pane size, screen resolution or even regional settings (for example, right-to-left writing system). As long
as the logical structure remains unchanged no interference from end-users is required for laying out the UI in different settings. On the
other hand these frameworks can be used as back-ends in our approach since they provide a similar set of layout primitives.

Constraint programming has already been used for deciding the placement of GUI controls. One of the examples are constraint reactive programming
language \textsc{Wallingford}~\cite{Wallingford2016} and the \textsc{Cassowary} system~\cite{Cassowary2001}. \textsc{Wallingford} allows to attach
constraints of various strength to different values in the program. The system reacts to the time changes and updates these values without violation
of the constraints. For example, one could calculate a width of a GUI control as the sine of current time. The \textsc{Cassowary} system and its
descendants allow to calculate the sizes and positions of controls dynamically, for example at the moment of canvas resize.
%The background theory is linear arithmetic.
It supports many different constraints, for example, Z-ordering, arithmetic operators (for example, a control's width can be the half
of another one's height), overlapping views, etc.  These systems are targeted for the tasks of dynamic adaptation the sizes of controls on resize.
Also, they don't provide support of expressing general rules about correct control placing, the constraints are added for values of concrete structure.
In our work we make a strict separation between GUI structure and rules of correct positioning of controls. We have doubts about expressing non-deterministic
layouts in these systems, for example, if vertical or horizontal placement of controls depend on their sizes. On the other hand, the number of various
constraint types is larger than in our approach. For example, they allow to express overlapping views, but based on our experience we initially ruled out
this option, and our current implementation doesn't generate such layouts at all.

Finally, in recent years methods and approaches from AI in ML/data-science sense start to percolate into the area. Some of the works address much more
ambitious objectives, than ours.

First of all, there is a direction of research on UI code generation from images~\cite{Cai2023}. Given a designer-drawn form, a code
generator recognizes UI controls and their relative placements and generates implementation code for one of GUI frameworks. This approach
is completely orthogonal and incompatible with what we suggest. Indeed, it requires an interaction with a designer when implementing every piece of
an interface while in our case a designer is only involved when guidelines are developed; then our system produces guideline-compatible
interfaces \emph{en masse} automatically. In~\cite{Robust} a slightly different task is addressed: given a picture of an interface synthesize its
implementable layout in terms of Android GUI primitives which would be scalable across various devices while avoiding a typical layout errors.
To achieve this goal, after recognizing UI controls and their locations a certain set of relational layout constraints is extracted. This
set of constraints resembles our layout primitives but is aligned with Android's \texttt{ConstraintLayout} widget~\cite{ConstraintLayout} semantics.
Given these constraints and a set of \emph{robustness properties} developed by the authors an implementation code is generated with
the aid of a probabilistic model trained on a large set of existing Android interfaces. This implementation is more stable under screen size
or resolution change, than that provided by image recognition only. Interestingly, one of the motivations for the work, as the authors
specify explicitly, is that ``the same layout needs to be rendered potentially on more than 15 000 Android devices with $\approx$100 different density
independent screen sizes. Requiring the user to provide and maintain input specifications for all of them is infeasible yet highly desirable.''
That is exactly what our system is capable of doing within a few minutes and with 100\% accuracy, so we consider our approach
much more general.

Automatic design of consistent (uniform throughout an application) GUI is addressed in~\cite{LearningGUI}. The approach is based on the idea of completion:
assuming there is already a set of consistent layouts a problem of adding yet another element is addressed; the addition should be consistent with the previous
designs. While this problem is, indeed, related to that we address (indeed, consider an existing set of designs as an implicitly specified guideline),
we can identify a number of potential weaknesses. First, only addition of a component is considered, but not removal; second, the addition/subtraction
of components does not necessarily lead to a ``monotonic'' change of the layout (adding yet another text field may result is a drastically different placement);
finally, the initial set of designs to be consistent with rarely comes out of a thin air; most likely it is a result of following some
existing (and perhaps implicit) guidelines, which should be specified and followed explicitly.

In~\cite{Grid} a much more ambitious problem of synthesizing a layout with no guidelines, based only on aesthetic, ergonomic, etc., metrics is considered.
Genetic algorithm is employed for synthesis, and users' feedback is used as a way to measure the quality of the synthesis; the synthesis itself can
sometimes take hours. While the approach presumably allows to synthesize aesthetically convincing layouts with no designer input, the problem of
layout consistency for sufficiently different structures is left unaddressed.

An interesting problem of layout exploration is considered in~\cite{Scout}. The objective is to help the \emph{designers} to develop convincing and
diverse layouts. A set of constraints is introduced which designers can use in order to specify some requirements for the design. Interestingly, in our terms the
set of constraints is a mixture of layout and structural ones: for example, both order and alignment constraints are present. Given
a number of constraints the system generates a set of layouts consistent with this constraints; by updating the constraints a designer can
continue the exploration. As a constraint solver modifier branch-and-bound algorithm is used; as the number of feasible solutions
turned out to be enormous a set of heuristic metrics was developed to rule aesthetically unfeasible designs out. While this work
shares some similarities with ours, being targeted on designers it can be considered as a mean to \emph{develop guidelines}, not to synthesize
guideline-consistent designs.

\section{Conclusion and future work}

We presented an approach for pattern matching implementation synthesis using relational programming. Currently, it demonstrates a good performance only
on a very small problems. The performance can be improved by searching for new ways to prune the search space and by speeding up the implementation of
relations and structural constraints. Also it could be interesting to integrate structural constraints more closely into \textsc{OCanren}'s core.
Discovering an optimal order of samples and reducing the complete set of samples is another direction for research.

The language of intermediate representation can be altered, too. It is interesting to add to an intermediate language so-called \emph{exit nodes}
described in~\cite{maranget2001}. The straightforward implementation of them might require nominal unification, but we are not aware of any
\textsc{miniKanren} implementation in which both disequality constraints and nominal unification~\cite{alphaKanren} coexist nicely.

At the moment we support only simple pattern matching without any extensions. It looks technically easy to extend our approach with
non-linear and disjunctive patterns. It will, however, increase the search space and might require more optimizations.





\bibliographystyle{ACM-Reference-Format}
\bibliography{main}

\end{document}
