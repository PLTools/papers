\section{Angelic Semantics and Fairness}
\label{sec:angelic-semantics}

In this section we introduce an \emph{angelic} operational semantics of \mk. Instead of a fixed order of conjunct evaluation this semantics chooses a conjunct nondeterministically
thus enumerating all possible orders of their evaluation. Thus, if a goal (in particular, a conjunction) diverges in the angelic semantics then no order of evaluation exists which
would make it to converge.

The development of this (and the following) section essentially relies on the results reported in~\cite{fair:semantics}. In the paper, two semantics for \mk are presented. The first one is
a denotational, set-theoretic semantics close to the least Herbrand model~\cite{Lloyd}; the second is operational semantics of relational search with interleaving in the form of a labeled transition
system~\cite{LTS}, which corresponds to the informal description given in the previous section. Then the soundness and completeness of interleaving search w.r.t. the denotational semantics
is formally established and mechanized using \textsc{Coq} proof assistant~\cite{Coq}. It also establishes a corollary that for a sound and complete operational semantics some
transformations of goals preserve their equivalence (w.r.t. the set of answers they provide). These transformations will play an important role in justifying the fairness of the semantics we present
in the next section. Due to space limitations, however, we do not present here these semantics and only sketch the proofs which use them.
 
As a first step we define a set of \emph{semantic} variables

\[
\mathcal{A}=\{\alpha_0,\alpha_1,\dots\}
\]

A set of \emph{semantic terms} $\mathcal{T}_{\mathcal A}$ then comes in naturally. We stipulate that all semantic variables are linearly ordered making it possible to take them one-by-one
in a deterministic manner. We will also use conventional notations of $\mathcal{FV}\,(\bullet)$ for the sets of free variables in terms and goals and $\bullet\,[x\gets t]$ for substitutions
of \emph{semantic} term $t$ for \emph{syntactic} variable $x$ in terms and goals. As the only binder in goals~--- the ``\lstinline|fresh|'' construct~--- can only bind \emph{syntactic}
variables no capturing can occur during substitutions. We omit the straightforward and standard definitions of free variables and substitutions.

Next, we define the following set of \emph{states} $\mathfrak{S}$:

\[
\begin{array}{rccll}
  \mathfrak{S} & = &      & \times         & \mbox{final state}\\
               &   & \mid & \mathfrak{S}^o & \mbox{non-final state}\\

  \mathfrak{S}^o & = &     & \inbr{\theta,\,i,\,r^*}            & \mbox{leaf state}\\
                 &   &\mid & \mathfrak{S}^o\oplus\mathfrak{S}^o & \mbox{disjunction}\\
  
  r              & = &     & R^k\,(t_1,\dots,t_k)               & \mbox{relation application}
\end{array}
\]

A non-final state is a tree of disjunctions with leaves of the form $\inbr{\theta,\,i,\,r^*}$, where $\theta$ is a substitution from semantic variables into semantic terms,
$i$ is the index of the first non-allocated semantic variable, and $r^*$ is a possibly empty list of relation applications $R^k\,(t_1,\dots,t_k)$, where $R^k$ is relation name,
$t_i$ are semantic terms. Informally, a state represents some point during goal evaluation when the results of (non-failed) unifications are collected in substitutions and the residual
goal is represented in DNF. If a list of relation applications in the leaf of the state is empty, then the substitution represents an answer. A final state corresponds to the end of evaluation.

We introduce two auxiliary functions to work with states. The first one, \primi{union}, joins two states into the one:

\[
\begin{array}{rcl}
  \primi{union}\,(\times,\, S) & = & S \\
  \primi{union}\,(S,\, \times) & = & S \\
  \primi{union}\,(S_1,\, S_2) & = & S_1\oplus S_2
\end{array}
\]

The next one, \primi{push}, is used to reconstruct a state after unfolding a single relation application:

\[
\begin{array}{rcl}
  \primi{push}\,(\_,\,\times) & = & \times\\
  \primi{push}\,(c,\,S_1\oplus S_2) & = & \primi{push}\,(c,\,S_1)\oplus\primi{push}\,(c,\,S_2)\\
  \primi{push}\,(\rho\Box\pi,\,\inbr{\theta,\,i,\,\sigma}) & = & \inbr{\theta,\,i,\,\rho\sigma\pi}
\end{array}
\]

The first argument of \primi{push} is a list of relation applications which contains a \emph{hole} ``$\Box$''. The hole indicates the place in which
an unfolded relation application was located. The second argument is a state which represents the result of unfolding. Thus, \primi{push} propagates the
context of the unfolding through its resulting local state.

Now we define the semantics for one-step unfolding of relation application. Informally, we take an application of a relation symbol to some semantic term, substitute these
terms for corresponding arguments in the relation definition body, perform all unifications and transform the rest of the body into DNF. As a result we acquire some state. Note, we
do not perform unfoldings of relation applications within the body. We describe one-step unfolding in terms of a big-step relation ``$\Rightarrow$'' with the following
single rule:

\[
\begin{array}{cr}
  \dfrac{R = \lambda \bar{x}. b \qquad \inbr{\theta,\, i,\, \epsilon} \vdash b\,[\bar{x} \leftarrow \bar{t}] \leadsto S}
      {\inbr{\theta,\,i} \vdash R\,(\bar{t}) \Rightarrow S}
      &     \ruleno{Unfold}
\end{array}
\]

We conventionally denote here $\bar{\bullet}$ vectors of terms or variables. In the conclusion we use a pair of a substitution and a number of first unallocated semantic variable as
environment; in the premise, however, we need to refer to a yet another, this time small-step, relation ``$\leadsto$'', which uses states as environments. This relation describes the
local computations within the body of unfolded application.  The definition of this relation is shown in the Fig.~\ref{fair:unfolding-semantics}.

\begin{figure}[t]
\[\begin{array}{cr}
{\times \vdash g \leadsto \times}
&     \ruleno{End} \\[3mm]
\dfrac{\not\exists \, mgu\,(t_1,\, t_2,\, \theta)}
      {\inbr{\theta,\, i,\, c} \vdash t_1 \equiv t_2 \leadsto \times}
&     \ruleno{UnifyFail}  \\[6mm]
\dfrac{\theta^\prime = mgu\,(t_1,\, t_2,\, \theta)}
      {\inbr{\theta,\, i,\, c} \vdash t_1 \equiv t_2 \leadsto \inbr{\theta^\prime,\, i,\, c}}
&     \ruleno{UnifySucc}  \\[6mm]
      {\inbr{\theta,\, i,\, c} \vdash R\,(\bar{t}) \leadsto \inbr{\theta,\, i,\,cR\,(\bar{t})}}
&     \ruleno{App} \\[3mm]
\dfrac{\inbr{\theta,\, i+1,\, c} \vdash g\,[x \leftarrow \alpha_i] \leadsto S}
      {\inbr{\theta,\, i,\, c} \vdash \mbox{\lstinline|fresh|} (x)\, g \leadsto S}
&     \ruleno{Fresh}  \\[6mm]
\dfrac{\inbr{\theta,\, i,\, c} \vdash g_1 \leadsto S_1 \qquad \inbr{\theta,\, i,\, c} \vdash g_2 \leadsto S_2}
      {\inbr{\theta,\, i,\, c} \vdash g_1 \lor g_2 \leadsto \primi{union}\,(S_1,\, S_2)}
&     \ruleno{DisjGoal}  \\[6mm]
\dfrac{S_1 \vdash g \leadsto S_3 \qquad S_2 \vdash g \leadsto S_4}
      {S_1 \oplus S_2 \vdash g \leadsto \primi{union}\,(S_3,\, S_4)}
&     \ruleno{DisjState}  \\[6mm]
\dfrac{\inbr{\theta,\, i,\, c} \vdash g_1 \leadsto S \qquad S \vdash g_2 \leadsto S^\prime}
      {\inbr{\theta,\, i,\, c} \vdash g_1 \land g_2 \leadsto S^\prime}
&     \ruleno{Conj}
\end{array}\]
\caption{Small-step semantics of local computations}
\label{fair:unfolding-semantics}
\end{figure}

The rule \rulen{End} propagates a final state through the rest of the computations. The rules \rulen{UnifyFail} and \rulen{UnifySucc} encode unification steps: if there exists the most general
unifier (MGU) for given terms in given substitution $\theta$, then we update the substitution according to the MGU; otherwise we produce a final state. The rule \rulen{App} preserves a relation
application, adding it to the list of applications in the current leaf state. The rule \rulen{Fresh} corresponds to a fresh semantic variable allocations. We take the first non-allocated semantic
variable and substitute it for the fresh-bound syntactic one; we also increment the number of non-allocated variables. The rules \rulen{DisjGoal} and \rulen{DisjState} describe the evaluation
of disjunctions. The first one evaluates both disjuncts and joins the resulting states by mean of primitive \primi{union}. The second handles the case when current environment state is a disjunction
itself. As in the first case, we perform independent evaluations and join the results. Finally, the rule \rulen{Conj} describes conjunction evaluation. Here we first evaluate the left subgoal,
acquiring a state $S$, and then evaluate the right one in the context $S$, obtaining the final result $S^\prime$. Note, as we do not perform any unfoldings, the process always converges with
some state.

Having one-step unfolding at our command, we can define the angelic semantics itself (Fig.~\ref{fair:angelic-semantics}). The semantics is defined in terms of a labeled transition
system over the set of states. The set of labels is defined as

\[
\mathcal{L} = \circ \mid \theta.
\]

where ``$\circ$'' denotes the absence of an answer, $\theta$~--- an answer in the form of a substitution.

\begin{figure}[h!]
\[\begin{array}{cr}
     {\inbr{\theta,\, i,\, \epsilon} \xrightarrow{\theta} \times}  
&     \ruleno{Answer} \\[3mm]
\dfrac{\inbr{\theta,\, i} \vdash c \Rightarrow S}
      {\inbr{\theta,\, i,\, \rho c \pi} \xrightarrow{\circ} \primi{push}\,(\rho \Box \pi, S)}
&     \ruleno{ConjUnfold} \\[6mm]
\dfrac{S_1 \xrightarrow{\alpha} \times}
      {S_1 \oplus S_2 \xrightarrow{\alpha} S_2}
&     \ruleno{Disj} \\[6mm]
\dfrac{S_1 \xrightarrow{\alpha} S^\prime_1 \qquad S^\prime_1 \not= \times}
      {S_1 \oplus S_2 \xrightarrow{\alpha} S_2 \oplus S^\prime_1}
&     \ruleno{DisjStep}
\end{array}\]
\caption{Angelic semantics of \mk}
\label{fair:angelic-semantics}
\end{figure}

The rule \rulen{Answer} corresponds to a situation when no relation application remain in the state under consideration; in this case we return current substitution as an answer.
If current state is a leaf with a non-empty list of relation applications, then we non-deterministically choose one for a one-step unfolding (rule \rulen{ConjUnfold}). After the
unfolding we construct a new state by pushing the remaining list of applications into the resulting state. Finally, if the current state is a disjunction, then we make a step in
the left disjunct $S_1$. If the result is a final state, we (by the rule \rulen{Disj}) set the residual state to $S_2$. Otherwise, by rule \rulen{DisjStep}, we construct a
new residual state by forming a new disjunction where the right substate of the original one is placed left and the residual state after the evaluation of the left one in
placed right. This switching, called \emph{interleaving}, is an important feature of relational search, which makes it complete~\cite{fair:interleaving}. These
two rules for disjunction literally repeat corresponding rules in~\cite{fair:semantics}.

In order to apply angelic semantics to a specification $D_1 D_2\ldots D_k \diamond g$ we need to replace all free variables in $g$ with distinct semantic
variables and convert in into the initial state $S^0\,(g)$:

\[
\inbr{\Lambda,\, n,\, \epsilon} \vdash g\,[x_0 \leftarrow \alpha_0, \ldots, x_{n-1} \leftarrow \alpha_{n-1}] \leadsto S^0\,(g)
\]

Here we assume $\mathcal{FV}\,(g)=\{x_0,\dots,x_{n-1}\}$, $\Lambda$~--- empty substitution. The definitions of the specification are used later during the unfolding steps.

While angelic semantics preserves the deterministic left-to-right interleaving of conventional \mk search, it treats the conjunctions nondeterministically, thus
describing all possible their evaluation scenarios. For example, the goal

\begin{lstlisting}
  fail$\!^o$ _ /\ div$\!^o$ _
\end{lstlisting}

from the previous section will converge according to the angelic semantics if we intentionally choose to unfold the left conjunct:

\[
\dfrac
{\dfrac
{\inbr{\Lambda, \, 1,\, \epsilon} \vdash A \equiv B \leadsto \times}
{\inbr{\Lambda,\, 1} \vdash\! \mbox{\lstinline|fail|}\!^o\, \alpha_0  \Rightarrow \times}}
{\inbr{\Lambda,\, 1, \, \mbox{\lstinline|fail|}\!^o\, \alpha_0 \land \mbox{\lstinline|div|}\!^o\, \alpha_0} \xrightarrow{\circ} \times}
\]

Indeed, the unfolding of  \lstinline|fail$\!^o$| will lead us to the final state in one step. At the same time if we choose to unfold only the right conjunct, the
evaluation will diverge:

\[
\dfrac
{\dfrac
{\inbr{\Lambda, \, 1, \epsilon} \vdash \mbox{\lstinline|div|}\!^o\, \alpha_0 \leadsto \inbr{\Lambda, \, 1, \mbox{\lstinline|div|}\!^o\, \alpha_0}}
{\inbr{\Lambda,\, 1} \vdash\! \mbox{\lstinline|div|}\!^o\, \alpha_0  \Rightarrow \inbr{\Lambda, \, 1, \mbox{\lstinline|div|}\!^o\, \alpha_0}}}
{\inbr{\Lambda, \, 1, \, \mbox{\lstinline|fail|}\!^o\, \alpha_0 \land \mbox{\lstinline|div|}\!^o\, \alpha_0} \xrightarrow{\circ} \inbr{\Lambda, \, 1, \, \mbox{\lstinline|fail|}\!^o\, \alpha_0 \land \mbox{\lstinline|div|}\!^o\, \alpha_0}}
\]

Since the states before and after the unfolding of \lstinline|div$\!^o$| coincide, the derivation will repeat itself forever.

The following lemma justifies the use of angelic semantics:

\begin{lemma}[Soundness and completeness]
  Angelic semantics is sound and complete.
\end{lemma}
\begin{proofsketch}
  The completeness follows immediately from the completeness of deterministic semantics from~\cite{fair:semantics}. Indeed, any derivation in deterministic semantics can be reproduced in
  the angelic one.

  The soundness, on the other hand, requires the whole soundness proof from~\cite{fair:semantics} to be repeated. 
\end{proofsketch}

Angelic semantics allows us to formally define what we can call the \emph{fairness} of conjunction evaluation strategy. First, we define the notion of \emph{convergence}:

\begin{definition}[Convergence]
  A goal $g$ \emph{converges} (notation: $g\Downarrow$) if $\inbr{S^0\,(g),\,\times}\in\,\rightarrow^*$; in other words, if there exists a derivation sequence from the initial state for $g$ to a
  final state.
\end{definition}

\begin{comment}
Let us have some other \emph{deterministic} semantics in the form of a labeled transition system; more specifically, let us have

\begin{itemize}
\item a set of states $\widetilde{\mathfrak{S}}$;
\item a set of labels $\widetilde{L}$;
\item a ternary transition relation $\mbox{``}\dashrightarrow\mbox{''}\subseteq\widetilde{\mathfrak{S}}\times\widetilde{L}\times\widetilde{\mathfrak{S}}$ such that for all $s, s^\prime, s^{\prime\prime}\in\widetilde{\mathfrak{S}}$ and $l^\prime, l^{\prime\prime}\in\widetilde{L}$

  \[
  s\overset{l^\prime}{\dashrightarrow}s^\prime\wedge s\overset{l^{\prime\prime}}{\dashrightarrow}s^{\prime\prime}\Longrightarrow s^\prime=s^{\prime\prime}\wedge l^\prime=l^{\prime\prime}
  \]
  
\item a final state $\widetilde{\times}\in\widetilde{\mathfrak{S}}$ such that $\inbr{\widetilde{\times},\,\_,\,\_}\not\in\mbox{``}\dashrightarrow\mbox{''}$;
\item an initial state $\widetilde{S}^0(g)$ for each top-level goal $g$.
\end{itemize}
\end{comment}

\begin{definition}[Fairness]
  A sound and complete semantics in the form of LTS is \emph{fair} if a goal $g$ converges to a final state whenever $g\Downarrow$.
\end{definition}

In the next section we present a fair and deterministic semantics for \mk.

