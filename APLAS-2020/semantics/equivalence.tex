\section{Equivalence of Semantics}
\label{equivalence}

Now we can relate two different kinds of semantics for \textsc{miniKanren} described in the previous sections and show that the results given by these two semantics are the same for any specification.
This will actually say something important about the search in the language: since operational semantics describes precisely the behavior of the search and denotational semantics
ignores the search and describes what we \emph{should} get from a mathematical point of view, by proving their equivalence we establish the \emph{completeness} of the search, which
means that the search will get all answers satisfying the described specification and only those.

But first, we need to relate the answers produced by these two semantics as they have different forms: a trace of substitutions (along with the numbers of allocated variables)
for the operational one and a set of representing functions for the denotational one. We can notice that the notion of a representing function is close to substitution, with only two differences:

\begin{itemize}
\item representing functions are total;
\item terms in the domain of representing functions are ground.
\end{itemize}

Therefore we can easily extend (perhaps ambiguously) any substitution to a representing function by composing it with an arbitrary representing function preserving
all variable dependencies in the substitution. So we can define a set of representing functions that correspond to a substitution as follows:

\[
\sembr{\sigma} = \{\overline{\mathfrak f} \circ \sigma \mid \mathfrak{f}:\mathcal{A}\mapsto\mathcal{D}\}
\]

\begin{comment}
In \textsc{Coq} this notion boils down to the following definition:

\begin{lstlisting}[language=Coq]
   Definition in_denotational_sem_subst
     (s : subst) (f : repr_fun) : Prop :=
       exists (f' : repr_fun),
         repr_fun_eq (subst_repr_fun_compose s f') f.
\end{lstlisting}

where ``\lstinline[language=Coq]|repr_fun_eq|'' stands for representing functions extensional equality, ``\lstinline[language=Coq]|subst_repr_fun_compose|''~---
for a composition of a substitution and a representing function.
\end{comment}

And the \emph{denotational analog} of operational semantics (a set of representing functions corresponding to the answers in the trace) for a given state $\hat{s}$ is
then defined as the union of sets for all substitutions in the trace:

\[
\sembr{\hat{s}}_{op} = \cup_{(\sigma, n) \in \tr{\hat{s}}} \sembr{\sigma}
\]

\begin{comment}
In \textsc{Coq} we again use a proposition instead:

\begin{lstlisting}[language=Coq]
   Definition in_denotational_analog
      (t : trace) (f : repr_fun) : Prop :=
      exists s n, in_stream (Answer s n) t /\
             in_denotational_sem_subst s f.
   Notation "{| t , f |}" := (in_denotational_analog t f).
\end{lstlisting}
\end{comment}

This allows us to state theorems relating the two semantics.

\begin{theorem}[Operational semantics soundness]
\label{lem:soundness}
If indices of all free variables in a goal $g$ are limited by some number $n$, then $\sembr{\inbr{g, \epsilon, n}}_{op} \subseteq \sembr{g}$.
\end{theorem}

It can be proven by nested induction, but first, we need to generalize the statement so that the inductive hypothesis is strong enough for the inductive step.
To do so, we define denotational semantics not only for goals but for arbitrary states. Note that this definition does not need to have any intuitive
interpretation, it is introduced only for the proof to go smoothly. The definition of the denotational semantics for extended states is shown on Fig.~\ref{denotational_semantics_of_states}.
The generalized version of the theorem uses it.

\begin{figure}[t]
  \[
  \begin{array}{ccl}
    \sembr{\Diamond}&=&\emptyset\\
    \sembr{\inbr{g, \sigma, n}}&=&\sembr{g}\cap\sembr{\sigma}\\
    \sembr{s_1 \oplus s_2}&=&\sembr{s_1}\cup\sembr{s_2}\\
    \sembr{s \otimes g}&=&\sembr{s}\cap\sembr{g}\\
  \end{array}
  \]
  \caption{Denotational semantics of states}
  \label{denotational_semantics_of_states}
\end{figure}

\begin{lemma}[Generalized soundness]
\label{lem:gen_soundness}
For any well-formed state $\hat{s}$

\[
\sembr{\hat{s}}_{op} \subseteq \sembr{\hat{s}}.
\]
\end{lemma}

It can be proven by the induction on the number of steps in which a given answer (more accurately, the substitution that contains it) occurs in the trace.
We break the proof in two parts and separately prove by induction on evidence that for every transition in our system the semantics of both the label (if there is one)
and the next state are subsets of the denotational semantics for the initial state.

\begin{lemma}[Soundness of the answer]
\label{lem:answer_soundness}
For any transition $s \xrightarrow{(\sigma, n)} \hat{s}$, \mbox{$\sembr{\sigma} \subseteq \sembr{s}$}.
\end{lemma}

\begin{lemma}[Soundness of the next state]
\label{lem:next_state_soundness}
For any transition $s \xrightarrow{l} \hat{s}$, \mbox{$\sembr{\hat{s}} \subseteq \sembr{s}$}.
\end{lemma}

It would be tempting to formulate the completeness of operational semantics as soundness with the inverted inclusion, but it does not hold in such generality.
The reason for this is that the denotational semantics encodes only the dependencies between free variables of a goal, which is reflected by the closedness condition,
while the operational semantics may also contain dependencies between semantic variables allocated in \lstinline|fresh| constructs. Therefore we formulate completeness
with representing functions restricted on the semantic variables allocated in the beginning (which includes all free variables of a goal). This does not
compromise our promise to prove the completeness of the search as \textsc{miniKanren} returns substitutions only for queried variables,
which are allocated in the beginning.

\begin{theorem}[Operational semantics completeness]
%\label{lem:gen_completeness}
If the indices of all free variables in a goal $g$ are limited by some number $n$, then

\[
\{\mathfrak{f}|_{\{\alpha_1,\dots,\alpha_n\}} \mid \mathfrak{f} \in \sembr{g}\} \subseteq \{\mathfrak{f}|_{\{\alpha_1,\dots,\alpha_n\}} \mid \mathfrak{f} \in \sembr{\inbr{g, \epsilon, n}}_{op}\}.
\]
\end{theorem}

Similarly to the soundness, this can be proven by nested induction, but the generalization is required. This time it is enough to generalize it from goals
to states of the shape $\inbr{g, \sigma, n}$. We also need to introduce one more auxiliary semantics~--- \emph{step-indexed denotational semantics} (denoted by $\sembr{\bullet}^i$). It is an implementation of the well-known approach~\cite{StepIndexing} of indexing typing or semantic logical relations by a number of permitted evaluation steps to allow inductive reasoning on it.
In our case, $\sembr{g}^i$ includes only those representing functions that one can get after no more than $i$ unfoldings of relational calls.

The step-indexed denotational semantics is an approximation of the conventional denotational semantics; it is clear that any answer in conventional denotational semantics will also be in step-indexed denotational semantics for some number of steps.

\begin{lemma}
$\sembr{g} \subseteq \cup_i \sembr{g}^i$
\end{lemma}

Now the generalized version of the completeness theorem is as follows.

\begin{lemma}[Generalized completeness]
\label{lem:gen_completeness}
For any set of relational definitions, for any number of unfoldings $i$, for any well-formed state $\inbr{g, \sigma, n}$,

\[
\{\mathfrak{f}|_{\{\alpha_1,\dots,\alpha_n\}} \mid \mathfrak{f} \in \sembr{g}^i \cap \sembr{\sigma}\} \subseteq \{\mathfrak{f}|_{\{\alpha_1,\dots,\alpha_n\}} \mid \mathfrak{f} \in \sembr{\inbr{g, \sigma, n}}_{op}\}.
\]
\end{lemma}

The proof is by the induction on nuber of unfoldings $i$. The induction step is proven by structural induction on goal $g$. We use lemmas~\ref{lem:sum_answers} and~\ref{lem:prod_answers} for evaluation of a disjunction and a conjunction respectively, and lemma~\ref{lem:den_sem_change_var} in the case of fresh variable introduction to move from an arbitrary semantic variable in denotational semantics to the next allocated fresh variable. The details of this proof may be found in the extended version of the paper.
