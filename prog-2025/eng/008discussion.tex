% !TEX TS-program = pdflatex
% !TeX spellcheck = en_US
% !TEX root = main.tex

We presented the results of our work on guideline-based synthesis of GUI layouts. The approach we take makes it possible to
automatically build GUI layouts which by construction comply with a designer-specified set of rules. The prototype implementation
we developed allows to synthesize the layouts for the real-word industrial GUI components w.r.t. the real-world guidelines in
appropriate time.

One interesting question which may arise is if the application of relational programming is essential for this problem to be solved. Indeed, the set of guidelines
describes a matching (or rewriting) system which, in principle, can be directly implemented without any use of relational techniques. We argue, however,
that in this case a whole piece of work on justification of the correctness of the solution would have been repeated anew. In our
case, the justification trivially follows from the completeness of the \textsc{miniKanren} search and refutational completeness of our solution.
We also speculate that such a solution would require reinventing of some implementation techniques to support nondeterminism and backtracking, which are already native to relational programming.
%Finally, the duality between patterns over structure and relational goals (initially unexpected for us), to our opinion, witnesses, that relational programming is a truly relevant technique for this problem.

In future, integration of coordinate constraints into \OCanren{} and relational solver should be studied in more detail.
It may seriously simplify the system, prune cases earlier and improve performance.
