% !TeX spellcheck = ru_RU
% !TEX root = main.tex

\section{Ограничения-неравенства}
\label{sec:diseq}

Зачастую для конкретной задачи пользователи расширяют реализацию реляционного программирования новыми ограничениями.
В этом разделе мы расскажем про ограничение неравенства, которое, строго говоря не обязательно, но полезно для широкого круга задач.

\subsection{Задача удаления элемента из списка}

Напишем отношение, которое принимает список и некоторый элемент, и возвращает список без первого вхождения этого элемента.

\begin{lstlisting}
let rec rembero1 x xs out =
  conde
    [ xs === Std.nil () &&& (xs === out)
    ; fresh (h tl) (xs === Std.List.cons h tl) (h === x) (tl === out)
    ; fresh (h tl res)
        (xs === Std.List.cons h tl)
        (out === Std.List.cons h res)
        (rembero1 x tl res)
    ]
\end{lstlisting}

\noindent Реализация повторяет оригинальную~\cite{WillThesis} для Scheme. Случай пустых списков тривиален, иначе нам надо либо удалить головной элемент \verb=h=, если он равен искомому \verb=x=, либо проигнорировать головной элемент и запуститься рекурсивно для хвоста списка.

Приведенная реализация успешно удаляет первое вхождение элемента и выдает ожидаемый ответ первым.
Но если запросить с помощью \verb=run= больше ответов, то, неожиданно, они найдутся.
Дело в том, что реляционный поиск попробует воспользоваться третьей ветвью поиска даже если искомый элемент находится в голове, и поэтому реляционная программа выше вернет по ответу на каждое удаление элемента \verb=x=, а также, не удалив ни одного \verb=x=, ещё один ответ --- исходный список.

\begin{lstlisting}
run * (fun q -> rembero1 !!2 [ 1; 2; 3; 2; 4 ] q)
  -> q=[1; 3; 2; 4]
  -> q=[1; 2; 3; 4]
  -> q=[1; 2; 3; 2; 4]
\end{lstlisting}
Чтобы решить эту проблему, необходимо сделать вторую и третью ветви \verb=conde= несовместными, сказав явно в третьей ветви, что \verb=x= не может быть равен \verb=h=.
После этой модификации реляционная программа \verb=rembero2= перестанет давать неожиданные ответы.



\begin{lstlisting}
let rec rembero2 x xs out =
  conde
    [ xs === Std.nil () &&& (xs === out)
    ; fresh (h tl) (xs === Std.List.cons h tl) (h === x) (tl === out)
    ; fresh
        (h tl res)
        (xs === Std.List.cons h tl)
        (h =/= x)
        (out === Std.List.cons h res)
        (rembero2 x tl res)
    ]
\end{lstlisting}

\subsection{Реализация проверки ограничений-неравенств}

Самым простым способом реализации ограничений-неравенств является их проверка с помощью унификации.
Для каждого ограничения из двух термов мы проводим их унификацию и действуем в зависимости от результата.
\begin{enumerate}
\item Если их унифицировать невозможно, то термы никогда не будут равными и ограничение можно выкинуть.
\item Если они унифицируются без расширения текущей подстановки, то термы являются равными в этой ветви поиска. Ограничение нарушено, поиск стоит продолжать в других ветвях.
\item Если термы унифицируются с путём добавления некоторых пар переменная-терм в текущую подстановку (\enquote{дельта}), то ограничение является содержательным, и его надо повторно проверять позже.
\end{enumerate}

К данной базовой реализации можно применить следующую оптимизацию, реализованную в OCanren и faster-miniKanren.
Заметим, что для полученной дельты если хотя бы одна пара переменная-терм унифицируется с непустой подстановкой, то скорее всего оригинальные два терма унифицируются с непустой подстановкой. Таким образом, можно сократить размер термов, которые надо унифицировать для проверки ограничений, что приводит к улучшениям производительности.

\subsection{Ограничения-неравенства и сложные типы данных}

Неаккуратное использование ограничений-неравенств для сложных типов данных может приводить к неожиданным результатам.
Рассмотрим реляционную спецификацию, которая постулирует, что переменная \verb=q= является натуральным числом (заданным в стиле Пеано) минимум с тремя конструкторами \verb=s= (следующий).
Другими словами, \verb=q= $\geq 3$.

\[
\verb|fun q -> fresh (tl) (q === s (s (s tl)))|
\]

При попытке выразить обратное (число меньше трех), заменив унификацию на ограничение неравенства, мы не добьёмся ожидаемого результата, так как для любого числа $>3$ найдется такое значение логической переменной \verb=q=, что ограничение будет выполненным, т.е. его можно просто выкинуть, оно не влияет на результат.
Ожидаемый результат получился бы, если бы конструкция \verb=fresh= здесь ввела бы универсально квантифицированную переменную, но она вводит экзистенциальную.
Добавление универсальной квантификации в miniKanren исследовалось\footnote{Ограничение eigen: \url{https://github.com/webyrd/miniKanren-with-symbolic-constraints}}, но на данный момент эффективная и корректная реализация не найдена.

Для решения этой проблемы в OCanren добавлен новый вид переменных --- шаблонные (англ. wildcard) логические переменные, которые обозначаются в синтаксисе с помощью двух подчеркиваний.
Их можно использовать на месте любого логического значения, но в реализации все шаблонные переменные представлены одинаково.
Они так названы в честь одноименной конструкции языка OCaml, которая позволяет при сопоставлении с образцом выразить, что пользователю не важно, как устроена часть терма.
Во время проверки ограничений-неравенств шаблонные переменные обрабатываются максимально пессимистичным образом: вместо них подразумевается терм, похожий на то, с чем унифицируется соответствующая шаблонная переменная.

Рассмотрим несколько примеров. Неравенство пар \verb|(1, _.10)| и \verb|(__, 2)| упрощается до неравенства переменной 10 и числа 2. Вместо шаблонной переменной пессимистично подставляется число 1, делая первые компоненты пар одинаковыми. Неравенство двух шаблонных переменных также упрощается максимально пессимистично: вместо переменные подставляется одинаковое значение и ограничение тривиально нарушается. Проверка неравенства свежих переменных и термов с шаблонными переменными внутри откладывается до момента уточнения свежих переменных. Например, ограничение \verb|(cons __ __ =/= _.10)| после уточнения 10й переменной с помощью унификации
 \verb|(cons _.11 _.12 === _.10)| упростится до \enquote{либо 11я, либо 12я переменная не унифицируется с шаблонной}. Разрешение неравенств свежих и шаблонных переменных откладывается на потом, до момента уточнения значений свежих переменных с помощью унификации.

Проверка неравенства шаблонных переменных и частично заданных термов заканчивается пессимистично и неуспешно. Именно поэтому пример со числами Пеано выше показывает ожидаемое поведение: достаточно большие числа \enquote{n} сведутся к проверке неравенства шаблонной переменной и \enquote{n-3}, для малых унификация закончится с ошибкой, не доходя до шаблонной переменной.
Данный пример можно переформулировать для других алгебраических типов данных, погруженных в реляционный домен.
В контексте оригинальной реализации на языке Scheme, шаблонные переменные также применимы, но с некоторыми оговорками. Например, встроенные в язык списки позволяют естественным образом представлять \enquote{списки длины хотя бы три}, поэтому в Schemе стоит демонстрировать шаблонные переменные с числами Пеано.

Шаблонные переменные применимы не только на \enquote{игрушечном} примере, показанном выше.
Автоматические синтез~\cite{Lozov2018} реляционных программ из функциональных может воспользоваться шаблонными переменными при обработке сопоставления с образцом.
На данные момент ветви сопоставления с образцом отображаются в набор реляционных спецификаций, соединенных дизъюнкцией с помощью \verb=conde= (рис.~\ref{fig:matching-example3}).
Это приводит к неожиданным результатам в виде дополнительных ответов, так как ветки conde независимы, а при сопоставлении с образцом выбирается всегда одна.
В этом случае шаблонные переменные позволяют описать (рис.~\ref{fig:matching-example4}) в спецификации набор ограничений, которые сделают ветви дизъюнктными.


\begin{figure*}[t]
  \centering
  \renewcommand\thesubfigure{\alph{subfigure}1}
  \begin{lstlisting}
match x, y, z with
| _, F, T -> 1
| F, T, _ -> 2
| _, _, F -> 3
| _, _, T -> 4
  \end{lstlisting}
  \vskip4.5mm
  \caption{Пример сопоставления с образцом в языке OCaml}
  \label{fig:matching-example1}
\end{figure*}

\begin{figure*}[t]
%\centering
\begin{subfigure}{0.49\textwidth}
\begin{lstlisting}
let enc_with_diseq q rez =
  let _T = !!true in
  let _F = !!false in
  let w = Std.triple in
  conde
    [ fresh (fresh1)
        (rez === !!1)
        (q === w fresh1 _F _T)
    ; fresh (fresh1 x)
        (rez === !!2)
        (q === w _F _T fresh1)
        (q =/= w x _F _T)
    ; fresh (fresh1 fresh2 x y z)
        (rez === !!3)
        (q === w fresh1 fresh2 _F)
        (q =/= w x _F _T)
        (q =/= w _F _T z)
    ; fresh (x y z fresh1 fresh2)
        (rez === !!4)
        (q =/= w x _F _T)
        (q =/= w _F _T z)
        (q =/= w x y _F)
        (q === w fresh1 fresh2 _T) ]
\end{lstlisting}
\caption{С ограничениями-неравенствами}
\label{fig:matching-example3}
\end{subfigure}
%\vskip13.5mm
\begin{subfigure}{0.49\textwidth}
\begin{lstlisting}
let enc_with_wc q rez =
    let _T = !!true in
    let _F = !!false in
    let w = Std.triple in
    conde
      [ fresh ()
          (rez === !!1)
          (q === w __ _F _T)
      ; fresh ()
          (rez === !!2)
          (q === w _F _T __)
          (q =/= w __ _F _T)
      ; fresh ()
          (rez === !!3)
          (q === w __ __ _F)
          (q =/= w __ _F _T)
          (q =/= w _F _T __)
      ; fresh ()
          (rez === !!4)
          (q =/= w __ _F _T)
          (q =/= w _F _T __)
          (q =/= w __ __ _F)
          (q === w __ __ _T) ]
\end{lstlisting}
\caption{С шаблонными переменными}
\label{fig:matching-example4}
\end{subfigure}
\caption{Две возможные трансляции примера с рис.~\ref{fig:matching-example1}}
\label{fig:maranget-example-compilation}
%\vskip13.5mm
\end{figure*}

