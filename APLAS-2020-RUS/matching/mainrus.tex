% !TEX TS-program = xelatex
% !TeX spellcheck = ru_RU
% !TEX root = mainrus.tex

%%
%% This is file `sample-acmsmall-conf.tex',
%% generated with the docstrip utility.
%%
%% The original source files were:
%%
%% samples.dtx  (with options: `all,proceedings,bibtex,acmsmall-conf')
%%
%% IMPORTANT NOTICE:
%%
%% For the copyright see the source file.
%%
%% Any modified versions of this file must be renamed
%% with new filenames distinct from sample-acmsmall-conf.tex.
%%
%% For distribution of the original source see the terms
%% for copying and modification in the file samples.dtx.
%%
%% This generated file may be distributed as long as the
%% original source files, as listed above, are part of the
%% same distribution. (The sources need not necessarily be
%% in the same archive or directory.)
%%
%%
%% Commands for TeXCount
%TC:macro \cite [option:text,text]
%TC:macro \citep [option:text,text]
%TC:macro \citet [option:text,text]
%TC:envir table 0 1
%TC:envir table* 0 1
%TC:envir tabular [ignore] word
%TC:envir displaymath 0 word
%TC:envir math 0 word
%TC:envir comment 0 0
%%
%%
%% The first command in your LaTeX source must be the \documentclass
%% command.
%%
%% For submission and review of your manuscript please change the
%% command to \documentclass[manuscript, screen, review]{acmart}.
%%
%% When submitting camera ready or to TAPS, please change the command
%% to \documentclass[sigconf]{acmart} or whichever template is required
%% for your publication.
%%
%%
\documentclass[acmsmall]{acmart}

% !TEX TS-program = pdflatex
% !TeX spellcheck = en_US
% !TEX root = main.tex
\newcommand{\OCanren}{\textsc{OCanren}}
\newcommand{\OCaml}{\textsc{OCaml}}
\newcommand{\Scheme}{\textsc{Scheme}}
\newcommand{\Kotlin}{\textsc{Kotlin}}
\newcommand{\Klogic}{\textsc{Klogic}}
%%
%% \BibTeX command to typeset BibTeX logo in the docs
\AtBeginDocument{%
  \providecommand\BibTeX{{%
    Bib\TeX}}}

%% Rights management information.  This information is sent to you
%% when you complete the rights form.  These commands have SAMPLE
%% values in them; it is your responsibility as an author to replace
%% the commands and values with those provided to you when you
%% complete the rights form.
%\setcopyright{acmlicensed}
%\copyrightyear{2018}
%\acmYear{2018}
%\acmDOI{XXXXXXX.XXXXXXX}

%% These commands are for a PROCEEDINGS abstract or paper.
%\acmConference[Conference acronym 'XX]{Make sure to enter the correct
%  conference title from your rights confirmation emai}{June 03--05,
%  2018}{Woodstock, NY}
%%
%%  Uncomment \acmBooktitle if the title of the proceedings is different
%%  from ``Proceedings of ...''!
%%
%%\acmBooktitle{Woodstock '18: ACM Symposium on Neural Gaze Detection,
%%  June 03--05, 2018, Woodstock, NY}
%\acmISBN{978-1-4503-XXXX-X/18/06}


%%
%% Submission ID.
%% Use this when submitting an article to a sponsored event. You'll
%% receive a unique submission ID from the organizers
%% of the event, and this ID should be used as the parameter to this command.
%%\acmSubmissionID{123-A56-BU3}

%%
%% For managing citations, it is recommended to use bibliography
%% files in BibTeX format.
%%
%% You can then either use BibTeX with the ACM-Reference-Format style,
%% or BibLaTeX with the acmnumeric or acmauthoryear sytles, that include
%% support for advanced citation of software artefact from the
%% biblatex-software package, also separately available on CTAN.
%%
%% Look at the sample-*-biblatex.tex files for templates showcasing
%% the biblatex styles.
%%

%%
%% The majority of ACM publications use numbered citations and
%% references.  The command \citestyle{authoryear} switches to the
%% "author year" style.
%%
%% If you are preparing content for an event
%% sponsored by ACM SIGGRAPH, you must use the "author year" style of
%% citations and references.
%% Uncommenting
%% the next command will enable that style.
%%\citestyle{acmauthoryear}


%%
%% end of the preamble, start of the body of the document source.
\begin{document}

%%
%% The "title" command has an optional parameter,
%% allowing the author to define a "short title" to be used in page headers.
\title{Relational Synthesis for Pattern Matching}

%%
%% The "author" command and its associated commands are used to define
%% the authors and their affiliations.
%% Of note is the shared affiliation of the first two authors, and the
%% "authornote" and "authornotemark" commands
%% used to denote shared contribution to the research.
\author{Косарев Дмитрий}
%\authornote{Both authors contributed equally to this research.}
\email{d.kosarev@spbu.ru,  Dmitrii.Kosarev@pm.me }
\orcid{0000-0002-6773-5322}
\author{Лозов Петр}
%\authornotemark[1]
\email{lozov.peter@gmail.com}
\orcid{0000-0003-3563-2828}
\affiliation{%
  \institution{СПбГУ}
  \city{Санкт-Петербург}
  \country{Россия}
}


\author{Дмитрий Булычев}
\affiliation{%
  \institution{СПбГУ}
  \city{Санкт-Петербург}
  \country{Россия}
}
\email{dboulytchev@math.spbu.ru}




%%
%% By default, the full list of authors will be used in the page
%% headers. Often, this list is too long, and will overlap
%% other information printed in the page headers. This command allows
%% the author to define a more concise list
%% of authors' names for this purpose.
\renewcommand{\shortauthors}{TODO et al.}

%%
%% The abstract is a short summary of the work to be presented in the
%% article.
\begin{abstract}
\documentclass[10pt, oneside, nocopyrightspace]{sigplanconf}

\usepackage{amssymb}
\usepackage{listings}
\usepackage{indentfirst}
\usepackage{verbatim}
\usepackage{amsmath, amsthm, amssymb}
\usepackage{graphicx}
\usepackage[hyphens]{url}
\usepackage[hidelinks]{hyperref}
\usepackage{verbatim}

\topmargin 2.0cm
\setlength{\textheight}{25.3cm}

\lstdefinelanguage{ocaml}{
keywords={fresh, let, begin, end, in, match, type, and, fun, function, try, with, when, class, 
object, method, of, rec, repeat, until, while, not, do, done, as, val, inherit, 
new, module, sig, deriving, datatype, struct, if, then, else, open, private, virtual},
sensitive=true,
basicstyle=\small,
commentstyle=\small\itshape\ttfamily,
keywordstyle=\ttfamily\underbar,
identifierstyle=\ttfamily,
basewidth={0.5em,0.5em},
columns=fixed,
fontadjust=true,
literate={->}{{$\;\;\to\;\;$}}1,
morecomment=[s]{(*}{*)}
}

\lstset{
basicstyle=\small,
identifierstyle=\ttfamily,
keywordstyle=\bfseries,
commentstyle=\scriptsize\rmfamily,
basewidth={0.5em,0.5em},
fontadjust=true,
%escapechar=~,
language=ocaml
}

\sloppy

\newcommand{\miniKanren}{\texttt{miniKanren}}

\begin{document}

\title{Typed Embedding of a Relational Language in OCaml}

\authorinfo{Dmitry Kosarev \and Dmitri Boulytchev}
{St.Petersburg State University \\ 
  Saint-Petersburg, Russia }
{$\mathtt{Dmitrii.Kosarev@protonmail.ch}$ \and $\mathtt{dboulytchev@math.spbu.ru}$}

\maketitle

\small
\begin{abstract}
\small
We present an implementation of relational programming language miniKanren as a set 
of combinators and syntax extension for OCaml. The key feature of our approach is 
\emph{polymorphic unification}, which can be used to unify data structures of almost 
arbitrary types. In addition we provide a useful generic programming pattern to 
systematically develop relational specifications in a typed manner, and address 
the problem of relational and functional code integration.
\end{abstract}

\section{Introduction}
\label{intro}

Relational programming~\cite{TRS} is an attractive technique, based on the idea 
of constructing programs as relations.  As a result, relational programs can be
``queried'' in various ``directions'', making it possible, for example, to simulate
reversed execution. 

\begin{comment}
Apart from being interesting from purely theoretical standpoint, 
this approach may have a practical value: some problems look much simpler, 
if they are considered as queries to relational specification. 
\end{comment}

\begin{comment}
There is a 
number of appealing examples, confirming this observation: a type checker 
for simply typed lambda calculus (and, at the same time, type inferencer and solver 
for the inhabitation problem), an interpreter (capable of generating ``quines''~--- 
programs, producing themselves as result)~\cite{Untagged}, list sorting (capable of 
producing all permutations) etc. 
\end{comment}

\begin{comment}
Many logic programming languages, such as Prolog, Mercury
%\footnote{\url{https://mercurylang.org}}, 
or Curry
%\footnote{\url{http://www-ps.informatik.uni-kiel.de/currywiki}} 
to some extent
can be considered as relational.
\end{comment}

We have chosen miniKanren\footnote{\url{http://minikanren.org}} 
as model language, because it was specifically designed as relational DSL, embedded in Scheme/Racket. 
Being rather a minimalistic language, which can be implemented with just a few data structures and
combinators, miniKanren found its way in dozens of host languages, including Haskell, 
Standard ML and OCaml.

There is, however, a predictable glitch in implementing miniKanren for strongly typed language. 
Designed in a metaprogramming-friendly and dynamically typed realm of Scheme/Racket, original 
miniKanren implementation pays very little attention to what has a significant importance in (specifically) 
ML or Haskell. In particular, one of the capstone constructs of miniKanren~--- unification~--- has to work for 
different data structures, which may have types, different beyond parametricity.

\begin{comment}
There are a few ways to overcome this problem. The first one is simply to follow the untyped paradigm and
provide unification for some concrete type, rich enough to represent any reasonable data structures.
Some Haskell miniKanren libraries\footnote{\url{https://github.com/JaimieMurdock/HK}, \url{https://github.com/rntz/ukanren}}
as well as existing OCaml implementation\footnote{\url{https://github.com/lightyang/minikanren-ocaml}} take this way. 
As a result, the original implementation can be retold with all its elegance; relational specifications, however,
become weakly typed. Another approach is to utilize \emph{ad hoc} polymorphism and provide type-specific
unification for each ``interesting'' type; Molog\footnote{\url{https://github.com/acfoltzer/Molog}} and 
MiniKanrenT\footnote{\url{https://github.com/jvranish/MiniKanrenT}}, both for Haskell, can be mentioned as examples.
While preserving strong typing, this approach requires a lot of ``boilerplate'' code to be written, so some
automation, for example, using Template Haskell\footnote{\url{https://wiki.haskell.org/Template_Haskell}},
is desirable. There is, actually, another potential approach, but we do not know, if anybody tried
it: to implement unification for generic representation of types as sum-of-products and fixpoints of 
functors~\cite{InstantGenerics, ALaCarte}. Thus, unification would work for any types, for which representation
is provided. We assume, that implementing representation would require less boilerplate code.

As it follows from given exposition, typed embedding of miniKanren in OCaml can be done with
a combination of datatype-generic programming~\cite{DGP} and \emph{ad hoc} polymorphism. There are a 
number of generic frameworks for OCaml (for example,~\cite{Deriving}). On the other hand, the support
for \emph{ad hoc} polymorphism in OCaml is weak; there is nothing comparable in power with Haskell 
type classes, and despite sometimes object-oriented layer of the language can be used to mimic
desirable behavior, the result as a rule is far from satisfactory. Existing proposals (for example, 
module implicits~\cite{Implicits}) require patching the compiler, which we tend to avoid.
\end{comment}

We present an implementation of a set of relational combinators in OCaml, which, 
technically speaking, corresponds to $\mu$Kanren~\cite{MicroKanren} with disequality 
constraints~\cite{CKanren}; syntax extension for the ``\lstinline{fresh}'' construct is
added as well. The contribution of our work is as follows:

\begin{enumerate}
\item Our implementation is based on \emph{polymorphic unification}, which, like polymorphic comparison,
can be used for almost arbitrary types. 

\begin{comment}
The implementation of polymorphic unification uses unsafe features and
relies on intrinsic knowledge of runtime representation of values; we show, however, that this does not
compromise type safety. Practically, we applied purely \emph{ad hoc} approach since the features, 
which would provide less \emph{ad hoc} solution are not yet integrated into the mainstream language.
\end{comment}

\item We describe a uniform and scalable pattern for using types for relational programming, which
helps in converting typed data to- and from relational domain. 

\begin{comment}
With this pattern, only one
generic feature (``\lstinline{map/morphism/Functor}'') is needed, and thus virtually any generic 
framework for OCaml can be used. Despite being rather a pragmatic observation, this pattern, as we
believe, would lead to a more regular and easy to maintain relational specifications.
\end{comment}

\item We provide a simplified way to integrate relational and functional code. Our approach utilizes
well-known pattern~\cite{Unparsing, DoWeNeed} for variadic function implementation and makes it
possible to hide refinement of the answers phase from an end-user.
\end{enumerate}

The source code of our implementation is accessible from \url{https://github.com/dboulytchev/OCanren}.
An extended version of this abstract can be downloaded from \url{http://oops.math.spbu.ru/papers/ocanren.pdf}.

\section{Polymorphic Unification}
\label{polyuni}

We consider it rather natural to employ polymorphic unification in the
language, already equipped with polymorphic comparison~--- a convenient, but
somewhat disputable
%\footnote{See, for example, \url{https://blogs.janestreet.com/the-perils-of-polymorphic-compare}} 
feature. 

\begin{comment}
Like polymorphic comparison, polymorphic unification performs traversal
of values, exploiting intrinsic knowledge of their runtime representation. 
The undeniable benefit of this solution is that in order to perform unification 
for user types no ``boilerplate'' code is needed. 
\end{comment}
\begin{comment}
On the other hand, all pitfalls of
polymorphic comparison are inherited as well; in particular, unification can loop 
for cyclic data structures and does not work for functional values. Since we generally 
do not expect any reasonable outcome in these cases, the only remaining problem is that
the compiler is incapable to provide any assistance in identifying 
and avoiding them. Another drawback is that the implementation of polymorphic unification
relies on runtime representation of values and have to be fixed every time the representation changes. 
Finally, as it written in unsafe manner using \lstinline{Obj} interface, it has to be
carefully developed and tested.
\end{comment}

An important difference between polymorphic comparison and unification is that the former 
only inspects its operands, while the results of unification are recorded in a substitution
(mapping from logical variables to terms), which later is used to refine answers and reify 
constraints. So, generally speaking, we have to show, that no ill-typed terms are constructed 
as a result.

Polymorphic unification is introduced via the following function:

\begin{lstlisting}[mathescape=true]
   val unify : $\alpha$ logic -> $\;\;\alpha$ logic -> $\;\;$subst option -> 
     $\;\;$subst option
\end{lstlisting}

\noindent where ``\lstinline[mathescape=true]{$\alpha$ logic}'' stands for the type $\alpha$, 
injected into the logic domain, ``\lstinline{subst}''~--- for the type of substitution. 

\begin{comment}
Unification can 
fail (hence ``\lstinline{option}'' in the result type), is performed in the context of
existing substitution (hence ``\lstinline{subst}'' in the third argument) and
can be chained (hence ``\lstinline{option}'' in the third argument). Note, that the 
type of substitution is not polymorphic, which means, that compiler completely looses the 
track of types for values, stored in a substitution. These types are recovered later during
refinement of answers.
\end{comment}

To justify the correctness of unification, we consider a set of typed terms, each of which
has one of two forms

$$
x^\tau \mid C^\tau(t_1^{\tau_1},\dots,t_k^{\tau_k})
$$

\noindent where $x^\tau$ denotes a logical variable of type $\tau$, 
$C^\tau$~--- some constructor of type $\tau$, $t_i^{\tau_i}$~--- some terms of types $\tau_i$.
We reflect by $t_1^\tau[t_2^\rho]$ the fact of $t_2^\rho$ being a subterm of $t_1^\tau$, and
assume, that $\rho$ is unambiguously determined by $t_1$, $\tau$, and a position of $t_2$ 
``inside'' $t_1$.

\begin{comment}
Outside unification the compiler maintains typing, which means, that all 
terms, subterms, and variables agree in their types in all contexts. However, as 
our implementation resorts to unsafe features, we have to make this work for unification
code on our own.
\end{comment}

We show, that the following three invariants are maintained for any substitution $s$, involved 
in unification:

\begin{enumerate}
\item if \mbox{$t_1^{\_}[x^{\tau}]$} and \mbox{$t_2^{\_}[x^{\rho}]$}~--- two arbitrary terms (in particular, 
$t_1^{\_}$ and $t_2^{\_}$ may be the same), bound in $s$ and containing occurrences of variable $x$, 
then $\rho=\tau$ (different occurrences of the same variable in $s$ are attributed with the same type);

\item if \mbox{$(s\;\;x^\tau)$} is defined, then \mbox{$(s\;\;x^\tau) = t^\tau$} (a substitution always
binds a variable to a term of the same type);

\item each variable in $s$ preserves its type, assigned by the compiler (from the first two invariants 
it follows, that this type is unique; note also, that all variables are created and have their 
types assigned outside unification, in a type-safe world).
\end{enumerate}

\begin{comment}
The initial (empty) substitution trivially fulfills these invariants; hence, it is sufficient
to show, that they are preserved by unification.

The following snippet presents the implementation of unification with triangular 
substitution in only a little bit more abstract form, than actual code (for example, 
``occurs check'' is omitted):

\begin{lstlisting}[mathescape=true,numbers=left,numberstyle=\tiny,stepnumber=1,numbersep=-5pt]
   let rec walk $s$ = function
   | $x^\tau$ when $x\in dom(s)$ -> $\;\;$walk $s$ $(s\;\;x)^\tau$
   | $t^\tau$ -> $\;\;t^\tau$

   let rec unify $t_1^\tau$ $t_2^\tau$ = function
   | None -> None
   | Some $s$ as $sub$  ->
       match walk $s$ $t_1$, walk $s$ $t_2$ with
       | $x_1^\tau$, $x_2^\tau$ when $x_1$ = $x_2$ -> $\;\;sub$
       | $x_1^\tau$, $(t_2^\prime)^\tau$ -> $\;\;$Some ($s[x_1 \gets t_2^\prime]$)
       | $(t_1^\prime)^\tau$, $x_2^\tau$ -> $\;\;$Some ($s[x_2 \gets t_1^\prime]$)
       | $C^\tau(t_1^{\tau_1},\dots,t_k^{\tau_k})$, $C^\tau(p_1^{\tau_1},\dots,p_k^{\tau_k})$ -> 
           unify $t_k^{\tau_k}$ $p_k^{\tau_k}$(.. (unify $t_1^{\tau_1}$ $p_1^{\tau_1}$ $sub$)$..$)
       | $\_$, $\_$ -> $\;\;$None
\end{lstlisting}

Type annotations, included in the snippet above, can be justified by the following 
reasonings\footnote{We omit the verbal description of unification algorithm; 
the details can be found in~\cite{MicroKanren}.}:

\begin{enumerate}
\item Line 2: the type of \mbox{$(s\;\;x^\tau)$} is $\tau$ due to the invariant 2; hence, 
the type of \lstinline{walk} result coincides with the type of its second argument (technically,
an induction on a number of recursive invocations of \lstinline{walk} is needed).

\item Line 9: the substitution is left unchanged, hence all invariants are preserved.

\item Line 10 (and, symmetrically, line 11): first, note, that \mbox{$(s\;\;x_1)$} is undefined
(otherwise \lstinline{walk} would not return $x_1$). Then, $x_1$ and $t_2^\prime$ have the
same type, which justifies the preservation of invariant 2. Finally, either \mbox{$x_1=t_1$}
(and, then, $\tau$ is the type of $x_1$, assigned by the compiler), or $x_1$ is retrieved
from $s$ with type $\tau$~--- both cases justify invariants 1 and 3. The same applies to 
the pair $t_2^\prime$ and $t_2$.

\item The previous paragraph justifies the base case for inductive proof on the number of
recursive invocations of \lstinline{unify}.
\end{enumerate}
\end{comment}

The function \lstinline{unify} is not directly accessible at the user level; it is used
to implement both unification (``\lstinline{===}'') and disequality (``\lstinline{=/=}'') 
goals. The implementation generally follows~\cite{CKanren}.

\section{Logic Variables and Injection}
\label{logics}

Unification, considered in Section~\ref{polyuni}, works for values of type \lstinline[mathescape=true]{$\alpha$ logic}. 
Any value of this type can be seen as either a value of type $\alpha$, or a logical variable of type $\alpha$. 

\begin{comment}
The type 
itself in made abstract, but its values can be uncovered after refinement (see Section~\ref{refinement}).
\end{comment}

Free variables can be created solely using the ``\lstinline{fresh}'' construct of miniKanren. Note, that 
since the unification is implemented in an untyped manner, we can not use simple pattern matching to
distinguish logical variables from other logical values. A special attention was paid to implement
variable recognition in constant time.

Apart from variables, other logical values can be obtained by injection; conversely, sometimes
a logical value can be projected to regular one. We supply two functions\footnote{``\lstinline{inj}'' and ``\lstinline{prj}'' in concrete syntax.}
for these purposes

\begin{lstlisting}[mathescape=true]
   val ($\uparrow$) : $\alpha$ -> $\;\;\alpha$ logic
   val ($\downarrow$) : $\alpha$ logic -> $\;\;\alpha$
\end{lstlisting}

As expected, injection is total, while projection is partial. Using these functions and type-specific
``\lstinline{map}'', which can be derived automatically using a number of existing frameworks for
generic programming, one can easily provide injection and projection for user-defined datatypes. We
consider user-defined list type as an example:

\begin{lstlisting}[mathescape=true]
   type ($\alpha$, $\beta$) list = Nil | Cons of $\alpha$ * $\beta$
   
   type $\alpha$ glist = ($\alpha$, $\alpha$ glist) list
   type $\alpha$ llist = ($\alpha$ logic, $\alpha$ llist) list logic

   let rec inj_list l = $\uparrow$(map$_{\mbox{\texttt{list}}}$ ($\uparrow$) inj_list l) 
   let rec prj_list l = map$_{\mbox{\texttt{list}}}$ ($\downarrow$) prj_list ($\downarrow$ l)
\end{lstlisting}

Here ``\lstinline{list}'' is a custom type for lists; note, that it made more
polymorphic, than usual~--- we abstracted it from itself and made it non-recursive. Then
we provided two specialized versions~--- ``\lstinline{glist}'' (``ground'' list), which 
corresponds to regular, non-logic lists, and ``\lstinline{llist}'' (``logical'' list), which
corresponds to logical lists with logical elements. Using single type-specific function
\lstinline[mathescape=true]{map$_{\mbox{\texttt{list}}}$}, we easily provided injection 
(of type {\lstinline[mathescape=true]{$\alpha$ glist -> $\;\;\alpha$ llist}}) and
projection (of type {\lstinline[mathescape=true]{$\alpha$ llist -> $\;\;\alpha$ glist}}).

\begin{comment}
In context of these definitions, we can implement relational list concatenation, 
which is often considered as one of first-step examples of miniKanren programming:

\begin{lstlisting}[mathescape=true]
   let rec append$^o$ x y xy =
     conde [
       (x === $\uparrow$ Nil) &&& (xy === y);
       fresh (h t ty)
         (x  === $\uparrow$(Cons (h, t))
         (xy === $\uparrow$(Cons (h, ty))
         (append$^o$ t y ty)
     ]
\end{lstlisting}
\end{comment}

\begin{comment}
Note, than in the definition of \lstinline[mathescape=true]{append$^o$} we
used only default injection (``$\uparrow$''). Customized version most likely would 
appear in some top-level goal, for example:

\begin{lstlisting}[mathescape=true]
   (fun q -> append$^o$ (inj_list [1; 2; 3]) 
                    (inj_list [4; 5; 6]) 
                    q
   )
\end{lstlisting}
\end{comment}

\section{Refinement and Top-Level Primitives}
\label{refinement}

The result of relational program is a stream of substitutions, each of which represents
a certain answer. As a rule, a substitution binds many intermediate logical variables, 
created by ``\lstinline{fresh}'' in the course of execution. A meaningful answer has to be
\emph{refined}.

In our implementation refinement is represented by the following function:

\begin{lstlisting}[mathescape=true]
   val refine : subst -> $\alpha$ logic -> $\;\;\alpha$ logic
\end{lstlisting}

This function takes a substitution and a logical value and recursively substitutes
all logical variables in that value w.r.t. the substitution until no occurrences of 
bound variables are left. Since in our implementation the type of substitution is
not polymorphic, \lstinline{refine} is also implemented in an unsafe manner. However,
it is easy to see, that \lstinline{refine} does not produce ill-typed terms. Indeed,
all original types of variables are preserved in a substitution due to invariant
3 from Section~\ref{polyuni}. Unification does not change unified terms, so all terms, 
bound in a substitution, are well-typed. Hence, \lstinline{refine} always substitutes
some subterm in a well-typed term with another term of the same type, which preserves
well-typedness.

In addition to performing substitutions, \lstinline{refine} also \emph{reifies} 
disequality constrains. Reification attaches to each free variable in a refined
term a list of \emph{refined} terms, describing disequality constraint for that
free variable. Note, that disequality can be established only for equally typed
terms, which justifies type-safety of reification. 

\begin{comment}
Note also, that additional care has 
to be taken to avoid infinite looping, since refinement and reification are
mutually recursive, and refinement of a variable can be potentially invoked from 
itself due to a chain of disequalify constraints.

After refinement the content of logical value can be inspected via the following 
function:

\begin{lstlisting}[mathescape=true]
   val destruct : $\alpha$ logic -> 
     [`Var of int * $\alpha$ logic list | `Value of $\alpha$]
\end{lstlisting}

Constructor \lstinline{`Var} corresponds to a free variable with unique
integer identifier and a list of terms, representing all disequality constraints
for this variable. These terms are refined as well.
\end{comment}

\begin{comment}
We did not make \lstinline{refine} accessible for an end-user; instead we provided
a set of top-level combinators, which should be used to surround relational code
and perform refinement in a transparent manner. Note, that from pragmatic
standpoint only variables, supplied as arguments for the top-level goal, have
to be refined (the original miniKanren implementation follows the same convention).
\end{comment}

The toplevel primitive in our implementation is \lstinline{run}, which takes three
arguments. The exact type of \lstinline{run} is rather complex and non-instructive, 
so we better describe the typical form of its application:

\begin{lstlisting}[mathescape=true]
   run $\overline{n}$ (fun $l_1\dots l_n$ -> $\;\;G$) (fun $a_1\dots a_n$ -> $\;\;H$)
\end{lstlisting}

Here $\overline{n}$ stands for a \emph{numeral}, which describes the number of
parameters for two other arguments of \lstinline{run}, \mbox{$l_1\dots l_n$}~---
free logical variables, $G$~--- a goal (which can make use of \mbox{$l_1\dots l_n$}), 
\mbox{$a_1\dots a_n$}~--- refined answers for \mbox{$l_1\dots l_n$}, respectively, and, 
finally, $H$~--- a \emph{handler} (which can make use of \mbox{$a_1\dots a_n$}). The types of 
\mbox{$l_1\dots l_n$} are inferred from $G$, and the types of \mbox{$a_1\dots a_n$} are
inferred from the types of \mbox{$l_1\dots l_n$}: if $l_i$ has type \lstinline[mathescape=true]{$t$ logic}, then
$a_i$ has type \lstinline[mathescape=true]{$t$ logic stream}. In other words, user-defined handler
takes streams of refined answers for all variables, supplied to the top-level goal. All streams $a_i$ contain
coherent elements, so they all have the same length and $n$-th elements of all streams correspond 
to the $n$-th answer, produced by the goal $G$.

There are a few predefined numerals for one, two, etc. arguments (called, by tradition, 
\lstinline{q}, \lstinline{qr}, \lstinline{qrs} etc.), and a successor function, which 
can be applied to existing numeral to increment the number of expected arguments. The
technique, used to implement them, generally follows~\cite{Unparsing, DoWeNeed}.

As a final example we consider a program, which calculates the list of all permutations 
of given list of integers, using relational sorting (some supplementary function 
definitions are omitted):

\begin{lstlisting}[mathescape=true]
let minmax$^o$ a b min max = conde [
    (min === a) &&& (max === b) &&& (le$^o$ a b);
    (max === a) &&& (min === b) &&& (gt$^o$ a b)]
let rec smallest$^o$ l s l' = conde [       
    (l === $\uparrow$(Cons (s, $\uparrow$Nil))) &&& (l' === $\uparrow$Nil);
    fresh (h t s' t' max)
      (l' === $\uparrow$(Cons(max,t')))
      (l === $\uparrow$(Cons(h,t)))
      (minmax$^o$ h s' s max)
      (smallest$^o$ t s' t')] 
let rec sort$^o$ x y = conde [
    (x === $\uparrow$Nil) &&& (y === $\uparrow$Nil);
    fresh (s xs xs')
      (y === $\uparrow$(Cons (s, xs')))
      (sort$^o$ xs xs')       
      (smallest$^o$ x s xs)]
let perm l = map prj_nat_list @@ run q 
  (fun q -> fresh (r) (sort$^o$ (inj_nat_list l) r)(sort$^o$ q r))
  (take ~n:(fact @@ length l))
\end{lstlisting}

\begin{comment}
\begin{lstlisting}[mathescape=true]
   let perm l = map prj_int_list @@ 
     run q 
       (fun q -> sort$^o$ q (inj_int_list (sort compare l))) 
       (take ~n:(fact @@ length l))
\end{lstlisting}
\end{comment}

Here \lstinline{take} is a function for a stream, which returns the specified 
number of its first items as regular list.

\begin{comment}
\section{Conclusion}

We presented strongly typed implementation of miniKanren for OCaml. Our implementation
passes all tests, written for miniKanren (including those for disequality constraints);
in addition we implemented many of interesting relational programs, known from
the literature. We claim, that our implementation can be used both as a convenient
relational DSL for OCaml and an experimental framework for future research in the area of
relational programming. 

The source code of our implementation is accessible from \url{https://github.com/dboulytchev/OCanren}.

We also want to express our gratitude to William Byrd, who infected us with relational programming, for 
the time he sacrificed to very instructive and clarifying conversations.
\end{comment}

\begin{thebibliography}{99}
\bibitem{TRS}
Daniel P. Friedman, William E.Byrd, Oleg Kiselyov. The Reasoned Schemer. The MIT
Press, 2005.

\bibitem{MicroKanren}
Jason Hemann, Daniel P. Friedman. $\mu$Kanren: A Minimal Core for Relational Programming //
Proceedings of the 2013 Workshop on Scheme and Functional Programming (Scheme '13).

\bibitem{CKanren}
Claire E. Alvis, Jeremiah J. Willcock, Kyle M. Carter, William E. Byrd, Daniel P. Friedman.
cKanren: miniKanren with Constraints // 
Proceedings of the 2011 Workshop on Scheme and Functional Programming (Scheme '11).

\bibitem{Untagged}
William E. Byrd, Eric Holk, Daniel P. Friedman.
miniKanren, Live and Untagged: Quine Generation via Relational Interpreters (Programming Pearl) //
Proceedings of the 2012 Workshop on Scheme and Functional Programming (Scheme '12).

\begin{comment}
\bibitem{Implicits}
Leo White, Fr\'ed\'eric Bour, Jeremy Yallop. 
Modular Implicits // Workshop on ML, 2014, arXiv:1512.01438.
\end{comment}

\bibitem{Unparsing}
Olivier Danvy.
Functional Unparsing // Journal of Functional Programming, Vol.~8, Issue~6, November 1998.

\bibitem{DoWeNeed}
Daniel Fridlender, Mia Indrika.
Do we need dependent types? // Journal of Functional Programming, Vol.~10, Issue~4, July 2000.

\begin{comment}
\bibitem{DGP}
Jeremy Gibbons. Datatype-generic Programming //
Proceedings of the 2006 International Conference on Datatype-generic Programming.

\bibitem{Deriving}
Jeremy Yallop. 
Practical Generic Programming in OCaml // Proceedings of 2007 Workshop on ML.
\end{comment}

\begin{comment}
\bibitem{InstantGenerics}
Manuel M. T. Chakravarty, Gabriel C. Ditu, Roman Leshchinskiy. 
Instant Generics: Fast and Easy. \url{http://www.cse.unsw.edu.au/~chak/papers/CDL09.html}, 2009.

\bibitem{ALaCarte}
Wouter Swierstra. Data Types \'a la Carte  // Journal of Functional Programming, Vol.~18, Issue~4, 2008.
\end{comment}

\end{thebibliography}

\end{document}


\end{abstract}

%%
%% The code below is generated by the tool at http://dl.acm.org/ccs.cfm.
%% Please copy and paste the code instead of the example below.
%%
%\begin{CCSXML}
%<ccs2012>
% <concept>
%  <concept_id>00000000.0000000.0000000</concept_id>
%  <concept_desc>Do Not Use This Code, Generate the Correct Terms for Your Paper</concept_desc>
%  <concept_significance>500</concept_significance>
% </concept>
% <concept>
%  <concept_id>00000000.00000000.00000000</concept_id>
%  <concept_desc>Do Not Use This Code, Generate the Correct Terms for Your Paper</concept_desc>
%  <concept_significance>300</concept_significance>
% </concept>
% <concept>
%  <concept_id>00000000.00000000.00000000</concept_id>
%  <concept_desc>Do Not Use This Code, Generate the Correct Terms for Your Paper</concept_desc>
%  <concept_significance>100</concept_significance>
% </concept>
% <concept>
%  <concept_id>00000000.00000000.00000000</concept_id>
%  <concept_desc>Do Not Use This Code, Generate the Correct Terms for Your Paper</concept_desc>
%  <concept_significance>100</concept_significance>
% </concept>
%</ccs2012>
%\end{CCSXML}

%\ccsdesc[500]{Do Not Use This Code~Generate the Correct Terms for Your Paper}
%\ccsdesc[300]{Do Not Use This Code~Generate the Correct Terms for Your Paper}
%\ccsdesc{Do Not Use This Code~Generate the Correct Terms for Your Paper}
%\ccsdesc[100]{Do Not Use This Code~Generate the Correct Terms for Your Paper}

%%
%% Keywords. The author(s) should pick words that accurately describe
%% the work being presented. Separate the keywords with commas.
%\keywords{Do, Not, Us, This, Code, Put, the, Correct, Terms, for,  Your, Paper}

%% A "teaser" image appears between the author and affiliation
%% information and the body of the document, and typically spans the
%% page.
\begin{teaserfigure}
%  \includegraphics[width=\textwidth]{sampleteaser}
%  \caption{Seattle Mariners at Spring Training, 2010.}
%  \Description{Enjoying the baseball game from the third-base
%  seats. Ichiro Suzuki preparing to bat.}
%  \label{fig:teaser}
\end{teaserfigure}

%\received{20 February 2007}
%\received[revised]{12 March 2009}
%\received[accepted]{5 June 2009}

%%
%% This command processes the author and affiliation and title
%% information and builds the first part of the formatted document.
\maketitle

\section{Introduction}
\label{intro}

Relational programming is an attractive technique, based on the idea of constructing programs as relations.
While in general some relational effects can be reproduced with a number of languages for logic programming, such as
Prolog, Mercury\footnote{\url{https://mercurylang.org}}, or Curry\footnote{\url{http://www-ps.informatik.uni-kiel.de/currywiki}}, in
a narrow sense relational programming amounts to writing relational specifications in \miniKanren~\cite{TRS}. \miniKanren\footnote{\url{http://minikanren.org}},
initially designed as a small relational DSL, embedded in Scheme/Racket, was later implemented for a number of general-purpose host languages,
including Scala, Haskell, Standard ML and OCaml.

With relational approach, it becomes possible to give simple and elegant solutions for the problems, otherwise
considered as tricky, tough, tedious, or boring~\cite{unified}. For example, relational interpreters can be used to derive
\emph{quines}~--- programs, which reduce to themselves, as well as \emph{twines} or \emph{thrines} (pairs or triples of
programs, reducing to each other)~\cite{Untagged}; a straightforward relational description of
simply typed lambda calculus~\cite{Lambda} inference rules works both as type inferencer and inhabitation problem solver~\cite{WillThesis};
relational list sorting can be used to generate all permutations~\cite{ocanren}, etc. 

On the other hand, writing relational specifications can sometimes be a tricky and error-prone task. Fortunately, many 
specifications can be written systematically by ``generalizing'' a certain functional program. From the very beginning, 
the conversion from functional to relational form was considered as an element of relational programming thesaurus~\cite{TRS}. However,
the traditional approach~--- \emph{unnesting}~--- was formulated for an untyped case, worked only for specifically written
programs and was never implemented.

We present a generalized form of relational conversion, which can be applied to typed terms in general form. We study the relational conversion 
for a small ML-like language (essentially, a certain subset of OCaml), equipped with Hindley-Milner type system with let-polymorphism~\cite{Types}. 
We start from retelling the syntax, typing rules, and operational semantics, and then extend the source language with a conventional set of 
relational constructs. This set corresponds to existing typed embedding of \miniKanren into OCaml~\cite{ocanren}. We then present typing rules and 
develop operational semantics for this relational extension; to our knowledge, this is the first attempt to specify formal semantics for
\miniKanren. Next, we develop formal rules for relational conversion and prove, that these rules respect both typing and
semantics. Finally, we describe the implementation of a relational converter and demonstrate its application for a number of problems, for some
of which we present a relational solution for the first time.

We would like to express our gratitude to William Byrd and the anonymous reviewers for their constructive remarks, which, we believe, led to the
improvement of the presentation. 

% !TEX TS-program = pdflatex
% !TeX spellcheck = en_US
% !TEX root = main.tex

\section{Related Work}
\label{related}

GUI design and implementation has been a hot topic for decades. Thus, to no surprise there is a lot of frameworks, approaches, papers
and reports on the subject. A fair share of them (if not all) present declarative and automatic solutions. A careful study, however,
discovers that this ``declarativeness'' and ``automation'' is understood differently then in our case.

First of all, we need to mention some software frameworks and tools for design and implementation of GUI and visualization of data, for example, \textsc{React}~\cite{react},
\textsc{Jetpack Compose}~\cite{Jetpack}, \textsc{SwiftUI}~\cite{SwiftUI}, \textsc{Streamlit}~\cite{Streamlit}, \textsc{D3}~\cite{D3} and others.
These frameworks provide a number of layout primitives which end-users can employ in order to render their data or UI. For example, \textsc{Streamlit}
provides a number of builtin layout primitives like ``columns'', ``container'', ``modal dialog'', etc.~\cite{StreamlitLayout} and an endless
variety of third-party external components. These primitives allow end-users to abstract away of concrete controls coordinate calculation and their
relative alignment; they also prescribe a reasonable behavior on enclosing pane resizing. However, which layout primitives to use is decided by
end-users, not the system. If due to any reason the layout needs to be changed these changes have to be implemented manually. In our case
end-users do not specify concrete layouts, only the logical structure of the UI. The guideline takes care of concrete layout, depending on
external constraints such as enclosing pane size, screen resolution or even regional settings (for example, right-to-left writing system). As long
as the logical structure remains unchanged no interference from end-users is required for laying out the UI in different settings. On the
other hand these frameworks can be used as back-ends in our approach since they provide a similar set of layout primitives.

Constraint programming has already been used for deciding the placement of GUI controls. One of the examples are constraint reactive programming
language \textsc{Wallingford}~\cite{Wallingford2016} and the \textsc{Cassowary} system~\cite{Cassowary2001}. \textsc{Wallingford} allows to attach
constraints of various strength to different values in the program. The system reacts to the time changes and updates these values without violation
of the constraints. For example, one could calculate a width of a GUI control as the sine of current time. The \textsc{Cassowary} system and its
descendants allow to calculate the sizes and positions of controls dynamically, for example at the moment of canvas resize.
%The background theory is linear arithmetic.
It supports many different constraints, for example, Z-ordering, arithmetic operators (for example, a control's width can be the half
of another one's height), overlapping views, etc.  These systems are targeted for the tasks of dynamic adaptation the sizes of controls on resize.
Also, they don't provide support of expressing general rules about correct control placing, the constraints are added for values of concrete structure.
In our work we make a strict separation between GUI structure and rules of correct positioning of controls. We have doubts about expressing non-deterministic
layouts in these systems, for example, if vertical or horizontal placement of controls depend on their sizes. On the other hand, the number of various
constraint types is larger than in our approach. For example, they allow to express overlapping views, but based on our experience we initially ruled out
this option, and our current implementation doesn't generate such layouts at all.

Finally, in recent years methods and approaches from AI in ML/data-science sense start to percolate into the area. Some of the works address much more
ambitious objectives, than ours.

First of all, there is a direction of research on UI code generation from images~\cite{Cai2023}. Given a designer-drawn form, a code
generator recognizes UI controls and their relative placements and generates implementation code for one of GUI frameworks. This approach
is completely orthogonal and incompatible with what we suggest. Indeed, it requires an interaction with a designer when implementing every piece of
an interface while in our case a designer is only involved when guidelines are developed; then our system produces guideline-compatible
interfaces \emph{en masse} automatically. In~\cite{Robust} a slightly different task is addressed: given a picture of an interface synthesize its
implementable layout in terms of Android GUI primitives which would be scalable across various devices while avoiding a typical layout errors.
To achieve this goal, after recognizing UI controls and their locations a certain set of relational layout constraints is extracted. This
set of constraints resembles our layout primitives but is aligned with Android's \texttt{ConstraintLayout} widget~\cite{ConstraintLayout} semantics.
Given these constraints and a set of \emph{robustness properties} developed by the authors an implementation code is generated with
the aid of a probabilistic model trained on a large set of existing Android interfaces. This implementation is more stable under screen size
or resolution change, than that provided by image recognition only. Interestingly, one of the motivations for the work, as the authors
specify explicitly, is that ``the same layout needs to be rendered potentially on more than 15 000 Android devices with $\approx$100 different density
independent screen sizes. Requiring the user to provide and maintain input specifications for all of them is infeasible yet highly desirable.''
That is exactly what our system is capable of doing within a few minutes and with 100\% accuracy, so we consider our approach
much more general.

Automatic design of consistent (uniform throughout an application) GUI is addressed in~\cite{LearningGUI}. The approach is based on the idea of completion:
assuming there is already a set of consistent layouts a problem of adding yet another element is addressed; the addition should be consistent with the previous
designs. While this problem is, indeed, related to that we address (indeed, consider an existing set of designs as an implicitly specified guideline),
we can identify a number of potential weaknesses. First, only addition of a component is considered, but not removal; second, the addition/subtraction
of components does not necessarily lead to a ``monotonic'' change of the layout (adding yet another text field may result is a drastically different placement);
finally, the initial set of designs to be consistent with rarely comes out of a thin air; most likely it is a result of following some
existing (and perhaps implicit) guidelines, which should be specified and followed explicitly.

In~\cite{Grid} a much more ambitious problem of synthesizing a layout with no guidelines, based only on aesthetic, ergonomic, etc., metrics is considered.
Genetic algorithm is employed for synthesis, and users' feedback is used as a way to measure the quality of the synthesis; the synthesis itself can
sometimes take hours. While the approach presumably allows to synthesize aesthetically convincing layouts with no designer input, the problem of
layout consistency for sufficiently different structures is left unaddressed.

An interesting problem of layout exploration is considered in~\cite{Scout}. The objective is to help the \emph{designers} to develop convincing and
diverse layouts. A set of constraints is introduced which designers can use in order to specify some requirements for the design. Interestingly, in our terms the
set of constraints is a mixture of layout and structural ones: for example, both order and alignment constraints are present. Given
a number of constraints the system generates a set of layouts consistent with this constraints; by updating the constraints a designer can
continue the exploration. As a constraint solver modifier branch-and-bound algorithm is used; as the number of feasible solutions
turned out to be enormous a set of heuristic metrics was developed to rule aesthetically unfeasible designs out. While this work
shares some similarities with ours, being targeted on designers it can be considered as a mean to \emph{develop guidelines}, not to synthesize
guideline-consistent designs.

\section{The Pattern Matching Synthesis Problem}

We start from a simplified view on pattern matching which does not incorporate some practically important aspects of the construct such as
name bindings in patterns, guards or even semantic actions in branches. In a purified form, however, it  represents the essence of pattern
matching as an ``inspect-and-branch'' procedure. Once we come up with the solution for the essential part of the problem we embellish it with
other features until it reaches a complete form.

First, we introduce a finite set of \emph{constructors} $\mathcal C$, equipped with arities, a set of values $\mathcal{V}$
and a set of patterns $\mathcal{P}$:
 
\[
 \begin{array}{rcll}
    \mathcal{C} & = & \{ C_1^{k_1}, \dots, C_n^{k_n} \}\\
    \mathcal{V} & = & \mathcal{C}\,\mathcal{V}^*\\  
    \mathcal{P} & = & \_ \mid \mathcal{C}\,\mathcal{P}^*
 \end{array}
\]

We define a matching of a value $v$ (\emph{scrutinee}) against an ordered non-empty sequence of patterns $p_1,\dots,p_k$ by means of the following
relation

\[
\setarrow{\xrightarrow}
\trans{\inbr{v;\,p_1,\dots,p_k}}{}{i},\,1\le i\le k+1
\]

which gives us the index of the leftmost matched pattern or $k+1$ if no such pattern exists. We use an auxiliary relation $\inbr{;}\subseteq\mathcal{V}\times\mathcal{P}$
to specify the notion of a value matched by an individual pattern (see Fig.~\ref{fig:match1pat}). The rule \ruleno{Wildcard} says that
a wildcard pattern ``\_'' matches any value, and \ruleno{Constructor} specifies that a constructor pattern matches exactly those values which
have the same constructor at the top level and all subvalues matched by corresponding subpatterns. The definition of ``$\xrightarrow{}{\!\!}$'' is
shown on Fig.~\ref{fig:matchpatts}. An auxiliary relation
 ``$\xrightarrow{}{}_{\!\!*}$'' 
is introduced to specify the left-to-right matching strategy, and we
use current index as an environment. An important rule, $\ruleno{MatchOtherwise}$ specifies that if we exhausted all the patterns with no matching we stop with
the current index (which in this case is equal to the number of patterns plus one).

\begin{figure}
   \renewcommand*{\arraystretch}{2}
   \[
   \begin{array}{cr}
     \inbr{v;\,\_} & \ruleno{Wildcard} \\
     \trule{\forall i\;\inbr{v_i;\,p_i}}{\inbr{C^k\,v_1\dots v_k;\,C^k\,p_1\dots p_k}},\,k\ge 0 & \ruleno{Constructor}
   \end{array}
   \]
   \caption{Matching against a single pattern}
   \label{fig:match1pat}
\end{figure}

\begin{figure}
   \renewcommand*{\arraystretch}{3}
   \setarrow{\xrightarrow}
   \setsubarrow{_*}
   \[
   \begin{array}{cr}
     \trule{\inbr{v;\,p_1}}
           {\withenv{i}{\trans{\inbr{v;\,p_1,\dots,p_k}}{}{i}}} & \ruleno{MatchHead}\\
     \trule{\neg\inbr{v;\,p_1}\qquad\withenv{i+1}{\trans{\inbr{v;\,p_2,\dots,p_k}}{}{j}}}
           {\withenv{i}{\trans{\inbr{v;\,p_1,\dots,p_k}}{}{j}}} & \ruleno{MatchTail}\\
     \withenv{i}{\trans{\inbr{v;\,\varepsilon}}{}{i}} & \ruleno{MatchOtherwise}\\
     \trule{\withenv{1}{\trans{\inbr{v;\,p_1,\dots,p_k}}{}{i}}}
           {\setsubarrow{}\trans{\inbr{v;\,p_1,\dots,p_k}}{}{i}} & \ruleno{Match}
   \end{array}
   \]
   \caption{Matching against an ordered sequence of patterns}
   \label{fig:matchpatts}
\end{figure}

The relation ``$\xrightarrow{}{}\!\!$'' gives us a \emph{declarative} semantics of pattern matching. Since we are interested in
synthesizing implementations, we need a \emph{programmatical} view on the same problem. Thus, we introduce a language $\mathcal S$
(the ``switch'' language) of test-and branch constructs:

\[
\begin{array}{rcl}
  \mathcal M & = & \bullet \\
  &   & \mathcal M\,[\mathbb{N}] \\
  \ir & = & \primi{return}\,\mathbb{N} \\
  &   & \primi{switch}\;\mathcal{M}\;\primi{with}\; [\mathcal{C}\; \primi{\rightarrow}\; \ir]^*\;\primi{otherwise}\;\ir
\end{array}
\]
 
Here $\mathcal{M}$ stands for a \emph{matching expression}, which is either a reference to a scrutinee ``$\bullet$'' or
an indexed subexpression of scrutinee. Programs in the switch language can discriminate based on the
structure of matching expressions, testing their top-level constructors and eventually returning natural numbers as results.
The switch language is similar to the intermediate representations for pattern matching code used in 
previous works on pattern matching implementation~\cite{maranget2001,maranget2008}.

The semantics of the switch language is given by mean of relations ``$\xrightarrow{}{}_{\!\!\!\mathcal M}$'' and ``$\xrightarrow{}{}_{\!\!\mathcal S}$''
(see Fig.~\ref{fig:matchexpr} and \ref{fig:test-and-branch}). The first one describes the semantics of matching expression, while
the second describes the semantics of the switch language itself. In both cases the scrutinee $v$ is used as an environment ($v\vdash$).


\begin{figure}
  \renewcommand*{\arraystretch}{2}
  \setarrow{\xrightarrow}
  \setsubarrow{_{\mathcal M}}
  \[
  \begin{array}{cr}
    \withenv{v}{\trans{\bullet}{}{v}} & \ruleno{Scrutinee} \\
    \trule{\withenv{v}{\trans{m}{}{C^k v_1\dots v_k}}}{\withenv{v}{\trans{m[i]}{}{v_i}}} & \ruleno{SubMatch} 
  \end{array}
  \]
  \caption{Semantics of matching expression}
  \label{fig:matchexpr}
\end{figure}

\begin{figure}
  \renewcommand*{\arraystretch}{3}
  \setarrow{\xrightarrow}
  \setsubarrow{_{\mathcal S}}
  \[
  \begin{array}{cr}
    \withenv{v}{\trans{\primi{return}\;i}{}{i}} & \ruleno{Return}\\
    \trule{{\setsubarrow{_{\mathcal M}}\withenv{v}{\trans{m}{}{C^k\ v_1 \dots v_k}}}\qquad \withenv{v}{\trans{s}{}{i}}}
          {\withenv{v}{\trans{\primi{switch}\;m\;\primi{with}\;[C^k\to s]s^*\;\primi{otherwise}\;s^\prime}{}{i}}} & \ruleno{SwitchMatched}\\
    \trule{{\setsubarrow{_{\mathcal M}}\withenv{v}{\trans{m}{}{D^n\  v_1 \dots v_n}}}\qquad C^k\ne D^n\qquad \withenv{v}{\trans{\primi{switch}\;m\;\primi{with}\;s^*\;\primi{otherwise}\;s^\prime}{}{i}}}
          {\withenv{v}{\trans{\primi{switch}\;m\;\primi{with}\;[C^k\to s]s^*\;\primi{otherwise}\;s^\prime}{}{i}}} & \ruleno{SwitchNotMatched}\\
    \trule{\withenv{v}{\trans{s}{}{i}}}{\withenv{v}{\trans{\primi{switch}\;m\;\primi{with}\;\varepsilon\;\primi{otherwise}\;s}{}{i}}} & \ruleno{SwitchOtherwise}
  \end{array}
  \]
  \caption{Semantics of switch programs}
  \label{fig:test-and-branch}
\end{figure}

The following observations can be easily proven by structural induction.

\begin{Observation}
  For arbitrary pattern the set of matching values is non-empty:

  \[
  \forall p\in\mathcal P : \{v\in\mathcal V\mid \inbr{v;\,p}\}\ne\emptyset
  \]
\end{Observation}

\begin{Observation}
  Relations ``$\xrightarrow{}{}\!\!$'' and ``$\xrightarrow{}{}_{\!\!\mathcal S}$'' are functional and deterministic respectively:

  \[
  \setarrow{\xrightarrow}
  \begin{array}{rcl}
    \forall p_1,\dots,p_k\in\mathcal P,\,\forall v\in \mathcal V,\,\forall \pi\in\mathcal S & : & |\{i\in\mathbb N\mid \trans{\inbr{v;\,p_1,\dots,p_k}}{}{i}\}|=1 \\
                                                                 &  & {\setsubarrow{_{\mathcal S}}|\{i\in\mathbb N\mid \withenv{v}{\trans{\pi}{}{i}}\}|\le 1}
  \end{array}
  \]
\end{Observation}

With these definitions, we can formulate the \emph{pattern matching synthesis problem} as follows: for a given ordered sequence of patterns $p_1,\dots,p_k$ find
a switch program $\pi$, such that

\[
\setarrow{\xrightarrow}
\forall v\in \mathcal V,\; \forall 1\le i\le n+1 : \trans{\inbr{v;\,p_1,\dots,p_n}}{}{i}\Longleftrightarrow{\setsubarrow{_{\mathcal S}}\withenv{v}{\trans{\pi}{}{i}}}\eqno{(\star)}
\]

In other words, program $\pi$ delivers a correct and complete implementation for pattern matching.

\section{Pattern Matching Synthesis, Relationally}
\label{sec:relationally}

In this section we describe a relational formulation for the pattern matching synthesis problem. Practically,
this amounts to constructing a goal with a free variable corresponding to the switch program to synthesize
for (arbitrary) list of patterns. In order to come up with a tractable goal certain steps have to be performed.
We first describe the general idea, and then consider these steps is details.

Our idea of using relational programming for pattern matching synthesis is based on the following observations:

\begin{itemize}
\item For the switch language we can implement a relational interpreter $eval^o_\ir$ with the following property: for
  arbitrary $v\in\mathcal V$, $\pi\in\ir$ and $i\in\mathbb N$
 
  \[
  \setarrow{\xrightarrow}
  \setsubarrow{_\ir}
   eval^o_\ir\, v\, \pi\, i \Longleftrightarrow \withenv{v}{\trans{\pi}{}{i}}
  \]

  In other words, $eval^o_\ir$ interprets a program $\pi$ for a scrutinee $v$ and returns exactly the same branch (if any)
  which is prescribed by the semantics of the switch language. 
  
\item On the other hand, we can directly encode the declarative semantics of pattern matching as a relational
  program $match^o$ such that for arbitrary $v\in\mathcal V$, $p_i\in\mathcal P$ and $i\in\mathbb N$

  \[
  \setarrow{\xrightarrow}
  match^o\,v\,p_1,\dots,p_k\,i \Longleftrightarrow \trans{\inbr{v;\,p_1,\dots,p_k}}{}{i}
  \]

  Again, $match^o$ succeeds with $1\le i\le k$ iff $p_i$ is the leftmost pattern, matching $v$; otherwise it
  succeeds with $i=k+1$.
\end{itemize}

We address the construction of relational interpreters for both semantics in Section~\ref{sec:relints}.

Being relational, both $eval^o_\ir$ and $match^o$ do not just succeed or fail for ground arguments, but also can be \emph{queried} for
arguments with free logical variables, thus performing a search for all substitutions for these variables which make the
relation hold. This observation leads us to the idea of utilizing the definition of the pattern matching
synthesis problem, replacing ``$\xrightarrow{}{}\!\!$'' with $match^o$, ``$\xrightarrow{}{}_{\!\!\!\mathcal S}$`` with $eval^o$,
and $\pi$ with a free logical variable $\circled{?}$, which gives us the goal

\[
\forall v\in \mathcal V,\; \forall 1\le i\le n+1 : match^o\,v\,p_1,\dots,p_n\,i\Longleftrightarrow eval^o\,v\,\circled{?}\,i
\]

\noindent This goal, however, is problematic from relational point of view due to a number of reasons.

First, \textsc{miniKanren} provides rather a limited support for universal quantification. Apart from being inefficient from
a performance standpoint, existing implementations either do not coexist with disequality constraints~\cite{eigen}
or do not support quantified goals with infinite number of answers~\cite{moiseenko}. As we will see below, both restrictions are
violated in our case. Second, there is no direct support for the equivalence of goals (``$\Leftrightarrow$''). Thus,
reducing the original synthesis problem to a viable relational goal involves some ``massaging''.

We eliminate the universal quantification over the infinite set of scrutinees, replacing it by a \emph{finite}
conjunction over a \emph{complete set of samples}. For a sequence of patterns $p_1,\dots,p_k$ a
complete set of samples is a finite set of values $\mathcal{E}(p_1,\dots,p_k)\subseteq\mathcal{V}$ with the following
property:

\[
\setarrow{\xrightarrow}
\begin{array}{rcl}
  \forall\pi\in\mathcal S & : & [\forall v\in\mathcal{E}(p_1,\dots,p_k),\,\forall i\in\mathbb{N} : \trans{\inbr{v;\,p_1,\dots,p_k}}{}{i} \Longleftrightarrow {\setsubarrow{_{\mathcal S}}\withenv{v}{\trans{\pi}{}{i}}}] \Longrightarrow \\
                          &   & [\forall v\in\mathcal V,\,\forall i\in\mathbb{N} : \trans{\inbr{v;\,p_1,\dots,p_k}}{}{i} \Longleftrightarrow  {\setsubarrow{_{\mathcal S}}\withenv{v}{\trans{\pi}{}{i}}}]
\end{array}
\]

In other words, if a program implements a correct and complete pattern matching for all values in a complete set of samples, then this
program implements a correct and complete pattern matching for all values. The idea of using a complete set of samples originates from the following observation: each pattern
describes a (potentially infinite) set of values, and pattern matching splits the set of all values into equivalence classes, each corresponding to a certain matching pattern.
Moreover, the values of different classes can be distinguished only by looking down to a \emph{finite} depth (as different patterns can be distinguished in this way).
The generation of complete sample set will be addressed below (see Section~\ref{sec:samples}).

\setarrow{\xrightarrow}

To eliminate the universal quantification over the set of answers we rely on the functionality of declarative pattern matching semantics. Indeed, given a fixed sequence $p_1,\dots,p_k$
of patterns, for every value $v$ there is exactly one answer value $i$, such that $\trans{\inbr{v;\,p_1,\dots,p_k}}{}{i}$. We can reformulate this property as

\[
\exists i:\, \trans{\inbr{v;\,p_1,\dots,p_k}}{}{i} \Longrightarrow  
\big(\forall j : \trans{\inbr{v;\,p_1,\dots,p_k}}{}{j} \Longrightarrow  j=i\large\big)
\]

Thus, we can replace universal quantification over the sets of answers by existential one, for which we have an efficient relational counterpart~--- the ``\lstinline|fresh|''
construct.

Following the same argument, we may replace the equivalence with conjunction: indeed, if

\[
\setarrow{\xrightarrow}
\trans{\inbr{v;\,p_1,\dots,p_k}}{}{i}
\]

for some $i$, then (by functionality), for any other $j\ne i$

\[
\setarrow{\xrightarrow}
\neg\;(\trans{\inbr{v;\,p_1,\dots,p_k}}{}{j})
\]

A correct pattern matching implementation $\pi$ should satisfy the condition

\[
\setarrow{\xrightarrow}
\setsubarrow{_{\mathcal S}}
\withenv{v}{\trans{\pi}{}{i}}
\]

But, by the determinism of the switch language semantics, it immediately follows, that for arbitrary $j\ne i$

\[
\setarrow{\xrightarrow}
\setsubarrow{_{\mathcal S}}
\neg\;(\withenv{v}{\trans{\pi}{}{j}})
\]

\begin{comment}
Alternatively\footnote{\color{red} Reviewer N1 said that passage about bool argument is unclear and may be omitted (or described with more details)}, we could switch to a more explicit relational representation of both semantics, adding an extra boolean argument to
both $eval^o_{\mathcal S}$ and $match^o$ and using the same fresh variable $b$ in the query of interest:

\[
match^o\,v\,p_1,\dots,p_k\,i\,b \wedge eval^o_{\mathcal S}\,v\,\pi\,i\,b
\]

\end{comment} 

Thus, the goal we eventually came up with is

\[
\bigwedge_{v\in\mathcal{E}\,(p_1,\dots,p_k)}\mbox{\lstinline|fresh ($i$)|}\; \{match^o\,v\,p_1,\dots,p_k\,i \wedge eval^o_{\mathcal S}\,v\,\circled{?}\,i\}
\eqno{(\star\star)}
\]

From relational point of view this is a pretty conventional goal which can be solved by virtually any decent \textsc{miniKanren} implementation in
which the relations $eval^o_{\mathcal S}$ and $match^o$ can be encoded.

Finally, we can make the following important observation. Obviously, any pattern matching synthesis problem has at least one trivial solution.
This, due to the completeness of relational interleaving search~\cite{search,certifiedSemantics}, means that the goal above \emph{can not diverge} with
no results. Actually it is rather easy to see that any pattern matching synthesis problem has \emph{infinitely many} solutions: indeed, having just
one it is always possible to ``pump'' it with superfluous ``$\primi{otherwise}$'' clauses; thus, the goal above is \emph{refutationally
complete}~\cite{WillThesis,DivergenceTest}. These observations justify the totality of our synthesis approach. In Section~\ref{sec:optimization} we show
how we can make it provide optimal solution.

\subsection{Constructing Relational Interpreters}
\label{sec:relints}

In this section we address the implementation of relations $eval^o_{\mathcal S}$ and $match^o$. In principle, it amounts to accurate encoding of
relations 
``$\xRightarrow{}{}\!\!$'' and ``$\xRightarrow{}{}_{\!\!\mathcal S}$'' 
in \textsc{miniKanren} (in our case, \textsc{OCanren}). We, however,
make use of a relational conversion~\cite{conversion} tool, called \textsc{noCanren}, which automatically converts a subset of \textsc{OCaml} into
\textsc{OCanren}. Thus, both interpreters are in fact implemented in \textsc{OCaml} and repeat corresponding inference rules almost
literally in a familiar functional style. For example, functional implementation of a declarative semantics looks like follows:

\begin{lstlisting}
   let rec $\inbr{v;\,p}$ =
     match ($v$, $p$) with
     | (_, Wildcard) -> true
     | ($C^k\;v^*$, $C^k\;p^*$) -> list_all $\inbr{;}$ (list_combine $v^*$ $p^*$)
     | _             -> false

  let $match^o$ $v$ $p^*$ =
    let rec inner $i$ $p^*$ =
      match $p^*$ with
      | []      -> $i$
      | $p$ :: $p^*$ -> if $\inbr{v;\,p}$ then $i$ else inner S($i$) $p^*$
    in inner O $p^*$
\end{lstlisting}

We mixed here the concrete syntax of \textsc{OCaml} and mathematical notation, used in the definition of the relation in question; the actual
implementation only a few lines of code longer. Note, we use here natural numbers in Peano form and custom list processing functions in order
to apply relational conversion later.

Using relational conversion saves a lot of efforts as \textsc{OCanren} specifications tend to be much more verbose; in addition
relational conversion implements some ``best practices'' in relational programming (for example, moves unifications forward in
conjunctions and puts recursive calls last). Finally, it has to be taken into account that relational conversion of pattern matching introduces
disequality constraints.

\subsection{Dealing with a Complete Set of Samples}
\label{sec:samples}

As we mentioned above, a complete set of samples plays an important role in our approach: it allows us to eliminate universal quantification over the
set of all values. As we replace the universal quantifier with a finite conjunction with one conjunct per sample value reducing the size of
the set is an important task. At the present time, however, we build an excessively large (worst case exponential of depth) number of samples. We discuss
the issues with this choice in Section~\ref{sec:eval} and consider developing a more advanced approach as the main direction for
improvement.

Our construction of a complete set of samples is based upon the following simple observations. We simultaneously define the \emph{depth} measure
for patterns and sequences of patterns as follows:

\[
\begin{array}{rcl}
   d\,(p_1,\dots,p_k)     & = & max\, \{ d\,(p_i)\}\\
   d\,(\_)                 & = & 0 \\
   d\,(C^k\,p_1,\dots p_k) & = & 1 + d\,(p_1,\dots,p_k)
\end{array}
\]

\noindent As a sequence of patterns is the single input in our synthesis approach we will call its depth \emph{synthesis depth}.

Similarly, we define the depth of matching expressions

\[
\begin{array}{rcl}
  d_{\mathcal M}\,(\bullet) & = & 1 \\
  d_{\mathcal M}\,(m\,[i]) & = & 1 + d_{\mathcal M}\,(m)\\
\end{array}
\]

and switch programs:

\[
\begin{array}{rcl}
  d_{\mathcal S}\,(\primi{return}\;i)&=&0\\
  d_{\mathcal S}\,(\primi{switch}\;m\;\primi{of}\;C_1\to s_1,\dots,C_k\to s_k\;\primi{otherwise}\;s)&=&max\,\{d_{\mathcal M}\,(m),\,d_{\mathcal S}\,(s_i),\,d_{\mathcal S}\,(s)\}
\end{array}
\]

Informally, the depth of a switch program tells us how deep the program can look into a value. 

From the definition of $\inbr{;}$ it immediately follows that a pattern $p$ can only discriminate values up to its depth $d\,(p)$: changing a value at the depth greater
or equal than $d\,(p)$ cannot affect the fact of matching/non matching. This means that we need only consider switch programs of depth no greater than the synthesis depth.
But for these programs the set of all values with height no greater than the synthesis depth forms a complete set of samples. Indeed, if the height of a value less or
equal to the synthesis depth, then this value is a member of complete set of samples and by definition the behavior of the synthesized program on this value is
correct. Otherwise there exists some value $s$ from the complete set of samples, such that given value can be obtained as an ``extension'' of $s$ at the
depth greater than the synthesis depth. Since neither declarative semantics nor switch programs can discriminate values at this depth, they behavior for a given value
will coincide with the correct-by-definition behavior for  $s$.

The implementation of complete set generation, again, is done using relational conversion. The enumeration of all terms up to a certain depth
can be acquired from a function which calculates the depth of a term: indeed, converting it into a relation and then running with \emph{fixed} depth
and \emph{free} term arguments delivers what we need. Thus, we add an extra conjunct which performs the enumeration of all values to the
relational goal $(\star\star)$, arriving at

\[
depth^o\,v\,n\wedge\mbox{\lstinline|fresh ($i$)|}\; \{match^o\,v\,p_1,\dots,p_k\,i \wedge eval^o_{\mathcal S}\,v\,\circled{?}\,i\}
\eqno{(\star\star\star)}
\]

Here $n$ is a precomputed synthesis depth in Peano form.

\begin{comment}
\begin{figure}[ht]
\begin{subfigure}[t]{0.2\linewidth}
  \[
  \{A^1,\,B^0,\,C^1,\,D^0\}
  \]
\vskip6mm
\caption{Constructors}
\label{fig:constructors}
\end{subfigure}
\hspace{0.5cm}
\begin{subfigure}[t]{0.26\linewidth}
  \[
  \begin{array}{c}
    C^1\,(A^1\,(B^0))\\
    C^1\,(\_)\\
    \_
  \end{array}
\]
\caption{Patterns}
\label{fig:patterns}
\end{subfigure}
\hspace{0.5cm}
\begin{subfigure}[t]{0.33\linewidth}
  \[
  \begin{array}{lcl}
     B             & \mapsto & 2 \\
     D             & \mapsto & 2 \\
     A\, (B)       & \mapsto & 2 \\
     A\, (D)       & \mapsto & 2 \\
     C\, (B)       & \mapsto & 1 \\
     C\, (D)       & \mapsto & 1 \\
     A\, (A\, (B)) & \mapsto & 2 \\
     A\, (A\, (D)) & \mapsto & 2 \\
     A\, (C\, (B)) & \mapsto & 2 \\
     A\, (C\, (D)) & \mapsto & 2 \\
     C\, (A\, (B)) & \mapsto & 0 \\
     C\, (A\, (D)) & \mapsto & 1 \\
     C\, (C\, (B)) & \mapsto & 1 \\
     C\, (C\, (D)) & \mapsto & 1 
  \end{array}
  \]
\caption{Generated samples}
\label{fig:samples}
\end{subfigure}
\caption{Complete set of samples example} 
\label{fig:complete-set-example}
\end{figure}
\end{comment}




\newcommand{\head}[2]{\multicolumn{1}{>{\centering\arraybackslash}m{#1}}{\textbf{#2}}}
\newcommand{\headll}[2]{\multicolumn{1}{>{\centering\arraybackslash}m{#1}||}{\textbf{#2}}}
\newcommand{\headl}[2]{\multicolumn{1}{>{\centering\arraybackslash}m{#1}|}{\textbf{#2}}}
\begin{figure}[t]
  %\begin{tabular}{|c|m{3cm}|cc||cccc}
  \begin{tabular}{|m{3cm}|cc||cccc|}
%    \headl{.5cm}{}    &
    \headl{3cm}{Patterns} & 
    \head{1.5cm}{Size constraint} & 
    \headll{1.7cm}{Answers requested} & 
    \head{2cm}{Number of samples} & 
    \head{1.5cm}{1st answer time (ms)} & 
    \head{1.5cm}{Answers found} & 
    \head{1.5cm}{Total search time (ms)}\\
    \hline
    \hline
    %1&
    \begin{lstlisting}[basicstyle=\scriptsize]
A
B
C
    \end{lstlisting} &100&all&3&1&1&1\\
        \hline
%        2&
    \begin{lstlisting}[basicstyle=\scriptsize]
true
false
    \end{lstlisting} &100&all&2&<1&1&<1\\
        \hline
%        3&
            \begin{lstlisting}[basicstyle=\scriptsize]
(true, _)
(_, true)
(false, false)
    \end{lstlisting} &100&all&4&6&2&10\\
        \hline
%        4&
     \begin{lstlisting}[basicstyle=\scriptsize]
(_, false, true)
(false, true, _)
(_, _, false)
(_, _, true)
    \end{lstlisting} &100&all&8&323&3&729\\
        \hline
%        5&
     \begin{lstlisting}[basicstyle=\scriptsize]
([], _)
(_, [])
(_ :: _, _ :: _)
    \end{lstlisting} &100&10&4&5&1&6\\
        \hline
%        6&
     \begin{lstlisting}[basicstyle=\scriptsize]
(Succ _, Succ _)
(Zero, _)
(_, Zero)
    \end{lstlisting} &1&all&4&53&2&108
    \\
        \hline
%        \mbox{7}&
        \multirow{3}{*}{
          \parbox{3cm}{
            \vskip2mm
\lstinline[basicstyle=\scriptsize]|(Nil, _)|\\[-1mm]
\lstinline[basicstyle=\scriptsize]|(_, Nil)|\\[-1mm]
\lstinline[basicstyle=\scriptsize]|(Nil2, _)|\\[-1mm]
\lstinline[basicstyle=\scriptsize]|(_, Nil2)|\\[-1mm]
\lstinline[basicstyle=\scriptsize]|(_ :: _, _ :: _)|}}
         & 1&10&9&643&2&3776\\[3mm]
        \cline{2-7}
      &10&10&9&95&2&540\\[3mm]
        \cline{2-7}
     &100&10&9&45&2&239                    \\ \hline
  \end{tabular}
  \caption{The results of synthesis evaluation}
  \label{fig:eval}

\end{figure}

%\FloatBarrier

\section{Implementation and Optimizations}
\label{sec:optimization}

In this section we address two aspects of our solution: a number of optimizations which make the search more efficient, and
the way it ends up with the optimal solution.

The relational goal in its final form, presented in the previous section, does not demonstrate good performance. Thus, we apply a number
of techniques, some of which require extending the implementation of the search. Namely, we apply the following optimizations:

\begin{itemize}
\item We make use of type information to restrict the subset of constructors which may appear in a certain branch of
  program being synthesized.
\item We implement \emph{structural constraints} which allow us to restrict the shape of terms during the search, and
  utilize them to implement pruning.  
\end{itemize}

In our formalization we do not make any use of types since as a rule type information does not affect matching. In addition,
utilizing the properties of a concrete type system would make our approach too coupled with this particular type system, hampering
its reusability for other languages. Nevertheless we may use a certain abstraction of type system which would deliver only
that part of information which is essential for our approach to function. Currently, we calculate the type of any matching expression in
the program being synthesized and from this type extract the subset of constructors which can appear when branching on this expression
is performed. The number of these constructors restricts the number of branches which a corresponding $\primi{switch}$ expression can have.
In our implementation we assume the constructor set ordered, and we consider only ordered branches, which restricts branching even more.

Our approach to finding an optimal solution in fact implements branch-and-bound strategy. The birds-eye view of our plan is as follows:

\FloatBarrier

\begin{itemize}
\item We construct a trivial solution, which gives us the first estimation.
\item During the search we prune all partial solutions whose size exceeds current estimation. We can do this due to
  the top-down nature of partial solution construction.
\item When we come up with a better solution we remember it and update current a estimate.
\end{itemize}

\noindent This strategy inevitably delivers us the optimal solution since there are only finitely many switch programs, shorter than trivial solution.

In order to implement this strategy we extended \textsc{OCanren} with a new primitive called \emph{structural constraint}, which may
fail on some terms depending on some criterion specified by an end-user. Structural constraints can be seen as a generalization of
some known constraints like \lstinline|absent$^o$| or \lstinline|symbol$^o$|~\cite{Untagged} in existing \textsc{miniKanren} implementations, 
so they can be widely used in solving other problems as well. Note, we could implement other constraints we considered (on the
depth of switch programs, on the type of scrutinee) as structural. However, our experience has shown that this leads to
a less efficient implementation. Since these constraints are inherent to the problem, we kept them ``hard coded''.

\section{Evaluation}

\label{sec:evaluation}

In this section, we present an evaluation of 
implemented constructive negation on a series of examples.

\subsection{If-then-else}

Using relational if-then-else operator, 
presented in section~\ref{sec:ifte},
we have implemented several 
higher-order relations over lists, namely 
\lstinline{find} (Listing~\ref{lst:eval-find}), 
\lstinline{remove}\footnote{Note, this implementation 
differs from the one in Section~\ref{sec:intro}, but 
it is easy to see that these two are semantically equivalent.} (Listing~\ref{lst:eval-remove}) 
and \lstinline{filter} (Listing~\ref{lst:eval-filter}).
These relations are almost identical (syntactically) to their
functional implementations.
We have tested that these relations can be run
in various directions and produce the expected results.
For example, the goal \lstinline{filter p q q}
with the predicate \lstinline{p} equal to

\begin{lstlisting}
  fun l -> fresh (x) (l === [x])
\end{lstlisting}

stating that the given list should be a singleton list,
starts to generate all singleton lists.
Vice versa, the goal \lstinline{filter p q []} 
with that same \lstinline{p} generates 
all lists, constrained to be not a singleton list.

Listings~\ref{lst:eval-p}-\ref{lst:eval-filter-queries} give 
more concrete examples of queries to these relations.
In the listing the syntax \lstinline{run n q g}
means running a goal \lstinline{g} with 
the free variable \lstinline{q}
taking the first \lstinline{n} answers (``\lstinline{*}'' denotes all answers).
After the sign $\leadsto$ the result of the query is given.
The result \lstinline{fail} means that the query has failed.
The result \lstinline[mathescape]|succ {{a$_1$}; ... {a$_n$}} |
means that the query has succeeded delivering $n$ answers.
Each answer represents a set of constraint on free variables.
Constraints are of two forms: equality constraints, e.g. \lstinline{q = (1, _.$_0$)}, 
or disequality constraints, e.g. \lstinline{q $\neq$ (1, _.$_0$)}.
The terms of the form \lstinline{_.$_i$} in the answer
denote some universally quantified variables.

\begin{minipage}[thb]{.3\textwidth}
\begin{lstlisting}[
  caption={A definition of \code{find} relation},
  label={lst:eval-find}
]
let find p e xs =
  fresh (x xs' ys') (
    xs === x::xs' /\
    ifte (p x)
      (e === x)
      (find p e xs')
  )
\end{lstlisting}
\end{minipage}\hfill
\begin{minipage}[thb]{.3\textwidth}
\begin{lstlisting}[
  caption={A definition of \code{remove} relation},
  label={lst:eval-remove}
]
let remove p xs ys =
  (xs === [] /\ ys === [])
  \/
  fresh (x xs' ys') (
    xs === x::xs' /\
    ifte (p x)
      (ys === xs')
      (ys === x::ys' /\ 
       remove p xs' ys')
  )
\end{lstlisting}
\end{minipage}\hfill
\begin{minipage}[thb]{.3\textwidth}
\begin{lstlisting}[
  caption={A definition of \code{filter} relation},
  label={lst:eval-filter}
]
let filter p xs ys =
  (xs === [] /\ ys === [])
  \/
  fresh (x xs' ys') (
    xs === x::xs' /\
    (ifte (p x)
      (ys === x :: ys')
      (ys === ys')) /\
    filter p xs' ys'
  )
\end{lstlisting}
\end{minipage}

% \vspace{3cm}

\begin{minipage}[thb]{0.4\textwidth}
\begin{lstlisting}[
  caption={Definition of the predicate \lstinline{p}},
  label={lst:eval-p}
]
let p l = fresh (x) (l === [x])
\end{lstlisting}
\begin{lstlisting}[
  caption={Example of queries to \lstinline{find}},
  label={lst:eval-find-queries}
]
run 3 q (fresh (e) find p e q) 
$\leadsto$ succ {
     { q = [_.$_0$] :: _.$_1$ }
     { q = _.$_0$ :: [_.$_1$] :: _.$_2$; 
         _.$_0$ $\neq$ [_.$_3$] }
     { q = _.$_0$ :: _.$_1$ :: [_.$_2$] :: _.$_3$; 
         _.$_0$ $\neq$ [_.$_4$]; _.$_1$ $\neq$ [_.$_5$] }
   }
\end{lstlisting}
\end{minipage}\hfill
\begin{minipage}[thb]{0.4\textwidth}
\begin{lstlisting}[
  caption={Example of queries to \lstinline{remove}},
  label={lst:eval-remove-queries}
]
run * q (fresh (e) remove p q [[ ]]) 
$\leadsto$ succ {
     { q = [[_.$_0$]; [ ]] }
     { q = [[ ]] }
     { q = [[ ]; [_.$_0$]] }
   }

run 3 q (fresh (e) remove p q q) 
$\leadsto$ succ {
     { q = [] }
     { q = [_.$_0$], _.$_0$ $\neq$ [_.$_1$] }
     { q = [_.$_0$; _.$_1$]; 
         _.$_0$ $\neq$ [_.$_2$]; _.$_1$ $\neq$ [_.$_3$] }
   }
\end{lstlisting}
\end{minipage}

\begin{minipage}[thb]{0.4\textwidth}
\begin{lstlisting}[
  caption={Example of queries to \lstinline{filter}},
  label={lst:eval-filter-queries}
]
run 3 q (filter p q q) 
$\leadsto$ succ {
     { q = [ ] }
     { q = [_.$_0$] }
     { q = [_.$_0$; _.$_1$] }
   }

run 3 q (filter p q [1]) 
$\leadsto$ succ {
     { q = [[1]] }
     { q = [_.$_0$; [1]]; _.$_0$ $\neq$ [_.$_1$] }
     { q = [[1]; _.$_0$]; _.$_0$ $\neq$ [_.$_1$] }
   }

run 3 q (filter p q [ ]) 
$\leadsto$ succ {
     { q = [] }
     { q = [_.$_0$]; _.$_0$ $\neq$ [_.$_1$] }
     { q = [_.$_0$; _.$_1$]; 
            _.$_0$ $\neq$ [_.$_2$]; _.$_1$ $\neq$ [_.$_3$] }
   }
\end{lstlisting}
\end{minipage}

\subsection{Universal quantification}

In the Section~\ref{sec:impl-univ} we presented 
the \lstinline{forall} goal constructor 
which is implemented through the double negation.
We have observed, that although \lstinline{forall g}
does not terminate when the goal \lstinline{g x} 
has an infinite number of answers 
(assuming \lstinline{x} is a fresh variable),
it does terminate in the case when \lstinline{g x} has 
a finite number of answers.
The behavior of \lstinline{forall} in this case is sound
even in the presence of disequality constraints or nested quantifiers. 

The Table~\ref{tab:univ} gives some concrete examples.
The left column contains the tested goals\footnote{
We typeset the goals in terms of first-order logic syntax 
instead of \textsc{OCanren} syntax for brevity and clarity.} 
and the right column gives the obtained results.
For the results we use the same notation 
as in the previous section.

\begin{table}[th]
  \centering
  \def\arraystretch{1.5}
  \begin{tabularx}{\textwidth}{|X|X|}
    \hline

    $\forall x\ldotp x = q$ & 
      \texttt{fail} \\
    \hline

    $\forall x\ldotp \exists y\ldotp x = y$ & 
      \texttt{succ \{[q = \_.$_0$]\}} \\
    \hline

    $\forall x\ldotp \exists y\ldotp x = y \wedge y = q$ &
      \texttt{fail} \\
    \hline

    $\forall x\ldotp q = (1, x)$ & 
      \texttt{fail} \\
    \hline

    $\forall x\ldotp \exists y\ldotp y = (1, x)$ & 
      \texttt{succ \{[q = \_.$_0$]\}} \\
    \hline

    $\forall x\ldotp \exists y\ldotp x = (1, y)$ &
      \texttt{fail} \\
    \hline

    $\forall x\ldotp x \neq q$ & \texttt{fail} \\
    \hline

    $\forall x\ldotp \exists y\ldotp x \neq y$ & 
      \texttt{succ \{[q = \_.$_0$]\}} \\
    \hline

    $\forall x\ldotp \exists y\ldotp x \neq y \wedge y = q$ & 
      \texttt{fail} \\
    \hline

    $\forall x\ldotp q \neq (1, x)$ & 
      \texttt{succ \{[q $\neq$ (1, \_.$_0$)]\}} \\
    \hline

    $(\exists x\ldotp q = (1, x)) \wedge (\forall x\ldotp q \neq (1, x))$ & 
      \texttt{fail} \\
    \hline

    $\forall x\ldotp (x, x) \neq (0, 1)$ & 
      \texttt{succ \{[q = \_.$_0$]\}} \\
    \hline

    $\forall x\ldotp (x, x) \neq (1, 1)$ & 
      \texttt{fail} \\
    \hline

    $\forall x\ldotp (x, x) \neq (q, 1)$ & 
      \texttt{succ \{[q $\neq$ 1]\}} \\
    \hline

    $\exists a~ b\ldotp q = (a, b) \wedge \forall x\ldotp (x, x) \neq (a, b)$ & 
      \texttt{succ \{[q = (\_.$_0$, \_.$_1$); \_.$_0$ $\neq$ \_.$_1$]\}} \\
    \hline

  \end{tabularx}
  \caption{\lstinline{forall} evaluation}
  \label{tab:univ}
\end{table}

\section{Conclusion and future work}

We presented an approach for pattern matching implementation synthesis using relational programming. Currently, it demonstrates a good performance only
on a very small problems. The performance can be improved by searching for new ways to prune the search space and by speeding up the implementation of
relations and structural constraints. Also it could be interesting to integrate structural constraints more closely into \textsc{OCanren}'s core.
Discovering an optimal order of samples and reducing the complete set of samples is another direction for research.

The language of intermediate representation can be altered, too. It is interesting to add to an intermediate language so-called \emph{exit nodes}
described in~\cite{maranget2001}. The straightforward implementation of them might require nominal unification, but we are not aware of any
\textsc{miniKanren} implementation in which both disequality constraints and nominal unification~\cite{alphaKanren} coexist nicely.

At the moment we support only simple pattern matching without any extensions. It looks technically easy to extend our approach with
non-linear and disjunctive patterns. It will, however, increase the search space and might require more optimizations.








%%
%% The next two lines define the bibliography style to be used, and
%% the bibliography file.
\bibliographystyle{ACM-Reference-Format}
\bibliography{references}


%%
%% If your work has an appendix, this is the place to put it.
\appendix


\end{document}
\endinput

