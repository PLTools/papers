%%
%% This is file `sample-sigplan.tex',
%% generated with the docstrip utility.
%%
%% The original source files were:
%%
%% samples.dtx  (with options: `sigplan')
%%
%% IMPORTANT NOTICE:
%%
%% For the copyright see the source file.
%%  
%% Any modified versions of this file must be renamed
%% with new filenames distinct from sample-sigplan.tex.
%%
%% For distribution of the original source see the terms
%% for copying and modification in the file samples.dtx.
%%
%% This generated file may be distributed as long as the
%% original source files, as listed above, are part of the
%% same distribution. (The sources need not necessarily be
%% in the same archive or directory.)
%%
%% The first command in your LaTeX source must be the \documentclass command.
\documentclass{article}[12pt]

\usepackage[T2A]{fontenc}
\usepackage[utf8]{inputenc}

\usepackage[english]{babel}
\usepackage{listings}
\usepackage[section]{placeins}
\usepackage{multirow}
\usepackage{pgfplots}
\usepackage{yfonts}
\usepackage{subcaption}
\usepackage{xspace}
\usepackage{amsmath}
\usepackage{mathtools}
\usepackage{comment}
\usepackage{url}
%\usepackage{tikz}
%\usetikzlibrary{trees}
\usepackage{amsthm}
\usepackage{amssymb}

\sloppy

\newcommand{\ruleno}[1]{\mbox{[\textsc{#1}]}}
\newcommand{\rulen}[1]{[\textsc{#1}]}
\newcommand{\mk}{\textsc{miniKanren}\xspace}
\newcommand{\inbr}[1]{\langle #1 \rangle}
\renewcommand{\emptyset}{\varnothing}
\newcommand{\primi}[1]{\mbox{\bf #1}}

\newtheorem{theorem}{Theorem}[section]
\newtheorem{corollary}{Corollary}[theorem]
\newtheorem{lemma}[theorem]{Lemma}


\begin{document}

\section{``Commutativity'' of Unification}

Let $p_{1,2},\,q_{1,2}$~--- some terms. Then

\[
MGU\,(p_1,\,p_2)\cdot MGU\,(q_1,\,q_2)=MGU\,(q_1,\,q_2)\cdot MGU\,(p_1,\,p_2)
\]

Indeed, both $MGU\,(p_1,\,p_2)\cdot MGU\,(q_1,\,q_2)$ and $MGU\,(q_1,\,q_2)\cdot MGU\,(p_1,\,p_2)$ are most general unifiers
for pairs $(p_1,\,p_2)$ and $(q_1,\,q_2)$ simultaneously. If there were different, then terms $C\,(p_1,\,q_1)$ and
$C\,(p_2,\,q_2)$, where $C$~--- some binary constructor, would have \emph{different} most general unifiers, which is impossible.

Let

\[
\theta\xrightarrow{u_1;\,u_2;\dots u_k}{\theta^\prime}
\]

be sequence of unifications which transforms a substitution $\theta$ into substitution $\theta^\prime$. Then

\[
\theta\xrightarrow{u_{\pi(1)};\,u_{\pi(2)};\dots u_{\pi(k)}}{\theta^\prime}
\]

for any permutation $\pi$.

\section{Properties of One-step Unfolding}

Let $r$ be an application (atomic formula), $\theta$~--- a substitution; one-step unfolding transforms a pair $\inbr{\theta,\,r}$ into
a set $\{\inbr{\theta_i,\,\rho_i}\}$, where $\rho_i$~--- a list (conjunction) of applications, $\theta_i$~--- some substitution. Then
for each $i$ there is a sequence of unifications $u^i_1\dots u^i_{k_i}$ such that

\[
\theta\xrightarrow{u^i_1\dots u^i_{k_i}}{\theta_i}
\]

Proof by derivation of one-step unfolding. Thus, we may introduce a denotation

\[
\theta_i=\theta\cdot[p\to\rho_i]
\]

\section{Weakenings in Angelic Semantics}

Let $\theta$ be a substitution, $\omega$~--- a list of applications, and let $\inbr{\theta,\,\omega}$ converges to $\{\}$. Then for arbitrary substitution $\sigma$ and
arbitrary list of applications $\psi$

\[
\inbr{\theta\cdot\sigma,\,\omega\psi}
\]

converges to $\{\}$.

\section{Properties of Unfolding in Angelic Semantics}

\begin{lemma}
  Let $\{\omega_i\}$ be a state (a set of lists (conjunctions) of applications). Then $\{\omega_i\}$ converges iff each $\omega_i$ converges.
\end{lemma}

\begin{lemma}
Let $\phi a \phi^\prime$ be a list of applications with one distinguished element $a$, $\theta$~--- a substitution. Let

\[
\{\inbr{\theta\cdot[a\to\alpha_i],\,\alpha_i}\}
\]

be the result of one-step unfolding for $\inbr{\theta,\,a}$. Then

\[
\inbr{\theta,\,\phi a \phi^\prime}
\]

converges iff all

\[
\{\inbr{\theta\cdot[a\to\alpha_i],\,\phi\alpha_i\phi^\prime}\}
\]

converge.
\end{lemma}

%\begin{proof}
Proof sketch.

\paragraph{$\Leftarrow$} Let all $\{\inbr{\theta\cdot[a\to\alpha_i],\,\phi\alpha_i\phi^\prime}\}$ converge. Then unfolding of $a$ delivers a converging derivation for $\inbr{\theta,\,\phi a \phi^\prime}$.
\paragraph{$\Rightarrow$} Let $\inbr{\theta,\,\phi a \phi^\prime}$ converges. Two cases are possible.
\begin{enumerate}
\item There is no unfolding of $a$ in the converging derivation. Then
\[
  \inbr{\theta,\,\phi a \phi^\prime}
\]
converges to $\{\}$ (each state contains $a$). Moreover
\[
  \inbr{\theta,\,\phi \phi^\prime}\eqno{(*)}
\]
converges to $\{\}$ (we can repeat the converging derivation of $\inbr{\theta,\,\phi a \phi^\prime}$).

By weakening all
\[
  \inbr{\theta\cdot[a\to\alpha_i],\,\phi\alpha_i\phi^\prime}
\]
 converges to $\{\}$ taking into account $(*)$.

\item There is a sequence of derivation
\[
    \inbr{\theta,\,\phi a \phi^\prime}\to \inbr{\theta\cdot\sigma,\,\pi a \rho}\eqno{(**)}
\]

where $\sigma$ is some sequence of unifications and the next step is unfolding of $a$. Then its result is
\[
\{\inbr{\theta\cdot\sigma\cdot[a\to\alpha_i],\,\pi \alpha_i \rho}\}
\]

Take the pair
\[
\inbr{\theta\cdot[a\to\alpha_i],\,\phi\alpha_i \phi^\prime}
\]

and repeat the unfoldings of $(**)$ we can (at most) arrive at state
\[
\inbr{\theta\cdot[a\to\alpha_i]\cdot\sigma,\,\pi \alpha_i \rho}
\]

which converges since
\[
\theta\cdot[a\to\alpha_i]\cdot\sigma=\theta\cdot\sigma\cdot[a\to\alpha_i]
\]

%\end{proof}

\end{enumerate}

\begin{corollary}
  Let $\inbr{\theta,\,\omega}$~--- some pair of a substitution and a list of applications. Let $\inbr{\theta^\prime,\,\omega^\prime}$~---
  some other pair, obtained from $\inbr{\theta,\,\omega}$ in a finite number of unfoldings. Then if $\inbr{\theta,\,\omega}$ converges, then
  $\inbr{\theta^\prime,\,\omega^\prime}$~--- converges.
\end{corollary}

\begin{lemma}
\label{one-conj-lemma}
Let $\mathcal{P}_\trianglelefteq$ is a well quasi-ordering predicate, $\theta$ is a substitution, $r$ and $r^\prime$ are relation applications, $h$ is a history of calls and $\inbr{\theta,\, (r, h) (r^\prime, \epsilon)}$ is a state. Then conjunct $r^\prime$ will be unfolded in all unfailed branches in a finite number of steps of semantics ``$\rightarrow_{\mathcal{P}_\trianglelefteq}$''.
\end{lemma}
\begin{proof}
Let's prove by contradiction. \\
Let exists a unfailed branch in which $r^\prime$ will never be unfolded. So we have an infinite sequence of states of the form
\[
\inbr{\theta,\, (r, h) (r^\prime, \epsilon)} \rightarrow \inbr{\theta_1,\, \pi_1 (r^\prime, \epsilon)} \rightarrow \ldots \rightarrow \inbr{\theta_d,\, \pi_d (r^\prime, \epsilon)} \rightarrow \ldots
\]

Consider $\max_{d \in \mathcal{N}}(\max_{(r,h) \in \pi_d}length(h))$. Two cases are possible: the maximum either does not exist, or it does exist.

\begin{itemize}
    \item The maximum doesn't exist. \\
    In this case, we have conjunct histories of unlimited length. It follows that the well quasi-ordering predicate $\mathcal{P}_\trianglelefteq$ is true for unlimited history. This is contrary to the definition of well-quasi ordering $\trianglelefteq$.
    \item The maximum exists and equals $K$,
    In this case the maximum possible number of unfoldings can be estimated from above. Indeed, the conjunct $r$ can be unfolded no more than a depth of $ T = K - length(h)$ (otherwise, the length of the history of descendants will become more than $K$). Also, each unfolding adds no more than $S$ new conjuncts, where $S$ is maximum number of conjuncts in all branches of all relations (the maximum exists, since there are a finite number of relations and branches). The conjunct $r$ requires one unfolding, its children require $S$ unfoldings, their children require $S^2$ unfoldings, and so on. Thus, total number of unfoldings is
    \[
    R = \sum _{k=0}^{T}S^k = \frac{S^{T + 1} - 1}{S - 1}.
    \]
    As a result, on the one hand, we have an infinite sequence of unfoldings, and on the other hand, we have an upper estimate $R$ for the number of unfoldings. We have got a contradiction.
\end{itemize}

\end{proof}


\begin{lemma}
Let $\mathcal{P}_\trianglelefteq$ is a well quasi-ordering predicate, $\inbr{\theta;\, (r_1,h_1)\ldots(r_k,h_k)\ldots(r_n,h_n)}$ is a leaf state and $k$ is index of conjunct. Then the conjunct $r_k$ will be unfolded in all unfailed branches in a finite number of steps of semantics ``$\rightarrow_{\mathcal{P}_\trianglelefteq}$''.
\end{lemma}
\begin{proof}
Determine which conjunct will be unfolded first. Let $m$ is index of this conjunct. We define the distance between the conjuncts in the current state as
\[
dist(m, k) =
\left\{
\begin{array}{ll}
 k - m     & \mbox{if } k \geq m \\
 k - m + n & \mbox{if } k < m \\
\end{array}
\right.
\]
$dist$ determines the number of conjuncts to unfold before conjunct $r_k$. If $r_m$ is to the right of $r_k$, we need to unfold the conjunct $r_m$ and all the more right conjuncts. Then we go to the leftmost conjunct and unfold all the conjuncts to the left of $r_k$. Total count of unfolded conjuncts is $k - m + n$.

We will prove the lemma by induction on $dist(m, k)$.
\begin{itemize}
    \item Base case: $dist(m, k) = 0$. \\
    In this case $m = k$. This means that the conjunct $r_k$ will be unfolded at current step.
    \item Induction step: $dist(m, k) = s + 1$. We also assume that the lemma is true for $dist(m, k) = s$.\\
    We will complete the unfolding in the conjunct $r_m$ in a finite number of steps in all unfailed branches by the lemma~\ref{one-conj-lemma}. Then we move on to the next conjunct and the distance $s + 1$ will decrease by one. Thus, all unfailed branches satisfy the induction hypothesis.

\end{itemize}
\end{proof}

\end{document}
