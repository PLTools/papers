% !TeX spellcheck = ru_RU
% !TEX root = main.tex

\section{\miniKanren~--- краткое изложение}
\label{sec:demo}

В этом разделе miniKanren будет кратко описан в его оригинальном виде на основе канонических примеров.
Предметно-ориентированный язык организован как набор комбинаторов и макросов для Scheme/Racket, которые должны описывать поиск решения для некоторой \enquote{цели} (англ. goal).
Ядро языка состоит из четырех реляционных конструкции для создания целей:
%In this section we briefly describe miniKanren in its original form, using a canonical example.
%\miniKanren is organized as a set of combinators and macros for Scheme/Racket, designated to describe
%a search for the solution of a certain \emph{goal}.
%There are four domain-specific constructs to build \emph{goals}:

\begin{itemize}
\item Ограничение синтаксической унификации~\cite{Unification} в форме \lstinline[language=scheme]|(== $t_1$ $t_2$)|, где  $t_1$, $t_2$ --- это некоторые \enquote{термы}.
Унификация является основой для проведения поиска: если существует унификатор двух термов, то цель считается выполненной, наиболее общий унификатор сохраняется как частичное решение, и поиск в данной ветви дерева поиска продолжается. Иначе происходит возврат назад (англ. backtracking) по дереву поиска.
%unification~\cite{Unification} in the form \lstinline[language=scheme]{(== $t_1$ $t_2$)}, where $t_1$, $t_2$ are
%some \emph{terms}; unification establishes a syntactic basis for all other goals. If there is a unifier for
%two given terms, the goal is considered satisfied, a most general unifier is kept as a partial solution, and the execution
%of current branch continues. Otherwise, the current branch backtracks.

\item Ограничения-неравенства ~\cite{CKanren} в форме \lstinline[language=scheme]|($\not\equiv$ $t_1$ $t_2$)|, где $t_1$, $t_2$ --- это некоторые \enquote{термы}.
Данное ограничения запрещает поиск в поддеревьях, где данные темы равны с учетом соответствующей подстановки.
%Disequality constraint~\cite{CKanren} in the form \lstinline[language=scheme]{($\not\equiv$ $t_1$ $t_2$)}, where
%$t_1$, $t_2$ are some terms; a disequality constraint prevents all branches (starting from the current), in which the
%specified terms are equal (w.r.t. the search state), from being considered.

\item Условная конструкция в форме %Conditional construct in the form

\begin{lstlisting}[language=scheme]
(conde
    [$g^1_1\;\;g^1_2\;\;\dots\;\;g^1_{k_1}$]
    [$g^2_1\;\;g^2_2\;\;\dots\;\;g^2_{k_2}$]
    $\ldots$
    [$g^n_1\;\;g^n_2\;\;\dots\;\;g^n_{k_n}$]),
\end{lstlisting}

где каждый $g^i_j$ является некоторой целью. Построенная с использованием \lstinline|conde| цель, рассматривает коллекцию подцелей в квадратных скобках как неявную конъюнкцию (т.е. \lstinline[language=scheme]|[$g^i_1\;\;g^i_2\;\;\dots\;\;g^i_{k_i}$]| считается конъюнкцией всех $g^i_j$) и пытается решать их независимо, т.е. \lstinline|conde| работает как дизъюнкция.
%where each $g^i_j$ is a goal. A \lstinline|conde| goal considers each collection of subgoals, surrounded by square brackets, as
%implicit conjunction (so \lstinline[language=scheme]|[$g^i_1\;\;g^i_2\;\;\dots\;\;g^i_{k_i}$] | is considered as a
%conjunction of all $g^i_j$) and tries to satisfy each of them independently~--- in other words, operates on them
%as a disjunction.

\item Создание новых (англ. fresh) логических переменных осуществляется с помощью конструкции

\begin{lstlisting}[language=scheme]
   (fresh ($x_1\;\;x_2\;\;\dots\;\;x_k$)
     $g_1$
     $g_2$
     $\ldots$
     $g_n$),
\end{lstlisting}

где каждый $g^i_j$ является некоторой целью. Создаются свежие логические переменные  $x_1\;\;x_2\;\;\dots\;\;x_k$ и
осуществляется поиск цели, являющейся конъюнкцией целей $g_1\;\;g_2\;\;\dots\;\;g_n$, внутри которых могут использоваться только что введенные логические переменные.
%This form introduces fresh variables $x_1\;\;x_2\;\;\dots\;\;x_k$ and
%tries to satisfy the conjunction of all subgoals $g_1\;\;g_2\;\;\dots\;\;g_n$ (these subgoals may contain introduced fresh variables).
\end{itemize}

В качестве примера рассмотрим отношение  \lstinline|append$^o$| для конкатенации списков (по традиции реляционные объекты именуются с суффиксом$^o$).
%As an example consider a list concatenation relation; by a well-established tradition, the names
%of relational objects are superscripted by ``$^o$'', hence \lstinline|append$^o$|:

\begin{lstlisting}[mathescape=true,language=scheme,numbers=left,numberstyle=\small,stepnumber=1,numbersep=-5pt]
  (define (append$^o$ x y xy)
    (conde
     [(== '() x) (== y xy)]
     [(fresh (h t ty)
        (== `(,h . ,t) x)
        (== `(,h . ,ty) xy)
        (append$^o$ t y ty))]))
\end{lstlisting}

Отношение \enquote{\lstinline|append$^o$ x y xy|} интерпретируется как \enquote{конкатенация  \lstinline|x| и \lstinline|y| даёт \lstinline|xy|}.
Если список \lstinline|x| пуст (строка 3), то, вне зависимости от содержимого \lstinline|y|, чтобы отношение выполнялось значение \lstinline|xy| должно быть таким же как \lstinline|y|.
Иначе, \lstinline|x| нужно разделить на голову \lstinline|h| и хвост \lstinline|t|, для этого понадобятся введённые свежие переменные.
Также понадобится дополнительная переменная \lstinline|ty|, чтобы хранить список, находящий в отношении \lstinline|append$^o$| с \lstinline|t| и \lstinline|y|.
Дополнив тривиальными соображениями (строки 5--7), получим полную реализацию отношения.

%We interpret the relation ``\lstinline|append$^o$ x y xy|'' as ``the concatenation of \lstinline|x| and \lstinline|y| equals \lstinline|xy|''.
%Indeed, if the list \lstinline|x| is empty, then (regardless the content of \lstinline|y|) in order for the relation to hold
%the value for \lstinline|xy| should by equal to that of \lstinline|y|~--- hence line 3.
%Otherwise, \lstinline|x| can be decomposed into the head \lstinline|h| and the tail \lstinline|t|~--- so we need some fresh variables.
%We also need the additional variable \lstinline|ty| to designate the list that is in the relation \lstinline|append$^o$| with \lstinline|t| and \lstinline|y|.
%Trivial relational reasonings complete the implementation (lines 5-7).

Цели, полученные с помощью упомянутых выше конструкций, можно запускать с помощью примитива \lstinline|run|:
%A goal, built with the aid of the aforementioned constructs, can be run by the following primitive:

\begin{lstlisting}[mathescape=true,language=scheme]
   run $n$ ($q_1\dots q_k$) $G$
\end{lstlisting}

\noindent Здесь $n$ --- это число желаемых ответов (для всех ответов используется \enquote{\texttt{*}}); $q_i$ --- свежие искомые переменные; а $G$ --- цель, использующая эти переменные.
%Here $n$ is the number of requested answers (or ``*'' for all answers), $q_i$ are fresh query variables, and $G$ is a goal, which can
%contain these variables.


Конструкция \lstinline|run| осуществляет поиск ответов для данной цели, и возвращает потенциально бесконечный список ответов, состоящий из значений искомых переменных, которые хранят ответы на данный запрос. Например
%The \lstinline|run| construct initiates the search for answers for a given goal and returns a (finite or infinite) list
%of answers~--- the bindings for query variables, which represent individual solutions for that query. For example,

\begin{lstlisting}[mathescape=true,language=scheme]
   run 1 (q) (append$^o$ q '(3 4) '(1 2 3 4) )
\end{lstlisting}

\noindent возвращает список \lstinline|((1 2))|, который хранит одно значение искомой переменной $q$.
Процесс реконструирования ответов из внутреннего представления называет \emph{реификацией}\footnote{Реификация (англ. reification) --- превращение чего-то абстрактного (т.е. существующего в виде идеи) во что-то конкретное. Кембриджский словарь. \url{https://dictionary.cambridge.org/dictionary/english/reification} (проверено 28 января 2025 г.)}~\cite{WillThesis}.
%\noindent returns a list \lstinline|((1 2))|, which constitutes the answer for a query variable $q$. The process of constructing the answers from internal data structures of miniKanren interpreter is called \emph{reification}~\cite{WillThesis}.

