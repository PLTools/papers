% !TeX spellcheck = ru_RU
% !TEX root = main.tex

\section{Реификация и примитивы верхнего уровня }
\label{sec:reification}


В разделе~\ref{sec:goals} был упомянут примитив верхнего уровня \lstinline|run|, который позволяет запускать цель и возвращать поток состояний.
Чтобы получить ответы на запрос, представленный целью, все переменные должны быть реифицированы в соответсвующих состояниях, с использованием примитивов реификации из раздела~\ref{sec:injection}.
Но состояния содержат ответы в нетипизированном представлении, и типы ответов восстанавливаются только на основе типов реифицируемых переменных.
Таким образом, корректность типов после реификации критически зависит от требования, чтобы переменные реифицировались только в тех состояниях, которые являеются потомками (в смысле дерева поиска) того состояния, где эти переменные были созданы.
В этом разделе   описан набор примитивов, который помогает поддерживать это требование.

%In Section~\ref{sec:goals} we presented a top-level function \lstinline|run|, which
%runs a goal and returns a stream of states. To acquire answers to the query,
%represented by that goal, its free variables have to be reified in these states, and
%we described the reification primitives in Section~\ref{sec:injection}. However,
%the states keep answers in an untyped form, and the types of answers are
%recovered solely on the basis of the types of variables being reified. So, the
%type safety of the reification critically depends on the requirement to
%reify each variable only in those states, which are descendants (w.r.t. the search tree)
%of the state, in which that variable was created. In this section we describe a set of
%top-level primitives, which enforce this requirement.

Ниже представлен набор комбинаторов, которые вызывают реляционный код и проводят реификацию только в правильных состояниях.
Примитив верхнего уровня \lstinline|run| реализован заново, и теперь принимает несколько аргументов.
Его конкретный тип довольно длинен, и поэтому описан пример   использования этого комбинатора:

%We provide a set of top-level combinators, which should be used to surround relational code
%and perform reification in a transparent manner only in correct states.
%We reimplement the top-level primitive \lstinline|run| to take three
%arguments. The exact type of \lstinline|run| is rather complex and non-instructive,
%so we prefer to describe the typical form of its application:

\begin{lstlisting}[mathescape=true]
   run $\overline{n}$ (fun $l_1\dots l_n$ -> $\;\;G$) (fun $a_1\dots a_n$ -> $\;\;H$)
\end{lstlisting}

\noindent Здесь $\overline{n}$ является \emph{нумералом}, который описывает количество аргументов у других параметров \lstinline|run|,
\mbox{$l_1\dots l_n$}~--- это логические переменные;
$G$~--- цель, которая может использовать упомянутые логические переменные \mbox{$l_1\dots l_n$};
\mbox{$a_1\dots a_n$}~---  реифицированные ответы для переменных \mbox{$l_1\dots l_n$} соответственно; и,
наконец, $H$~--- \emph{обработчик}, который может использовать \mbox{$a_1\dots a_n$}.


%Здесь $\overline{n}$ stands for a \emph{numeral}, which describes the number of parameters for two other arguments of \lstinline|run|,
%\mbox{$l_1\dots l_n$}~--- free logical variables,
%$G$~--- a goal (which can make use of \mbox{$l_1\dots l_n$}),
%\mbox{$a_1\dots a_n$}~--- reified answers for \mbox{$l_1\dots l_n$}, respectively, and,
%finally, $H$~--- a \emph{handler} (which can make use of \mbox{$a_1\dots a_n$}).

The types of \mbox{$l_1\dots l_n$} are inferred from $G$ and always have a form

\begin{lstlisting}
   $\{\alpha,\;[\beta]\}$
\end{lstlisting}

\noindent since the types of variables can be constrained only in unification or disequality constraints.

The types of \mbox{$a_1\dots a_n$} are inferred from the types of \mbox{$l_1\dots l_n$} and
have the form

\begin{lstlisting}
   $(\alpha,\;\beta)$ reified stream
\end{lstlisting}

\noindent where the type \lstinline|reified|, in turn, is

\begin{lstlisting}
   type ($\alpha$, $\beta$) reified = $<\;$prj : $\alpha$; reify : (helper -> $\{\alpha,\;\beta\}$ -> $\beta$) -> $\beta>$
\end{lstlisting}

Two methods of this type can be used to perform two different styles of reification: first, a value without
free variables can be returned as is (using the method \lstinline|prj| which checks that in the value of
interest no free variables occur, and raises an exception otherwise). If the value contains some free
variables, it has to be properly injected into the logic domain~--- this is what \lstinline|reify| stands
for. It takes as an argument a type-specific tagging function, constructed using generic
primitives described in the previous section.

In other words a user-defined handler takes streams of reified answers for all variables supplied to the top-level
goal. All streams $a_i$ contain coherent elements, so they all have the same length and $n$-th elements of all
streams correspond to the $n$-th answer, produced by the goal $G$.

Несколько нумералов (с названиями \lstinline|q|, \lstinline|qr|, \lstinline|qrs|) для 1,2 и 3-аргументных отношений объявлены заранее,
остальные получаются с помощью применения функции следующего нумерала, которая принимает нумерал и увеличивают количество ожидаемых аргументов на единицу.
Реализация этой функции сделана на основе работ~\cite{Unparsing, DoWeNeed}.

%There are a few predefined numerals for one, two, etc. arguments (called, traditionally,
%\lstinline|q|, \lstinline|qr|, \lstinline|qrs| etc.), and a successor function, which
%can be applied to existing numeral to increment the number of expected arguments.
%The implementation technique generally follows~\cite{Unparsing, DoWeNeed}.

Итого, поиск и реификация тесно связаны: невозможно выполнить реификацию в произвольном состоянии для произвольной переменной.
Такое решение гарантирует корректность типов и освобождает программиста от проведения реификации полностью вручную.

%the search and reification are tightly coupled; it is simply impossible to perform the reification
%for arbitrarily-taken state and variable. This solution both guarantees the type safety and frees an end
%user from the necessity to call reification primitives manually.
