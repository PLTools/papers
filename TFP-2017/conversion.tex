\section{Relational Conversion}
\label{conversion}
\def\arraystretch{1}

Before we describe the relational conversion itself, we formulate some limitations for the source
programs. Functional programs tend to operate with higher-order values, while miniKanren is
limited by a first-order unification. Therefore, it would be unreasonable to expect, that arbitrary
functional program can be converted into a relational form (at least using reasonably simple 
transformations). 

We introduce the set of ground types $\mathcal G$:

$$
\mathcal G=\alpha \mid T^k(g_1,\dots,g_k)
$$

Informally, a value of a ground type cannot contain closures. Then we formulate the following limitations for
the programs to be converted into a relational form:

\begin{itemize}
  \item all constructor parameter types must be type variables;
  \item constructors and polymorphic equality can only be applied to the values of ground types;
  \item all \lstinline|match|-expressions must be of ground types.
\end{itemize}

The first condition means, that all algebraic datatypes (which we consider as defined implicitly, see Section~\ref{source_language}) 
have to be fully-polymorphic. The first two limitations then allow us to specify the polymorphism restriction for 
relational programs, which we mentioned informally in Section~\ref{ocanren}: all type variables are bounded to
range only over ground types (this condition, of course, is sufficient, but not necessary).

The third limitation is not essential and introduced only to simplify the presentation. If a \lstinline|match|-expression does not
have a ground type, it can always be transformed to have one by applying $\eta$-expansion:

\begin{lstlisting}
   match $e$ with {$p_i$ -> $e_i$} $\leadsto$ fun $\bar{x}$.match $e$ with {$p_i$ -> $e_i\,\bar{x}$}
\end{lstlisting}

\noindent where $\bar{x}$ is a vector of new variables, different from those in $e$, $e_i$, and $p_i$. In fact, our implementation,
described in Section~\ref{evaluation}, performs this expansion as long as a non-ground type \lstinline|match|-expression is encountered. 
This is the single case when we actually inspect types and perform $\eta$-expansion.

The general idea behind the conversion can be illustrated on a type level: an expression of type $t$ in the source
language is transformed into the expression of type $\sembr{t}^t$ in relational extension, where
the transformation $\sembr{\bullet}^t$ is defined as follows:

$$
\begin{array}{rcl}
\sembr{g}^t                     & = & g \to \G \\
\sembr{t_1 \to t_2}^t           & = & \sembr{t_1}^t \to \sembr{t_2}^t \\
%\sembr{\forall \alpha. \: t} & = & \forall \alpha. \: \sembr{t}
\end{array}
$$

In other words, an expression of a ground type is converted into a goal-returning function. The informal semantics
of this function is to make its argument respect a certain contract. As the argument can have some free variable occurrences, 
the goal tries to substitute these variables with some values in order to respect the contract this goal represents. 
For example, a constant \lstinline|Nil| is converted into a function \lstinline|fun $q$ . $q\,$=== ^Nil|.

The conversion itself is described in terms of transformation $\sembr{\bullet}^c$, see Fig.~7. %\ref{relational_conversion}. 
The first five rules
simply propagate the conversion through the expression; the last three actually do the work. These rules themselves may look complicated,
but the idea is rather simple.

\begin{figure}[t]
  \centering
  \begin{tabular}{rcp{6cm}}
     $\sembr{x}^c$                &=&$x$\\
     $\sembr{\lambda x.e}^c$      &=&$\lambda x.\sembr{e}^c$\\
     $\sembr{f\;e}^c$             &=&$\sembr{f}^c\;\sembr{e}^c$\\
     $\sembr{\lstinline|let $\;x\;$ = $\;e_1\;$ in $\;e_2$|}^c$&=&\lstinline|let $x$ = $\sembr{e_1}^c$ in $\sembr{e_2}^c$|\\
     $\sembr{\lstinline|let rec $\;f\;$ = $\lambda x.e_1\;$ in $\;e_2$|}^c$&=&\lstinline|let rec $f$ = $\sembr{\lambda x.e_1}^c$ in $\sembr{e_2}^c$|\\[2mm]
     $\sembr{C^k (e_1,\dots,e_k)}^c$&=&\lstinline|fun $q$.fresh ($q_1 \dots q_k$)|
\begin{lstlisting}
  ($\sembr{e_1}^c\; q_1$) /\
  ...
  ($\sembr{e_k}^c\; q_k$) /\
  ($q$ === $\;\uparrow(C^n (q_1, \dots, q_k)$))
\end{lstlisting}\\[-2mm]
     $\sembr{\lstinline|match $\;e\;$ with \{$C^{n_i}_i(x^i_1,\dots,x^i_{n_i})\;$ -> $\;e_i$\}|}^c$&=&\lstinline|fun $q$.fresh ($q_e$)|
\begin{lstlisting}
    ($\sembr{e}^c\;q_e$) /\
    $\bigvee_i$ ((fresh ($q^i_1\dots q^i_{n_i}$)
           ($q_e$ === $\;\uparrow C^{n_i}_i(q^i_1,\dots,q^i_{n_i})$) /\
           (fun $x^i_1\dots x^i_{n_i}$.$\sembr{e_i}^c$) ($\equiv q^i_1$) ... ($\equiv q^i_{n_i}$) $q$
     ) 
    )
\end{lstlisting}\\[-2mm]
     $\sembr{\lstinline|$e_1\,$=$\,e_2$|}^c$&=&\lstinline|fun $q$.fresh ($q_1\,q_2$)|
\begin{lstlisting}
  $\sembr{e_1}^c\,q_1$ /\
  $\sembr{e_2}^c\,q_2$ /\
  (($q_1$ === $\;q_2$ /\ $q$ === $\;$^true) |||
   ($q_1$ =/= $\;q_2$ /\ $q$ === $\;$^false)
  )
\end{lstlisting}
  \end{tabular}
\label{relational_conversion}
\caption{Relational conversion}
\end{figure}

In the case of constructor we know, that all expressions $e_i$ have ground types. Thus, their relational images are goal-returning
functions. We create a set of fresh variables (one for each expression) and pass them as arguments to these functions to associate
them with the values of the expressions. The result of conversion for the constructor application itself has to be a 
goal-returning function as well. We surround expression constructed so far with abstraction and unify its argument $q$ with the
constructor, applied to corresponding logical variables. We also apply logical constructor $\uparrow$ to respect the typing rule
for unification.

The rule for pattern-matching conversion operates similarly. First, the scrutinee must have a ground type (since it is matched against
constructors). We create a fresh variable $q_e$ and associate it with the value of the scrutinee exactly as in the previous
case. Then, for each branch we create a number of fresh variables (one for each variable in the pattern for the branch) and
express pattern-matching in terms of unification, using these variables and corresponding constructor. Finally, the body $e_i$ of the branch
is an expression with free variables, corresponding to those in the pattern. We, therefore, convert $e_i$ and surround the result with
lambdas, closing all these variables. To pass the bindings $q^i_j$ for pattern variables to the body, we apply this function to
 goal-returning functions $(\equiv q^i_j)$. This, again, gives us a goal-returning function, which we apply to the topmost result variable $q$.

The last rule follows the same pattern: both arguments of polymorphic equality are transformed into goal-returning functions, and we know, that
the arguments of these functions are of some ground type. We apply these functions to fresh variables and perform case analysis. Note, this is
the only case when we actually use disequality constraints.

An interesting property of relational conversion is that it does not change terms, which do not use constructors, equality, and pattern-matching. Thus,
a lot of useful higher-order functions~--- application, composition, fixed point, etc.~--- are already relational and can be used in
relational specifications.

Another observation is that our transformation is compositional (a relational image of application is an application of relational
images). This means, that relational conversion is compatible with separate compilation~--- multiple source files can be
converted independently without losing the possibility to work properly when combined.

Then, it is interesting, that the result of relational conversion runs in a forward direction
deterministically. Thus, relational conversion imposes only a constant-time slowdown in a forward
direction.

Finally, we formulate the following properties for relational conversion:

\begin{itemize}
\item Static correctness: if an expression $e$ has a type $t$ in the source language, then $\sembr{e}^c$ has a 
type $\sembr{t}^t$ in relational extension. In other words, relational conversion transforms properly typed
programs into properly typed. Proof is by structural induction (and trivial).
\item Partial semantic correctness: if an expression $e$ has a ground type $t$ and \mbox{$e \leadsto^f v$} for some
  value $v$, then \mbox{$\lstinline|fresh($x$)($\sembr{e}^c\;x$)| \leadsto^r (\theta,\emptyset)$}, and 
\mbox{$\theta(\mathfrak{s})=v$}, where $\mathfrak{s}$ is a semantic variable, associated with $x$ on the
first step of the relational evaluation.
%The essential part of the proof is given in the Appendix~\ref{appendix}.
%Proof
%is by induction on the length of derivation sequence (a number of lemmas have to be justified on the way).
\end{itemize}

In order to prove the complete correctness, we need some means to interpret the results of relational 
derivation with free variables in functional case. This is a subject of future research.
