% !TEX TS-program = pdflatex
% !TeX spellcheck = en_US
% !TEX root = main.tex
\documentclass[acmsmall,screen,review,anonymous]{acmart}

\AtBeginDocument{%
  \providecommand\BibTeX{{%
    Bib\TeX}}}

%% Rights management information.  This information is sent to you
%% when you complete the rights form.  These commands have SAMPLE
%% values in them; it is your responsibility as an author to replace
%% the commands and values with those provided to you when you
%% complete the rights form.
\setcopyright{acmlicensed}
\copyrightyear{2024}
\acmYear{2024}
\acmDOI{XXXXXXX.XXXXXXX}

%% These commands are for a PROCEEDINGS abstract or paper.
\acmConference[miniKanren'24]{Make sure to enter the correct
  conference title from your rights confirmation emai}{September 02--07,
  2024}{Politecnico di Milano, Italy}
%%
%%  Uncomment \acmBooktitle if the title of the proceedings is different
%%  from ``Proceedings of ...''!
%%
%%\acmBooktitle{Woodstock '18: ACM Symposium on Neural Gaze Detection,
%%  June 03--05, 2018, Woodstock, NY}
%\acmISBN{978-1-4503-XXXX-X/18/06}
\acmISBN{XXXXXXXXXXXXXXXXXXXXXXX}


%%
%% Submission ID.
%% Use this when submitting an article to a sponsored event. You'll
%% receive a unique submission ID from the organizers
%% of the event, and this ID should be used as the parameter to this command.
%%\acmSubmissionID{123-A56-BU3}

%%
%% For managing citations, it is recommended to use bibliography
%% files in BibTeX format.
%%
%% You can then either use BibTeX with the ACM-Reference-Format style,
%% or BibLaTeX with the acmnumeric or acmauthoryear sytles, that include
%% support for advanced citation of software artefact from the
%% biblatex-software package, also separately available on CTAN.
%%
%% Look at the sample-*-biblatex.tex files for templates showcasing
%% the biblatex styles.
%%

%%
%% The majority of ACM publications use numbered citations and
%% references.  The command \citestyle{authoryear} switches to the
%% "author year" style.
%%
%% If you are preparing content for an event
%% sponsored by ACM SIGGRAPH, you must use the "author year" style of
%% citations and references.
%% Uncommenting
%% the next command will enable that style.
%%\citestyle{acmauthoryear}


%%
%% end of the preamble, start of the body of the document source.

%
%\usepackage[T1]{fontenc}

% The preceding line is only needed to identify funding in the first footnote. If that is unneeded, please comment it out.
%\usepackage{cite,easyReview}
\usepackage{amsmath}%,amssymb}
\usepackage{amsfonts}
\usepackage{algorithmic}
\usepackage{graphicx}
\usepackage{textcomp}
\usepackage{xcolor}
\usepackage{hyperref}
\usepackage{csquotes}
%\usepackage{lmodern}     % Fonts will be nicer 'sudo apt install lmodern'
%\def\BibTeX{{\rm B\kern-.05em{\sc i\kern-.025em b}\kern-.08em
%    T\kern-.1667em\lower.7ex\hbox{E}\kern-.125emX}}

% ======== Custom definitions =============
\usepackage[normalem]{ulem} % \sout
\usepackage{graphicx}
\usepackage{tikz}
\usetikzlibrary{shapes.geometric,backgrounds,fit,
  %positioning-plus,
  %node-families,
  calc}
\usepackage{xcolor}
\usepackage{mathtools}
%\usepackage[draft]{hyperref}
\usepackage{marginnote}
\usepackage{verbatim}
\usepackage{listings}
\usepackage{multirow}
\usepackage{array}
\usepackage{url}
\usepackage{caption}
\usepackage{subcaption}
\usepackage{xspace}
\usepackage{flushend} % Equalizes columns on the last page
\usepackage{placeins}
\usepackage{verbatimbox}
\usepackage[title]{appendix}

\newcommand{\term}[1]{\mbox{\texttt{\textbf{#1}}}}
\newcommand{\run}[2]{\term{run}^{#1}\,\left[#2\right]}
\newcommand{\todo}[1]{{\bf\color{red}#1}}
\newcommand{\rel}[3]{{#1}\xrightarrow{#2}{#3}}
\newcommand{\prg}[1]{\mbox{\lstinline|#1|}}
\renewcommand{\leq}{\ensuremath{\leqslant}}
\renewcommand{\geq}{\ensuremath{\geqslant}}

% !TEX TS-program = pdflatex
% !TeX spellcheck = en_US
% !TEX root = main.tex
\newcommand{\OCanren}{\textsc{OCanren}}
\newcommand{\OCaml}{\textsc{OCaml}}
\newcommand{\Scheme}{\textsc{Scheme}}
\newcommand{\Kotlin}{\textsc{Kotlin}}
\newcommand{\Klogic}{\textsc{Klogic}}

\lstdefinelanguage{guideline}{
keywords={},
sensitive=true,
commentstyle=\small\itshape\ttfamily,
keywordstyle=\ttfamily\underbar,
identifierstyle=\ttfamily,
basewidth={0.5em,0.5em},
columns=fixed,
fontadjust=true,
literate={->}{{$\to$}}3 {=>}{{$\Rightarrow$}}3,
morecomment=[s]{(*}{*)}
}

\lstdefinelanguage{ocanren}{
keywords={run, conde, fresh, let, in, match, with, when, class, type,
object, method, of, rec, repeat, until, while, not, do, done, as, val, inherit,
new, module, sig, deriving, datatype, struct, if, then, else, open, private, virtual, include, success, failure, switch,
true, false, ocanren},
sensitive=true,
commentstyle=\small\itshape\ttfamily,
keywordstyle=\ttfamily\textbf,
identifierstyle=\ttfamily,
basewidth={0.5em,0.5em},
columns=fixed,
mathescape=true,
fontadjust=true,
literate={fun}{{$\lambda$}}1 {->}{{$\to$}}3 {===}{{$\equiv$}}1 {=/=}{{$\not\equiv$}}1 {|>}{{$\triangleright$}}3 {\\/}{{$\vee$}}2 {/\\}{{$\wedge$}}2 {^}{{$\uparrow$}}1
%{[]}{{\texttt|[]|}}1
,
morecomment=[s]{(*}{*)}
}

\lstdefinelanguage{ocaml}{
keywords={type, struct},
sensitive=true,
commentstyle=\small\itshape\ttfamily,
keywordstyle=\ttfamily\textbf,
%keywordstyle=\ttfamily\underbar,
identifierstyle=\ttfamily,
basewidth={0.5em,0.5em},
columns=fixed,
fontadjust=true,
literate={->}{{$\to$}}3 {=>}{{$\Rightarrow$}}3,
morecomment=[s]{(*}{*)}
}

\lstdefinelanguage{algo}{
basicstyle=\small,
keywords={if, for, return},
sensitive=true,
commentstyle=\small\itshape\rmfamily,
keywordstyle=\rmfamily\textbf,
%keywordstyle=\ttfamily\underbar,
identifierstyle=\rmfamily,
basewidth={0.5em,0.5em},
columns=fixed,
fontadjust=true,
literate={->}{{$\to$}}3 {=>}{{$\Rightarrow$}}3,
morecomment=[l]{//},
morecomment=[s]{(*}{*)}
}

\lstset{
mathescape=true,
%basicstyle=\small,
identifierstyle=\ttfamily,
keywordstyle=\bfseries,
commentstyle=\scriptsize\rmfamily,
basewidth={0.5em,0.5em},
fontadjust=true,
language=guideline
}

%\setlength{\belowcaptionskip}{-10pt}

% =========================================

\sloppy
\begin{document}

\title{TODO}

%%
%% The "author" command and its associated commands are used to define
%% the authors and their affiliations.
%% Of note is the shared affiliation of the first two authors, and the
%% "authornote" and "authornotemark" commands
%% used to denote shared contribution to the research.

\author{Peter Lozov}
\email{lozov.peter@gmail.com}
\author{Dmitrii Kosarev}
\email{Dmitrii.Kosarev@pm.me}
\affiliation{%
  \city{St.~Petersburg}
  \country{Russia}}

%\author{Dmitrii Kosarev}
%\email{Dmitrii.Kosarev@pm.me}
%\author{Dmitry Boulytchev}
%\email{dboulytchev@math.spbu.ru}
%\affiliation{%
%  \institution{St.~Petersburg State University}
%  \city{St.~Petersburg}
%  \country{Russia}
%}

%%
%% By default, the full list of authors will be used in the page
%% headers. Often, this list is too long, and will overlap
%% other information printed in the page headers. This command allows
%% the author to define a more concise list
%% of authors' names for this purpose.
\renewcommand{\shortauthors}{Kosarev and Lozov}

\begin{abstract}
The myriad of various miniKanren implementation raised an interest in comparison between them, usually by expressive power and performance.
The straightforward solution to rewrite typical benchmarks for another host language
could be cumbersome: you need to have a familiarity with new host language.
The increasing size and number of benchmarks makes this task error prone: it's easy to occasionally make a spelling mistake when rewriting low-level constructions like unification. In this paper we discuss work-in-progress tool which translates programs written with miniKanren in OCaml (\OCanren{}) to \Scheme{} and miniKanren in \Kotlin{} (\Klogic{}).

\end{abstract}

%\begin{CCSXML}
%<ccs2012>
%   <concept>
%       <concept_id>10003752.10010070</concept_id>
%       <concept_desc>Theory of computation~Theory and algorithms for application domains</concept_desc>
%       <concept_significance>300</concept_significance>
%       </concept>
%   <concept>
%       <concept_id>10003752.10003790.10003795</concept_id>
%       <concept_desc>Theory of computation~Constraint and logic programming</concept_desc>
%       <concept_significance>500</concept_significance>
%       </concept>
%   <concept>
%       <concept_id>10002950.10003705.10003707</concept_id>
%       <concept_desc>Mathematics of computing~Solvers</concept_desc>
%       <concept_significance>500</concept_significance>
%       </concept>
%   <concept>
%       <concept_id>10003120.10003121.10003129.10011757</concept_id>
%       <concept_desc>Human-centered computing~User interface toolkits</concept_desc>
%       <concept_significance>500</concept_significance>
%       </concept>
% </ccs2012>
%\end{CCSXML}
%
%\ccsdesc[300]{Theory of computation~Theory and algorithms for application domains}
%\ccsdesc[500]{Theory of computation~Constraint and logic programming}
%\ccsdesc[500]{Mathematics of computing~Solvers}
%\ccsdesc[500]{Human-centered computing~User interface toolkits}
%
%\keywords{graphic user interface, relational programming, constraint solving}


\maketitle

%\begin{IEEEkeywords}
%component, formatting, style, GUI, constraint satisfaction, relational programming, synthesis
%\end{IEEEkeywords}

% !TEX TS-program = pdflatex
% !TeX spellcheck = en_US
% !TEX root = main.tex


\section{Introduction}
\label{sec:intro}


One of distinguishable features of \miniKanren{} is the fact that it is a family of languages:
many languages may host different \miniKanren{} implementations.
For example, faster-\miniKanren{} for \Scheme{} and \Racket{}, \CoreLogic{} for \textsc{Closure}, \OCanren{} for \OCaml{}, \Klogic{} for \Kotlin{} and others.
The users of these DSLs may want to compare expressive power of various flavor of miniKanren, specifics due to host language, and performance implications of choosing a different host language.


The straightforward solution is to rewrite a number of significant benchmarks for many implementation, as it done for other languages\footnote{\url{https://benchmarksgame-team.pages.debian.net/benchmarksgame/index.html}}.
Doing it manually is time consuming and error prone.
Due to low-level nature of relational programs, it is easy to make spelling mistakes, for example accidentally write wrong identifier in unification arguments.
(We did many of mistakes of this kind while porting programs from \OCanren{} to \Klogic{}.) Moreover, \miniKanren{} doesn't pardon relational programs that do the same thing: it was reported, that the order of conjuncts significantly affects performance even if the search does the same unifications~\cite{scheduling2022}.

% !TEX TS-program = pdflatex
% !TeX spellcheck = en_US
% !TEX root = main.tex

\section{Related Work}
\label{related}
%
%GUI design and implementation has been a hot topic for decades. Thus, to no surprise there is a lot of frameworks, approaches, papers
%and reports on the subject. A fair share of them (if not all) present declarative and automatic solutions. A careful study, however,
%discovers that this ``declarativeness'' and ``automation'' is understood differently then in our case.
%
%First of all, we need to mention some software frameworks and tools for design and implementation of GUI and visualization of data, for example, \textsc{React}~\cite{react},
%\textsc{Jetpack Compose}~\cite{Jetpack}, \textsc{SwiftUI}~\cite{SwiftUI}, \textsc{Streamlit}~\cite{Streamlit}, \textsc{D3}~\cite{D3} and others.
%These frameworks provide a number of layout primitives which end-users can employ in order to render their data or UI. For example, \textsc{Streamlit}
%provides a number of builtin layout primitives like ``columns'', ``container'', ``modal dialog'', etc.~\cite{StreamlitLayout} and an endless
%variety of third-party external components. These primitives allow end-users to abstract away of concrete controls coordinate calculation and their
%relative alignment; they also prescribe a reasonable behavior on enclosing pane resizing. However, which layout primitives to use is decided by
%end-users, not the system. If due to any reason the layout needs to be changed these changes have to be implemented manually. In our case
%end-users do not specify concrete layouts, only the logical structure of the UI. The guideline takes care of concrete layout, depending on
%external constraints such as enclosing pane size, screen resolution or even regional settings (for example, right-to-left writing system). As long
%as the logical structure remains unchanged no interference from end-users is required for laying out the UI in different settings. On the
%other hand these frameworks can be used as back-ends in our approach since they provide a similar set of layout primitives.
%
%Constraint programming has already been used for deciding the placement of GUI controls. One of the examples are constraint reactive programming
%language \textsc{Wallingford}~\cite{Wallingford2016} and the \textsc{Cassowary} system~\cite{Cassowary2001}. \textsc{Wallingford} allows to attach
%constraints of various strength to different values in the program. The system reacts to the time changes and updates these values without violation
%of the constraints. For example, one could calculate a width of a GUI control as the sine of current time. The \textsc{Cassowary} system and its
%descendants allow to calculate the sizes and positions of controls dynamically, for example at the moment of canvas resize.
%%The background theory is linear arithmetic.
%It supports many different constraints, for example, Z-ordering, arithmetic operators (for example, a control's width can be the half
%of another one's height), overlapping views, etc.  These systems are targeted for the tasks of dynamic adaptation the sizes of controls on resize.
%Also, they don't provide support of expressing general rules about correct control placing, the constraints are added for values of concrete structure.
%In our work we make a strict separation between GUI structure and rules of correct positioning of controls. We have doubts about expressing non-deterministic
%layouts in these systems, for example, if vertical or horizontal placement of controls depend on their sizes. On the other hand, the number of various
%constraint types is larger than in our approach. For example, they allow to express overlapping views, but based on our experience we initially ruled out
%this option, and our current implementation doesn't generate such layouts at all.
%
%Finally, in recent years methods and approaches from AI in ML/data-science sense start to percolate into the area. Some of the works address much more
%ambitious objectives, than ours.
%
%First of all, there is a direction of research on UI code generation from images~\cite{Cai2023}. Given a designer-drawn form, a code
%generator recognizes UI controls and their relative placements and generates implementation code for one of GUI frameworks. This approach
%is completely orthogonal and incompatible with what we suggest. Indeed, it requires an interaction with a designer when implementing every piece of
%an interface while in our case a designer is only involved when guidelines are developed; then our system produces guideline-compatible
%interfaces \emph{en masse} automatically. In~\cite{Robust} a slightly different task is addressed: given a picture of an interface synthesize its
%implementable layout in terms of Android GUI primitives which would be scalable across various devices while avoiding a typical layout errors.
%To achieve this goal, after recognizing UI controls and their locations a certain set of relational layout constraints is extracted. This
%set of constraints resembles our layout primitives but is aligned with Android's \texttt{ConstraintLayout} widget~\cite{ConstraintLayout} semantics.
%Given these constraints and a set of \emph{robustness properties} developed by the authors an implementation code is generated with
%the aid of a probabilistic model trained on a large set of existing Android interfaces. This implementation is more stable under screen size
%or resolution change, than that provided by image recognition only. Interestingly, one of the motivations for the work, as the authors
%specify explicitly, is that ``the same layout needs to be rendered potentially on more than 15 000 Android devices with $\approx$100 different density
%independent screen sizes. Requiring the user to provide and maintain input specifications for all of them is infeasible yet highly desirable.''
%That is exactly what our system is capable of doing within a few minutes and with 100\% accuracy, so we consider our approach
%much more general.
%
%Automatic design of consistent (uniform throughout an application) GUI is addressed in~\cite{LearningGUI}. The approach is based on the idea of completion:
%assuming there is already a set of consistent layouts a problem of adding yet another element is addressed; the addition should be consistent with the previous
%designs. While this problem is, indeed, related to that we address (indeed, consider an existing set of designs as an implicitly specified guideline),
%we can identify a number of potential weaknesses. First, only addition of a component is considered, but not removal; second, the addition/subtraction
%of components does not necessarily lead to a ``monotonic'' change of the layout (adding yet another text field may result is a drastically different placement);
%finally, the initial set of designs to be consistent with rarely comes out of a thin air; most likely it is a result of following some
%existing (and perhaps implicit) guidelines, which should be specified and followed explicitly.
%
%In~\cite{Grid} a much more ambitious problem of synthesizing a layout with no guidelines, based only on aesthetic, ergonomic, etc., metrics is considered.
%Genetic algorithm is employed for synthesis, and users' feedback is used as a way to measure the quality of the synthesis; the synthesis itself can
%sometimes take hours. While the approach presumably allows to synthesize aesthetically convincing layouts with no designer input, the problem of
%layout consistency for sufficiently different structures is left unaddressed.
%
%An interesting problem of layout exploration is considered in~\cite{Scout}. The objective is to help the \emph{designers} to develop convincing and
%diverse layouts. A set of constraints is introduced which designers can use in order to specify some requirements for the design. Interestingly, in our terms the
%set of constraints is a mixture of layout and structural ones: for example, both order and alignment constraints are present. Given
%a number of constraints the system generates a set of layouts consistent with this constraints; by updating the constraints a designer can
%continue the exploration. As a constraint solver modifier branch-and-bound algorithm is used; as the number of feasible solutions
%turned out to be enormous a set of heuristic metrics was developed to rule aesthetically unfeasible designs out. While this work
%shares some similarities with ours, being targeted on designers it can be considered as a mean to \emph{develop guidelines}, not to synthesize
%guideline-consistent designs.

% !TEX TS-program = pdflatex
% !TeX spellcheck = en_US
% !TEX root = main.tex

\section{Discussion and Future Work}

We presented the results of our work on guideline-based synthesis of GUI layouts. The approach we take makes it possible to
automatically build GUI layouts which by construction comply with a designer-specified set of rules. The prototype implementation
we developed allows to synthesize the layouts for the real-word industrial GUI components w.r.t. the real-world guidelines in
appropriate time.

One interesting question which may arise is if the application of relational programming is essential for this problem to be solved. Indeed, the set of guidelines
describes a matching (or rewriting) system which, in principle, can be directly implemented without any use of relational techniques. We argue, however,
that in this case a whole piece of work on justification of the correctness of the solution would have been repeated anew. In our
case, the justification trivially follows from the completeness of the \textsc{miniKanren} search and refutational completeness of our solution. We also speculate
that such a solution would require reinventing of some implementation techniques to support nondeterminism and backtracking, which are already native to relational
programming. Finally, the duality between patterns over structure and relational goals (initially unexpected for us), to our opinion, witnesses, that relational
programming is a truly relevant technique for this problem. %, and we are not stretching an owl over a globe.

%We can also consider the task of getting rid of extra solver (\textsc{Z3}) as relevant; this would not only simplify the infrastructure of the system, but also allow to
%integrate the constraint resolution phase into the constraint synthesis, improving the performance of the whole system.
%, and finally provide the completeness
%of the synthesizer, which we strictly speaking do not yet have. To do this, we plan to develop a relational integer inequality solver with a binary representation
%of fixed-size numbers. Note that for satisfactory performance it will also be necessary to extend \OCanren{} with inequality operations for such numbers in the
%form of a new type of constraint.




\bibliographystyle{ACM-Reference-Format}
\bibliography{main}


%% !TEX TS-program = pdflatex
% !TeX spellcheck = en_US
% !TEX root = main.tex
\appendix
\section{Relational Interpreter for Quines, Translated from OCanren}
\label{appendix:synQuines}

An implementation of relational interpreter, translated from \OCanren{}. Extra tagging is represented with \textbf{\underbar{underlined bold}} style.
\todo{TODO Link}

\lstinputlisting[language=miniKanren, numbers=left, ndkeywords={symb,seq,val}]{interpreter_syn.scm}

\section{Default Quines Implementation}
\label{appendix:defaultQuines}
\todo{TODO Link}

\lstinputlisting[language=miniKanren,numbers=left]{default_quines.scm}
%\end{subappendices}
%\appendix

\end{document}
