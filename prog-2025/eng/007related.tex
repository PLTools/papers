% !TEX TS-program = pdflatex
% !TeX spellcheck = en_US
% !TEX root = main.tex

GUI design and implementation has been a hot topic for decades. Thus, to no surprise there is a lot of frameworks, approaches, papers
and reports on the subject. A fair share of them (if not all) present declarative and automatic solutions. A careful study, however,
discovers that this ``declarativeness'' and ``automation'' is understood differently then in our case.

First of all, we need to mention some software frameworks and tools for design and implementation of GUI and visualization of data, for example, \textsc{React}~\cite{react},
\textsc{Jetpack Compose}~\cite{Jetpack}, % \textsc{SwiftUI}~\cite{SwiftUI},
\textsc{Streamlit}~\cite{Streamlit}%, \textsc{D3}~\cite{D3}
and others.
These frameworks provide a number of layout primitives which end-users can employ in order to render their data or UI. For example, \textsc{Streamlit}
provides a number of builtin layout primitives like ``columns'', ``container'', ``modal dialog'', etc.~\cite{StreamlitLayout} and an endless
variety of third-party external components. These primitives allow end-users to abstract away of concrete controls coordinate calculation and their
relative alignment; they also prescribe a reasonable behavior on enclosing pane resizing. However, which layout primitives to use is decided by
end-users, not the system. If due to any reason the layout needs to be changed these changes have to be implemented manually. In our case
end-users do not specify concrete layouts, only the logical structure of the UI. The guideline takes care of concrete layout, depending on
external constraints such as enclosing pane size, screen resolution or even regional settings (for example, right-to-left writing system). As long
as the logical structure remains unchanged no interference from end-users is required for laying out the UI in different settings. On the
other hand these frameworks can be used as back-ends in our approach since they provide a similar set of layout primitives.
In this paper we used Qt/QML as a backend.

Constraint programming has already been used for deciding the placement of GUI controls. One of the examples are constraint reactive programming
language \textsc{Wallingford}~\cite{Wallingford2016} and the \textsc{Cassowary} system~\cite{Cassowary2001}. \textsc{Wallingford} allows to attach
constraints of various strength to different values in the program. The system reacts to the time changes and updates these values without violation
of the constraints. For example, one could calculate a width of a GUI control as the sine of current time. The \textsc{Cassowary} system and its
descendants allow to calculate the sizes and positions of controls dynamically, for example at the moment of canvas resize.
%The background theory is linear arithmetic.
It supports many different constraints, for example, Z-ordering, arithmetic operators (for example, a control's width can be the half
of another one's height), overlapping views, etc.  These systems are targeted for the tasks of dynamic adaptation the sizes of controls on resize.
Also, they don't provide support of expressing general rules about correct control placing, the constraints are added for values of concrete structure.
In our work we make a strict separation between GUI structure and rules of correct positioning of controls. We have doubts about expressing non-deterministic
layouts in these systems, for example, if vertical or horizontal placement of controls depend on their sizes. On the other hand, the number of various
constraint types is larger than in our approach. For example, they allow to express overlapping views, but based on our experience we initially ruled out
this option, and our current implementation doesn't generate such layouts at all.

Finally, in recent years methods and approaches from AI in ML/data-science sense start to percolate into the area. Some of the works address much more
ambitious objectives, than ours.

First of all, there is a direction of research on UI code generation from images~\cite{Cai2023}. Given a designer-drawn form, a code
generator recognizes UI controls and their relative placements and generates implementation code for one of GUI frameworks. This approach
is completely orthogonal and incompatible with what we suggest. Indeed, it requires an interaction with a designer when implementing every piece of
an interface while in our case a designer is only involved when guidelines are developed; then our system produces guideline-compatible
interfaces \emph{en masse} automatically. In~\cite{Robust} a slightly different task is addressed: given a picture of an interface synthesize its
implementable layout in terms of Android GUI primitives which would be scalable across various devices while avoiding a typical layout errors.
To achieve this goal, after recognizing UI controls and their locations a certain set of relational layout constraints is extracted. This
set of constraints resembles our layout primitives but is aligned with Android's \textsc{ConstraintLayout} widget~\cite{ConstraintLayout} semantics.
Given these constraints and a set of \emph{robustness properties} developed by the authors an implementation code is generated with
the aid of a probabilistic model trained on a large set of existing Android interfaces. This implementation is more stable under screen size
or resolution change, than that provided by image recognition only. Interestingly, one of the motivations for the work, as the authors
specify explicitly, is that ``the same layout needs to be rendered potentially on more than 15 000 Android devices with $\approx$100 different density
independent screen sizes. Requiring the user to provide and maintain input specifications for all of them is infeasible yet highly desirable.''
That is exactly what our system is capable of doing within a few minutes and with 100\% accuracy, so we consider our approach
much more general.

Automatic design of consistent (uniform throughout an application) GUI is addressed in~\cite{LearningGUI}. The approach is based on the idea of completion:
assuming there is already a set of consistent layouts a problem of adding yet another element is addressed; the addition should be consistent with the previous
designs. While this problem is, indeed, related to that we address (indeed, consider an existing set of designs as an implicitly specified guideline),
we can identify a number of potential weaknesses. First, only addition of a component is considered, but not removal; second, the addition/subtraction
of components does not necessarily lead to a ``monotonic'' change of the layout (adding yet another text field may result is a drastically different placement);
finally, the initial set of designs to be consistent with rarely comes out of a thin air; most likely it is a result of following some
existing (and perhaps implicit) guidelines, which should be specified and followed explicitly.

In~\cite{Grid} a much more ambitious problem of synthesizing a layout with no guidelines, based only on aesthetic, ergonomic, etc., metrics is considered.
Genetic algorithm is employed for synthesis, and users' feedback is used as a way to measure the quality of the synthesis; the synthesis itself can
sometimes take hours. While the approach presumably allows to synthesize aesthetically convincing layouts with no designer input, the problem of
layout consistency for sufficiently different structures is left unaddressed.

An interesting problem of layout exploration is considered in~\cite{Scout}. The objective is to help the \emph{designers} to develop convincing and
diverse layouts. A set of constraints is introduced which designers can use in order to specify some requirements for the design. Interestingly, in our terms the
set of constraints is a mixture of layout and structural ones: for example, both order and alignment constraints are present. Given
a number of constraints the system generates a set of layouts consistent with this constraints; by updating the constraints a designer can
continue the exploration. As a constraint solver modifier branch-and-bound algorithm is used; as the number of feasible solutions
turned out to be enormous a set of heuristic metrics was developed to rule aesthetically unfeasible designs out. While this work
shares some similarities with ours, being targeted on designers it can be considered as a mean to \emph{develop guidelines}, not to synthesize
guideline-consistent designs.
