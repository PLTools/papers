
% !TeX spellcheck = ru_RU
% !TEX root = mainrus.tex


\section{Introduction}
\label{sec:intro}

Графические интерфейсы пользователя (GUI) --- один из самых распространенных способов взаимодействия с программами. Как на персональном компьютере, так и на ноутбуке, мобильном телефоне, банкомате, игровой консоли и т.д. пользователь встретится с некоторым набором графических элементов управления и интуитивно понятным смыслом. За такой низкий порог вхождения мы платим некоторую цену: графические интерфейсы управления не так просто разрабатывать~\cite{UI1,UI12}.


%Graphical user interface (GUI) is by far the most common device people use to interact with software. Be it desktop
%or laptop computer, mobile device, ATM, game console, etc., an end-user will probably find a certain set
%of graphical controls with intuitively clear meanings. This low entry threshold, however, comes at some
%price~--- GUI is rather a subtle thing to design~\cite{UI1,UI12}.

Обычные пользователи как правило не осознают, что делает графический интерфейс приятно выглядящим и удобным в использовании. Чтобы получить качественный интерфейс необходимо учесть множество эстетических, эргономических и психологических аспектов, а это требует экспертизы, которой программисты обладают не всегда. Из-за этого появляются должности не только по разработке интерфейсов (UI), но и по дизайну взаимодействия с пользователем (UX)~\cite{UI5}, где специалисты разрабатывают правила (англ. guidelines) разработки хороших интерфейсов. Данные правила часто неполны, двусмысленны и неформальны, а следование им требует интуиции, которой разработчики могут и не обладать.
Взаимодействие между разработчиками и дизайнерами часто осложняется разницей в экспертизе, подходах и складах мышления~\cite{UI6}. Более того, упомянутые правила не всегда естественно переносятся между разными устройствами и даже разрешениями экрана. Поэтому, чтобы поддерживать все желаемые платформы, разработчикам придется следовать нескольких наборам правил проектирования интерфейсов одновременно.

%Regular people rarely ponder what makes GUI so good-looking and easy to use. In order to make it such a lot of
%aesthetic, ergonomic, and psychological aspects have to be taken into account which require an expertise
%regular software developers can hardly possess. This led to the establishing the roles of not only UI, but also UX
%(user experience) designers~\cite{UI5}, who devise the \emph{guidelines} which developers have to
%follow when implementing interfaces. These guidelines, however, are often incomplete, ambiguous and informal, and understanding
%and following them require intuition the developers do not always have.  The communication between these two
%sides~--- developers and designers~---  often constitutes an important issue due to the difference in
%expertise, approaches, and mindsets~\cite{UI6}. Moreover, the guidelines as a rule
%do not simply scale between various devices or even screen resolutions, so in order to support all desirable
%platforms a developer might have to follow different guidelines separately.

В данной работе представлен подход к синтезу расположения (англ. layout) элементов управления GUI  с учетом правил (гайдлайна\textcolor{red}{???}). Наша модель GUI отделяет логическую структуру элементов управления (часть предельно понятная программистам), от конкретного расположения элементов, отступов и выравниваний (понятных дизайнеру).
%In this paper we present a framework for synthesizing the layouts for GUI controls with respect to design guidelines. We give a model for GUI
%which separates its logical structure (the part software developers understand well) from layout, i.e. concrete placement
%of GUI controls with insets, alignments and indentation (the part the designers are responsible for).
Мы разработали автоматический подход, который по логической структуре (предоставляемой программистом) и набору правил (предоставленных дизайнером) синтезирует раскладку элементов управления, удовлетворяющую этим правилам.
В общем случае правила неоднозначны, поэтому несколько конкретных раскладок может им удовлетворять.
Мы рассматриваем синтез интерфейсов как задачу удовлетворения ограничений (англ. сonstraint satisfaction), и используем реляционное программирование~\cite{TRS} и подход
верификатор-из-решателя~\cite{searchproblems} \textcolor{red}{???} (англ. verifier-to-solver), чтобы получить решатель, располагающий элементы управления с помощью примитивов.
В примитивной форме такой решатель неэффективен, и мы применили набор оптимизаций (в том числе специализацию), чтобы его ускорить. В результате получился решатель, написанный на смеси из реляционного и функционального кода. Затем набор примитивов расположения превращается в линейные ограничения на координаты и решается с помощью SMT решателя Z3~\cite{Zthree}, что дает нам абсолютные координаты всех элементов управления.
Этот подход позволяет получить \emph{корректный} и \emph{полный} решатель: по указанной логической структуре он синтезирует все расположения элементов, удовлетворяющие правилам дизайна.
% We devise a completely automatic approach
%which for a given logical structure (provided by a software developer) and a set of guideline rules (provided by a UI/UX designer) synthesizes
%concrete layouts admissible w.r.t. these rules.
%In general the guidelines are ambiguous, so multiple concrete layouts can
%comply with them. We consider layout synthesis as a constraint satisfaction problem and use relational programming~\cite{TRS}
%and the verifier-to-solver~\cite{searchproblems} approach to construct a solver which synthesizes layouts in terms of layout primitives. In its
%vanilla form the solver turned out to be inefficient so we applied a number of refinements, including specialization; in effect our
%final solver is a mixture of functional and relational code. The set of synthesized layout primitives, in turn, is reified into linear
%integer constraints which are then solved using \textsc{Z3} solver~\cite{Zthree} providing absolute coordinates for all GUI controls.
%Our approach delivers a \emph{sound} and \emph{complete} solver~--- for a given logical GUI structure it synthesizes a set of all layouts each of which is
%admissible w.r.t. given guidelines.

При этом солвер достаточно эффективен, чтобы запускаться на обычном ноутбуке.
%The solver is also efficient and can be run on a regular laptop.

