% !TEX TS-program = pdflatex
% !TeX spellcheck = en_US
% !TEX root = main.tex

\section{Showcase: Relational Interpreter}
\label{sec:interpreter}

Current largest application of the translator is a solver for \Java{} subtyping problems~\cite{JavaGenericsSolver2023} which has both \OCaml{} and (partially synthesized, halfway hand-written) \Kotlin{} implementation.
For \Scheme{} we tested relational arithmetic~\cite{Kiselyov2008PureDA} and interpreters~\cite{Untagged}.
The synthesized code for the former is almost identical to hand-written one, due to special treatment of lists in the translator.
On the contrary, relational interpreter is different in it's value representation.

\OCanren{} representation use algebraic data types to describe terms and result of interpretation.
This level of indirection (i.e. tagging) is necessary for \OCaml{}: you can't store either a closure with its contents or arbitrary value in a single data type.
After translation this tags appear in the synthesized relational interpreter (appendix~\ref{appendix:synQuines}) and play the role of second layer tagging (if we consider built-in \Scheme{} lists as a first layer).
And more tagging means higher unification cost and lower performance.

On the other hand explicit tagging (\texttt{val\_} and \verb=closure= for interpretation results, \verb=seq= and \verb=symb= for sequences (lists) and symbols) allow to avoid \emph{absento} constraint.
With tagging we always are able to distinguish closure returned from interpreter and application of function named ``closure'' to its arguments.
This feature was previously reported~\cite{OCanren}, but as far as we know nobody tried to benchmark the default \Scheme{} implementation and rewritten in \OCanren{} like manner.
It is difficult to say a priory which one performs better: the original has an extra constraint that does deep traversal of the values, the another one has more tagging.
The results~\ref{fig:benchmarks} were collected  on the desktop machine Intel\copyright~Core\texttrademark~i7-4790K (16Gb RAM).
\Racket{} 8.7 was taken from the official Ubuntu 23.04 repository (compiled to native code using \Chez{} backend).
%\OCaml{} compiler 4.14.2+flambda was installed using \textsc{OPAM}\footnote{\url{https://opam.ocaml.org/}  (access date: \DTMdate{2024-06-02})} package manager.


\begin{figure}
\begin{tabular}{c|c|c|c }
Name &     Min &    Mean &     Max \\\hline
100 quines &   554&   577&   747\\
15 twines5 &   403&   426&   566\\
2 thrines &   723&   767&   939\\
100 quines (translated) &   291&   325&   364\\
15 twines (translated) &   267&   287&   330\\
2 thrines (translated) &   489&   527&   679\\
\end{tabular}
\label{fig:benchmarks}
\caption{Benchmarking relational original relational interpreter (with \emph{absento}) and the one translated from \OCanren{}.
The table contains minimum, maximum and mean synthesis time (in ms, less is better).
The numbers show that on these tests the implementation without absento is faster.}
\end{figure}

%
%\begin{figure}
%\begin{lstlisting}[language=ocaml]
%fresh (t)
%    (term === seq (symb !!"quote" % (t % nil ())))
%    (r === val_ t)
%    (not_in_envo2 !!"quote" env)
%\end{lstlisting}
%\end{figure}