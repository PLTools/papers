\documentclass[submission,copyright,creativecommons]{eptcs}
\providecommand{\event}{TEASE-LP 2020} % Name of the event you are submitting to
\usepackage{breakurl}             % Not needed if you use pdflatex only.
\usepackage{underscore}           % Only needed if you use pdflatex.

\usepackage{amsmath,amssymb}
%\usepackage[english]{babel}
\usepackage{amssymb}
\usepackage{mathtools}
\usepackage{listings}
\usepackage{comment}
\usepackage{indentfirst}
\usepackage{hyperref}
\usepackage{amsthm}
\usepackage{xcolor}
\usepackage{stmaryrd}
\usepackage{eufrak}
\usepackage{placeins}
\newtheorem{theorem}{Theorem}
%\newtheorem{lemma}{Lemma}
%\newtheorem{corollary}{Corollary}
\newtheorem{hyp}{Hypothesis}
%\newtheorem{definition}{Definition}

\lstdefinelanguage{minikanren}{
basicstyle=\normalsize,
keywords={fresh},
sensitive=true,
commentstyle=\itshape\ttfamily, % \footnotesize\itshape\ttfamily,
keywordstyle=\textbf,
identifierstyle=\ttfamily,
basewidth={0.5em,0.5em},
columns=fixed,
fontadjust=true,
literate={fun}{{$\lambda\;\;$}}1 {->}{{$\to$}}3 {===}{{$\,\equiv\,$}}1 {=/=}{{$\not\equiv$}}1 {|>}{{$\triangleright$}}3 {/\\}{{$\wedge$}}2 {\\/}{{$\vee$}}2,
morecomment=[s]{(*}{*)}
}

\lstset{
mathescape=true,
language=minikanren
}

\usepackage{letltxmacro}
\newcommand*{\SavedLstInline}{}
\LetLtxMacro\SavedLstInline\lstinline
\DeclareRobustCommand*{\lstinline}{%
  \ifmmode
    \let\SavedBGroup\bgroup
    \def\bgroup{%
      \let\bgroup\SavedBGroup
      \hbox\bgroup
    }%
  \fi
  \SavedLstInline
}

\def\transarrow{\xrightarrow}
\newcommand{\setarrow}[1]{\def\transarrow{#1}}

\def\padding{\phantom{X}}
\newcommand{\setpadding}[1]{\def\padding{#1}}

\def\subarrow{}
\newcommand{\setsubarrow}[1]{\def\subarrow{#1}}

\newcommand{\trule}[2]{\frac{#1}{#2}}
\newcommand{\crule}[3]{\frac{#1}{#2},\;{#3}}
\newcommand{\withenv}[2]{{#1}\vdash{#2}}
\newcommand{\trans}[3]{{#1}\transarrow{\padding{\textstyle #2}\padding}\subarrow{#3}}
\newcommand{\ctrans}[4]{{#1}\transarrow{\padding#2\padding}\subarrow{#3},\;{#4}}
\newcommand{\llang}[1]{\mbox{\lstinline[mathescape]|#1|}}
\newcommand{\pair}[2]{\inbr{{#1}\mid{#2}}}
\newcommand{\inbr}[1]{\left<{#1}\right>}
\newcommand{\highlight}[1]{\color{red}{#1}}
%\newcommand{\ruleno}[1]{\eqno[\scriptsize\textsc{#1}]}
\newcommand{\ruleno}[1]{\mbox{[\textsc{#1}]}}
\newcommand{\rulename}[1]{\textsc{#1}}
\newcommand{\inmath}[1]{\mbox{$#1$}}
\newcommand{\lfp}[1]{fix_{#1}}
\newcommand{\gfp}[1]{Fix_{#1}}
\newcommand{\vsep}{\vspace{-2mm}}
\newcommand{\supp}[1]{\scriptsize{#1}}
\newcommand{\sembr}[1]{\llbracket{#1}\rrbracket}
\newcommand{\cd}[1]{\texttt{#1}}
\newcommand{\free}[1]{\boxed{#1}}
\newcommand{\binds}{\;\mapsto\;}
\newcommand{\dbi}[1]{\mbox{\bf{#1}}}
\newcommand{\sv}[1]{\mbox{\textbf{#1}}}
\newcommand{\bnd}[2]{{#1}\mkern-9mu\binds\mkern-9mu{#2}}
\newcommand{\meta}[1]{{\mathcal{#1}}}
\newcommand{\dom}[1]{\mathtt{dom}\;{#1}}
\newcommand{\primi}[2]{\mathbf{#1}\;{#2}}
\renewcommand{\dom}[1]{\mathcal{D}om\,({#1})}
\newcommand{\ran}[1]{\mathcal{VR}an\,({#1})}
\newcommand{\fv}[1]{\mathcal{FV}\,({#1})}
\newcommand{\tr}[1]{\mathcal{T}r_{#1}}
\newcommand{\diseq}{\not\equiv}
\newcommand{\reprfunset}{\mathcal{D}}
\newcommand{\reprfun}{\mathfrak{f}}
\newcommand{\cstore}{\Omega}
\newcommand{\cstoreinit}{\cstore_\epsilon^{init}}
\newcommand{\csadd}[3]{\mathbf{add}\,(#1, #2 \diseq #3)}  %{#1 + [#2 \diseq #3]}
\newcommand{\csupdate}[2]{\mathbf{update}\,(#1, #2)}  %{#1 \cdot #2}

\let\emptyset\varnothing
\let\eps\varepsilon

\title{Certified Semantics for Disequality Constraints}
\author{Dmitry Rozplokhas
\institute{Higher School of Economics and \\ JetBrains Research, Russia}
\email{rozplokhas@edu.hse.ru}
\and
%Andrey Vyatkin
%\institute{Saint Petersburg State University, Russia}
%\email{dewshick@gmail.com}
%\and
Dmitry Boulytchev
\institute{Saint Petersburg State University and \\ JetBrains Research, Russia}
\email{dboulytchev@math.spbu.ru}
}
\def\titlerunning{Certified Semantics for Disequality Constraints}
\def\authorrunning{D. Rozplokhas \& D. Boulytchev}
\begin{document}
\maketitle

We present a certified semantics for disequality constraints in \textsc{miniKanren}. In its initial form~\cite{TRS,MicroKanren} \textsc{miniKanren} introduces a single
form of constraint~--- unification of terms. While from a theoretical standpoint unification together with other primitive constructs (conjunction, disjunction and
fresh variable introduction) form a Turing-complete basis, in practice of relational programming a number of extensions are oftenly used to make specifications more
expressive, concise or efficient. One of the most important extensions is \emph{disequality constraint}.

Disequality constraint~\cite{Disunification} introduces one additional type of base goal~--- a disequality of two terms

\[
t_1 \diseq t_2
\]

The informal semantics of disequality constraint is complementary to that of unification: it puts certain restrictions on free variables in the terms which
prevent them from turning into syntactically equal. Similarly to unification, whose evaluation results in a substitution, which is then threaded through
the rest of computations, the effect of disequalify constraint is recorded in a \emph{constraint store} which is later used to check the violation of
disequalities~\cite{CKanren}.

%It has a natural interpretation: we want all solutions (the values of the free variables in the terms) for which the two given terms are not syntactically equal.
%In our framework, we formalize a denotational semantics of a goal as some subset of the set $\reprfunset$ of all \emph{representing functions}, i.e. the functions
%that map every free variable into a ground term. Although for convenient usage the domain of a representing function is the set of all free variables, we maintain
%the important \emph{completeness condition}, which states that only the values on free variables that occur in a goal determine whether a representing function belongs
%to the denotational semantics of this goal. In such terms, we have the following definition of denotational semantics for a disequality goal:

%\[ \sembr{t_1 \diseq t_2}  =  \{\reprfun \in \reprfunset \mid \overline{\reprfun}\,(t_1)=\overline{\reprfun}\,(t_2)\}, \]

%where $\overline{\reprfun}\,(t)$ applies $\reprfun$ to all free variables in $t$. This definition for a new type of goals fits nicely into the general inductive
%definition of denotational semantics of an arbitrary goal and preserve its properties such as completeness condition.

We present an extension of our prior work on certified semantics for core \textsc{miniKanren}~\cite{CertifiedSemantics}. In that work we
defined denotational semantics, similar to the least Herbrand model, and operational semantics, corresponding to conventional for \textsc{miniKanren}
\emph{interleaving search}, and proved the soundness and completeness of the latter w.r.t. the former. The development was formally certified in
\textsc{Coq} proof assistant, and a correct-by-construction interpreter was extracted.

The contribution of our current work is as follows:

\begin{itemize}
\item we extend our denotational semantics to handle disequality constraints;
\item we introduce a new abstraction layer (a constraint store with a number of abstract operations) in our operational semantics;
\item we formulate a set of \emph{sufficient conditions for completeness}, expressed as algebraic properties of constraint store and
  abstract operators, and prove the soundness and completeness of operational semantics w.r.t. the denotational one;
\item we present a concrete implementation of constraint store and abstract operators and show that they satisfy the
  sufficient conditions; thus, the soundness and completeness of the implementation with disequality constraints follow
  immediately, and correct-by-construction interpreter for \textsc{miniKanren} with disequalify constraints
  can be extracted.
\end{itemize}

%At the same time, there is variability in how the interpreter should be extended to be able to deal with disequalities. Different MiniKanren versions use different ways,
%which vary in simplicity, efficiency, and usability, and we want our operational semantics to capture many of these implementations at once. There is a common part: in
%standard MiniKanren the state of evaluation consists of a substitution (the information accumulated from unifications up to this point) and a number of fresh variables
%allocated, MiniKanren with disequality constraints augments this pair with a \emph{constraint store}, which is updated as new unifications and disequalities are encountered
%and in the end is used to presents information about disequalities in some form. However, the definition of a constraint store and its updates differ together with its
%denotational interpretation. We, therefore, parametrize our operational semantics with all this.

The extension of denotational semantics is straightforward (as disequality constraint is complementary to unification). In operational case,
we assume that we are given a set of constraint store objects, which we denote by $\cstore_\sigma$ (indexing every constraint store with
some substitution $\sigma$ and assuming the store and the substitution are consistent with each other), and three following operations:

\begin{enumerate}
\item Initial constraint store $\cstoreinit$ (where $\epsilon$ is empty substitution), which does not contain any constraints yet.
\item Adding a disequality constraint to a store $\csadd{\cstore_\sigma}{t_1}{t_2}$, which may result in a new constraint store $\cstore^\prime_\sigma$ or a failure $\bot$,
  if the new constraint store is inconsistent with the substitution $\sigma$.
\item Updating a substitution in a constraint store $\csupdate{\cstore_\sigma}{\delta}$ to intergate a new substitution $\delta$ into the current one,
  which may result in a new constraint store $\cstore^\prime_{\sigma \delta}$ or a failure $\bot$, if the constraint store is inconsistent with the new substitution.
\end{enumerate}

The definition of operational semantics for the language with disequality constraints is now straightforward: for unification we use $\mathbf{update}$ operation and for
disequality constraint we use $\mathbf{add}$. In both cases the search in the current branch is pruned if these primitives return $\bot$.

To prove the soundness and completeness result we need a mean to relate both denotational and operational semantics. As in our prior work, this can be done
by prescribing a denotational interpretation (denoted by ``$\sembr{\bullet}$'') not only to goals, but also to substitutions and constraint stores. Thus, we
may formulate the following set of \emph{sufficient conditions} for soundness and completeness:

%The main objective for our two semantics of MiniKanren was to prove the soundness and completeness of the operational one with respect to the denotational one to
%ensure that the set of solutions that interleaving search in MiniKanren finds for a given program is actually equivalent to its mathematical model. To extend this
%result on the language with disequalities we first need a way to interpret the answers in the new form. For that, we also assume we are provided with a denotational
%interpretation of given constraint stores: $\sembr{\cstore_\sigma} \subset \reprfunset$, which we regard as the set of all solutions satisfying the constraints in a
%given store. Then every answer state $(\sigma, n_r, \cstore_\sigma)$ will represent the set $[\sigma] \cap \sembr{\cstore_\sigma}$, where $[\sigma]$ is the set of
%all representing functions extending the substitution $\sigma$. And we can formulate the soundness and completeness as follows.

%\begin{theorem}[Operational semantics soundness and completeness]
%If all free variables in a goal $g$ belong to the set $\{\alpha_1,\dots,\alpha_n\}$, then

%\[
%\bigcup_{(\sigma, n_r, \cstore_\sigma) \in Tr(g, (\epsilon, n, \cstoreinit))}  {[\sigma] \cap \sembr{\cstore_\sigma}}  \qquad =_n \qquad \sembr{g},
%\]

%where $Tr(g, st)$ is the stream of states that is the operational semantics of a goal $g$ with initial state $st$, and `$=_n$' means the sets are equal if we
%restrict the domains of representing functions in them to free variables $\{\alpha_1,\dots,\alpha_n\}$.
%\end{theorem}

%To be able to prove it we, of course, need certain requirements for the given operations on constraint stores. We came up with the following list of sufficient
%conditions on them for soundness and completeness.

\begin{enumerate}
\item $\sembr{\cstoreinit} = \reprfunset$ (where $\reprfunset$ is the whole domain in our denotational semantics);
\item $\csadd{\cstore_\sigma}{t_1}{t_2} = \cstore^\prime_\sigma \implies \sembr{\cstore^\prime_\sigma} \cap \sembr{\sigma} = \sembr{\cstore_\sigma} \cap \sembr{t_1 \diseq t_2} \cap \sembr{\sigma}$;
\item $\csadd{\cstore_\sigma}{t_1}{t_2} = \bot \implies \sembr{\cstore_\sigma} \cap \sembr{t_1 \diseq t_2} \cap \sembr{\sigma} = \emptyset$;
\item $\csupdate{\cstore_\sigma}{\delta} = \cstore^\prime_{\sigma \delta} \implies \sembr{\cstore^\prime_{\sigma \delta}} \cap \sembr{\sigma \delta} = \sembr{\cstore_\sigma} \cap \sembr{\sigma \delta}$;
\item $\csupdate{\cstore_\sigma}{\delta} = \bot \implies \sembr{\cstore_\sigma} \cap \sembr{\sigma \delta} = \emptyset$.
\end{enumerate}

These conditions state that given denotational interpretation and the given operations on constraint stores are adequate to each other.
The conditions 2-5 describe exactly what we need to prove the soundness and completeness for base goals (unification and disequality); at the same time,
these conditions have relatively simple intuitive meaning in terms of these two operations and are expected to hold naturally in all reasonable
implementations of constraint stores.

%The certified proof of soundness and correctness 

%We can prove that this is enough for soundness and completeness to hold for an arbitrary goal. However,
%contrary to our expectations, the proof for other types of goals, except base ones, is not working for this new formulation and has to be modified
%significantly in the case of \lstinline|fresh|. Specifically, we need one additional condition on constraint store in state $(\sigma, n, \cstore_\sigma)$:
%only the values on the first $n$ fresh variables determine whether a representing function belongs to the denotational semantics $[\sigma] \cap \sembr{\cstore_\sigma}$
%of the state (note the similarity to the completeness condition). Luckily, we can infer this property for all states that can be constructed by our operational
%semantics from the necessary conditions in the list above.

%Thus, for an arbitrary implementation, we need to give a formal definition of constraint store object and its denotational interpretation, provide three
%operations for it and prove five conditions on them, and by this we ensure that for arbitrary goal the interpretations of all solutions found by the
%search in this version of MiniKanren will cover exactly the mathematical interpretation of this goal. One thing we can not guarantee though is that
%the solutions will be presented to the user in a way that reflects their interpretation correctly. So we expect provided denotational interpretation
%for constraint store to be exactly what is put on the screen. {\color{red} Something that sounds more impressive is needed here probably.}

%As well as our previous development this extension is certified in Coq\footnote{\color{red} TODO: create a new branch in the miniKanren-coq repository
%and put a link here}. We describe operational semantics and its soundness and completeness as modules parametrized by the definitions of constraint
%stores and our necessary conditions. We also provide a module with an example of a very obvious (and very impractical) suitable implementation of
%disequality constraints (with lists of pairs of terms as constraint stores) and extract a correct-by-construction reference interpreter for MiniKanren
%with disequality constraints from the operational semantics with this simple implementation passed as the parameter.

%In addition to verification of correctness of different implementation of disequality constraints, we potentially can use our framework to formally
%state and prove some of its other important properties (for example, we currently allow meaningless answers with empty interpretations, which don't
%affect soundness or completeness, but we probably want ``good'' implementations not to have such things) or correctness of some optimizations inside
%specific implementation (for example, throwing out a constraint that is subsumed by some other constraint in the store).

%\nocite{*}
\bibliographystyle{eptcs}
\bibliography{main}

\end{document}
