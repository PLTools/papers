\documentclass[acmlarge]{acmart}
\usepackage[
    type={CC},           % your choice
    modifier={by-sa},    % your choice
    version={4.0},       % your choice
]{doclicense}            % your choice, see \doclicenseThis below

\settopmatter{printacmref=false}
\fancyfoot{}

\makeatletter
\def\@formatdoi#1{}
\def\@permissionCodeOne{miniKanren.org/workshop}
\def\@copyrightpermission{\doclicenseThis} % your choice of text
\def\@copyrightowner{Copyright held by the author(s).} % your choice
\makeatother

\copyrightyear{2019}
\setcopyright{rightsretained}

\acmMonth{8}
\acmArticle{5}

%% Bibliography style
\bibliographystyle{ACM-Reference-Format}
%% Citation style
%% Note: author/year citations are required for papers published as an
%% issue of PACMPL.
\citestyle{acmauthoryear}   %% For author/year citations


%%%%%%%%%%%%%%%%%%%%%%%%%%%%%%%%%%%%%%%%%%%%%%%%%%%%%%%%%%%%%%%%%%%%%%
%% Note: Authors migrating a paper from PACMPL format to traditional
%% SIGPLAN proceedings format must update the '\documentclass' and
%% topmatter commands above; see 'acmart-sigplanproc-template.tex'.
%%%%%%%%%%%%%%%%%%%%%%%%%%%%%%%%%%%%%%%%%%%%%%%%%%%%%%%%%%%%%%%%%%%%%%


%% Some recommended packages.
\usepackage{booktabs}   %% For formal tables:
                        %% http://ctan.org/pkg/booktabs
\usepackage{subcaption} %% For complex figures with subfigures/subcaptions
                        %% http://ctan.org/pkg/subcaption


\usepackage{amsmath,amssymb}
\usepackage[russian,english]{babel}
\usepackage{amssymb}
\usepackage{mathtools}
\usepackage{listings}
\usepackage{comment}
\usepackage{indentfirst}
\usepackage{hyperref}
\usepackage{amsthm}
\usepackage{stmaryrd}
\usepackage{eufrak}
\usepackage{lstcoq}

\newtheorem{theorem}{Theorem}
\newtheorem{lemma}{Lemma}
\newtheorem{corollary}{Corollary}
\newtheorem{hyp}{Hypethesis}
\newtheorem{definition}{Definition}

\lstdefinelanguage{minikanren}{
keywords={fresh},
sensitive=true,
commentstyle=\small\itshape\ttfamily,
keywordstyle=\textbf,
identifierstyle=\ttfamily,
basewidth={0.5em,0.5em},
columns=fixed,
fontadjust=true,
literate={fun}{{$\lambda\;\;$}}1 {->}{{$\to$}}3 {===}{{$\,\equiv\,$}}1 {=/=}{{$\not\equiv$}}1 {|>}{{$\triangleright$}}3 {/\\}{{$\wedge$}}2 {\\/}{{$\vee$}}2,
morecomment=[s]{(*}{*)}
}

\lstset{
mathescape=true,
language=minikanren
}

\usepackage{letltxmacro}
\newcommand*{\SavedLstInline}{}
\LetLtxMacro\SavedLstInline\lstinline
\DeclareRobustCommand*{\lstinline}{%
  \ifmmode
    \let\SavedBGroup\bgroup
    \def\bgroup{%
      \let\bgroup\SavedBGroup
      \hbox\bgroup
    }%
  \fi
  \SavedLstInline
}

\def\transarrow{\xrightarrow}
\newcommand{\setarrow}[1]{\def\transarrow{#1}}

\def\padding{\phantom{X}}
\newcommand{\setpadding}[1]{\def\padding{#1}}

\def\subarrow{}
\newcommand{\setsubarrow}[1]{\def\subarrow{#1}}

\newcommand{\trule}[2]{\frac{#1}{#2}}
\newcommand{\crule}[3]{\frac{#1}{#2},\;{#3}}
\newcommand{\withenv}[2]{{#1}\vdash{#2}}
\newcommand{\trans}[3]{{#1}\transarrow{\padding{\textstyle #2}\padding}\subarrow{#3}}
\newcommand{\ctrans}[4]{{#1}\transarrow{\padding#2\padding}\subarrow{#3},\;{#4}}
\newcommand{\llang}[1]{\mbox{\lstinline[mathescape]|#1|}}
\newcommand{\pair}[2]{\inbr{{#1}\mid{#2}}}
\newcommand{\inbr}[1]{\left<{#1}\right>}
\newcommand{\highlight}[1]{\color{red}{#1}}
%\newcommand{\ruleno}[1]{\eqno[\scriptsize\textsc{#1}]}
\newcommand{\ruleno}[1]{\mbox{[\textsc{#1}]}}
\newcommand{\rulename}[1]{\textsc{#1}}
\newcommand{\inmath}[1]{\mbox{$#1$}}
\newcommand{\lfp}[1]{fix_{#1}}
\newcommand{\gfp}[1]{Fix_{#1}}
\newcommand{\vsep}{\vspace{-2mm}}
\newcommand{\supp}[1]{\scriptsize{#1}}
\newcommand{\sembr}[1]{\llbracket{#1}\rrbracket}
\newcommand{\cd}[1]{\texttt{#1}}
\newcommand{\free}[1]{\boxed{#1}}
\newcommand{\binds}{\;\mapsto\;}
\newcommand{\dbi}[1]{\mbox{\bf{#1}}}
\newcommand{\sv}[1]{\mbox{\textbf{#1}}}
\newcommand{\bnd}[2]{{#1}\mkern-9mu\binds\mkern-9mu{#2}}
\newcommand{\meta}[1]{{\mathcal{#1}}}
\newcommand{\dom}[1]{\mathtt{dom}\;{#1}}
\newcommand{\primi}[2]{\mathbf{#1}\;{#2}}
\renewcommand{\dom}[1]{\mathcal{D}om\,({#1})}
\newcommand{\ran}[1]{\mathcal{VR}an\,({#1})}
\newcommand{\fv}[1]{\mathcal{FV}\,({#1})}
\newcommand{\tr}[1]{\mathcal{T}r_{#1}}

\newcommand{\searchRule}[6] {
  #1, #2 \vdash (#3, #4) \xRightarrow{#5} #6}
\newcommand{\extSearchRule}[8] {
  #1, #2, #3, #4 \vdash (#5, #6) \xRightarrow{#7}_{e} #8}
\newcommand{\q}{\hspace{0.5em}}
\newcommand{\bigcdot}{\boldsymbol{\cdot}}
\newcommand{\bigslant}[2]{{\raisebox{.2em}{$#1$}\left/\raisebox{-.2em}{$#2$}\right.}}

\let\emptyset\varnothing
\let\eps\varepsilon

\sloppy

\begin{document}

%% Title information
\title{Certified Semantics for \textsc{miniKanren}} %% [Short Title] is optional;
                                           %% when present, will be used in
                                           %% header instead of Full Title.
\titlenote{This work was partially suppored by the grant 18-01-00380 from The Russian Foundation for Basic Research} %% \titlenote is optional;
                                        %% can be repeated if necessary;
                                        %% contents suppressed with 'anonymous'
%\subtitle{Subtitle}                     %% \subtitle is optional
%\subtitlenote{with subtitle note}       %% \subtitlenote is optional;
                                        %% can be repeated if necessary;
                                        %% contents suppressed with 'anonymous'


%% Author information
%% Contents and number of authors suppressed with 'anonymous'.
%% Each author should be introduced by \author, followed by
%% \authornote (optional), \orcid (optional), \affiliation, and
%% \email.
%% An author may have multiple affiliations and/or emails; repeat the
%% appropriate command.
%% Many elements are not rendered, but should be provided for metadata
%% extraction tools.

\author{Dmitry Rozplokhas}
\affiliation{%
  \institution{Higher School of Economics}}
\affiliation{%
  \institution{JetBrains Research}
  \country{Russia}}
\email{darozplokhas@edu.hse.ru}

\author{Andrey Vyatkin}
\affiliation{%
  \institution{Saint Petersburg State University}
  \country{Russia}}
\email{dewshick@gmail.com}

\author{Dmitry Boulytchev}
\affiliation{%
  \institution{Saint Petersburg State University}}
\affiliation{%
  \institution{JetBrains Research}
  \country{Russia}}
\email{dboulytchev@math.spbu.ru}



%% Abstract
%% Note: \begin{abstract}...\end{abstract} environment must come
%% before \maketitle command
\begin{abstract}
  We present two formal semantics for the core \textsc{miniKanren}. First, we give denotational
  variant which corresponds to the minimal Herbrand model for definite logic programs. Second,
  we present operational semantics which models interleaving, and prove its soundness and
  completeness w.r.t. denotational semantics. Our development is supported by formal \textsc{Coq}
  specification, thus making it certified.
\end{abstract}


%% 2012 ACM Computing Classification System (CSS) concepts
%% Generate at 'http://dl.acm.org/ccs/ccs.cfm'.
\begin{CCSXML}
<ccs2012>
<concept>
<concept_id>10003752.10003790.10003795</concept_id>
<concept_desc>Theory of computation~Constraint and logic programming</concept_desc>
<concept_significance>500</concept_significance>
</concept>
<concept>
<concept_id>10003752.10010124.10010131.10010133</concept_id>
<concept_desc>Theory of computation~Denotational semantics</concept_desc>
<concept_significance>500</concept_significance>
</concept>
<concept>
<concept_id>10003752.10010124.10010131.10010134</concept_id>
<concept_desc>Theory of computation~Operational semantics</concept_desc>
<concept_significance>500</concept_significance>
</concept>
</ccs2012>
\end{CCSXML}
\ccsdesc[500]{Theory of computation~Constraint and logic programming}
\ccsdesc[500]{Theory of computation~Denotational semantics}
\ccsdesc[500]{Theory of computation~Operational semantics}
%% End of generated code


%% Keywords
%% comma separated list
\keywords{Relational programming, denotational semantics, operational semantics, certified programming}  %% \keywords are mandatory in final camera-ready submission


%% \maketitle
%% Note: \maketitle command must come after title commands, author
%% commands, abstract environment, Computing Classification System
%% environment and commands, and keywords command.
\maketitle
\thispagestyle{empty}

% !TEX TS-program = pdflatex
% !TeX spellcheck = en_US
% !TEX root = main.tex


\section{Introduction}
\label{sec:intro}


One of distinguishable features of \miniKanren{} is the fact that it is a family of languages:
many languages may host different \miniKanren{} implementations.
For example, \faster{}\footnote{\url{https://github.com/michaelballantyne/faster-minikanren} (access date: \DTMdate{2024-06-06})} for \Scheme{} and \Racket{}, \CoreLogic{} for \textsc{Closure}, \OCanren{}~\cite{OCanren} for \OCaml{}, \Klogic{}~\cite{Klogic2023} for \Kotlin{} and others.
The users of these DSLs may want to compare expressive power of various flavor of miniKanren, specifics due to host language, and performance implications of choosing a different host language.


The straightforward solution is to rewrite a number of significant benchmarks for many implementation, as it done for other languages\footnote{\url{https://benchmarksgame-team.pages.debian.net/benchmarksgame/index.html} (access date: \DTMdate{2024-06-06})}.
Doing it manually is time consuming and error prone.
Due to low-level nature of relational programs, it is easy to make spelling mistakes, for example accidentally write wrong identifier in unification arguments.
(We did many of mistakes of this kind while porting programs from \OCanren{} to \Klogic{}.) Moreover, \miniKanren{} doesn't pardon relational programs that solve the same task: it was reported, that the order of conjuncts significantly affects~\cite{scheduling2022} performance even if the search does the same unifications.

The differences between host languages also complicate porting relational code.
For example, \Kotlin{} doesn't support currying and partial applications comparatively to \OCaml{}, and sometimes full $\eta$-expansion is needed.
Also, porting from dynamically typed languages like \Scheme{} to statically typed ones like \OCaml{} could be uneasy for newcomers to statically typed languages.
This porting could be not straightforward:
basic data representations in \OCaml{}/\Klogic{} (algebraic data types and classes with subclasses~--- sum types) is different from \Scheme{} (lists, i.e. arbitrary length tuples~--- product types).
This fact in some cases requires special constraints~\cite{Wildcards2023} to level the expressivity, and in other cases (like relational interpreters) allows to get rid of \emph{absento/symbolo} constraints.

Things could get even more complicated where we want to port larger projects which are using functional/relational approach where relational parts are intermixed with straightforward programming.
The developer is obliged to know relational approach, the original host general purpose language and to have experience  with a new host language.

In this paper we describe current status of our converter from relational \OCanren{} to \Klogic{} and \miniKanren{} in \Scheme{}.
At the moment only relational subset of \OCanren{} is supported, we don't support whole \OCaml{} language.
In next section we describe technical aspects of our approach and currently supported features.
In section \ref{sec:interpreter} we discuss transformation in relational interpreter~\cite{Untagged} from \OCanren{} to \Scheme{} and peculiarities of autogenerated implementation.



\section{The Source Language and Relational Extension}

Our development of relational conversion is based on the idea of transforming a program in a functional
language into a program in \emph{relational extension} of that language. In the context of 
miniKanren, this approach looks quite natural, since miniKanren itself, as a DSL, reuses 
many important features (for example, function definitions) from a host language.

In this section, we present a formal description of a small functional language, taken as a source
for relational conversion. We describe its syntax, typing rules, and semantics, and then extend it
with relational constructs. We specify the typing rules and semantics for the extension as well.


\subsection{The Source Language}
\label{source_language}

The syntax of our source functional language is shown on Figure~\ref{functional_syntax}. It consists of a lambda calculus, 
enriched with constructors with fixed arities $C^n$, patterns $p$ and pattern-matching constructs, and  
expressions for recursive/non-recursive let-bindings.
Among the constructors we distinguish two nullary interpreted constructors \lstinline|true| and \lstinline|false|, and add a boolean equality
operator ``$=$''. 

\begin{wrapfigure}{r}{0.5\textwidth}
\centering
%\scalebox{0.9}{
$$
\begin{array}{rcl}
 \mathcal E &=&x\\
     & &\lambda x.e\\
     & &e_1\;e_2\\
     & &C^n(e_1,\dots, e_n)\\
     & &\lstinline|true|\\
     & &\lstinline|false|\\
     & &\lstinline|let $x$ = $e_1$ in $e_2$|\\
     & &\lstinline|let rec $f$ = $\lambda x.e_1$ in $e_2$|\\
     & &e_1\,=\,e_2\\
     & &\lstinline|match $e$ with $\{p_i$ -> $e_i\}$|\\
     & &\\
 \mathcal P &=&C^n(x_1,\dots,x_n)\\
\end{array}
$$
%}
\caption{The syntax of source language}
\label{functional_syntax}
\end{wrapfigure}

In a pattern matching, we only allow shallow patterns (which is not an essential limitation) and do not allow wildcards (which is 
important~--- converting wildcard pattern matching into relational form would require essentially different projections).

\begin{comment}
This choice of a language may 
look quite a restrictive. However, in terms of relational programming, the language contains virtually everything one would need. Indeed, from
a relational conversion standpoint the standard built-in integer arithmetics, for example, is of no use~--- 
there is simply no way to convert integer expression into relational form, using integer expressions. In order to use relational 
arithmetics, one needs to reimplement everything from scratch, using, for example, Peano encoding; but Peano arithmetics can be
easily expressed in the language we present.
\end{comment}

Our language is equipped with Hindley-Milner type system, and we present the typing rules in a conventional syntax-directed form 
on Fig.~\ref{functional_typing}. Besides type variables and function types, our system contains a number of implicitly defined 
algebraic datatypes $T^k$, and we stipulate, that each constructor $C^n$ belongs to the exactly one
datatype. In the rule \textsc{Constr$_T$}, we assume that type $t^C$ has the form $T^k(t_1,\dots,t_k)$, where each of the types
$t_i$ is recovered from the types $t_i^C$ of arguments of constructor $C^n$ and, moreover, these types agree in the sense of
constructor application. Similarly, in the rule \textsc{Match$_T$}, the types of all $C_i^{k_i}(x^i_1,\dots,x^i_{k_i})$ are expected
to be equal $t^C$, and $t^{C_i}_j$ is a type of $j$-th argument of constructor $C_i$, used in the pattern. The rule \textsc{Eq$_T$}
specifies that both operands of equality operator must have the same (but arbitrary) type. Thus, we can call this operator
``polymorphic equality''.

\setarrow{:}
\newcommand{\typed}[3]{\withenv{#1}{\trans{#2}{}{#3}}}

\begin{figure}
\centering
{\bf Types:}
$$
\begin{array}{rcll}
  \mathcal X &=&\alpha, \beta, \dots                            &\mbox{\supp{(type variables)}}\\
  \mathcal D &=&\lstinline|bool|,\,T^n,...                      &\mbox{\supp{(datatype constructors)}}\\
  \mathcal T &=&\alpha\mid T^k(t_1,\dots,t_k)\mid t_1\to t_2 &\mbox{\supp{(types)}}\\
  \mathcal S &=&\forall\bar{\alpha}.t                           &\mbox{\supp{(type schemas)}}
\end{array}
$$
{\bf Typing rules:}
\def\arraystretch{0}
\begin{tabular}{p{7cm}p{7cm}}
$$
\typed{\Gamma}{\lstinline|true|,\;\lstinline|false|}{\lstinline|bool|}
\ruleno{Bool$_T$}
$$ 
&
$$
\trule{\typed{\Gamma}{e_1}{t}\;\;\;\;\typed{\Gamma}{e_2}{t}}
      {\typed{\Gamma}{e_1=e_2}{\lstinline|bool|}}
\ruleno{Eq$_T$}
$$
\\
$$
\trule{\typed{\Gamma}{e_i}{t^C_i}}
      {\typed{\Gamma}{C^n(e_1,\dots,e_n)}{t^C}}
\ruleno{Constr$_T$}
$$
&
$$
\typed{\Gamma,x:\forall\bar{\alpha}.t}{x}{t[\bar{\alpha}\gets\bar{t^\prime}]}
\ruleno{Var$_T$}
$$
\\[-2mm]
$$
\trule{\typed{\Gamma}{f}{t_1\to t_2}\;\;\;\;\typed{\Gamma}{e}{t_1}}
      {\typed{\Gamma}{f\;e}{t_2}}
\ruleno{App$_T$}
$$
&
$$
\trule{\typed{\Gamma,\,x:t_1}{f}{t_2}}
      {\typed{\Gamma}{\lambda x.f}{t_1\to t_2}}
\ruleno{Abs$_T$}
$$
\\[-2mm]
\multicolumn{2}{p{14cm}}{
$$
\trule{\typed{\Gamma}{e_1}{t_1}\;\;\;\;\typed{\Gamma,x:\forall\bar{\alpha}.t_1}{e_2}{t}}
      {\typed{\Gamma}{\lstinline|let $\;x\;$ = $\;e_1\;$ in $\;e_2$|}{t}},\;\bar{\alpha}=FV(t_1)\setminus FV(\Gamma)
\ruleno{Let$_T$}
$$}\\[-2mm]
\multicolumn{2}{p{14cm}}{
$$
\trule{\typed{\Gamma,f:t_1}{\lambda x.e_1}{t_1}\;\;\;\;\typed{\Gamma,f:\forall\bar{\alpha}.t_1}{e_2}{t}}
      {\typed{\Gamma}{\lstinline|let rec $\;f\;$ = $\;\lambda x.e_1\;$ in $\;e_2$|}{t}},\;\bar{\alpha}=FV(t_1)\setminus FV(\Gamma)
\ruleno{LetRec$_T$}
$$}\\[-2mm]
\multicolumn{2}{p{14cm}}{
$$
\trule{\typed{\Gamma}{e}{t^C}\;\;\;\;\typed{\Gamma,x^i_1:t^{C_i}_1,\dots,x^i_{k_i}:t^{C_i}_{k_i}}{e_i}{t}}
      {\typed{\Gamma}{\lstinline|match $\;e\;$ with $\;\{C_i^{k_i}(x^i_1,\dots,x^i_{k_i})$ -> $e_i\}$|}{t}}
\ruleno{Match$_T$}
$$}
\end{tabular}
\caption{Typing rules for the source language}
\label{functional_typing}
\end{figure}


\setarrow{\to}
\newcommand{\step}[2]{\trans{\inbr{#1}}{}{\inbr{#2}}}

\begin{figure}[t]
\centering
{\bf Values:}
$$
\mathcal V = C^n(v_1,\dots,v_n)\mid\lambda x.e\mid\mu f\lambda x.e\mid\lstinline|true|\mid\lstinline|false|
$$
{\bf Contexts:}
$$
\mathcal C = \Box\;e\mid v\;\Box\mid\lstinline|let $x$ = $\Box$ in $e$|\mid\lstinline|match $\;\Box\;$ with $\{p_i$->$e_i\}$|\mid C^n(\bar{v},\Box,\bar{e})\mid\Box\lstinline|=e|\mid\lstinline|v=|\Box
$$
{\bf Stack of contexts:}
$$
\mathcal S=\epsilon\mid\mathcal C : \mathcal S
$$
{\bf States:}
$$
\inbr{\mathcal S, e}\mbox{\supp{(stack of contexts, expression)}};\;\inbr{\epsilon,e}\mbox{\supp{(initial state)}};\;\inbr{\epsilon,v}\mbox{\supp{(final state)}}
$$
{\bf Transitions:}
\vskip2mm
\bgroup
\def\arraystretch{0}
\begin{tabular}{p{7cm}p{7cm}}
\multicolumn{2}{p{14cm}}{
$$
\step{C:\mathcal S,\, v}{\mathcal S,\, C[v]}\ruleno{Value}
$$}\\[-4mm]
$$
\step{\mathcal S,\, f\;e}{\Box\;e:\mathcal S,\, f}\ruleno{AppL}
$$&
$$
\step{\mathcal S,\, v\;e_2}{v\;\Box:\mathcal S,\, e_2}\ruleno{AppR}
$$\\[-4mm]
$$
\step{\mathcal S,\,e_1=e_2}{\Box=e_2:\mathcal S,\,e_1}\ruleno{EqL}
$$&
$$
\step{\mathcal S,\,v=e}{v=\Box:\mathcal S,\,e}\ruleno{EqR}
$$\\[-4mm]
\multicolumn{2}{p{14cm}}{
$$
\step{\mathcal S,\,v=v}{\mathcal S,\,\lstinline|true|}\ruleno{EqTrue}
$$}\\[-4mm]
\multicolumn{2}{p{14cm}}{
$$
\step{\mathcal S,\,v_1=v_2}{\mathcal S,\,\lstinline|false|},\;v_1\ne v_2\ruleno{EqFalse}
$$}\\[-4mm]
\multicolumn{2}{p{14cm}}{
$$
\step{\mathcal S,\, (\lambda x.e)\;v}{\mathcal S,\, e[x\gets v]}\ruleno{Beta}
$$}\\[-4mm]
\multicolumn{2}{p{14cm}}{
$$
\step{\mathcal S,\, (\mu f\lambda x.e)\;v}{\mathcal S,\, e[f\gets\mu f\lambda x.e,\, x\gets v]}\ruleno{Mu}
$$}\\[-4mm]
\multicolumn{2}{p{14cm}}{
$$
\step{\mathcal S,\, C^n(v_1,\dots,v_{k-1},e_k,\dots,e_n)}{C^n(v_1,\dots,v_{k-1},\Box,\dots,e_n):\mathcal S,\, e_k}\ruleno{Constr}
$$}\\[-4mm]
\multicolumn{2}{p{14cm}}{
$$
\step{\mathcal S,\, \lstinline|let $\;x\;$ = $\;e_1\;$ in $\;e_2$|}{\lstinline|let $\;x\;$ = $\;\Box\;$ in $\;e_2$|:\mathcal S,\, e_1}\ruleno{Let}
$$}\\[-4mm]
\multicolumn{2}{p{14cm}}{
$$
\step{\mathcal S,\, \lstinline|let $\;x\;$ = $\;v\;$ in $\;e$|}{\mathcal S,\,e[x\gets v]}\ruleno{LetVal}
$$}\\[-4mm]
\multicolumn{2}{p{14cm}}{
$$
\step{\mathcal S,\, \lstinline|let rec $\;f\;$ = $\;\lambda x.e_1\;$ in $\;e_2$|}{\mathcal S,\, e_2[f\gets\mu f\lambda x.e_1]}\ruleno{LetRec}
$$}\\[-4mm]
\multicolumn{2}{p{14cm}}{
$$
\step{\mathcal S,\,\lstinline|match $\;e\;$ with $\;\{p_i$->$e_i\}$|}{\lstinline|match $\;\Box\;$ with $\;\{p_i$->$e_i\}$|:\mathcal S,\, e}\ruleno{Match}
$$}\\[-4mm]
\multicolumn{2}{p{14cm}}{
$$
\step{\mathcal S,\,\lstinline|match $\;C_k^{n_k}(v_1,\dots,v_{n_k})\;$ with $\;\{C_i^{n_i}(x^i_1,\dots,x^i_{n_i})\to e_i\}$|}{\mathcal S,\,e_k[x^k_j\gets v_j]}\ruleno{MatchVal}
$$}
\end{tabular}
\egroup
\caption{Semantics for the source language}
\label{functional_semantics}
\end{figure}

We describe the semantics of our language in the form of transition system. The transition relation

$$
\step{\mathcal S,\,e}{\mathcal S^\prime,\,e^\prime}
$$

\noindent describes a one step of evaluation of expression $e$ with a stack of contexts $\mathcal S$, which results in
a new stack $\mathcal S^\prime$ and a new expression $e^\prime$. A context is an expression with a unique hole; informally speaking, 
a stack of contexts describes a path in the expression being evaluated from the topmost construct to the point, where the evaluation 
currently is taking place. For a context $C$ and an expression $e$, we denote by $C[e]$ a complete expression with no holes, which is 
obtained by plugging $e$ into the unique hole of $C$. From each state $\inbr{C_1:C_2:\dots:C_k,e}$ we can build an 
expression $C_k[\dots[C_2[C_1[e]]]\dots]$, which represents an intermediate result of evaluation according to a small-step semantics. 
This form of semantic description originates from Felleisen-style~\cite{Felleisen} approach for small-step semantics, and we've
chosen it since it can be naturally extended for a relational case.

Our semantics describes call-by-value left-to-right evaluation; in the rules $\textsc{Beta}$, $\textsc{Mu}$, $\textsc{LetVal}$,
$\textsc{LetRec}$ and $\textsc{MatchVal}$, we perform capture-avoiding substitutions, which respect the names in abstractions and let-bindings.
In the rule $\textsc{MatchVal}$ we assume, that at most one pattern matches the scrutinee~--- this is an important difference from the usual 
semantics of pattern matching, when the patterns are examined in a top-down manner until the matching succeeds. In the rules $\textsc{EqTrue}$
and $\textsc{EqFalse}$ we assume, that the values $v$, $v_1$, $v_2$ do not have the forms $\lambda x\dots$ or $\mu f\dots$.

Finally, for a closed expression $e$ and a value $v$, we write $e \leadsto^f v$, iff 

$$\inbr{\epsilon,e}\to^*\inbr{\epsilon,v}$$

\noindent where $\epsilon$~--- an empty stack, and ``$\to^*$'' is a reflexive-transitive closure of ``$\to$''. 


\subsection{Relational Extension}
\label{relational_extension}

The relational extension adds five conventional miniKanren expressions for constructing goals; the syntax is shown on Fig.~\ref{relational_syntax}.
Since relational constructs are added to regular functional ones, it becomes possible to construct expressions like \lstinline|fun x.(x /\ fun y.y)|, etc.
In order to rule such pathological expressions out, we devised an extension for the type system of the source language. In fact, this approach follows the
actual implementation for OCaml, where a careful choice of types for representing terms and goals made it possible to reject the majority of non-well-formed
programs at compile-time.

Our extension for the type system introduces one interpreted datatype constructor $\Box^o$ with one data constructor $\uparrow$~--- a polymorphic type and
a constructor for logical terms. In addition, we introduce an interpreted type of goals $\G$, which is distinct from all other types. The typing rules for the relational 
extension are shown on Fig.~\ref{relational_typing}. These rules describe rather expected typing: in unification and disequality constraints only
terms of the same logical type can be used, and conjunction and disjunction can only be taken for goals. Note, in our extension a term can be calculated as
a result of arbitrary expression in initial functional language (as long as this expression has expected logical type), but such ``higher-order'' terms will
never appear as a result of relational conversion, so, in fact, relational extension we describe here defines a richer language, than we actually need.

The semantics of extended language is shown on Fig.~\ref{relational_semantics}. First, the state is extended: besides the stack of contexts and
current expression it now contains a set of used \emph{semantic variables} $\Sigma$ and a \emph{logical state} $\sigma$. 
Semantic variables are allocated and substituted for syntactic logic variable occurrences, when \lstinline|fresh| expression is evaluated 
(see rule \textsc{Fresh}). Logical states are affected, when unification or disequality constraint is evaluated; we explain them
in details below. All existing rules for the initial language are considered rewritten to propagate newly added components of states unchanged.
Then, we modify the substitution to respect names, bound in \lstinline|fresh| as well. 
Next, we consider two new kinds of values: a semantic variable and a special value \lstinline|success|. The former is a result of evaluation for
a free logic variable, the latter~--- the result of evaluation for a succeeded goal.

\begin{wrapfigure}[11]{r}{0.5\textwidth}  
  \centering
  \vspace{-11pt}
  $$
  \begin{array}{rl}
    \mathcal E\mathrel{{+}{=}}&\lstinline|fresh ($x$) $\;e$|\\
    &e_1\equiv e_2\\
    &e_1\not\equiv e_2\\
    &e_1\vee e_2\\
    &e_1\wedge e_2
  \end{array}
  $$
  \caption{The syntax of relational extension}
  \label{relational_syntax}
\end{wrapfigure}

We also extend the definition of context to handle the new kinds of expressions. In unification and disequality constraint, the terms are evaluated left-to-right.
Conjunction and disjunction, however, evaluate nondeterministically: in disjunction only one subgoal is chosen (see rules \textsc{DisjL} and \textsc{DisjR}),
a conjunction can evaluate either left, or right subgoal first (see rules \textsc{ConjStartL} and \textsc{ConjStartR}). When chosen subgoal is evaluated
to the value \lstinline|success|, the other subgoal starts its evaluation (rules \textsc{ConjL} and \textsc{ConjR}).
We have chosen a nondeterministic variant for the semantics, since different existing miniKanren implementations use (a little bit) different search, and we do 
not want to depend on the implementation details. An opposite side of this solution is that for a concrete program and a concrete miniKanren implementation,
the result of the evaluation might not coincide with that, prescribed by the semantics: in some concrete implementation a program can diverge, while
nondeterministic semantics may still define a certain scenario to complete with a result. We argue, that in this case, it will always be possible to
rewrite a program or/and interpreter to converge according to that scenario.
\FloatBarrier

\setarrow{:}
\begin{figure}[t]
\centering
{\bf Types:}
$$
\begin{array}{rcl}
 \mathcal L &=               &\alpha^o \mid (T^n(l_1,\dots,l_n))^o\;\;\mbox{\supp{(logical types)}}\\
 \mathcal T &\mathrel{{+}{=}}&\G
\end{array}
$$
{\bf Typing rules:}
\def\arraystretch{0}
\begin{tabular}{p{7cm}p{7cm}}
\multicolumn{2}{p{14cm}}{
$$
\trule{\typed{\Gamma,x:l}{e}{\G}}
      {\typed{\Gamma}{\lstinline|fresh ($x$) $\;e$|}{\G}}
\ruleno{Fresh$_T$}
$$}\\[-2mm]
$$
\trule{\typed{\Gamma}{e_1}{l}\;\;\;\;\typed{\Gamma}{e_2}{l}}
      {\typed{\Gamma}{e_1\equiv e_2}{\G}}
\ruleno{Unify$_T$}
$$&
$$
\trule{\typed{\Gamma}{e_1}{l}\;\;\;\;\typed{\Gamma}{e_2}{l}}
      {\typed{\Gamma}{e_1\not\equiv e_2}{\G}}
\ruleno{Disequality$_T$}
$$\\[-2mm]
$$
\trule{\typed{\Gamma}{e_1}{\G}\;\;\;\;\typed{\Gamma}{e_2}{\G}}
      {\typed{\Gamma}{e_1\wedge e_2}{\G}}
\ruleno{Conjunction$_T$}
$$&
$$
\trule{\typed{\Gamma}{e_1}{\G}\;\;\;\;\typed{\Gamma}{e_2}{\G}}
      {\typed{\Gamma}{e_1\vee e_2}{\G}}
\ruleno{Disjunction$_T$}
$$
\end{tabular}
\caption{Typing rules for the relational extension}
\label{relational_typing}
\end{figure}

\setarrow{\leadsto}
\def\arraystretch{0}
\begin{figure}[t]
\centering
{\bf Semantic variables:}\vspace{-2mm}
\begin{gather*}
\mathfrak S = \mathfrak s_1, \mathfrak s_2, \dots\\[-2mm]
\Sigma, \Sigma^\prime\dots \subset 2^{\mathcal S}\;\mbox{\supp{(sets of allocated semantics variables)}}\\[-1mm]
\inbr{\Sigma^\prime, \mathfrak s}\gets\lstinline|new|\;\Sigma,\;\Sigma^\prime=\Sigma\cup\{\mathfrak s\},\;{\mathfrak s}\notin\Sigma\;\mbox{\supp{(allocation of a new semantic variable)}}\vspace{-2mm}
\end{gather*}
{\bf Values:}\vspace{-2mm}
$$
\mathcal V \mathrel{{+}{=}} \lstinline|success|\mid\mathfrak s
$$\vspace{-2mm}
{\bf Contexts:}\vspace{-2mm}
$$
\mathcal C \mathrel{{+}{=}}\Box\equiv e\mid v\equiv\Box\mid\Box\not\equiv e\mid v\not\equiv\Box\mid\Box\wedge e\mid e\wedge\Box
$$\vspace{-2mm}
{\bf States:}\vspace{-2mm}
\begin{gather*}
\inbr{\Sigma,\mathcal S,e,\sigma}\mbox{\supp{(set of allocated semantic variables, stack of contexts, expression, logical state)}}\\[-1mm]
\inbr{\emptyset,\epsilon,e,\iota}\mbox{\supp{(initial state)}}
\end{gather*}\vspace{-2mm}
{\bf Transitions:}\vspace{1mm}
{\def\arraystretch{0}
\begin{tabular}{p{14cm}}
$$
\step{\Sigma,\,\mathcal S,\,\lstinline|fresh($x$) $\;e$|,\,\sigma}{\Sigma^\prime,\,\mathcal S,\,e[x\gets\mathfrak s],\,\sigma},\,\inbr{\Sigma^\prime,\mathfrak s}\gets\lstinline|new|\;\Sigma\ruleno{Fresh}
$$\\[-4mm]
$$
\step{\Sigma,\,\mathcal S,\,e_1\equiv e_2,\,\sigma}{\Sigma,\,\Box\equiv e_2:\mathcal S,\,e_1,\,\sigma}\ruleno{UnifyL}
$$\\[-4mm]
$$
\step{\Sigma,\,\mathcal S,\,v\equiv e,\,\sigma}{\Sigma,\,v\equiv\Box:\mathcal S,\,e,\,\sigma}\ruleno{UnifyR}
$$\\[-4mm]
$$
\step{\Sigma,\,\mathcal S,\,v_1\equiv v_2,\,\sigma}{\Sigma,\,\mathcal S,\,\lstinline|success|,\,\sigma^\prime},\,{\bf unify}\,(\sigma,\,v_1,\,v_2)=\sigma^\prime\ruleno{Unify}
$$\\[-4mm]
$$
\step{\Sigma,\,\mathcal S,\,e_1\not\equiv e_2,\,\sigma}{\Sigma,\,\Box\not\equiv e_2:\mathcal S,\,e_1,\,\sigma}\ruleno{DisEqL}
$$\\[-4mm]
$$
\step{\Sigma,\,\mathcal S,\,v\not\equiv e,\,\sigma}{\Sigma,\,v\not\equiv\Box:\mathcal S,\,e,\,\sigma}\ruleno{DisEqR}
$$\\[-4mm]
$$
\step{\Sigma,\,\mathcal S,\,v_1\not\equiv v_2,\,\sigma}{\Sigma,\,\mathcal S,\,\lstinline|success|,\,\sigma^\prime},\,{\bf diseq}\,(\sigma,\,v_1,\,v_2)=\sigma^\prime\ruleno{DisEq}
$$\\[-4mm]
$$
\step{\Sigma,\,\mathcal S,\,e_1\vee e_2,\,\sigma}{\Sigma,\,\mathcal S,\,e_1,\,\sigma}\ruleno{DisjL}
$$\\[-4mm]
$$
\step{\Sigma,\,\mathcal S,\,e_1\vee e_2,\,\sigma}{\Sigma,\,\mathcal S,\,e_2,\,\sigma}\ruleno{DisjR}
$$\\[-4mm]
$$
\step{\Sigma,\,\mathcal S,\,e_1\wedge e_2,\,\sigma}{\Sigma,\,\Box\wedge e_2:\mathcal S,\,e_1,\,\sigma}\ruleno{ConjStartL}
$$\\[-4mm]
$$
\step{\Sigma,\,\mathcal S,\,e_1\wedge e_2,\,\sigma}{\Sigma,\,e_1\wedge\Box:\mathcal S,\,e_2,\,\sigma}\ruleno{ConjStartR}
$$\\[-4mm]
$$
\step{\Sigma,\,\mathcal S,\,\lstinline|success|\wedge e,\,\sigma}{\Sigma,\,\mathcal S,\,e,\,\sigma}\ruleno{ConjL}
$$\\[-4mm]
$$
\step{\Sigma,\,\mathcal S,\,e\wedge\lstinline|success|,\,\sigma}{\Sigma,\,\mathcal S,\,e,\,\sigma}\ruleno{ConjR}
$$
\end{tabular}}
\caption{Semantics for the relational extension}
\label{relational_semantics}
\end{figure}

Finally, we describe the structure of a logical state and the implementation of unification and disequality constraint. The development is mainly based on the existing implementation~\cite{CKanren} and standard approaches for implementing unification~\cite{Unification,UnificationRevisited}. We, therefore, assume the familiarity of the reader with the following notions:

\begin{itemize}
  \item substitution ($\theta$);
  \item application of substitution $\theta$ to a term $t$ ($t\,\theta$);
  \item composition of substitutions ($\theta\theta^\prime$);
  \item most general unifier of two terms ($mgu\,(t_1, t_2)$).
\end{itemize}

\begin{comment}

As it can be seen from the semantics and typing rules, a unification (or disequality constraint) can only
be applied to equally-typed logical values, and we consider substitutions to be partial functions from
semantic variables ($\mathfrak S$) to logical values.

Before giving the detailed description, we consider the following example, which is called to reveal the essence of unification and
disequality constraint coexistence. Note, only these two kinds of goals deliver new information~--- all other kinds (conjunction, disjunction, etc.) are rather needed to
put them in a right context or order. Let us have the following sequence of goals, which we have to evaluate one after another (we assume all logical 
variables are already properly allocated):

$$
\def\arraystretch{1}
\begin{array}{rcl}
x&\not\equiv&y\\
z&\not\equiv&y\\
y&\equiv&\lstinline|A|\\
x&\equiv&\lstinline|B|\\
z&\equiv&\lstinline|A|
\end{array}
$$

Since in the beginning we know nothing yet, we are unable to determine, if the first goal~--- \mbox{$x\not\equiv y$}~--- succeeds or fails right now. All we can do is to
remember the substitution \mbox{$[x\binds y]$} with the note, that we \emph{do not want} to allow it. We call this kind of substitutions \emph{negative}. Similarly,
after the next goal we have another negative substitution \mbox{$[z\binds y]$}. 

The next goal is the unification \mbox{$y\equiv\lstinline|A|$}. Obviously, it provides us with the substitution \mbox{$[y\binds\lstinline|A|]$}. What effect (if any)
should it have on the set of negative substitutions? Since the unification completely eliminates the variable $y$, replacing negative substitutions with
\mbox{$[x\binds\lstinline|A|]$}, \mbox{$[z\binds\lstinline|A|]$} looks reasonable.

The next goal is \mbox{$x\equiv\lstinline|B|$}. It extends the current substitution, making it \mbox{$[y\binds\lstinline|A|,\,x\binds\lstinline|B|]$}. Now we may
note, that since $x$ already became \lstinline|B|, it can never be \lstinline|A| anymore. Thus, we can drop \mbox{$[x\binds\lstinline|A|]$} from the set of
negative substitutions.

Finally, we unify $z$ with \lstinline|A|. This, however, gets us a substitution \mbox{$[z\binds\lstinline|A|]$}, which we have to disallow since it
matches with the negative one.

In other words, disequality constraints may not succeed or fail right away, and the results of unification have to be additionally matched against the set of 
previously encountered disequality constraints, which, in turn, can be altered by a unification or another disequality constraint. Now we can continue.
\end{comment}

A logical state contains two components

$$
\sigma=(\theta,\Theta^-)
$$

\noindent where $\theta$ is a substitution, $\Theta^-$~--- a set of negative substitutions, describing disequality constraints, 
which can potentially be violated. The initial state contains undefined substitution and empty set:

$$
\iota=(\bot,\emptyset)
$$

The effect of unification is described by the following primitive:

$$
{\bf unify}\,(\sigma,\,t_1,\,t_2)={\bf unify}\,((\theta,\Theta^-),\,t_1,\,t_2)
$$

First, it calculates the most general unifier for the terms under consideration w.r.t. current substitution:

$$
\rho=mgu\,(t_1\,\theta,t_2\,\theta)
$$

If there is no such $\rho$, the unification fails, and the evaluation terminates unsuccessfully. Otherwise,
$\rho$ has to be checked against the disequality constraints, represented by $\Theta^-$ (if $\Theta^-$ is empty, the
check succeeds immediately).

Being a substitution, $\rho$ at the same time can be considered as the following unification problem: we can try to unify a pair of terms 

$$
\begin{array}{rcl}
t_l&=&(\mathfrak s_1,\dots,\mathfrak s_k)\\
t_r&=&(\rho(\mathfrak s_1),\dots,\rho(\mathfrak s_k))
\end{array}
$$

\noindent where $\{\mathfrak s_i\}=dom\,(\rho)$. We pick every substitution $\theta^-\in\Theta^-$ and calculate 
the $mgu\,(t_l\,\theta^-,t_r\,\theta^-)$. There are three possible outcomes:

\begin{enumerate}
\item The unification fails. This means, that disequality constraint, represented by $\theta^-$, can no
longer be violated. We remove $\theta^-$ from $\Theta^-$ and continue with the next disequality constraint.
\item The unification succeeds with the empty substitution. This means, that 
disequality constraint, represented by $\theta^-$, is violated. The check stops, and the whole top-level 
unification fails.
\item The unification succeeds with a non-empty substitution $\theta^{\prime-}$. This means, that in order not to 
violate disequality constraint, represented by $\theta^-$, $\theta^{\prime-}$ has to be respected. We replace
$\theta^-$ with $\theta^{\prime-}$ in $\Theta^-$ and continue with the next disequality constraint.
\end{enumerate}

If the disequality check succeeds, by the end we have a modified set $\Theta^{\prime-}$, and we assume

$$
{\bf unify}\,((\theta,\Theta^-),\,t_1,\,t_2)=(\theta\rho,\Theta^{\prime-})
$$

The evaluation of disequality constraint is performed in a similar manner using the primitive

$$
{\bf diseq}\,(\sigma,\,t_1,\,t_2)={\bf diseq}\,((\theta,\Theta^-),\,t_1,\,t_2)
$$

First, the $mgu\,(t_1\,\theta,t_2\,\theta)$ is calculated. Again, there are three
possible cases:

\begin{enumerate}
\item The unification fails. This means, that disequality constraint is satisfied.
\item The unification succeeds with the empty substitution. This means, that disequality
constraint is violated.
\item The unification succeeds with a non-empty substitution $\theta^{\prime-}$. This means, that 
this substitution describes the disequality constraint, which has to be respected in
the future, so we add it to $\Theta^-$. 
\end{enumerate}

If disequality constraint succeeds, we obtain a (potentially) modified set $\Theta^{\prime-}$, and we
assume

$$
{\bf diseq}\,((\theta,\Theta^-),\,t_1,\,t_2)=(\theta,\Theta^{\prime-})
$$

Finally, for a closed goal $g$ and a logical state $\sigma$, we define $g \leadsto^r \sigma$, iff

$$
\inbr{\emptyset,\epsilon,g,\iota}\leadsto^*\inbr{\Sigma,\epsilon,\lstinline|success|,\sigma}\mbox{ for some $\Sigma$}
$$
 
\noindent where ``$\leadsto^*$'' is a reflexive-transitive closure of ``$\leadsto$''. 

One may notice, that the typing rules for the relational extension add nothing more than some
interpreted types and symbols w.r.t. the type system of the substrate language. Thus, it 
is rather expected, that the relational extension inherits all its useful properties (like progress and
type preservation). Surprisingly, this is not completely so. Indeed, the only value for goals is
\lstinline|success|, but, obviously, not every goal succeeds (for example, \lstinline|A === B| always
fails). Thus, our relational extension lacks the progress property~--- a decently typed non-value
goal sometimes cannot make a step. This makes no harm in the context of the paper; in any case,
a failure value for goals can be added to the language together with the failure propagation rules. 




\begin{figure}[t]
\[
\begin{array}{rcll}
  x\,[t/x] &=& t &\\
  y\,[t/x] &=& y & y\ne x\\
  C_i^{k_i}\,(t_1,\dots,t_{k_i})\,[t/x]&=&C_i^{k_i}\,(t_1\,[t/x],\dots,t_{k_i}\,[t/x])&\\[2mm]
  (t_1 \equiv t_2)\,[t/x]&=&t_1\,[t/x] \equiv t_2\,[t/x]&\\
  (g_1 \wedge g_2)\,[t/x]&=&g_1\,[t/x] \wedge g_2\,[t/x]&\\
  (g_1 \vee g_2)\,[t/x]&=&g_1\,[t/x] \vee g_2\,[t/x]&\\
  (\mbox{\lstinline|fresh|}\;x\,.\,g)\,[t/x]&=&\mbox{\lstinline|fresh|}\;x\,.\,g&\\
  (\mbox{\lstinline|fresh|}\;y\,.\,g)\,[t/x]&=&\mbox{\lstinline|fresh|}\;y\,.\,(g\,[t/x])&y\ne x\\
  (R_i^{k_i}\,(t_1,\dots,t_{k_i})\,[t/x]&=&R_i^{k_i}\,(t_1\,[t/x],\dots,t_{k_i}\,[t/x])&
\end{array}
\]
  \caption{Substitutions for terms and goals}
  \label{substitution}
\end{figure}

\section{Denotational Semantics}
\label{denotational}

In this section we present a denotational semantics for the language we defined above. We use a simple set-theoretic
approach which can be considered as an analogy to the least Herbrand model for definite logic programs~\cite{LHM}.
Strictly speaking, instead of developing it from scratch we could have just described the conversion of specifications
into definite logic form and took their least Herbrand model. However, in that case we would still need to define
the least Herbrand model semantics for definite logic programs in a certified way. In addition, while for
this concrete language the conversion to definite logic form is trivial, it may become less trivial for
its extensions (with, for examples, nominal constructs~\cite{AlphaKanren}) which we plan to do in future.

To motivate further development, we first consider the following example. Let us have the following goal:

\begin{lstlisting}
   x === Cons (y, z)
\end{lstlisting}

There are three free variables, and solving the goal delivers us the following single answer:

\begin{lstlisting}
   $\alpha\mapsto\;$ Cons ($\beta$, $\gamma$)
\end{lstlisting}

where semantic variables $\alpha$, $\beta$ and $\gamma$ correspond to the syntactic ones ``\lstinline|x|'', ``\lstinline|y|'', ``\lstinline|z|''. The
goal does not put any constraints on ``\lstinline|y|'' and ``\lstinline|z|'', so there are no bindings for ``$\beta$'' and ``$\gamma$'' in the answer.
This answer can be seen as the following ternary relation over the set of all ground terms:

\[
\{(\mbox{\lstinline|Cons ($\beta$, $\,\gamma$)|}, \beta, \gamma) \mid \beta\in\mathcal{D},\,\gamma\in\mathcal{D}\}\subset\mathcal{D}^3
\]

The order of ``dimensions'' is important, since each dimension corresponds to a certain free variable. Our main idea is to represent this relation as a set of total
functions 

\[
\mathfrak{f}:\mathcal{A}\mapsto\mathcal{D}
\]

from semantic variables to ground terms. We call these functions \emph{representing functions}. Thus, we may reformulate the same relation as

\[
\{(\mathfrak{f}\,(\alpha),\mathfrak{f}\,(\beta),\mathfrak{f}\,(\gamma))\mid\mathfrak{f}\in\sembr{\mbox{\lstinline|$\alpha$ === Cons ($\beta$, $\,\gamma$)|}}\}
\]

where we use conventional semantic brackets ``$\sembr{\bullet}$'' to denote the semantics. For the top-level goal we need to substitute its free syntactic
variables with distinct semantic ones, calculate the semantics, and build the explicit representation for the relation as shown above. The relation, obviously,
does not depend on concrete choice of semantic variables, but depends on the order in which the values of representing functions are tupled. This order can be
conventionalized, which gives us a completely deterministic semantics.

Now we implement this idea. First, for a representing function

\[
\mathfrak{f} : \mathcal{A}\to\mathcal{D}
\]

we introduce its homomorphic extension 

\[
  \overline{\mathfrak{f}}:\mathcal{T_A}\to\mathcal{D}
\]

which maps terms to terms:

\[
\begin{array}{rcl}

  \overline{\mathfrak f}\,(\alpha) & = & \mathfrak f\,(\alpha)\\
  \overline{\mathfrak f}\,(C_i^{k_i}\,(t_1,\dots.t_{k_i})) & = & C_i^{k_i}\,(\overline{\mathfrak f}\,(t_1),\dots \overline{\mathfrak f}\,(t_{k_i}))
\end{array}
\]

Let us have two terms $t_1, t_2\in\mathcal{T_A}$. If there is a unifier for $t_1$ and $t_2$ then, clearly, there is a substitution $\theta$ which
turns both $t_1$ and $t_2$ into the same \emph{ground} term (we do not require $\theta$ to be the most general). Thus, $\theta$ maps
(some) ground variables into ground terms, and its application to $t_{1(2)}$ is exactly $\overline{\theta}(t_{1(2)})$. This reasoning can be
performed in the opposite direction: a unification $t_1\equiv t_2$ defines the set of all representing functions $\mathfrak{f}$ for which
$\overline{\mathfrak{f}}(t_1)=\overline{\mathfrak{f}}(t_2)$. 

Then, the semantic function for goals is parameterized over environments which prescribe semantic functions to relational symbols:

\[
  \Gamma : \mathcal{R} \to (\mathcal{T_A}^*\to 2^{\mathcal{A}\to\mathcal{D}})
\]

An environment associates with relational symbol a function which takes a string of terms (the arguments of the relation) and returns a set of
representing functions. The signature for semantic brackets for goals is as follows:

\[
\sembr{\bullet}_{\Gamma} : \mathcal{G}\to 2^{\mathcal{A}\to\mathcal{D}}
\]

It maps a goal into the set of representing functions w.r.t. an environment $\Gamma$.

We formulate the following important \emph{completeness condition} for the semantics of a goal $g$:

\[
\forall\alpha\not\in FV\,(g)\; \forall d \in \mathcal{D}\; \forall\mathfrak{f} \in \sembr{g}\; \exists \mathfrak{f'} \in \sembr{g} \;:\; \mathfrak{f'}\,(\alpha)\; = d \wedge \forall \beta \neq \alpha:\; \mathfrak{f'}\,(\beta)\; = \mathfrak{f}\,(\beta)\; 
\]

In other words, representing functions for a goal $g$ restrict only the values of free variables of $g$ and do not introduce any ``hidden'' correlations.
This condition guarantees that our semantics is complete in the sense that it does not introduce artificial restrictions for the relation it defines. It
can be proven that the semantics of goals always satisfy this condition.

We remind conventional notions of pointwise modification of a function

\[
f\,[x\gets v]\,(z)=\left\{
\begin{array}{rcl}
  f\,(z) &,& z \ne x \\
  v      &,& z = x
\end{array}
\right.
\]

and substitution of a free variable with a term in terms and goals (see Figure~\ref{substitution}).

For a representing function $\mathfrak{f}:\mathcal{A}\to\mathcal{D}$ and a semantic variable $\alpha$ we define
the following \emph{generalization} operation:

\[
\mathfrak{f}\uparrow\alpha = \{ \mathfrak{f}\,[\alpha\gets d] \mid d\in\mathcal D\}
\]

Informally, this operation generalizes a representing function into a set of representing functions in such a way that the
values of these functions for a given variable cover the whole $\mathcal{D}$. We extend the generalization operation for sets of
representing functions $\mathfrak{F}\subseteq\mathcal{A}\to\mathcal{D}$:

\[
  \mathfrak{F}\uparrow\alpha = \bigcup_{\mathfrak{f}\in\mathfrak{F}}(\mathfrak{f}\uparrow\alpha)
\]

Now we are ready to specify the semantics for goals (see Figure~\ref{denotational_semantics_of_goals}). We've already given the motivation for
the semantics of unification: the condition $\overline{\mathfrak{f}}(t_1)=\overline{\mathfrak{f}}(t_2)$ gives us the set of all (otherwise
unrestricted) representing functions which ``equate'' terms $t_1$ and $t_2$. Set union and intersection provide a conventional interpretation
for disjunction and conjunction of goals, and the semantics of relational invocation reduces to the application of corresponding
function from the environment. The only interesting case is ``\lstinline|fresh $x$ . $g$|''. First, we take an arbitrary semantic variable $\alpha$,
not free in $g$, and substitute $x$ with $\alpha$. Then we calculate the semantics of $g\,[\alpha/x]$. The interesting part is the next step:
as $x$ can not be free in ``\lstinline|fresh $x$ . $g$|'', we need to generalize the result over $\alpha$ since in our model the semantics of a
goal specifies a relation over its free variables. We introduce some nondeterminism, by choosing arbitrary $\alpha$, but it can be proven by structural induction, that with different choices of free variable, semantics of a goal won't change. Consider the following example:

\begin{lstlisting}
   fresh y . ($\alpha$ ===  y) /\ (y === Zero)
\end{lstlisting}

As there is no invocations involved, we can safely omit the environment. Then:

\[
\begin{array}{lcr}
  \sembr{\mbox{\lstinline|fresh y . ($\alpha$ === y) $\,\wedge\,$ (y === Zero)|}}&=&\mbox{(by \textsc{Fresh$_D$})}\\[1mm]
  (\sembr{\mbox{\lstinline|($\alpha$ === $\beta$) $\,\wedge\,$ ($\beta$ === Zero)|}})\uparrow\beta&=&\mbox{(by \textsc{Conj$_D$})}\\[1mm]
  (\sembr{\mbox{\lstinline|$\alpha$ === $\beta$|}} \,\cap\, \sembr{\mbox{\lstinline|$\beta$ === Zero)|}})\uparrow\beta&=&\mbox{(by \textsc{Unify$_D$})}\\[1mm]
  (\{\mathfrak{f}\mid \overline{\mathfrak{f}}\,(\alpha)=\overline{\mathfrak{f}}\,(\beta)\} \,\cap\, \{\mathfrak{f}\mid \overline{\mathfrak{f}}\,(\beta)=\overline{\mathfrak{f}}\,(\mbox{\lstinline|Zero|})\})\uparrow\beta&=&\mbox{(by the definition of ``$\overline{\mathfrak{f}}$'')}\\[1mm]
  (\{\mathfrak{f}\mid \mathfrak{f}\,(\alpha)=\mathfrak{f}\,(\beta)\} \,\cap\, \{\mathfrak{f}\mid \mathfrak{f}\,(\beta)=\mbox{\lstinline|Zero|}\})\uparrow\beta&=&\mbox{(by the definition of ``$\cap$'')}\\[1mm]
  (\{\mathfrak{f}\mid \mathfrak{f}\,(\alpha)=\mathfrak{f}\,(\beta)=\mbox{\lstinline|Zero|}\})\uparrow\beta&=&\mbox{(by the definition of ``$\uparrow$'')}\\[1mm]
  \{\mathfrak{f}\mid \mathfrak{f}\,(\alpha)=\mbox{\lstinline|Zero|}, \mathfrak{f}\,(\beta)=d, d\in\mathcal{D}\}&=&\mbox{(by the totality of representing functions)}\\[1mm]
  \{\mathfrak{f}\mid \mathfrak{f}\,(\alpha)=\mbox{\lstinline|Zero|}\}&&
\end{array}
\]

In the end we've got the set of representing functions, each of which restricts only the value of free variable $\alpha$. 

\begin{figure}[t]
  \[
  \begin{array}{cclr}
    \sembr{t_1\equiv t_2}_\Gamma&=&\{\mathfrak f : \mathcal{A}\to\mathcal{D}\mid \overline{\mathfrak{f}}\,(t_1)=\overline{\mathfrak{f}}\,(t_2)\}& \ruleno{Unify$_D$}\\
    \sembr{g_1\wedge g_2}_\Gamma&=&\sembr{g_1}_\Gamma\cap\sembr{g_1}_\Gamma&\ruleno{Conj$_D$}\\
    \sembr{g_1\vee g_2}_\Gamma&=&\sembr{g_1}_\Gamma\cup\sembr{g_1}_\Gamma&\ruleno{Disj$_D$}\\
    \sembr{\mbox{\lstinline|fresh|}\,x\,.\,g}_\Gamma&=&(\sembr{g\,[\alpha/x]}_\Gamma)\uparrow\alpha,\;\alpha\not\in FV(g)& \ruleno{Fresh$_D$}\\
    \sembr{R\,(t_1,\dots,t_k)}_\Gamma&=&(\Gamma\,R)\,t_1\dots t_k & \ruleno{Invoke$_D$}
  \end{array}
  \]
  \caption{Denotational semantics of goals}
  \label{denotational_semantics_of_goals}
\end{figure}

The final component is the semantics of specifications. Given a specification

\[
\{R_i=\lambda\,x_1^i\dots x_{k_i}^i\,.\,g_i;\}_{i=1}^n\;g
\]

we have to construct a correct environment $\Gamma_0$ and then take the semantics of the top-level goal:

\[
\sembr{\{R_i=\lambda\,x_1^i\dots x_{k_i}^i\,.\,g_i;\}_{i=1}^n\;g}=\sembr{g}_{\Gamma_0}
\]

As the set of definitions can be mutually recursive we apply the fixed point approach. We consider the following
function

\[
\mathcal{F} : (\mathcal{T_A}^*\to 2^{\mathcal{A}\to\mathcal{D}})^n\to (\mathcal{T_A}^*\to 2^{\mathcal{A}\to\mathcal{D}})^n
\]

which represents a semantic for the set of definitions abstracted over themselves. The definition of this function is
rather standard:

\begin{gather*}
    \begin{array}{rcl}
      \mathcal{F}\,(p_1,\dots,p_n)& = &(t^1_1\dots t^1_{k_1}\mapsto\sembr{g^1\,[t^1_1/x^1_1,\dots,t^1_{k_1}/x^1_{k_1}]}_\Gamma,\\
                                  &  &\phantom{(}\dots\\
                                  &  &\phantom{(}t^n_1\dots t^n_{k_n}\mapsto\sembr{g^n\,[t^n_1/x^n_1,\dots,t^n_{k_n}/x^n_{k_n}]}_\Gamma)
    \end{array}\\
    \mbox{where}\;\Gamma\, R_i=p_i
\end{gather*}

Here $p_i$ is a semantic function for $i$-th definition; we build an environment $\Gamma$ which associates each relational symbol
$R_i$ with $p_i$ and construct a $n$-dimensional vector-function, where $i$-th component corresponds to a function which
calculates the semantics of $i$-th relational definition application to terms w.r.t. the environment $\Gamma$. Finally,
we take the least fixed point of $\mathcal{F}$ and define the top-level environment as follows:

\[
\Gamma_0\,R_i=(fix\;\mathcal{F})\,[i]
\]

where ``$[i]$'' denotes the $i$-th component of a vector-function.

The least fixed point exists by Knaster-Tarski~\cite{TarskiKnaster} theorem~--- the set $(\mathcal{T_A}^*\to 2^{\mathcal{A}\to\mathcal{D}})^n$
forms a complete lattice, and $\mathcal{F}$ is monotonic. 

To formalize denotational semantics in \textsc{Coq} we can define representing functions simply as \textsc{Coq} functions:

\begin{lstlisting}[language=Coq]
   Definition repr_fun : Set := var -> ground_term.
\end{lstlisting}

We define the semantics via inductive proposition ``\lstinline|in_denotational_sem_goal|'' such that

\[
\forall g,\mathfrak{f}\;:\;\mbox{\lstinline|in_denotational_sem_goal|}\;g\;\mathfrak{f}\Longleftrightarrow\mathfrak{f}\in\sembr{g}_\Gamma
\]

The definition is as follows:

\begin{lstlisting}[language=Coq]
   Inductive in_denotational_sem_goal : goal -> repr_fun -> Prop :=
   | dsgUnify  : forall f t1 t2, apply_repr_fun f t1 = apply_repr_fun f t2 ->
                            in_denotational_sem_goal (Unify t1 t2) f

   | dsgDisjL  : forall f g1 g2, in_denotational_sem_goal g1 f ->
                            in_denotational_sem_goal (Disj g1 g2) f

   | dsgDisjR  : forall f g1 g2, in_denotational_sem_goal g2 f ->
                            in_denotational_sem_goal (Disj g1 g2) f

   | dsgConj   : forall f g1 g2, in_denotational_sem_goal g1 f ->
                            in_denotational_sem_goal g2 f ->
                            in_denotational_sem_goal (Conj g1 g2) f

   | dsgFresh  : forall f fn a fg, (~ is_fv_of_goal a (Fresh fg)) ->
                              in_denotational_sem_goal (fg a) fn ->
                              (forall x, x <> a -> fn x = f x) ->
                              in_denotational_sem_goal (Fresh fg) f

   | dsgInvoke : forall r t f, in_denotational_sem_goal (proj1_sig (Prog r) t) f ->
                          in_denotational_sem_goal (Invoke r t) f.
\end{lstlisting}

Here we refer to a fixpoint ``\lstinline[language=Coq]|apply_repr_fun|'' which calculates the extension ``$\overline{\bullet}$'' for a representing
function, and inductive proposition ``\lstinline[language=Coq]|is_fv_of_goal|'' which encodes the set of free variables for a goal.

Recall that the environment ``\lstinline[language=Coq]|Prog|'' maps every relational symbol to the definition of relation,
which is a pair of a function from terms to goals and a proof that it has no unbound variables.
So in the last case ``\lstinline[language=Coq]|(proj1_sig (Prog r) t)|'' simply takes the body of the corresponding relation;
thus ``\lstinline[language=Coq]|Prog|'' in \textsc{Coq} specification plays role of a global environment $\Gamma$.

It is interesting that in \textsc{Coq} implementation we do not need to refer to Tarski-Knaster theorem explicitly since
the least fixpoint semantic is implicitly provided by inductive definitions.

\section{Operational Semantics}
\label{operational}

In this section we describe the operational semantics of \textsc{miniKanren}, which corresponds to the known
implementations with interleaving search. The semantics is given in the form of a labeled transition system (LTS)~\cite{LTS}. From now on we
assume the set of semantic variables to be linearly ordered ($\mathcal{A}=\{\alpha_1,\alpha_2,\dots\}$).

We introduce the notion of substitution

\[
  \sigma : \mathcal{A}\to\mathcal{T_A}
\]

\noindent as a (partial) mapping from semantic variables to terms over the set of semantic variables. We denote $\Sigma$ the
set of all substitutions, $\dom{\sigma}$~--- the domain for a substitution $\sigma$,
$\ran{\sigma}=\bigcup_{\alpha\in\mathcal{D}om\,(\sigma)}\fv{\sigma\,(\alpha)}$~--- its range (the set of all free variables in the image).

The \emph{non-terminal states} in the transition system have the following shape:

\[
S = \mathcal{G}\times\Sigma\times\mathbb{N}\mid S\oplus S \mid S \otimes \mathcal{G}
\]

As we will see later, an evaluation of a goal is separated into elementary steps, and these steps are performed interchangeably for different subgoals. 
Thus, a state has a tree-like structure with intermediate nodes corresponding to partially-evaluated conjunctions (``$\otimes$'') or
disjunctions (``$\oplus$''). A leaf in the form $\inbr{g, \sigma, n}$ determines a goal in a context, where $g$ is a goal, $\sigma$ is a substitution accumulated so far,
and $n$ is a natural number, which corresponds to a number of semantic variables used to this point. For a conjunction node, its right child is always a goal since
it cannot be evaluated unless some result is provided by the left conjunct.

The full set of states also include one separate terminal state (denoted by $\diamond$), which symbolizes the end of the evaluation.

\[
\hat{S} = \diamond \mid S
\]

We will operate with the well-formed states only, which are defined as follows.

\begin{definition}
  Well-formedness condition for extended states:
  
  \begin{itemize}
  \item $\diamond$ is well-formed;
  \item $\inbr{g, \sigma, n}$ is well-formed iff $\fv{g}\cup\dom{\sigma}\cup\ran{\sigma}\subseteq\{\alpha_1,\dots,\alpha_n\}$;
  \item $s_1\oplus s_2$ is well-formed iff $s_1$ and $s_2$ are well-formed;
  \item $s\otimes g$ is well-formed iff $s$ is well-formed and for all leaf triplets $\inbr{\_,\_,n}$ in $s$ it is true that $\fv{g}\subseteq\{\alpha_1,\dots,\alpha_n\}$.
  \end{itemize}
  
\end{definition}

Informally the well-formedness restricts the set of states to those in which all goals use only allocated variables.

Finally, we define the set of labels:

\[
L = \step \mid \Sigma\times \mathbb{N}
\]

The label ``$\step$'' is used to mark those steps which do not provide an answer; otherwise, a transition is labeled by a pair of a substitution and a number of allocated
variables. The substitution is one of the answers, and the number is threaded through the derivation to keep track of allocated variables.

\begin{figure*}[t]
  \renewcommand{\arraystretch}{1.6}
  \[
  \begin{array}{cr}
    \inbr{t_1 \equiv t_2, \sigma, n} \xrightarrow{\step} \Diamond , \, \, \nexists\; mgu\,(t_1 \sigma, t_2 \sigma) &\ruleno{UnifyFail} \\
    \inbr{t_1 \equiv t_2, \sigma, n} \xrightarrow{(mgu\,(t_1 \sigma, t_2 \sigma) \circ \sigma,\, n)} \Diamond & \ruleno{UnifySuccess} \\
    \inbr{g_1 \lor g_2, \sigma, n} \xrightarrow{\step} \inbr{g_1, \sigma, n} \oplus \inbr{g_2, \sigma, n} & \ruleno{Disj} \\
    \inbr{g_1 \land g_2, \sigma, n} \xrightarrow{\step} \inbr{ g_1, \sigma, n} \otimes g_2 & \ruleno{Conj} \\
    \inbr{\mbox{\lstinline|fresh|}\, x\, .\, g, \sigma, n} \xrightarrow{\step} \inbr{g\,[\bigslant{\alpha_{n + 1}}{x}], \sigma, n + 1} & \ruleno{Fresh} \\
    \dfrac{R_i^{k_i}=\lambda\,x_1\dots x_{k_i}\,.\,g}{\inbr{R_i^{k_i}\,(t_1,\dots,t_{k_i}),\sigma,n} \xrightarrow{\step} \inbr{g\,[\bigslant{t_1}{x_1}\dots\bigslant{t_{k_i}}{x_{k_i}}], \sigma, n}} & \ruleno{Invoke}\\
    \dfrac{s_1 \xrightarrow{\step} \Diamond}{(s_1 \oplus s_2) \xrightarrow{\step} s_2} & \ruleno{SumStop}\\
    \dfrac{s_1 \xrightarrow{r} \Diamond}{(s_1 \oplus s_2) \xrightarrow{r} s_2} & \ruleno{SumStopAns}\\
    \dfrac{s \xrightarrow{\step} \Diamond}{(s \otimes g) \xrightarrow{\step} \Diamond} &\ruleno{ProdStop}\\
    \dfrac{s \xrightarrow{(\sigma, n)} \Diamond}{(s \otimes g) \xrightarrow{\step} \inbr{g, \sigma, n}}  & \ruleno{ProdStopAns}\\
    \dfrac{s_1 \xrightarrow{\step} s'_1}{(s_1 \oplus s_2) \xrightarrow{\step} (s_2 \oplus s'_1)} &\ruleno{SumStep}\\
    \dfrac{s_1 \xrightarrow{r} s'_1}{(s_1 \oplus s_2) \xrightarrow{r} (s_2 \oplus s'_1)} &\ruleno{SumStepAns}\\
    \dfrac{s \xrightarrow{\step} s'}{(s \otimes g) \xrightarrow{\step} (s' \otimes g)} &\ruleno{ProdStep}\\
    \dfrac{s \xrightarrow{(\sigma, n)} s'}{(s \otimes g) \xrightarrow{\step} (\inbr{g, \sigma, n} \oplus (s' \otimes g))} & \ruleno{ProdStepAns} 
  \end{array}
  \]
  \caption{Operational semantics of interleaving search}
  \label{lts}
\end{figure*}

The transition rules are shown in Fig.~\ref{lts}. The first two rules specify the semantics of unification. If two terms are not unifiable under the current substitution
$\sigma$ then the evaluation stops with no answer; otherwise, it stops with the most general unifier applied to a current substitution as an answer.

The next two rules describe the steps performed when disjunction or conjunction is encountered on the top level of the current goal. For disjunction, it schedules both goals (using ``$\oplus$'') for
evaluating in the same context as the parent state, for conjunction~--- schedules the left goal and postpones the right one (using ``$\otimes$'').

The rule for ``\lstinline|fresh|'' substitutes bound syntactic variable with a newly allocated semantic one and proceeds with the goal.

The rule for relation invocation finds a corresponding definition, substitutes its formal parameters with the actual ones, and proceeds with the body.

The rest of the rules specify the steps performed during the evaluation of two remaining types of the states~--- conjunction and disjunction. In all cases, the left state
is evaluated first. If its evaluation stops, the disjunction evaluation proceeds with the right state, propagating the label (\textsc{SumStop} and \textsc{SumStep}), and the conjunction schedules the right goal for evaluation in the context of the returned answer (\textsc{ProdStopAns}) or stops if there is no answer (\textsc{ProdStop}).

The last four rules describe \emph{interleaving}, which occurs when the evaluation of the left state suspends with some residual state (with or without an answer). In the case of disjunction
the answer (if any) is propagated, and the constituents of the disjunction are swapped (\textsc{SumStep}, \textsc{SumStepAns}). In the case of conjunction, if the evaluation step in
the left conjunct did not provide any answer, the evaluation is continued in the same order since there is still no information to proceed with the evaluation of the right
conjunct (\textsc{ProdStep}); if there is some answer, then the disjunction of the right conjunct in the context of the answer and the remaining conjunction is
scheduled for evaluation (\textsc{ProdStepAns}).

The introduced transition system is completely deterministic: there is exactly one transition from any non-terminal state.
There is, however, some freedom in choosing the order of evaluation for conjunction and
disjunction states. For example, instead of evaluating the left substate first, we could choose to evaluate the right one, etc.
\begin{comment}
In each concrete case, we would
end up with a different (but still deterministic) system that would prescribe different semantics to a concrete goal.
\end{comment}
This choice reflects the inherent non-deterministic nature of search in relational (and, more generally, logical) programming.
Although we could introduce this ambiguity into the semantics (by replacing specific rules for disjunctions and conjunctions evaluation with some conditions on it), we want an operational semantics that would be easy to present and easy to employ to describe existing language extensions (already described for a specific implementation of interleaving search), so we instead base the semantics on one canonical search strategy.
At the same time, as long as deterministic search procedures are sound and complete, we can consider them ``equivalent''.\footnote{There still can be differences in observable behavior of concrete goals under different sound and complete search strategies.
For example, a goal can be refutationally complete~\cite{WillThesis} under one strategy and non-complete under another.}

It is easy to prove that transitions preserve well-formedness of states.

\begin{lemma}{(Well-formedness preservation)}
\label{lem:well_formedness_preservation}
For any transition $s \xrightarrow{l} \hat{s}$, if $s$ is well-formed then $\hat{s}$ is also well-formed.
\end{lemma}

A derivation sequence for a certain state determines a \emph{trace}~--- a finite or infinite sequence of answers. The trace corresponds to the stream of answers
in the reference \textsc{miniKanren} implementations. We denote a set of answers in the trace for state $\hat{s}$ by $\tr{\hat{s}}$.

We can relate sets of answers for the partially evaluated conjunction and disjunction with sets of answers for their constituents by the two following lemmas.

\begin{lemma}
\label{lem:sum_answers}
For any non-terminal states $s_1$ and $s_2$, $\tr{s_1 \oplus s_2} = \tr{s_1} \cup \tr{s_2}$.
\end{lemma}

\begin{lemma}
\label{lem:prod_answers}
For any non-terminal state $s$ and goal $g$,  \mbox{$\tr{s \otimes g} \supseteq \bigcup_{(\sigma, n) \in \tr{s}} \tr{\inbr{g, \sigma, n}}$}.
\end{lemma}

These two lemmas constitute the exact conditions on definition of these operators that we will use to prove the completeness of an operational semantics.

We also can easily describe the criterion of termination for disjunctions.

\begin{lemma}
\label{lem:disj_termination}
For any goals $g_1$ and $g_2$, sunbstitution $\sigma$, and number $n$, the trace from the state $\inbr{g_1 \vee g_2, \sigma, n}$ is finite iff the traces from both $\inbr{g_1, \sigma, n}$ and $\inbr{g_2, \sigma, n}$ are finite.
\end{lemma}

These simple statements already allow us to prove two important properties of interleaving search as corollaries: the ``fairness'' of disjunction~--- the fact that the trace for disjunction contains all the answers from both streams for disjuncts~--- and the ``commutativity'' of disjunctions~--- the fact that swapping two disjuncts (at the top level) does not change the termination of the goal evaluation. 

\section{Equivalence of Semantics}
\label{equivalence}

Now we can relate two different kinds of semantics for \textsc{miniKanren} described in the previous sections and show that the results given by these two semantics are the same for any specification.
This will actually say something important about the search in the language: since operational semantics describes precisely the behavior of the search and denotational semantics
ignores the search and describes what we \emph{should} get from a mathematical point of view, by proving their equivalence we establish the \emph{completeness} of the search, which
means that the search will get all answers satisfying the described specification and only those.

But first, we need to relate the answers produced by these two semantics as they have different forms: a trace of substitutions (along with the numbers of allocated variables)
for the operational one and a set of representing functions for the denotational one. We can notice that the notion of a representing function is close to substitution, with only two differences:

\begin{itemize}
\item representing functions are total;
\item terms in the domain of representing functions are ground.
\end{itemize}

Therefore we can easily extend (perhaps ambiguously) any substitution to a representing function by composing it with an arbitrary representing function preserving
all variable dependencies in the substitution. So we can define a set of representing functions that correspond to a substitution as follows:

\[
\sembr{\sigma} = \{\overline{\mathfrak f} \circ \sigma \mid \mathfrak{f}:\mathcal{A}\mapsto\mathcal{D}\}
\]

\begin{comment}
In \textsc{Coq} this notion boils down to the following definition:

\begin{lstlisting}[language=Coq]
   Definition in_denotational_sem_subst
     (s : subst) (f : repr_fun) : Prop :=
       exists (f' : repr_fun),
         repr_fun_eq (subst_repr_fun_compose s f') f.
\end{lstlisting}

where ``\lstinline[language=Coq]|repr_fun_eq|'' stands for representing functions extensional equality, ``\lstinline[language=Coq]|subst_repr_fun_compose|''~---
for a composition of a substitution and a representing function.
\end{comment}

And the \emph{denotational analog} of operational semantics (a set of representing functions corresponding to the answers in the trace) for a given state $\hat{s}$ is
then defined as the union of sets for all substitutions in the trace:

\[
\sembr{\hat{s}}_{op} = \cup_{(\sigma, n) \in \tr{\hat{s}}} \sembr{\sigma}
\]

\begin{comment}
In \textsc{Coq} we again use a proposition instead:

\begin{lstlisting}[language=Coq]
   Definition in_denotational_analog
      (t : trace) (f : repr_fun) : Prop :=
      exists s n, in_stream (Answer s n) t /\
             in_denotational_sem_subst s f.
   Notation "{| t , f |}" := (in_denotational_analog t f).
\end{lstlisting}
\end{comment}

This allows us to state theorems relating the two semantics.

\begin{theorem}[Operational semantics soundness]
\label{lem:soundness}
If indices of all free variables in a goal $g$ are limited by some number $n$, then $\sembr{\inbr{g, \epsilon, n}}_{op} \subseteq \sembr{g}$.
\end{theorem}

It can be proven by nested induction, but first, we need to generalize the statement so that the inductive hypothesis is strong enough for the inductive step.
To do so, we define denotational semantics not only for goals but for arbitrary states. Note that this definition does not need to have any intuitive
interpretation, it is introduced only for the proof to go smoothly. The definition of the denotational semantics for extended states is shown on Fig.~\ref{denotational_semantics_of_states}.
The generalized version of the theorem uses it.

\begin{figure}[t]
  \[
  \begin{array}{ccl}
    \sembr{\Diamond}&=&\emptyset\\
    \sembr{\inbr{g, \sigma, n}}&=&\sembr{g}\cap\sembr{\sigma}\\
    \sembr{s_1 \oplus s_2}&=&\sembr{s_1}\cup\sembr{s_2}\\
    \sembr{s \otimes g}&=&\sembr{s}\cap\sembr{g}\\
  \end{array}
  \]
  \caption{Denotational semantics of states}
  \label{denotational_semantics_of_states}
\end{figure}

\begin{lemma}[Generalized soundness]
\label{lem:gen_soundness}
For any well-formed state $\hat{s}$

\[
\sembr{\hat{s}}_{op} \subseteq \sembr{\hat{s}}.
\]
\end{lemma}

It can be proven by the induction on the number of steps in which a given answer (more accurately, the substitution that contains it) occurs in the trace.
We break the proof in two parts and separately prove by induction on evidence that for every transition in our system the semantics of both the label (if there is one)
and the next state are subsets of the denotational semantics for the initial state.

\begin{lemma}[Soundness of the answer]
\label{lem:answer_soundness}
For any transition $s \xrightarrow{(\sigma, n)} \hat{s}$, \mbox{$\sembr{\sigma} \subseteq \sembr{s}$}.
\end{lemma}

\begin{lemma}[Soundness of the next state]
\label{lem:next_state_soundness}
For any transition $s \xrightarrow{l} \hat{s}$, \mbox{$\sembr{\hat{s}} \subseteq \sembr{s}$}.
\end{lemma}

It would be tempting to formulate the completeness of operational semantics as soundness with the inverted inclusion, but it does not hold in such generality.
The reason for this is that the denotational semantics encodes only the dependencies between free variables of a goal, which is reflected by the closedness condition,
while the operational semantics may also contain dependencies between semantic variables allocated in \lstinline|fresh| constructs. Therefore we formulate completeness
with representing functions restricted on the semantic variables allocated in the beginning (which includes all free variables of a goal). This does not
compromise our promise to prove the completeness of the search as \textsc{miniKanren} returns substitutions only for queried variables,
which are allocated in the beginning.

\begin{theorem}[Operational semantics completeness]
%\label{lem:gen_completeness}
If the indices of all free variables in a goal $g$ are limited by some number $n$, then

\[
\{\mathfrak{f}|_{\{\alpha_1,\dots,\alpha_n\}} \mid \mathfrak{f} \in \sembr{g}\} \subseteq \{\mathfrak{f}|_{\{\alpha_1,\dots,\alpha_n\}} \mid \mathfrak{f} \in \sembr{\inbr{g, \epsilon, n}}_{op}\}.
\]
\end{theorem}

Similarly to the soundness, this can be proven by nested induction, but the generalization is required. This time it is enough to generalize it from goals
to states of the shape $\inbr{g, \sigma, n}$. We also need to introduce one more auxiliary semantics~--- \emph{step-indexed denotational semantics} (denoted by $\sembr{\bullet}^i$). It is an implementation of the well-known approach~\cite{StepIndexing} of indexing typing or semantic logical relations by a number of permitted evaluation steps to allow inductive reasoning on it.
In our case, $\sembr{g}^i$ includes only those representing functions that one can get after no more than $i$ unfoldings of relational calls.

The step-indexed denotational semantics is an approximation of the conventional denotational semantics; it is clear that any answer in conventional denotational semantics will also be in step-indexed denotational semantics for some number of steps.

\begin{lemma}
$\sembr{g} \subseteq \cup_i \sembr{g}^i$
\end{lemma}

Now the generalized version of the completeness theorem is as follows.

\begin{lemma}[Generalized completeness]
\label{lem:gen_completeness}
For any set of relational definitions, for any number of unfoldings $i$, for any well-formed state $\inbr{g, \sigma, n}$,

\[
\{\mathfrak{f}|_{\{\alpha_1,\dots,\alpha_n\}} \mid \mathfrak{f} \in \sembr{g}^i \cap \sembr{\sigma}\} \subseteq \{\mathfrak{f}|_{\{\alpha_1,\dots,\alpha_n\}} \mid \mathfrak{f} \in \sembr{\inbr{g, \sigma, n}}_{op}\}.
\]
\end{lemma}

The proof is by the induction on nuber of unfoldings $i$. The induction step is proven by structural induction on goal $g$. We use lemmas~\ref{lem:sum_answers} and~\ref{lem:prod_answers} for evaluation of a disjunction and a conjunction respectively, and lemma~\ref{lem:den_sem_change_var} in the case of fresh variable introduction to move from an arbitrary semantic variable in denotational semantics to the next allocated fresh variable. The details of this proof may be found in the extended version of the paper.

\section{Conclusion and future work}

We presented an approach for pattern matching implementation synthesis using relational programming. Currently, it demonstrates a good performance only
on a very small problems. The performance can be improved by searching for new ways to prune the search space and by speeding up the implementation of
relations and structural constraints. Also it could be interesting to integrate structural constraints more closely into \textsc{OCanren}'s core.
Discovering an optimal order of samples and reducing the complete set of samples is another direction for research.

The language of intermediate representation can be altered, too. It is interesting to add to an intermediate language so-called \emph{exit nodes}
described in~\cite{maranget2001}. The straightforward implementation of them might require nominal unification, but we are not aware of any
\textsc{miniKanren} implementation in which both disequality constraints and nominal unification~\cite{alphaKanren} coexist nicely.

At the moment we support only simple pattern matching without any extensions. It looks technically easy to extend our approach with
non-linear and disjunctive patterns. It will, however, increase the search space and might require more optimizations.





%Text of paper \ldots


%% Acknowledgments
%\begin{acks}                            %% acks environment is optional
                                        %% contents suppressed with 'anonymous'
  %% Commands \grantsponsor{<sponsorID>}{<name>}{<url>} and
  %% \grantnum[<url>]{<sponsorID>}{<number>} should be used to
  %% acknowledge financial support and will be used by metadata
  %% extraction tools.
 % This material is based upon work supported by the
  %\grantsponsor{GS100000001}{National Science
   % Foundation}{http://dx.doi.org/10.13039/100000001} under Grant
  %No.~\grantnum{GS100000001}{nnnnnnn} and Grant
  %No.~\grantnum{GS100000001}{mmmmmmm}.  Any opinions, findings, and
  %conclusions or recommendations expressed in this material are those
  %of the author and do not necessarily reflect the views of the
  %National Science Foundation.
%\end{acks}


%% Bibliography
\bibliography{main}


%% Appendix
%\appendix
%\section{Appendix}

%Text of appendix \ldots

\end{document}
