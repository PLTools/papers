% !TEX TS-program = pdflatex
% !TeX spellcheck = en_US
% !TEX root = main.tex

\section{Implementation}
\label{sec:implementation}

In this work we translate a subset of \OCanren{} (miniKanren in \OCaml{})  to \miniKanren in \Scheme{} and to \Klogic{} in \Kotlin{}.
\OCanren{} was picked as input implementation for various reasons, and the main is that we are the most familiar with \OCaml{} language and ecosystem.
Also we can observe an interest of porting \OCanren{} programs to JVM around us.
If we would use \Scheme{} implementation for fast prototyping, we expect more issue with automated translation of \Scheme{} to \Kotlin{} because of absence of types in relational programs in \Scheme{}.
There is an option to translate dynamically typed implementation to a weakly typed implementation in a typed language, but we believe that it is bad idea. (TODO Why?)

The translator takes typed syntax tree of \OCaml{} file, detects relational code by checking the types of declared values, reconstructs a representation for relational parts and prints them to \Kotlin{} and \Scheme{}.
All macro expansion are performed before typing phase.
As a result, typed relational AST in \OCaml{} uses low-level constructs, like \verb=call/fresh= and binary conjunction \verb=&&&=.
In this aspect, \OCanren{} implementation stands in between \Klogic{} and \miniKanren in \Scheme{}: the former doesn't use macro at all; the latter uses macro for everything: even \verb=conde= is expanded to \verb=mplus*=.
The task of translator is to collect the enriched AST with high-level relational constructs, reverse applications of macros, and print back to \Klogic{} and \Scheme{} a readable relational code, ideally without changes to a search order.

In some cases the typed representation of a relation is not enough, and we need to add user information that is skipped while type checking phase.
This is implemented using \OCaml{}'s attributes, for example \verb=[@@@name <OCaml expression>]=, placed in the top level of OCaml's modules.

\subsection{Kotlin-related features}

Code in \Kotlin{} and \Java{} usually requires importing many names to the global namespace using the \verb=import= keyword.
\Scheme{} is affected of this problem by a significantly lesser extent, because we can \verb=(include "filename")= to textually substitute relational implementation, where we require it.
Because of that we have special attributes \verb=[@@@klogic.preamble ...]= and \verb=[@@@klogic.epilogue ...]= that contain string literal with \Kotlin{} code that should be placed in the beginning and in the end of generated file.

Translation from one typed language to another requires also generating types of values.
In principle, we could engineer a general convention to mangle \OCaml{} types to \Kotlin{} ones, but ensuring that purely Kotlin developers will follow this convention is a huge task.
Also, automatic straightforward mangling sometimes hit \Kotlin{}-specific keywords, which leads to non-compilable \Klogic{} code.
At the moment we support only naive mangling where \OCaml{} developer specifies conversion from string representations of \OCaml{} types to string representation of \Klogic{} types.
For our needs it is enough, the general solution is left for the future.
The same approach is currently used for mangling \OCaml{}'s to \Kotlin{}'s functions' names.

We have an initial support for modularity.
On Fig.~\ref{fig:modularity} there is an example how OCaml's functors (functions from modules to modules) are translated to \Kotlin{} functions from objects to another object.
This feature is a huge help for testing relational code, because it is essentially a static version of dependency injection\footnote{\url{https://martinfowler.com/articles/injection.html} (accessed: \DTMdate{2024-06-09})}.

 \begin{figure}[ht]
   \begin{subfigure}[t]{\textwidth}
      \begin{lstlisting}[language=ocaml]
module type VERIFIER = sig
  val params : int ilogic -> int ilogic Jtype.injected Std.List.injected -> goal
  ...
end

module Verifier (CT : CLASSTABLE) : VERIFIER = struct
  let params : int ilogic -> int ilogic Jtype.injected Std.List.injected -> goal =
    fun id p -> ...
  ...
end
      \end{lstlisting}
            \subcaption{\OCaml{}'s module types and functors (functions from modules to modules)...}
   \end{subfigure}
   \begin{subfigure}[t]{\textwidth}
      \begin{lstlisting}[language=kotlin]
interface VERIFIER {
  context(RelationalContext)
  fun params(v1: Term<LogicInt>, v2: Term<LogicList<Jtype<LogicInt>>>): Goal
  ...
}

val Verifier : (CLASSTABLE) -> VERIFIER = { CT: CLASSTABLE ->
object: VERIFIER {
  context(RelationalContext)
  override fun  params(id: Term<LogicInt>,
      p: Term<LogicList<Jtype<LogicInt>>>): Goal = ...
  ...
}
      \end{lstlisting}
      \subcaption{...are being translated to \Klogic{} as interfaces, and functions that take \Kotlin{} objects and return other objects.}
   \end{subfigure}
   \caption{Modularity in \OCaml{} language supported.}
   \label{fig:modularity}
 \end{figure}

\subsection{Scheme-related features}

\Scheme{} is not a primary target for our translation.
We added it to check applicability of our approach for one more implementation of \miniKanren{} and picking the default one\footnote{\url{https://github.com/michaelballantyne/faster-minikanren} (accessed: \DTMdate{2024-06-09})} looks like a decent idea.

Translating to \Scheme{} comparatively to \Kotlin{} is easier and more complex at the same time.
All types from \OCanren{} code are being erased, and do not appear in \Scheme{}.
On the other hand, \OCanren{} and \Klogic{} use data representation that is distinct from \Scheme{}'s one.

\OCanren{}~\cite{OCanren} relies on algebraic data types to represent terms for relational programming, because in statically typed functional language \OCaml{} they are prolific.
User-defined values are tagged by a constructor and may contain arguments.
Unification checks that constructors and its arity are the same, and unifies arguments recursively.
All constructors are associated with a data types, and unification of values of different types is prohibited by \OCaml{}'s type system.
The construction of values from relational domain is performed by adding a constructor and injection using predefined primitive \verb=!!=.
The implementation is quite intricate: relational domain stands in between of domains of ground values and reified values.
This approach reduces amount of tagging; without it terms would have extra constructors (representation in memory by pointers) that increase size of terms and slowdown unification.

In \miniKanren{} we use lists as the basic data structures and put various symbols into this lists to distinguish values of one sort from another.
For example, relational interpreters could evaluate into normal values or closures, where latter is represented using tag \verb=closure=, variable name, body and captured environment: \verb=`(closure ,x ,body ,env)=.

In \Klogic{} and \OCanren{} terms are constructed either by explicit usage of constructor, or by using of smart construction functions.
With  \miniKanren{} in \Scheme{} it is possible too, but most of the papers in the topic prefer quotation syntax.
We currently support only this approach, because it allows to compare original and translated implementations purely by eyesight.

\begin{figure}
\begin{lstlisting}[language=ocaml]
type ('a, 'l) list_like = Nil | Cons of 'a * 'l
type 'a injected = ('a, 'a injected) list_like ilogic
...
let nil () : 'a injected = !!Nil
let cons : 'a -> 'a injected -> 'a injected = fun x y -> !!(Cons (x, y))
\end{lstlisting}
\caption{Declaration of list-like algebraic data type and smart constructors.
Type definition is written in open recursion style to allow a logic variable in tail position (not really relevant for this paper).
Smart constructor functions are basically a composition of an injection \texttt{!!} after a constructor application.
The function \texttt{nil} has an extra argument to avoid typing artifact called value restriction~\cite{Relaxed2004}.
}
\end{figure}

