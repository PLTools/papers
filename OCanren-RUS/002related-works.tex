% !TeX spellcheck = ru_RU
% !TEX root = main.tex

\section{Обзор}
\label{sec:relworks}

При реализации miniKanren в строго статически типизированном языке сложности возникают естественным образом, так как реляционное программирование родилось в среде Scheme/Racket --- динамически типизированной и удобной для метапрограммирования среде.
Оригинальная реализация miniKanren не обращает внимания на особенности, важные для таких языков как ML и Haskell.
Например, одна из основных конструкций miniKanren --- унификация --- обязана исполняться для разных данных и одного только параметрического полиморфизма может быть не достаточно.

%There is a predictable difficulty in implementing \miniKanren for a strongly typed language.
%Designed in the metaprogramming-friendly and dynamically typed realm of Scheme/Racket, the original
%\miniKanren implementation pays very little attention to what has a significant importance in (specifically)
%ML or Haskell. In particular, one of the capstone constructs of \miniKanren~--- unification~--- has to work for
%different data structures, which may have types different beyond parametricity.

Существуют несколько способов преодолеть эту проблему.
Во-первых, можно повторять нетипизированный подход и предоставить реализацию унификации для некоторого конкретного типа, достаточно богатого, чтобы представить все другие типы данных.
Некоторые библиотеки~\footnote{\url{https://github.com/JaimieMurdock/HK}, \url{https://github.com/rntz/ukanren}} реляционного программирования для Haskell и предыдущая~\footnote{\url{https://github.com/lightyang/minikanren-ocaml}} реализация для OCaml  пошли по этому пути.
В результате оригинальная реализация на Scheme может быть повторена со всей её краткостью, но реляционные спецификации  станут слабо типизированы.
Сходный подход встречался ранее при встраивании Prolog в  Haskell~\cite{PrologInHaskell}.

%There are a few ways to overcome this problem. The first one is simply to follow the untyped paradigm and
%provide unification for some concrete type, rich enough to represent any reasonable data structures.
%Some Haskell \miniKanren libraries\footnote{\url{https://github.com/JaimieMurdock/HK}, \url{https://github.com/rntz/ukanren}}
%as well as the previous OCaml implementation\footnote{\url{https://github.com/lightyang/minikanren-ocaml}} take this approach.
%As a result, the original implementation can be retold with all its elegance; the relational specifications, however,
%become weakly typed. A similar approach was taken in early works on embedding Prolog into Haskell~\cite{PrologInHaskell}.

Другим подходом является использование \emph{ad hoc} полиморфизма и написание специфичной унификации для каждого из интересующих типов.
Некоторые реализации miniKanren, например  Molog\footnote{\url{https://github.com/acfoltzer/Molog}} and
MiniKanrenT\footnote{\url{https://github.com/jvranish/MiniKanrenT}} для Haskell, выбрали данный подход.
Строгая типизация сохраняется, но требуется написания большого количества стереотипного (англ. boilerplate) кода и применяется некоторая оптимизация этого, например
Template Haskell~\cite{SheardTMH}.
Можно заводить~\cite{TypedLogicalVariables} отдельные классы типов для выполнения и унификации, и выделения свободных логических переменных в пользовательских типах.
Требования выделять логические переменные в пользовательских типах данных особым способом выглядит для нас несколько искусственно, так как эти логические переменные придется обрабатывать и в модулях программы, не имеющих отношения к логическому программированию.

%Another approach is to utilize \emph{ad hoc} polymorphism and provide a type-specific unification for each ``interesting'' type.
%Some \miniKanren{} implementations, such as Molog\footnote{\url{https://github.com/acfoltzer/Molog}} and
%MiniKanrenT\footnote{\url{https://github.com/jvranish/MiniKanrenT}}, both for Haskell, can be mentioned as examples.
%While preserving strong typing, this approach requires a lot of ``boilerplate''
%code to be written, so some automation --- for example, using
%Template Haskell~\cite{SheardTMH}~---
%is desirable. In~\cite{TypedLogicalVariables} a separate type class was introduced to both perform the unification
%and detect free logical variables in end-user data structures. The requirement for end user to provide a way to represent
%logical variables in custom data structures looks superfluous for us since these logical variables would require proper
%handling in the rest of the code outside the logical programming subsystem.

Существует ещё один подход, но нам неизвестны реализации miniKanren  на его основе.
Можно реализовать унификацию для обобщённого представления типов данных в виде сумм произведений и неподвижных точек функторов для них~\cite{InstantGenerics, ALaCarte}.
Унификация сможет работать для любого типа данных, для которого сделано представление.
Мы полагаем, что с таким подходом будет меньше стереотипного кода.

%There is, actually, another potential approach, but we do not know if anybody has tried
%it: implementing unification for a generic representation of types as sum-of-products and fixpoints of
%functors~\cite{InstantGenerics, ALaCarte}. Thus, unification would work for any type for which a representation
%is provided. We believe that implementing this representation would require less boilerplate code to be written.
\def\adhoc{\emph{ad hoc}}

Из выше сказанного можно заключить, что типизированное встраивание miniKanren в OCaml состоит из сочетания обобщённого программирования~\cite{DGP} и \adhoc{}  полиморфизма.
Использования обобщённого программирования в OCaml уже довольно хорошо исследовано (например,~\cite{Deriving}).
А возможности для \emph{ad hoc} полиморфизма в OCaml развиты слабо: нет ничего сравнимого с классами типов языка Haskell, использование объектно-ориентированного программирования позволяет проэмулировать желаемое поведение, но не без недостатков.
Существующие предложения по развитию \adhoc{} полиморфизма (например, неявные модули~\cite{Implicits}) требуют модификации компилятора, чего мы хотели бы избегать.
Поэтому мы пошли другим путём и реализовали полиморфную унификацию один раз и для всех логических типов, и эта реализация существенно завязан на реализацию OCaml,
так как возможности для менее \adhoc{} подхода пока не интегрированы в язык.
Для работы с пользовательскими типами данных и реляционных части программы, было предложено использовать специальные логические представления (глава~\ref{sec:injection}), которые освобождают пользователя от задачи поддержки переменных в его типах данных.
Использование же обобщённого программирования позволяет систематическим образом получать преобразования в логические представления и обратно.

%As follows from this exposition, a typed embedding of \miniKanren{} in OCaml can be done with a combination of datatype-generic programming~\cite{DGP} and \emph{ad hoc} polymorphism.
%There are a number of generic frameworks for OCaml (for example,~\cite{Deriving}).
%On the other hand, the support for \emph{ad hoc} polymorphism in OCaml is weak; there is nothing comparable in power to Haskell type classes, and even though sometimes the object-oriented layer of the language can be used to mimic desirable behavior, the result, as a rule, is far from satisfactory.
%Existing proposals for \emph{ad hoc} polymorphism (for example, modular implicits~\cite{Implicits}) require patching the compiler, which we want to avoid. Therefore, we take a different approach, implementing polymorphic unification once and for all logical types~--- a purely \emph{ad hoc} approach, since the features which would provide a less \emph{ad hoc} solution are not yet well integrated into the language.
%To deal with user-defined types in the relational subsystem, we propose to use their logical representations (see Section~\ref{sec:injection}),
%which free an end user from the burden of maintaining logical variables, and we use generic programming to build conversions from and to logical
%representations in a systematic manner.
%
%
%
