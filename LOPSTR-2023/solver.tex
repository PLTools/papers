\section{Relational Subtyping Solver}
\label{sec:solver}

The approach of verifier-to-solver conversion relies on the implementation of functional verifier for the problem in question. In our
case such verifier should test if two given ground types are in the subtyping relation. This, in particular, requires implementation
of capture conversion, ``contains'' relation, and direct subtyping verifier. Then a reflexive-transitive closure of the latter has to
be implemented. The functional verifier is then converted into relational form which by construction delivers a subtyping solver for
non-ground types with free variables.

A simple observation, however, makes its obvious that it is in fact much easier to implement reflexive-transitive closure directly
in relational language. Indeed, given relation $R$ its reflexive-transitive $R^*$ closure can be expressed by just

\[
R^*\,(x,\, y) = R\, (x,\, y)\vee x\equiv y\vee\exists\, z\,.\,R\,(x,\,z)\wedge R^*\,(z,\,y)\eqno{(\star)}
\]

which can be easily directly encoded in \textsc{OCanren}. However, from the implementation standpoint the application of this technique
imposes a certain problem since ``$\prec$'' and ``$\precprec$'' are mutually recursive, and we expect ``$\prec$'' to be obtained as a
result of relational conversion of functional implementation.

Functional verifier is implemented in \textsc{OCaml} in a straightforward manner: we encode types as data structures of certain types and give a
direct implementation for all components of subtyping relation except the transitive closure. All these definitions are ``wrapped'' in a
functor (an \textsc{OCaml} module parameterized by a module) which takes \emph{class table} as a parameter. The class table
contains the definitions of relevant set of classes with their direct supertyping encoding, description of their type parameters, etc.
The concrete contents of the class table is supplied by the symbolic execution engine.

To build a mixture of relationally-converted code and relational implementation of transitive closure we use open recursion: the functional
implementation of direct supertyping relation ``$\prec$'' is parameterized by subtyping relation  ``$\precprec$''. In functional implementation
the knot is tied not by transitive closure of ``$\prec$'' but by ``$\prec$'' itself:

\begin{lstlisting}[language=ocanren,mathescape=true]
  let rec ($\precprec$) ta tb = ta $\prec$ tb 
  and ($\prec$) ta tb     = Verify.($\prec$) ($\precprec$) ta tb
\end{lstlisting}

Here ``\lstinline|Verify|'' is a module acquired by an instantiation of the verifier for a simple hand-written testing class table.
Thus, functional verifier can only check the subtyping of types no more then two ``steps'' away from each other.

The next step is applying relational conversion for this functional verifier. For technical reasons this requires a mild
``massaging'' of initial \textsc{OCaml} implementation: arrays (more efficient in functional world) have to be replaced
with lists, and some changes have to be made in order to compensate for the incompleteness of current relational
conversion implementation.
After the relational conversion and the parameterization of verifier functor with a class table a proper recursive knot is tied by a transitive closure of ``$\prec$'', which
completes the construction of the solver. While functional verifier can only work for ground types and return a boolean value, its relational counterpart
searches for all substitution for free variables in incomplete types is order to make them subtype each other.

Our first evaluation discovered the fact that the solver was unsound~--- it established the subtyping relation for two arbitrary types. This finding constitutes a drastic
contradiction with the theory which predicts that the solver has to be correct by construction. The careful analysis, however, discovered that functional
verifier was unsound as well! The reason was very simple: let us have two arbitrary types $A$ and $B$. To establish, for example, that $A \precprec B$, take a
type variable $\alpha^B_A$ with upper bound $B$ and lower bound $A$. Then, by definition

\[
A\precprec \alpha^B_A \wedge \alpha^B_A\precprec B \Rightarrow A\precprec B
\]

In the JLS there were no explicit requirement for lower/upper bounds of type variables to respect the subtyping relation, thus this requirement was not
encoded in the verifier. Another finding of the evaluation was that, contrary to the expectation, the direct supertyping relation is actually
reflexive (in full accordance with the JLS).

After fixing the unsoundness we found that the performance of the solver was very low: it took seconds to come up with the first solution even for
small hard-written class tables. After a careful analysis we come up with the following optimizations.

First, we changed the representation of class/interface identifiers. In functional verifier all identifiers were represented by integer numbers which,
under relational conversion, were tuned into natural numbers in Peano form (essentially, lists). At the same time as no transformation are
performed on identifiers (besides equality checks) this representation is excessive. We manually reverted the identifiers representation to
integers.

Then, we optimized the search in class tables. Functional verifier sometimes needs to find a declaration of a class/interface by identifier, or
find a superclass/superinterface for a class given this class identifier, etc. Under relational conversion all these procedures
take form of disjunctions of the length proportional to the size of the class table. However, when some of the involved identifiers
are ground the search can be optimized: we can query a precompiled optimized map-like structure of the class table to filter out
only relevant information and synthesize the residual disjunction at the runtime. We call this optimization \emph{dynamic class table specialization}.

Finally, we optimize the evaluation of transitive closure based on the groundness analysis of the arguments. For example, if we need to find a certain
$x$ such that $R^*\,(x,\,y)$ where $y$ is \emph{ground}, then the following (equivalent to $(\star)$) procedure of finding transitive closure is
found to be more efficient:

\[
R^*\,(x,\, y) = R\, (x,\, y)\vee x\equiv y\vee\exists\, z\,.\,R\,(z,\,y)\wedge R^*\,(x,\,z)
\]

So, in our implementation the groundness analysis (performed on the fly) chooses one of a few versions of transitive closure calculations.
We call this optimization \emph{dynamic transitive closure evaluation}.

With all of these optimizations enabled our solver finally can demonstrate a reasonable performance. We present the results of the evaluation
in the next section\footnote{Source code is available online: \url{https://github.com/Lozov-Petr/JGS/tree/LOPSTR-2023}}. 
