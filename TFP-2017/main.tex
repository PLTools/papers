\documentclass{llncs}

\usepackage{makeidx}
\usepackage{amssymb}
\usepackage{listings}
\usepackage{indentfirst}
\usepackage{verbatim}
\usepackage{amsmath}
\usepackage{graphicx}
\usepackage{xcolor}
\usepackage{url}
\usepackage{stmaryrd}
\usepackage{xspace}
\usepackage{comment}
\usepackage{wrapfig}
\usepackage{placeins}
\usepackage{tabularx}
\usepackage{ragged2e}
\usepackage{subcaption}
\captionsetup{compatibility=false}
%\usepackage{natbib}
%\usepackage [sorting = none] {biblatex}
%\addbibresource {main.bib}

\def\transarrow{\xrightarrow}
\newcommand{\setarrow}[1]{\def\transarrow{#1}}

\newcommand{\trule}[2]{\frac{#1}{#2}}
\newcommand{\crule}[3]{\frac{#1}{#2},\;{#3}}
\newcommand{\withenv}[2]{{#1}\vdash{#2}}
\newcommand{\trans}[3]{{#1}\transarrow{#2}{#3}}
\newcommand{\ctrans}[4]{{#1}\transarrow{#2}{#3},\;{#4}}
\newcommand{\llang}[1]{\mbox{\lstinline[mathescape]|#1|}}
\newcommand{\pair}[2]{\inbr{{#1}\mid{#2}}}
\newcommand{\inbr}[1]{\left<{#1}\right>}
\newcommand{\highlight}[1]{\color{red}{#1}}
\newcommand{\ruleno}[1]{\eqno[\scriptsize\textsc{#1}]}
\newcommand{\inmath}[1]{\mbox{$#1$}}
\newcommand{\lfp}[1]{fix_{#1}}
\newcommand{\gfp}[1]{Fix_{#1}}
\newcommand{\vsep}{\vspace{-2mm}}
\newcommand{\supp}[1]{\scriptsize{#1}}
\newcommand{\G}{\mathfrak G}
\newcommand{\sembr}[1]{\llbracket{#1}\rrbracket}
\newcommand{\cd}[1]{\texttt{#1}}
\newcommand{\miniKanren}{miniKanren\xspace}
\newcommand{\ocanren}{OCanren\xspace}
\newcommand{\free}[1]{\boxed{#1}}
\newcommand{\binds}{\;\mapsto\;}
\newcommand{\dbi}[1]{\mbox{\bf{#1}}}

\let\emptyset\varnothing
\setlength{\abovecaptionskip}{3pt plus 3pt minus 2pt}
\setlength{\belowcaptionskip}{-10pt plus 3pt minus 2pt}

\lstdefinelanguage{ocanren}{
keywords={fresh, let, in, match, with, when, class, type,
object, method, of, rec, repeat, until, while, not, do, done, as, val, inherit,
new, module, sig, deriving, datatype, struct, if, then, else, open, private, virtual, include, success, failure,
true, false},
sensitive=true,
commentstyle=\small\itshape\ttfamily,
keywordstyle=\ttfamily\underbar,
identifierstyle=\ttfamily,
basewidth={0.5em,0.5em},
columns=fixed,
fontadjust=true,
literate={fun}{{$\lambda$}}1 {->}{{$\to$}}3 {===}{{$\equiv$}}1 {=/=}{{$\not\equiv$}}1 {|>}{{$\triangleright$}}3 {|||}{{$\vee$}}2 {/\\}{{$\wedge$}}2 {^}{{$\uparrow$}}1,
morecomment=[s]{(*}{*)}
}

\lstset{
mathescape=true,
%basicstyle=\small,
identifierstyle=\ttfamily,
keywordstyle=\bfseries,
commentstyle=\scriptsize\rmfamily,
basewidth={0.5em,0.5em},
fontadjust=true,
language=ocanren
}

\usepackage{letltxmacro}
\newcommand*{\SavedLstInline}{}
\LetLtxMacro\SavedLstInline\lstinline
\DeclareRobustCommand*{\lstinline}{%
  \ifmmode
    \let\SavedBGroup\bgroup
    \def\bgroup{%
      \let\bgroup\SavedBGroup
      \hbox\bgroup
    }%
  \fi
  \SavedLstInline
}
%\addtolength{\parskip}{-2pt}
%\setlength{\parskip}{0pt}
\setlength{\belowcaptionskip}{-15pt}

\pagestyle{plain}
\begin{document}
\sloppy
\mainmatter

\title{Typed Relational Conversion\thanks{This work is supported by RFBR grant No 18-01-00380.}}

\author{
  Petr Lozov\inst{1} \and Andrei Vyatkin\inst{2} \and Dmitry Boulytchev\inst{3}
}

\institute{
St.Petersburg State University,\\
Universitetski pr., 28, 198504, St.Petersburg, Russia,\\
\email{lozov.peter@gmail.com}
\and
St.Petersburg State University,\\
Universitetski pr., 28, 198504, St.Petersburg, Russia,\\
\email{dewshick@gmail.com}
\and
St.Petersburg State University,\\
Universitetski pr., 28, 198504, St.Petersburg, Russia,\\
JetBrains Research\\
Universitetskaya emb., 7-9-11, bldg. 5A, 199034, St.Petersburg, Russia,\\
\email{dboulytchev@math.spbu.ru}}

\maketitle

\begin{abstract}
We address the problem of transforming typed functional programs into relational form. 
In this form, a program can be run in various ``directions'' with some arguments left free, 
making it possible to acquire different behaviors from a single specification. We specify the 
syntax, typing rules and semantics for the source language as well as its relational extension, 
describe the conversion and prove its correctness both in terms of typing and dynamic semantics. 
We also discuss the limitations of our approach, present the implementation of the conversion for 
the subset of OCaml and evaluate it on a number of realistic examples.
\end{abstract}

\section{Introduction}
\label{intro}

Relational programming is an attractive technique, based on the idea of constructing programs as relations.
While in general some relational effects can be reproduced with a number of languages for logic programming, such as
Prolog, Mercury\footnote{\url{https://mercurylang.org}}, or Curry\footnote{\url{http://www-ps.informatik.uni-kiel.de/currywiki}}, in
a narrow sense relational programming amounts to writing relational specifications in \miniKanren~\cite{TRS}. \miniKanren\footnote{\url{http://minikanren.org}},
initially designed as a small relational DSL, embedded in Scheme/Racket, was later implemented for a number of general-purpose host languages,
including Scala, Haskell, Standard ML and OCaml.

With relational approach, it becomes possible to give simple and elegant solutions for the problems, otherwise
considered as tricky, tough, tedious, or boring~\cite{unified}. For example, relational interpreters can be used to derive
\emph{quines}~--- programs, which reduce to themselves, as well as \emph{twines} or \emph{thrines} (pairs or triples of
programs, reducing to each other)~\cite{Untagged}; a straightforward relational description of
simply typed lambda calculus~\cite{Lambda} inference rules works both as type inferencer and inhabitation problem solver~\cite{WillThesis};
relational list sorting can be used to generate all permutations~\cite{ocanren}, etc. 

On the other hand, writing relational specifications can sometimes be a tricky and error-prone task. Fortunately, many 
specifications can be written systematically by ``generalizing'' a certain functional program. From the very beginning, 
the conversion from functional to relational form was considered as an element of relational programming thesaurus~\cite{TRS}. However,
the traditional approach~--- \emph{unnesting}~--- was formulated for an untyped case, worked only for specifically written
programs and was never implemented.

We present a generalized form of relational conversion, which can be applied to typed terms in general form. We study the relational conversion 
for a small ML-like language (essentially, a certain subset of OCaml), equipped with Hindley-Milner type system with let-polymorphism~\cite{Types}. 
We start from retelling the syntax, typing rules, and operational semantics, and then extend the source language with a conventional set of 
relational constructs. This set corresponds to existing typed embedding of \miniKanren into OCaml~\cite{ocanren}. We then present typing rules and 
develop operational semantics for this relational extension; to our knowledge, this is the first attempt to specify formal semantics for
\miniKanren. Next, we develop formal rules for relational conversion and prove, that these rules respect both typing and
semantics. Finally, we describe the implementation of a relational converter and demonstrate its application for a number of problems, for some
of which we present a relational solution for the first time.

We would like to express our gratitude to William Byrd and the anonymous reviewers for their constructive remarks, which, we believe, led to the
improvement of the presentation. 

\section{Relational Programming in \miniKanren}
\label{ocanren}

In the context of this paper, we will use a certain concrete implementation of \miniKanren~--- a shallow DSL for 
OCaml\footnote{https://github.com/dboulytchev/ocanren}, called \ocanren~\cite{ocanren}. \ocanren corresponds to \miniKanren with
disequality constraints~\cite{CKanren}, and (modulo typing) follows the original implementation~\cite{MicroKanren,SmallEmbedding}. Here we describe the external view 
on \ocanren, giving the only intuitive meaning of its constructs; the formal description will be presented in Section~\ref{relational_extension}.
We also use a simplified syntax, which is a little bit different from the concrete syntax in actual implementation, but assumed to
be easier to read.

The central notion of \miniKanren is \emph{goal}; in \ocanren a goal can be an arbitrary expression of reserved goal type, which we denote $\G$.
There are only five syntactic forms of goals (denoted below as $g, g_1, g_2$, etc.):

\begin{itemize}
  \item conjunction $g_1\wedge g_2$;
  \item disjunction $g_1\vee g_2$;
  \item fresh variable introduction $\lstinline|fresh ($x$) $\;g$|$;
  \item unification $t_1\;\equiv\;t_2$;
  \item disequality constraint $t_1\;\not\equiv\;t_2$.
\end{itemize}

Two last forms of goals constitute a basis for goal construction; here $t_1$ and $t_2$ are \emph{terms}. In \ocanren a term is an arbitrary expression of polymorphic logic type $\alpha^o$. The postfix notation $\Box^o$ is a traditional way to denote relational entities, and we will use it for types as well\footnote{In the real implementation the terms have a more complex two-parametric type, which encodes a tagging, needed to be performed when the results of the relational program are returned into the functional world; these details, however, are irrelevant to the objectives of the paper, and we stick with the simplified version.}.

The simplest expression of logic type is a variable, bound in \lstinline|fresh|. Another example is a primitive value, \emph{injected} into the logic domain with a built-in primitive ``$\uparrow$'', such as $\uparrow\!3$ (of type \lstinline|int$^o$|) or \lstinline|$\uparrow$true| (of type \lstinline|bool$^o$|).
Other types (pairs, lists, user-defined algebraic datatypes, etc.) can be used in relational specifications as well, being injected by the same primitive. For example, expression \lstinline{^(1, "abc")} has the type \lstinline{(int * string)$^o$}, \lstinline{^[1; 2; 3]}~--- the type \lstinline{(int list)$^o$}, etc. The subtle part is that (since the unification only works for logical types) the placement of ``$^o$'' determines the granularity of unification. Indeed, a logical variable can only be placed where logical type is expected. Thus, in unification one can use a value of type \lstinline{(int * int)$^o$} as \emph{a whole}, but in order to control the \emph{contents} of the pair relationally, the type \lstinline{(int$^o$, int$^o$)$^o$} is required. This makes it impossible to reuse some built-in or standard types in relational code~--- for example, predefined list type is not flexible enough, since it does not allow the tail of the list to be logical. Instead, logical list type has to be introduced:

\begin{lstlisting}
   type $\alpha$ llist = Nil | Cons of $\alpha^o$ * ($\alpha$ llist)$^o$
\end{lstlisting}

With logical list type, we can implement some relations for lists:

\begin{lstlisting}
   val append : ($\alpha$ llist)$^o$ -> ($\alpha$ llist)$^o$ -> ($\alpha$ llist)$^o$ -> $\G$
   let rec append$^o$ x y xy =
     (x === ^Nil /\ xy === y) |||
     (fresh (h t ty)
        x  === ^(Cons (h, t)) /\
        xy === ^(Cons (h, ty)) /\
        append$^o$ t y ty
     ) 
\end{lstlisting}

Here we defined relational list concatenation \lstinline{append$^o$}, a canonical example in the field. This ternary relation is constructed,
using case analysis and recursion:

\begin{enumerate}
\item If the first list is empty, then the second and the third lists must be equal.
\item Otherwise, the first list can be split into a head and a tail, and two fresh variables \lstinline{h} and \lstinline{t} are needed to denote them.
We also need a fresh variable \lstinline{ty} to denote the list, such that appending \lstinline{y} to \lstinline{t} equals \lstinline{ty}. To ensure this
property, we use a recursive call to \lstinline{append$^o$}. Finally, we acquire the final result by consing \lstinline{h} and \lstinline{ty}. 
\end{enumerate}

The definition of \lstinline{append$^o$} takes three logical lists \lstinline{x}, \lstinline{y} and \lstinline{xy} as arguments, and constructs a goal,
which can be executed or combined with other goals. In the former case, a stream of \emph{answers} is returned. An element of the stream contains the description
of certain constraints for logical variables, which have to be respected in order for the relation to hold. We denote the running primitive ``$\leadsto$'', so

\begin{lstlisting}
  fresh (q) append$^o$ ^(Cons (^1, ^Nil)) q ^Nil $\leadsto$ []
\end{lstlisting}

\noindent returns an empty stream, since there is no list \lstinline{q}, such that appending \lstinline{Cons (1, Nil)} and \lstinline{q} gives empty list, while

\begin{lstlisting}
  fresh (q) append$^o$ q ^Nil ^(Cons (^1, ^Nil)) $\leadsto$ [q$\binds$Cons (1, Nil)]
\end{lstlisting}

\noindent discovers the expected constraint for the variable \lstinline{q}.

As it can be seen from the type, relational concatenation is polymorphic, like its functional counterpart. However, the query

\begin{lstlisting}
   append$^o$ ^(Cons (^$\lambda$x.x, ^Nil)) q ^(Cons (^$\lambda$y.y, ^Nil))  
\end{lstlisting}

\noindent ends with a run-time error due to inability to unify closures. This is a fundamental limitation in original \miniKanren as well,
as it deals only with first-order syntactic unification~\cite{Unification}. This example demonstrates, that, unlike pure OCaml, the typing
in OCanren is somewhat weak. In order to restore the strong typing, some of the type variables have to be bounded to range over only non-functional
types. The lack of direct support for bounded polymorphism~\cite{cardelli} in OCaml makes this step problematic. Our experience, however, shows, that
in practice this deficiency rarely gets in the way. In the following development, we assume, that in polymorphic types some type variables may be
implicitly bounded by the set of non-function types, and these boundings are respected in all instantiations of those type variables.

\begin{figure}[t]
  \centering
  \begin{subfigure}[t]{0.4\textwidth}
    \centering
\begin{lstlisting}
let rec append x y =
  match x with
  | Nil -> y
  | Cons (h, t) ->
     Cons (h, append t y)     
\end{lstlisting}
\caption{}
  \end{subfigure}
  ~
  \begin{subfigure}[t]{0.4\textwidth}
        \centering
\begin{lstlisting}
let rec append x y =
  match x with 
  | Nil -> y
  | Cons (h, t) -> 
     let ty = append t y in
     Cons (h, ty)
\end{lstlisting}
\vspace{-1\baselineskip}
\caption{}
  \end{subfigure}
  \vskip2mm
  \begin{subfigure}[t]{0.4\textwidth}
        \centering
\begin{lstlisting}
let rec append$^o$ x y xy =
  (t === ^Nil /\ xy === y) |||
  (fresh (h t ty)
    (x  === ^Cons (h, t)) /\
    (xy === ^Cons (h, ty)) /\
    (append$^o$ t y ty)
  )
\end{lstlisting}
\vspace{-1\baselineskip}
\caption{}
  \end{subfigure}
  \vskip4mm
\caption{Unnesting example}
\label{unnesting_example}
\end{figure}

Finally, we describe the unnesting technique~\cite{TRS}, which was introduced as a method for manual transformation
of functional programs into relational form. Unnesting introduces a new name for each nested subexpression; now, when the value of
each subexpression is bound to a certain variable, the conversion is straightforward: each pattern-matching construct is
transformed into a disjunction, new names, introduced in pattern bindings and unnestings, are transformed into \lstinline|fresh| variables, and
each converted function is supplied with the additional argument, unified with the result. As a result we consider, again, the list
concatenation function (see Fig.~\ref{unnesting_example}a). The result of unnesting is shown on Fig.~\ref{unnesting_example}b, while the
final relational form~--- on Fig.~\ref{unnesting_example}c.

\begin{figure}[t]
\centering
\begin{subfigure}[t]{0.4\textwidth}
 \centering
\begin{lstlisting}
let bar y =
  let f x = x in
  let g a = f in
  g A y
\end{lstlisting}
\vspace{-1\baselineskip}
    \caption{}
  \end{subfigure}
  ~
  \begin{subfigure}[t]{0.4\textwidth}
    \centering
\begin{lstlisting}
let bar$^o$ y r =
  let f x r = x === r in
  let g a r = f === r in
  g ^A y r
\end{lstlisting}
\vspace{-1\baselineskip}
  \caption{}
  \end{subfigure}
  \vskip4mm
\caption{Unnesting: invalid case}
\label{unnesting_invalid}
\end{figure}

However, not every definition can be converted to a relational form by unnesting. Consider, for example, the definition on Fig.~\ref{unnesting_invalid}a.
Unnesting would transform this program into the form, shown on Fig.~\ref{unnesting_invalid}b, which is obviously invalid, since it unifies a
function $f$ with a logical variable $r$. In order to apply unnesting, one needs to $\eta$-expand the definition of $g$, making the functional nature of
its return type syntactically visible.
We stress, that relational conversion, described in Section~\ref{conversion}, is essentially different from unnesting. In particular, 
we use $\eta$-expansion in a very limited manner (only in one case).

\begin{comment}

and for the aforementioned example the result of relational
conversion looks as follows:

\begin{lstlisting}
   let bar$^o$ y =
     let f x = x in
     let g a = f in
     g (fun q. q === ^A) y
\end{lstlisting}

Note, the majority of definitions are left intact; the only difference with the functional version comes from the use of the constructor 
\lstinline|A|, which was transformed into a goal-returning function.

\end{comment}


\section{The Source Language and Relational Extension}

Our development of relational conversion is based on the idea of transforming a program in a functional
language into a program in \emph{relational extension} of that language. In the context of 
miniKanren, this approach looks quite natural, since miniKanren itself, as a DSL, reuses 
many important features (for example, function definitions) from a host language.

In this section, we present a formal description of a small functional language, taken as a source
for relational conversion. We describe its syntax, typing rules, and semantics, and then extend it
with relational constructs. We specify the typing rules and semantics for the extension as well.


\subsection{The Source Language}
\label{source_language}

The syntax of our source functional language is shown on Figure~\ref{functional_syntax}. It consists of a lambda calculus, 
enriched with constructors with fixed arities $C^n$, patterns $p$ and pattern-matching constructs, and  
expressions for recursive/non-recursive let-bindings.
Among the constructors we distinguish two nullary interpreted constructors \lstinline|true| and \lstinline|false|, and add a boolean equality
operator ``$=$''. 

\begin{wrapfigure}{r}{0.5\textwidth}
\centering
%\scalebox{0.9}{
$$
\begin{array}{rcl}
 \mathcal E &=&x\\
     & &\lambda x.e\\
     & &e_1\;e_2\\
     & &C^n(e_1,\dots, e_n)\\
     & &\lstinline|true|\\
     & &\lstinline|false|\\
     & &\lstinline|let $x$ = $e_1$ in $e_2$|\\
     & &\lstinline|let rec $f$ = $\lambda x.e_1$ in $e_2$|\\
     & &e_1\,=\,e_2\\
     & &\lstinline|match $e$ with $\{p_i$ -> $e_i\}$|\\
     & &\\
 \mathcal P &=&C^n(x_1,\dots,x_n)\\
\end{array}
$$
%}
\caption{The syntax of source language}
\label{functional_syntax}
\end{wrapfigure}

In a pattern matching, we only allow shallow patterns (which is not an essential limitation) and do not allow wildcards (which is 
important~--- converting wildcard pattern matching into relational form would require essentially different projections).

\begin{comment}
This choice of a language may 
look quite a restrictive. However, in terms of relational programming, the language contains virtually everything one would need. Indeed, from
a relational conversion standpoint the standard built-in integer arithmetics, for example, is of no use~--- 
there is simply no way to convert integer expression into relational form, using integer expressions. In order to use relational 
arithmetics, one needs to reimplement everything from scratch, using, for example, Peano encoding; but Peano arithmetics can be
easily expressed in the language we present.
\end{comment}

Our language is equipped with Hindley-Milner type system, and we present the typing rules in a conventional syntax-directed form 
on Fig.~\ref{functional_typing}. Besides type variables and function types, our system contains a number of implicitly defined 
algebraic datatypes $T^k$, and we stipulate, that each constructor $C^n$ belongs to the exactly one
datatype. In the rule \textsc{Constr$_T$}, we assume that type $t^C$ has the form $T^k(t_1,\dots,t_k)$, where each of the types
$t_i$ is recovered from the types $t_i^C$ of arguments of constructor $C^n$ and, moreover, these types agree in the sense of
constructor application. Similarly, in the rule \textsc{Match$_T$}, the types of all $C_i^{k_i}(x^i_1,\dots,x^i_{k_i})$ are expected
to be equal $t^C$, and $t^{C_i}_j$ is a type of $j$-th argument of constructor $C_i$, used in the pattern. The rule \textsc{Eq$_T$}
specifies that both operands of equality operator must have the same (but arbitrary) type. Thus, we can call this operator
``polymorphic equality''.

\setarrow{:}
\newcommand{\typed}[3]{\withenv{#1}{\trans{#2}{}{#3}}}

\begin{figure}
\centering
{\bf Types:}
$$
\begin{array}{rcll}
  \mathcal X &=&\alpha, \beta, \dots                            &\mbox{\supp{(type variables)}}\\
  \mathcal D &=&\lstinline|bool|,\,T^n,...                      &\mbox{\supp{(datatype constructors)}}\\
  \mathcal T &=&\alpha\mid T^k(t_1,\dots,t_k)\mid t_1\to t_2 &\mbox{\supp{(types)}}\\
  \mathcal S &=&\forall\bar{\alpha}.t                           &\mbox{\supp{(type schemas)}}
\end{array}
$$
{\bf Typing rules:}
\def\arraystretch{0}
\begin{tabular}{p{7cm}p{7cm}}
$$
\typed{\Gamma}{\lstinline|true|,\;\lstinline|false|}{\lstinline|bool|}
\ruleno{Bool$_T$}
$$ 
&
$$
\trule{\typed{\Gamma}{e_1}{t}\;\;\;\;\typed{\Gamma}{e_2}{t}}
      {\typed{\Gamma}{e_1=e_2}{\lstinline|bool|}}
\ruleno{Eq$_T$}
$$
\\
$$
\trule{\typed{\Gamma}{e_i}{t^C_i}}
      {\typed{\Gamma}{C^n(e_1,\dots,e_n)}{t^C}}
\ruleno{Constr$_T$}
$$
&
$$
\typed{\Gamma,x:\forall\bar{\alpha}.t}{x}{t[\bar{\alpha}\gets\bar{t^\prime}]}
\ruleno{Var$_T$}
$$
\\[-2mm]
$$
\trule{\typed{\Gamma}{f}{t_1\to t_2}\;\;\;\;\typed{\Gamma}{e}{t_1}}
      {\typed{\Gamma}{f\;e}{t_2}}
\ruleno{App$_T$}
$$
&
$$
\trule{\typed{\Gamma,\,x:t_1}{f}{t_2}}
      {\typed{\Gamma}{\lambda x.f}{t_1\to t_2}}
\ruleno{Abs$_T$}
$$
\\[-2mm]
\multicolumn{2}{p{14cm}}{
$$
\trule{\typed{\Gamma}{e_1}{t_1}\;\;\;\;\typed{\Gamma,x:\forall\bar{\alpha}.t_1}{e_2}{t}}
      {\typed{\Gamma}{\lstinline|let $\;x\;$ = $\;e_1\;$ in $\;e_2$|}{t}},\;\bar{\alpha}=FV(t_1)\setminus FV(\Gamma)
\ruleno{Let$_T$}
$$}\\[-2mm]
\multicolumn{2}{p{14cm}}{
$$
\trule{\typed{\Gamma,f:t_1}{\lambda x.e_1}{t_1}\;\;\;\;\typed{\Gamma,f:\forall\bar{\alpha}.t_1}{e_2}{t}}
      {\typed{\Gamma}{\lstinline|let rec $\;f\;$ = $\;\lambda x.e_1\;$ in $\;e_2$|}{t}},\;\bar{\alpha}=FV(t_1)\setminus FV(\Gamma)
\ruleno{LetRec$_T$}
$$}\\[-2mm]
\multicolumn{2}{p{14cm}}{
$$
\trule{\typed{\Gamma}{e}{t^C}\;\;\;\;\typed{\Gamma,x^i_1:t^{C_i}_1,\dots,x^i_{k_i}:t^{C_i}_{k_i}}{e_i}{t}}
      {\typed{\Gamma}{\lstinline|match $\;e\;$ with $\;\{C_i^{k_i}(x^i_1,\dots,x^i_{k_i})$ -> $e_i\}$|}{t}}
\ruleno{Match$_T$}
$$}
\end{tabular}
\caption{Typing rules for the source language}
\label{functional_typing}
\end{figure}


\setarrow{\to}
\newcommand{\step}[2]{\trans{\inbr{#1}}{}{\inbr{#2}}}

\begin{figure}[t]
\centering
{\bf Values:}
$$
\mathcal V = C^n(v_1,\dots,v_n)\mid\lambda x.e\mid\mu f\lambda x.e\mid\lstinline|true|\mid\lstinline|false|
$$
{\bf Contexts:}
$$
\mathcal C = \Box\;e\mid v\;\Box\mid\lstinline|let $x$ = $\Box$ in $e$|\mid\lstinline|match $\;\Box\;$ with $\{p_i$->$e_i\}$|\mid C^n(\bar{v},\Box,\bar{e})\mid\Box\lstinline|=e|\mid\lstinline|v=|\Box
$$
{\bf Stack of contexts:}
$$
\mathcal S=\epsilon\mid\mathcal C : \mathcal S
$$
{\bf States:}
$$
\inbr{\mathcal S, e}\mbox{\supp{(stack of contexts, expression)}};\;\inbr{\epsilon,e}\mbox{\supp{(initial state)}};\;\inbr{\epsilon,v}\mbox{\supp{(final state)}}
$$
{\bf Transitions:}
\vskip2mm
\bgroup
\def\arraystretch{0}
\begin{tabular}{p{7cm}p{7cm}}
\multicolumn{2}{p{14cm}}{
$$
\step{C:\mathcal S,\, v}{\mathcal S,\, C[v]}\ruleno{Value}
$$}\\[-4mm]
$$
\step{\mathcal S,\, f\;e}{\Box\;e:\mathcal S,\, f}\ruleno{AppL}
$$&
$$
\step{\mathcal S,\, v\;e_2}{v\;\Box:\mathcal S,\, e_2}\ruleno{AppR}
$$\\[-4mm]
$$
\step{\mathcal S,\,e_1=e_2}{\Box=e_2:\mathcal S,\,e_1}\ruleno{EqL}
$$&
$$
\step{\mathcal S,\,v=e}{v=\Box:\mathcal S,\,e}\ruleno{EqR}
$$\\[-4mm]
\multicolumn{2}{p{14cm}}{
$$
\step{\mathcal S,\,v=v}{\mathcal S,\,\lstinline|true|}\ruleno{EqTrue}
$$}\\[-4mm]
\multicolumn{2}{p{14cm}}{
$$
\step{\mathcal S,\,v_1=v_2}{\mathcal S,\,\lstinline|false|},\;v_1\ne v_2\ruleno{EqFalse}
$$}\\[-4mm]
\multicolumn{2}{p{14cm}}{
$$
\step{\mathcal S,\, (\lambda x.e)\;v}{\mathcal S,\, e[x\gets v]}\ruleno{Beta}
$$}\\[-4mm]
\multicolumn{2}{p{14cm}}{
$$
\step{\mathcal S,\, (\mu f\lambda x.e)\;v}{\mathcal S,\, e[f\gets\mu f\lambda x.e,\, x\gets v]}\ruleno{Mu}
$$}\\[-4mm]
\multicolumn{2}{p{14cm}}{
$$
\step{\mathcal S,\, C^n(v_1,\dots,v_{k-1},e_k,\dots,e_n)}{C^n(v_1,\dots,v_{k-1},\Box,\dots,e_n):\mathcal S,\, e_k}\ruleno{Constr}
$$}\\[-4mm]
\multicolumn{2}{p{14cm}}{
$$
\step{\mathcal S,\, \lstinline|let $\;x\;$ = $\;e_1\;$ in $\;e_2$|}{\lstinline|let $\;x\;$ = $\;\Box\;$ in $\;e_2$|:\mathcal S,\, e_1}\ruleno{Let}
$$}\\[-4mm]
\multicolumn{2}{p{14cm}}{
$$
\step{\mathcal S,\, \lstinline|let $\;x\;$ = $\;v\;$ in $\;e$|}{\mathcal S,\,e[x\gets v]}\ruleno{LetVal}
$$}\\[-4mm]
\multicolumn{2}{p{14cm}}{
$$
\step{\mathcal S,\, \lstinline|let rec $\;f\;$ = $\;\lambda x.e_1\;$ in $\;e_2$|}{\mathcal S,\, e_2[f\gets\mu f\lambda x.e_1]}\ruleno{LetRec}
$$}\\[-4mm]
\multicolumn{2}{p{14cm}}{
$$
\step{\mathcal S,\,\lstinline|match $\;e\;$ with $\;\{p_i$->$e_i\}$|}{\lstinline|match $\;\Box\;$ with $\;\{p_i$->$e_i\}$|:\mathcal S,\, e}\ruleno{Match}
$$}\\[-4mm]
\multicolumn{2}{p{14cm}}{
$$
\step{\mathcal S,\,\lstinline|match $\;C_k^{n_k}(v_1,\dots,v_{n_k})\;$ with $\;\{C_i^{n_i}(x^i_1,\dots,x^i_{n_i})\to e_i\}$|}{\mathcal S,\,e_k[x^k_j\gets v_j]}\ruleno{MatchVal}
$$}
\end{tabular}
\egroup
\caption{Semantics for the source language}
\label{functional_semantics}
\end{figure}

We describe the semantics of our language in the form of transition system. The transition relation

$$
\step{\mathcal S,\,e}{\mathcal S^\prime,\,e^\prime}
$$

\noindent describes a one step of evaluation of expression $e$ with a stack of contexts $\mathcal S$, which results in
a new stack $\mathcal S^\prime$ and a new expression $e^\prime$. A context is an expression with a unique hole; informally speaking, 
a stack of contexts describes a path in the expression being evaluated from the topmost construct to the point, where the evaluation 
currently is taking place. For a context $C$ and an expression $e$, we denote by $C[e]$ a complete expression with no holes, which is 
obtained by plugging $e$ into the unique hole of $C$. From each state $\inbr{C_1:C_2:\dots:C_k,e}$ we can build an 
expression $C_k[\dots[C_2[C_1[e]]]\dots]$, which represents an intermediate result of evaluation according to a small-step semantics. 
This form of semantic description originates from Felleisen-style~\cite{Felleisen} approach for small-step semantics, and we've
chosen it since it can be naturally extended for a relational case.

Our semantics describes call-by-value left-to-right evaluation; in the rules $\textsc{Beta}$, $\textsc{Mu}$, $\textsc{LetVal}$,
$\textsc{LetRec}$ and $\textsc{MatchVal}$, we perform capture-avoiding substitutions, which respect the names in abstractions and let-bindings.
In the rule $\textsc{MatchVal}$ we assume, that at most one pattern matches the scrutinee~--- this is an important difference from the usual 
semantics of pattern matching, when the patterns are examined in a top-down manner until the matching succeeds. In the rules $\textsc{EqTrue}$
and $\textsc{EqFalse}$ we assume, that the values $v$, $v_1$, $v_2$ do not have the forms $\lambda x\dots$ or $\mu f\dots$.

Finally, for a closed expression $e$ and a value $v$, we write $e \leadsto^f v$, iff 

$$\inbr{\epsilon,e}\to^*\inbr{\epsilon,v}$$

\noindent where $\epsilon$~--- an empty stack, and ``$\to^*$'' is a reflexive-transitive closure of ``$\to$''. 


\subsection{Relational Extension}
\label{relational_extension}

The relational extension adds five conventional miniKanren expressions for constructing goals; the syntax is shown on Fig.~\ref{relational_syntax}.
Since relational constructs are added to regular functional ones, it becomes possible to construct expressions like \lstinline|fun x.(x /\ fun y.y)|, etc.
In order to rule such pathological expressions out, we devised an extension for the type system of the source language. In fact, this approach follows the
actual implementation for OCaml, where a careful choice of types for representing terms and goals made it possible to reject the majority of non-well-formed
programs at compile-time.

Our extension for the type system introduces one interpreted datatype constructor $\Box^o$ with one data constructor $\uparrow$~--- a polymorphic type and
a constructor for logical terms. In addition, we introduce an interpreted type of goals $\G$, which is distinct from all other types. The typing rules for the relational 
extension are shown on Fig.~\ref{relational_typing}. These rules describe rather expected typing: in unification and disequality constraints only
terms of the same logical type can be used, and conjunction and disjunction can only be taken for goals. Note, in our extension a term can be calculated as
a result of arbitrary expression in initial functional language (as long as this expression has expected logical type), but such ``higher-order'' terms will
never appear as a result of relational conversion, so, in fact, relational extension we describe here defines a richer language, than we actually need.

The semantics of extended language is shown on Fig.~\ref{relational_semantics}. First, the state is extended: besides the stack of contexts and
current expression it now contains a set of used \emph{semantic variables} $\Sigma$ and a \emph{logical state} $\sigma$. 
Semantic variables are allocated and substituted for syntactic logic variable occurrences, when \lstinline|fresh| expression is evaluated 
(see rule \textsc{Fresh}). Logical states are affected, when unification or disequality constraint is evaluated; we explain them
in details below. All existing rules for the initial language are considered rewritten to propagate newly added components of states unchanged.
Then, we modify the substitution to respect names, bound in \lstinline|fresh| as well. 
Next, we consider two new kinds of values: a semantic variable and a special value \lstinline|success|. The former is a result of evaluation for
a free logic variable, the latter~--- the result of evaluation for a succeeded goal.

\begin{wrapfigure}[11]{r}{0.5\textwidth}  
  \centering
  \vspace{-11pt}
  $$
  \begin{array}{rl}
    \mathcal E\mathrel{{+}{=}}&\lstinline|fresh ($x$) $\;e$|\\
    &e_1\equiv e_2\\
    &e_1\not\equiv e_2\\
    &e_1\vee e_2\\
    &e_1\wedge e_2
  \end{array}
  $$
  \caption{The syntax of relational extension}
  \label{relational_syntax}
\end{wrapfigure}

We also extend the definition of context to handle the new kinds of expressions. In unification and disequality constraint, the terms are evaluated left-to-right.
Conjunction and disjunction, however, evaluate nondeterministically: in disjunction only one subgoal is chosen (see rules \textsc{DisjL} and \textsc{DisjR}),
a conjunction can evaluate either left, or right subgoal first (see rules \textsc{ConjStartL} and \textsc{ConjStartR}). When chosen subgoal is evaluated
to the value \lstinline|success|, the other subgoal starts its evaluation (rules \textsc{ConjL} and \textsc{ConjR}).
We have chosen a nondeterministic variant for the semantics, since different existing miniKanren implementations use (a little bit) different search, and we do 
not want to depend on the implementation details. An opposite side of this solution is that for a concrete program and a concrete miniKanren implementation,
the result of the evaluation might not coincide with that, prescribed by the semantics: in some concrete implementation a program can diverge, while
nondeterministic semantics may still define a certain scenario to complete with a result. We argue, that in this case, it will always be possible to
rewrite a program or/and interpreter to converge according to that scenario.
\FloatBarrier

\setarrow{:}
\begin{figure}[t]
\centering
{\bf Types:}
$$
\begin{array}{rcl}
 \mathcal L &=               &\alpha^o \mid (T^n(l_1,\dots,l_n))^o\;\;\mbox{\supp{(logical types)}}\\
 \mathcal T &\mathrel{{+}{=}}&\G
\end{array}
$$
{\bf Typing rules:}
\def\arraystretch{0}
\begin{tabular}{p{7cm}p{7cm}}
\multicolumn{2}{p{14cm}}{
$$
\trule{\typed{\Gamma,x:l}{e}{\G}}
      {\typed{\Gamma}{\lstinline|fresh ($x$) $\;e$|}{\G}}
\ruleno{Fresh$_T$}
$$}\\[-2mm]
$$
\trule{\typed{\Gamma}{e_1}{l}\;\;\;\;\typed{\Gamma}{e_2}{l}}
      {\typed{\Gamma}{e_1\equiv e_2}{\G}}
\ruleno{Unify$_T$}
$$&
$$
\trule{\typed{\Gamma}{e_1}{l}\;\;\;\;\typed{\Gamma}{e_2}{l}}
      {\typed{\Gamma}{e_1\not\equiv e_2}{\G}}
\ruleno{Disequality$_T$}
$$\\[-2mm]
$$
\trule{\typed{\Gamma}{e_1}{\G}\;\;\;\;\typed{\Gamma}{e_2}{\G}}
      {\typed{\Gamma}{e_1\wedge e_2}{\G}}
\ruleno{Conjunction$_T$}
$$&
$$
\trule{\typed{\Gamma}{e_1}{\G}\;\;\;\;\typed{\Gamma}{e_2}{\G}}
      {\typed{\Gamma}{e_1\vee e_2}{\G}}
\ruleno{Disjunction$_T$}
$$
\end{tabular}
\caption{Typing rules for the relational extension}
\label{relational_typing}
\end{figure}

\setarrow{\leadsto}
\def\arraystretch{0}
\begin{figure}[t]
\centering
{\bf Semantic variables:}\vspace{-2mm}
\begin{gather*}
\mathfrak S = \mathfrak s_1, \mathfrak s_2, \dots\\[-2mm]
\Sigma, \Sigma^\prime\dots \subset 2^{\mathcal S}\;\mbox{\supp{(sets of allocated semantics variables)}}\\[-1mm]
\inbr{\Sigma^\prime, \mathfrak s}\gets\lstinline|new|\;\Sigma,\;\Sigma^\prime=\Sigma\cup\{\mathfrak s\},\;{\mathfrak s}\notin\Sigma\;\mbox{\supp{(allocation of a new semantic variable)}}\vspace{-2mm}
\end{gather*}
{\bf Values:}\vspace{-2mm}
$$
\mathcal V \mathrel{{+}{=}} \lstinline|success|\mid\mathfrak s
$$\vspace{-2mm}
{\bf Contexts:}\vspace{-2mm}
$$
\mathcal C \mathrel{{+}{=}}\Box\equiv e\mid v\equiv\Box\mid\Box\not\equiv e\mid v\not\equiv\Box\mid\Box\wedge e\mid e\wedge\Box
$$\vspace{-2mm}
{\bf States:}\vspace{-2mm}
\begin{gather*}
\inbr{\Sigma,\mathcal S,e,\sigma}\mbox{\supp{(set of allocated semantic variables, stack of contexts, expression, logical state)}}\\[-1mm]
\inbr{\emptyset,\epsilon,e,\iota}\mbox{\supp{(initial state)}}
\end{gather*}\vspace{-2mm}
{\bf Transitions:}\vspace{1mm}
{\def\arraystretch{0}
\begin{tabular}{p{14cm}}
$$
\step{\Sigma,\,\mathcal S,\,\lstinline|fresh($x$) $\;e$|,\,\sigma}{\Sigma^\prime,\,\mathcal S,\,e[x\gets\mathfrak s],\,\sigma},\,\inbr{\Sigma^\prime,\mathfrak s}\gets\lstinline|new|\;\Sigma\ruleno{Fresh}
$$\\[-4mm]
$$
\step{\Sigma,\,\mathcal S,\,e_1\equiv e_2,\,\sigma}{\Sigma,\,\Box\equiv e_2:\mathcal S,\,e_1,\,\sigma}\ruleno{UnifyL}
$$\\[-4mm]
$$
\step{\Sigma,\,\mathcal S,\,v\equiv e,\,\sigma}{\Sigma,\,v\equiv\Box:\mathcal S,\,e,\,\sigma}\ruleno{UnifyR}
$$\\[-4mm]
$$
\step{\Sigma,\,\mathcal S,\,v_1\equiv v_2,\,\sigma}{\Sigma,\,\mathcal S,\,\lstinline|success|,\,\sigma^\prime},\,{\bf unify}\,(\sigma,\,v_1,\,v_2)=\sigma^\prime\ruleno{Unify}
$$\\[-4mm]
$$
\step{\Sigma,\,\mathcal S,\,e_1\not\equiv e_2,\,\sigma}{\Sigma,\,\Box\not\equiv e_2:\mathcal S,\,e_1,\,\sigma}\ruleno{DisEqL}
$$\\[-4mm]
$$
\step{\Sigma,\,\mathcal S,\,v\not\equiv e,\,\sigma}{\Sigma,\,v\not\equiv\Box:\mathcal S,\,e,\,\sigma}\ruleno{DisEqR}
$$\\[-4mm]
$$
\step{\Sigma,\,\mathcal S,\,v_1\not\equiv v_2,\,\sigma}{\Sigma,\,\mathcal S,\,\lstinline|success|,\,\sigma^\prime},\,{\bf diseq}\,(\sigma,\,v_1,\,v_2)=\sigma^\prime\ruleno{DisEq}
$$\\[-4mm]
$$
\step{\Sigma,\,\mathcal S,\,e_1\vee e_2,\,\sigma}{\Sigma,\,\mathcal S,\,e_1,\,\sigma}\ruleno{DisjL}
$$\\[-4mm]
$$
\step{\Sigma,\,\mathcal S,\,e_1\vee e_2,\,\sigma}{\Sigma,\,\mathcal S,\,e_2,\,\sigma}\ruleno{DisjR}
$$\\[-4mm]
$$
\step{\Sigma,\,\mathcal S,\,e_1\wedge e_2,\,\sigma}{\Sigma,\,\Box\wedge e_2:\mathcal S,\,e_1,\,\sigma}\ruleno{ConjStartL}
$$\\[-4mm]
$$
\step{\Sigma,\,\mathcal S,\,e_1\wedge e_2,\,\sigma}{\Sigma,\,e_1\wedge\Box:\mathcal S,\,e_2,\,\sigma}\ruleno{ConjStartR}
$$\\[-4mm]
$$
\step{\Sigma,\,\mathcal S,\,\lstinline|success|\wedge e,\,\sigma}{\Sigma,\,\mathcal S,\,e,\,\sigma}\ruleno{ConjL}
$$\\[-4mm]
$$
\step{\Sigma,\,\mathcal S,\,e\wedge\lstinline|success|,\,\sigma}{\Sigma,\,\mathcal S,\,e,\,\sigma}\ruleno{ConjR}
$$
\end{tabular}}
\caption{Semantics for the relational extension}
\label{relational_semantics}
\end{figure}

Finally, we describe the structure of a logical state and the implementation of unification and disequality constraint. The development is mainly based on the existing implementation~\cite{CKanren} and standard approaches for implementing unification~\cite{Unification,UnificationRevisited}. We, therefore, assume the familiarity of the reader with the following notions:

\begin{itemize}
  \item substitution ($\theta$);
  \item application of substitution $\theta$ to a term $t$ ($t\,\theta$);
  \item composition of substitutions ($\theta\theta^\prime$);
  \item most general unifier of two terms ($mgu\,(t_1, t_2)$).
\end{itemize}

\begin{comment}

As it can be seen from the semantics and typing rules, a unification (or disequality constraint) can only
be applied to equally-typed logical values, and we consider substitutions to be partial functions from
semantic variables ($\mathfrak S$) to logical values.

Before giving the detailed description, we consider the following example, which is called to reveal the essence of unification and
disequality constraint coexistence. Note, only these two kinds of goals deliver new information~--- all other kinds (conjunction, disjunction, etc.) are rather needed to
put them in a right context or order. Let us have the following sequence of goals, which we have to evaluate one after another (we assume all logical 
variables are already properly allocated):

$$
\def\arraystretch{1}
\begin{array}{rcl}
x&\not\equiv&y\\
z&\not\equiv&y\\
y&\equiv&\lstinline|A|\\
x&\equiv&\lstinline|B|\\
z&\equiv&\lstinline|A|
\end{array}
$$

Since in the beginning we know nothing yet, we are unable to determine, if the first goal~--- \mbox{$x\not\equiv y$}~--- succeeds or fails right now. All we can do is to
remember the substitution \mbox{$[x\binds y]$} with the note, that we \emph{do not want} to allow it. We call this kind of substitutions \emph{negative}. Similarly,
after the next goal we have another negative substitution \mbox{$[z\binds y]$}. 

The next goal is the unification \mbox{$y\equiv\lstinline|A|$}. Obviously, it provides us with the substitution \mbox{$[y\binds\lstinline|A|]$}. What effect (if any)
should it have on the set of negative substitutions? Since the unification completely eliminates the variable $y$, replacing negative substitutions with
\mbox{$[x\binds\lstinline|A|]$}, \mbox{$[z\binds\lstinline|A|]$} looks reasonable.

The next goal is \mbox{$x\equiv\lstinline|B|$}. It extends the current substitution, making it \mbox{$[y\binds\lstinline|A|,\,x\binds\lstinline|B|]$}. Now we may
note, that since $x$ already became \lstinline|B|, it can never be \lstinline|A| anymore. Thus, we can drop \mbox{$[x\binds\lstinline|A|]$} from the set of
negative substitutions.

Finally, we unify $z$ with \lstinline|A|. This, however, gets us a substitution \mbox{$[z\binds\lstinline|A|]$}, which we have to disallow since it
matches with the negative one.

In other words, disequality constraints may not succeed or fail right away, and the results of unification have to be additionally matched against the set of 
previously encountered disequality constraints, which, in turn, can be altered by a unification or another disequality constraint. Now we can continue.
\end{comment}

A logical state contains two components

$$
\sigma=(\theta,\Theta^-)
$$

\noindent where $\theta$ is a substitution, $\Theta^-$~--- a set of negative substitutions, describing disequality constraints, 
which can potentially be violated. The initial state contains undefined substitution and empty set:

$$
\iota=(\bot,\emptyset)
$$

The effect of unification is described by the following primitive:

$$
{\bf unify}\,(\sigma,\,t_1,\,t_2)={\bf unify}\,((\theta,\Theta^-),\,t_1,\,t_2)
$$

First, it calculates the most general unifier for the terms under consideration w.r.t. current substitution:

$$
\rho=mgu\,(t_1\,\theta,t_2\,\theta)
$$

If there is no such $\rho$, the unification fails, and the evaluation terminates unsuccessfully. Otherwise,
$\rho$ has to be checked against the disequality constraints, represented by $\Theta^-$ (if $\Theta^-$ is empty, the
check succeeds immediately).

Being a substitution, $\rho$ at the same time can be considered as the following unification problem: we can try to unify a pair of terms 

$$
\begin{array}{rcl}
t_l&=&(\mathfrak s_1,\dots,\mathfrak s_k)\\
t_r&=&(\rho(\mathfrak s_1),\dots,\rho(\mathfrak s_k))
\end{array}
$$

\noindent where $\{\mathfrak s_i\}=dom\,(\rho)$. We pick every substitution $\theta^-\in\Theta^-$ and calculate 
the $mgu\,(t_l\,\theta^-,t_r\,\theta^-)$. There are three possible outcomes:

\begin{enumerate}
\item The unification fails. This means, that disequality constraint, represented by $\theta^-$, can no
longer be violated. We remove $\theta^-$ from $\Theta^-$ and continue with the next disequality constraint.
\item The unification succeeds with the empty substitution. This means, that 
disequality constraint, represented by $\theta^-$, is violated. The check stops, and the whole top-level 
unification fails.
\item The unification succeeds with a non-empty substitution $\theta^{\prime-}$. This means, that in order not to 
violate disequality constraint, represented by $\theta^-$, $\theta^{\prime-}$ has to be respected. We replace
$\theta^-$ with $\theta^{\prime-}$ in $\Theta^-$ and continue with the next disequality constraint.
\end{enumerate}

If the disequality check succeeds, by the end we have a modified set $\Theta^{\prime-}$, and we assume

$$
{\bf unify}\,((\theta,\Theta^-),\,t_1,\,t_2)=(\theta\rho,\Theta^{\prime-})
$$

The evaluation of disequality constraint is performed in a similar manner using the primitive

$$
{\bf diseq}\,(\sigma,\,t_1,\,t_2)={\bf diseq}\,((\theta,\Theta^-),\,t_1,\,t_2)
$$

First, the $mgu\,(t_1\,\theta,t_2\,\theta)$ is calculated. Again, there are three
possible cases:

\begin{enumerate}
\item The unification fails. This means, that disequality constraint is satisfied.
\item The unification succeeds with the empty substitution. This means, that disequality
constraint is violated.
\item The unification succeeds with a non-empty substitution $\theta^{\prime-}$. This means, that 
this substitution describes the disequality constraint, which has to be respected in
the future, so we add it to $\Theta^-$. 
\end{enumerate}

If disequality constraint succeeds, we obtain a (potentially) modified set $\Theta^{\prime-}$, and we
assume

$$
{\bf diseq}\,((\theta,\Theta^-),\,t_1,\,t_2)=(\theta,\Theta^{\prime-})
$$

Finally, for a closed goal $g$ and a logical state $\sigma$, we define $g \leadsto^r \sigma$, iff

$$
\inbr{\emptyset,\epsilon,g,\iota}\leadsto^*\inbr{\Sigma,\epsilon,\lstinline|success|,\sigma}\mbox{ for some $\Sigma$}
$$
 
\noindent where ``$\leadsto^*$'' is a reflexive-transitive closure of ``$\leadsto$''. 

One may notice, that the typing rules for the relational extension add nothing more than some
interpreted types and symbols w.r.t. the type system of the substrate language. Thus, it 
is rather expected, that the relational extension inherits all its useful properties (like progress and
type preservation). Surprisingly, this is not completely so. Indeed, the only value for goals is
\lstinline|success|, but, obviously, not every goal succeeds (for example, \lstinline|A === B| always
fails). Thus, our relational extension lacks the progress property~--- a decently typed non-value
goal sometimes cannot make a step. This makes no harm in the context of the paper; in any case,
a failure value for goals can be added to the language together with the failure propagation rules. 




\section{Relational Conversion}
\label{conversion}
\def\arraystretch{1}

Before we describe the relational conversion itself, we formulate some limitations for the source
programs. Functional programs tend to operate with higher-order values, while miniKanren is
limited by a first-order unification. Therefore, it would be unreasonable to expect, that arbitrary
functional program can be converted into a relational form (at least using reasonably simple 
transformations). 

We introduce the set of ground types $\mathcal G$:

$$
\mathcal G=\alpha \mid T^k(g_1,\dots,g_k)
$$

Informally, a value of a ground type cannot contain closures. Then we formulate the following limitations for
the programs to be converted into a relational form:

\begin{itemize}
  \item all constructor parameter types must be type variables;
  \item constructors and polymorphic equality can only be applied to the values of ground types;
  \item all \lstinline|match|-expressions must be of ground types.
\end{itemize}

The first condition means, that all algebraic datatypes (which we consider as defined implicitly, see Section~\ref{source_language}) 
have to be fully-polymorphic. The first two limitations then allow us to specify the polymorphism restriction for 
relational programs, which we mentioned informally in Section~\ref{ocanren}: all type variables are bounded to
range only over ground types (this condition, of course, is sufficient, but not necessary).

The third limitation is not essential and introduced only to simplify the presentation. If a \lstinline|match|-expression does not
have a ground type, it can always be transformed to have one by applying $\eta$-expansion:

\begin{lstlisting}
   match $e$ with {$p_i$ -> $e_i$} $\leadsto$ fun $\bar{x}$.match $e$ with {$p_i$ -> $e_i\,\bar{x}$}
\end{lstlisting}

\noindent where $\bar{x}$ is a vector of new variables, different from those in $e$, $e_i$, and $p_i$. In fact, our implementation,
described in Section~\ref{evaluation}, performs this expansion as long as a non-ground type \lstinline|match|-expression is encountered. 
This is the single case when we actually inspect types and perform $\eta$-expansion.

The general idea behind the conversion can be illustrated on a type level: an expression of type $t$ in the source
language is transformed into the expression of type $\sembr{t}^t$ in relational extension, where
the transformation $\sembr{\bullet}^t$ is defined as follows:

$$
\begin{array}{rcl}
\sembr{g}^t                     & = & g \to \G \\
\sembr{t_1 \to t_2}^t           & = & \sembr{t_1}^t \to \sembr{t_2}^t \\
%\sembr{\forall \alpha. \: t} & = & \forall \alpha. \: \sembr{t}
\end{array}
$$

In other words, an expression of a ground type is converted into a goal-returning function. The informal semantics
of this function is to make its argument respect a certain contract. As the argument can have some free variable occurrences, 
the goal tries to substitute these variables with some values in order to respect the contract this goal represents. 
For example, a constant \lstinline|Nil| is converted into a function \lstinline|fun $q$ . $q\,$=== ^Nil|.

The conversion itself is described in terms of transformation $\sembr{\bullet}^c$, see Fig.~7. %\ref{relational_conversion}. 
The first five rules
simply propagate the conversion through the expression; the last three actually do the work. These rules themselves may look complicated,
but the idea is rather simple.

\begin{figure}[t]
  \centering
  \begin{tabular}{rcp{6cm}}
     $\sembr{x}^c$                &=&$x$\\
     $\sembr{\lambda x.e}^c$      &=&$\lambda x.\sembr{e}^c$\\
     $\sembr{f\;e}^c$             &=&$\sembr{f}^c\;\sembr{e}^c$\\
     $\sembr{\lstinline|let $\;x\;$ = $\;e_1\;$ in $\;e_2$|}^c$&=&\lstinline|let $x$ = $\sembr{e_1}^c$ in $\sembr{e_2}^c$|\\
     $\sembr{\lstinline|let rec $\;f\;$ = $\lambda x.e_1\;$ in $\;e_2$|}^c$&=&\lstinline|let rec $f$ = $\sembr{\lambda x.e_1}^c$ in $\sembr{e_2}^c$|\\[2mm]
     $\sembr{C^k (e_1,\dots,e_k)}^c$&=&\lstinline|fun $q$.fresh ($q_1 \dots q_k$)|
\begin{lstlisting}
  ($\sembr{e_1}^c\; q_1$) /\
  ...
  ($\sembr{e_k}^c\; q_k$) /\
  ($q$ === $\;\uparrow(C^n (q_1, \dots, q_k)$))
\end{lstlisting}\\[-2mm]
     $\sembr{\lstinline|match $\;e\;$ with \{$C^{n_i}_i(x^i_1,\dots,x^i_{n_i})\;$ -> $\;e_i$\}|}^c$&=&\lstinline|fun $q$.fresh ($q_e$)|
\begin{lstlisting}
    ($\sembr{e}^c\;q_e$) /\
    $\bigvee_i$ ((fresh ($q^i_1\dots q^i_{n_i}$)
           ($q_e$ === $\;\uparrow C^{n_i}_i(q^i_1,\dots,q^i_{n_i})$) /\
           (fun $x^i_1\dots x^i_{n_i}$.$\sembr{e_i}^c$) ($\equiv q^i_1$) ... ($\equiv q^i_{n_i}$) $q$
     ) 
    )
\end{lstlisting}\\[-2mm]
     $\sembr{\lstinline|$e_1\,$=$\,e_2$|}^c$&=&\lstinline|fun $q$.fresh ($q_1\,q_2$)|
\begin{lstlisting}
  $\sembr{e_1}^c\,q_1$ /\
  $\sembr{e_2}^c\,q_2$ /\
  (($q_1$ === $\;q_2$ /\ $q$ === $\;$^true) |||
   ($q_1$ =/= $\;q_2$ /\ $q$ === $\;$^false)
  )
\end{lstlisting}
  \end{tabular}
\label{relational_conversion}
\caption{Relational conversion}
\end{figure}

In the case of constructor we know, that all expressions $e_i$ have ground types. Thus, their relational images are goal-returning
functions. We create a set of fresh variables (one for each expression) and pass them as arguments to these functions to associate
them with the values of the expressions. The result of conversion for the constructor application itself has to be a 
goal-returning function as well. We surround expression constructed so far with abstraction and unify its argument $q$ with the
constructor, applied to corresponding logical variables. We also apply logical constructor $\uparrow$ to respect the typing rule
for unification.

The rule for pattern-matching conversion operates similarly. First, the scrutinee must have a ground type (since it is matched against
constructors). We create a fresh variable $q_e$ and associate it with the value of the scrutinee exactly as in the previous
case. Then, for each branch we create a number of fresh variables (one for each variable in the pattern for the branch) and
express pattern-matching in terms of unification, using these variables and corresponding constructor. Finally, the body $e_i$ of the branch
is an expression with free variables, corresponding to those in the pattern. We, therefore, convert $e_i$ and surround the result with
lambdas, closing all these variables. To pass the bindings $q^i_j$ for pattern variables to the body, we apply this function to
 goal-returning functions $(\equiv q^i_j)$. This, again, gives us a goal-returning function, which we apply to the topmost result variable $q$.

The last rule follows the same pattern: both arguments of polymorphic equality are transformed into goal-returning functions, and we know, that
the arguments of these functions are of some ground type. We apply these functions to fresh variables and perform case analysis. Note, this is
the only case when we actually use disequality constraints.

An interesting property of relational conversion is that it does not change terms, which do not use constructors, equality, and pattern-matching. Thus,
a lot of useful higher-order functions~--- application, composition, fixed point, etc.~--- are already relational and can be used in
relational specifications.

Another observation is that our transformation is compositional (a relational image of application is an application of relational
images). This means, that relational conversion is compatible with separate compilation~--- multiple source files can be
converted independently without losing the possibility to work properly when combined.

Then, it is interesting, that the result of relational conversion runs in a forward direction
deterministically. Thus, relational conversion imposes only a constant-time slowdown in a forward
direction.

Finally, we formulate the following properties for relational conversion:

\begin{itemize}
\item Static correctness: if an expression $e$ has a type $t$ in the source language, then $\sembr{e}^c$ has a 
type $\sembr{t}^t$ in relational extension. In other words, relational conversion transforms properly typed
programs into properly typed. Proof is by structural induction (and trivial).
\item Partial semantic correctness: if an expression $e$ has a ground type $t$ and \mbox{$e \leadsto^f v$} for some
  value $v$, then \mbox{$\lstinline|fresh($x$)($\sembr{e}^c\;x$)| \leadsto^r (\theta,\emptyset)$}, and 
\mbox{$\theta(\mathfrak{s})=v$}, where $\mathfrak{s}$ is a semantic variable, associated with $x$ on the
first step of the relational evaluation.
%The essential part of the proof is given in the Appendix~\ref{appendix}.
%Proof
%is by induction on the length of derivation sequence (a number of lemmas have to be justified on the way).
\end{itemize}

In order to prove the complete correctness, we need some means to interpret the results of relational 
derivation with free variables in functional case. This is a subject of future research.

\section{Evaluation}

\label{sec:evaluation}

In this section, we present an evaluation of 
implemented constructive negation on a series of examples.

\subsection{If-then-else}

Using relational if-then-else operator, 
presented in section~\ref{sec:ifte},
we have implemented several 
higher-order relations over lists, namely 
\lstinline{find} (Listing~\ref{lst:eval-find}), 
\lstinline{remove}\footnote{Note, this implementation 
differs from the one in Section~\ref{sec:intro}, but 
it is easy to see that these two are semantically equivalent.} (Listing~\ref{lst:eval-remove}) 
and \lstinline{filter} (Listing~\ref{lst:eval-filter}).
These relations are almost identical (syntactically) to their
functional implementations.
We have tested that these relations can be run
in various directions and produce the expected results.
For example, the goal \lstinline{filter p q q}
with the predicate \lstinline{p} equal to

\begin{lstlisting}
  fun l -> fresh (x) (l === [x])
\end{lstlisting}

stating that the given list should be a singleton list,
starts to generate all singleton lists.
Vice versa, the goal \lstinline{filter p q []} 
with that same \lstinline{p} generates 
all lists, constrained to be not a singleton list.

Listings~\ref{lst:eval-p}-\ref{lst:eval-filter-queries} give 
more concrete examples of queries to these relations.
In the listing the syntax \lstinline{run n q g}
means running a goal \lstinline{g} with 
the free variable \lstinline{q}
taking the first \lstinline{n} answers (``\lstinline{*}'' denotes all answers).
After the sign $\leadsto$ the result of the query is given.
The result \lstinline{fail} means that the query has failed.
The result \lstinline[mathescape]|succ {{a$_1$}; ... {a$_n$}} |
means that the query has succeeded delivering $n$ answers.
Each answer represents a set of constraint on free variables.
Constraints are of two forms: equality constraints, e.g. \lstinline{q = (1, _.$_0$)}, 
or disequality constraints, e.g. \lstinline{q $\neq$ (1, _.$_0$)}.
The terms of the form \lstinline{_.$_i$} in the answer
denote some universally quantified variables.

\begin{minipage}[thb]{.3\textwidth}
\begin{lstlisting}[
  caption={A definition of \code{find} relation},
  label={lst:eval-find}
]
let find p e xs =
  fresh (x xs' ys') (
    xs === x::xs' /\
    ifte (p x)
      (e === x)
      (find p e xs')
  )
\end{lstlisting}
\end{minipage}\hfill
\begin{minipage}[thb]{.3\textwidth}
\begin{lstlisting}[
  caption={A definition of \code{remove} relation},
  label={lst:eval-remove}
]
let remove p xs ys =
  (xs === [] /\ ys === [])
  \/
  fresh (x xs' ys') (
    xs === x::xs' /\
    ifte (p x)
      (ys === xs')
      (ys === x::ys' /\ 
       remove p xs' ys')
  )
\end{lstlisting}
\end{minipage}\hfill
\begin{minipage}[thb]{.3\textwidth}
\begin{lstlisting}[
  caption={A definition of \code{filter} relation},
  label={lst:eval-filter}
]
let filter p xs ys =
  (xs === [] /\ ys === [])
  \/
  fresh (x xs' ys') (
    xs === x::xs' /\
    (ifte (p x)
      (ys === x :: ys')
      (ys === ys')) /\
    filter p xs' ys'
  )
\end{lstlisting}
\end{minipage}

% \vspace{3cm}

\begin{minipage}[thb]{0.4\textwidth}
\begin{lstlisting}[
  caption={Definition of the predicate \lstinline{p}},
  label={lst:eval-p}
]
let p l = fresh (x) (l === [x])
\end{lstlisting}
\begin{lstlisting}[
  caption={Example of queries to \lstinline{find}},
  label={lst:eval-find-queries}
]
run 3 q (fresh (e) find p e q) 
$\leadsto$ succ {
     { q = [_.$_0$] :: _.$_1$ }
     { q = _.$_0$ :: [_.$_1$] :: _.$_2$; 
         _.$_0$ $\neq$ [_.$_3$] }
     { q = _.$_0$ :: _.$_1$ :: [_.$_2$] :: _.$_3$; 
         _.$_0$ $\neq$ [_.$_4$]; _.$_1$ $\neq$ [_.$_5$] }
   }
\end{lstlisting}
\end{minipage}\hfill
\begin{minipage}[thb]{0.4\textwidth}
\begin{lstlisting}[
  caption={Example of queries to \lstinline{remove}},
  label={lst:eval-remove-queries}
]
run * q (fresh (e) remove p q [[ ]]) 
$\leadsto$ succ {
     { q = [[_.$_0$]; [ ]] }
     { q = [[ ]] }
     { q = [[ ]; [_.$_0$]] }
   }

run 3 q (fresh (e) remove p q q) 
$\leadsto$ succ {
     { q = [] }
     { q = [_.$_0$], _.$_0$ $\neq$ [_.$_1$] }
     { q = [_.$_0$; _.$_1$]; 
         _.$_0$ $\neq$ [_.$_2$]; _.$_1$ $\neq$ [_.$_3$] }
   }
\end{lstlisting}
\end{minipage}

\begin{minipage}[thb]{0.4\textwidth}
\begin{lstlisting}[
  caption={Example of queries to \lstinline{filter}},
  label={lst:eval-filter-queries}
]
run 3 q (filter p q q) 
$\leadsto$ succ {
     { q = [ ] }
     { q = [_.$_0$] }
     { q = [_.$_0$; _.$_1$] }
   }

run 3 q (filter p q [1]) 
$\leadsto$ succ {
     { q = [[1]] }
     { q = [_.$_0$; [1]]; _.$_0$ $\neq$ [_.$_1$] }
     { q = [[1]; _.$_0$]; _.$_0$ $\neq$ [_.$_1$] }
   }

run 3 q (filter p q [ ]) 
$\leadsto$ succ {
     { q = [] }
     { q = [_.$_0$]; _.$_0$ $\neq$ [_.$_1$] }
     { q = [_.$_0$; _.$_1$]; 
            _.$_0$ $\neq$ [_.$_2$]; _.$_1$ $\neq$ [_.$_3$] }
   }
\end{lstlisting}
\end{minipage}

\subsection{Universal quantification}

In the Section~\ref{sec:impl-univ} we presented 
the \lstinline{forall} goal constructor 
which is implemented through the double negation.
We have observed, that although \lstinline{forall g}
does not terminate when the goal \lstinline{g x} 
has an infinite number of answers 
(assuming \lstinline{x} is a fresh variable),
it does terminate in the case when \lstinline{g x} has 
a finite number of answers.
The behavior of \lstinline{forall} in this case is sound
even in the presence of disequality constraints or nested quantifiers. 

The Table~\ref{tab:univ} gives some concrete examples.
The left column contains the tested goals\footnote{
We typeset the goals in terms of first-order logic syntax 
instead of \textsc{OCanren} syntax for brevity and clarity.} 
and the right column gives the obtained results.
For the results we use the same notation 
as in the previous section.

\begin{table}[th]
  \centering
  \def\arraystretch{1.5}
  \begin{tabularx}{\textwidth}{|X|X|}
    \hline

    $\forall x\ldotp x = q$ & 
      \texttt{fail} \\
    \hline

    $\forall x\ldotp \exists y\ldotp x = y$ & 
      \texttt{succ \{[q = \_.$_0$]\}} \\
    \hline

    $\forall x\ldotp \exists y\ldotp x = y \wedge y = q$ &
      \texttt{fail} \\
    \hline

    $\forall x\ldotp q = (1, x)$ & 
      \texttt{fail} \\
    \hline

    $\forall x\ldotp \exists y\ldotp y = (1, x)$ & 
      \texttt{succ \{[q = \_.$_0$]\}} \\
    \hline

    $\forall x\ldotp \exists y\ldotp x = (1, y)$ &
      \texttt{fail} \\
    \hline

    $\forall x\ldotp x \neq q$ & \texttt{fail} \\
    \hline

    $\forall x\ldotp \exists y\ldotp x \neq y$ & 
      \texttt{succ \{[q = \_.$_0$]\}} \\
    \hline

    $\forall x\ldotp \exists y\ldotp x \neq y \wedge y = q$ & 
      \texttt{fail} \\
    \hline

    $\forall x\ldotp q \neq (1, x)$ & 
      \texttt{succ \{[q $\neq$ (1, \_.$_0$)]\}} \\
    \hline

    $(\exists x\ldotp q = (1, x)) \wedge (\forall x\ldotp q \neq (1, x))$ & 
      \texttt{fail} \\
    \hline

    $\forall x\ldotp (x, x) \neq (0, 1)$ & 
      \texttt{succ \{[q = \_.$_0$]\}} \\
    \hline

    $\forall x\ldotp (x, x) \neq (1, 1)$ & 
      \texttt{fail} \\
    \hline

    $\forall x\ldotp (x, x) \neq (q, 1)$ & 
      \texttt{succ \{[q $\neq$ 1]\}} \\
    \hline

    $\exists a~ b\ldotp q = (a, b) \wedge \forall x\ldotp (x, x) \neq (a, b)$ & 
      \texttt{succ \{[q = (\_.$_0$, \_.$_1$); \_.$_0$ $\neq$ \_.$_1$]\}} \\
    \hline

  \end{tabularx}
  \caption{\lstinline{forall} evaluation}
  \label{tab:univ}
\end{table}

\section{Conclusion and future work}

We presented an approach for pattern matching implementation synthesis using relational programming. Currently, it demonstrates a good performance only
on a very small problems. The performance can be improved by searching for new ways to prune the search space and by speeding up the implementation of
relations and structural constraints. Also it could be interesting to integrate structural constraints more closely into \textsc{OCanren}'s core.
Discovering an optimal order of samples and reducing the complete set of samples is another direction for research.

The language of intermediate representation can be altered, too. It is interesting to add to an intermediate language so-called \emph{exit nodes}
described in~\cite{maranget2001}. The straightforward implementation of them might require nominal unification, but we are not aware of any
\textsc{miniKanren} implementation in which both disequality constraints and nominal unification~\cite{alphaKanren} coexist nicely.

At the moment we support only simple pattern matching without any extensions. It looks technically easy to extend our approach with
non-linear and disjunctive patterns. It will, however, increase the search space and might require more optimizations.





%\begin{comment}
\begin{thebibliography}{10}

\bibitem{CKanren}
C.~E. Alvis, J.~J. Willcock, K.~M. Carter, W.~E. Byrd, and D.~P. Friedman.
\newblock {cKanren}: {miniKanren} with Constraints.
\newblock In {\em Proceedings of the 2011 Annual Workshop on Scheme and
  Functional Programming}, Oct. 2011.

\bibitem{Lambda}
H.~P. Barendregt.
\newblock Lambda Calculi with Types.
\newblock In {\em Handbook of Logic in Computer Science (vol. 2)}, 
pages 117--309. Oxford University Press, Inc., New York, NY, USA, 1992.

\bibitem{WillOnHM}
W.~E. Byrd.
\newblock Private communications.

\bibitem{WillThesis}
W.~E. Byrd.
\newblock Relational Programming in miniKanren: Techniques, Applications,
  and Implementations.
\newblock PhD thesis, Indiana University, September 2009.

\bibitem{unified}
W.~E. Byrd, M.~Ballantyne, G.~Rosenblatt, and M.~Might.
\newblock A Unified Approach to Solving Seven Programming Problems (functional
  pearl).
\newblock {\em Proc. ACM Program. Lang.}, 1(ICFP):8:1--8:26, Aug. 2017.

\bibitem{alphaKanren}
W.~E. Byrd and D.~P. Friedman.
\newblock {$\alpha$Kanren}: A Fresh Name in Nominal Logic Programming.
\newblock In {\em Proceedings of the 2007 Annual Workshop on Scheme and
  Functional Programming}, pages 79--90, 2007.

\bibitem{Untagged}
W.~E. Byrd, E.~Holk, and D.~P. Friedman.
\newblock miniKanren, Live and Untagged: Quine Generation via Relational
  Interpreters (programming pearl).
\newblock In {\em Proceedings of the 2012 Annual Workshop on Scheme and
  Functional Programming}, Scheme '12, pages 8--29, New York, NY, USA, 2012.
  ACM.

\bibitem{cardelli}
L.~Cardelli and P.~Wegner.
\newblock On Understanding Types, Data Abstraction, and Polymorphism.
\newblock {\em ACM Comput. Surv.}, 17(4):471--523, Dec. 1985.

\bibitem{TRS}
D.~P. Friedman, W.~E. Byrd, and O.~Kiselyov.
\newblock The Reasoned Schemer.
\newblock The MIT Press, 2005.

\bibitem{MicroKanren}
J.~Hemann and D.~P. Friedman.
\newblock $\mu$Kanren: A Minimal Functional Core for Relational Programming.
\newblock In {\em Proceedings of the 2013 Annual Workshop on Scheme and
  Functional Programming}, 2013.

\bibitem{SmallEmbedding}
J.~Hemann, D.~P. Friedman, W.~E. Byrd, and M.~Might.
\newblock A Small Embedding of Logic Programming with a Simple Complete Search.
\newblock {\em SIGPLAN Not.}, 52(2):96--107, Nov. 2016.

\bibitem{ocanren}
D.~Kosarev and D.~Boulytchev.
\newblock Typed Embedding of a Relational Language in OCaml.
\newblock {\em ACM SIGPLAN Workshop on ML}, 2016.

\bibitem{UnificationRevisited}
J.-L. Lassez, M.~J. Maher, and K.~Marriott.
\newblock Unification Revisited.
\newblock In {\em Foundations of Deductive Databases and Logic Programming},
pages 587--625. Morgan Kaufmann Publishers Inc., San Francisco, CA, USA, 1988.

\bibitem{Types}
B.~C. Pierce.
\newblock Types and Programming Languages.
\newblock The MIT Press, 1st edition, 2002.

\bibitem{Unification}
F.~Baader and W.~Snyder. 
\newblock{Unification Theory.}
\newblock In {\em Handbook of Automated Reasoning},
Elsevier Science Publishers B. V., Amsterdam, The Netherlands, The Netherlands, 2001.

\bibitem{Felleisen}
A.~Wright and M.~Felleisen.
\newblock A Syntactic Approach to Type Soundness.
\newblock {\em Inf. Comput.}, 115(1):38--94, Nov. 1994.

\end{thebibliography}
%\end{comment}

%\renewcommand{\clearpage}{} 
%\bibliographystyle{abbrv}
%\bibliography{main}

%\clearpage
%\appendix
%% !TEX TS-program = pdflatex
% !TeX spellcheck = en_US
% !TEX root = main.tex
\appendix
\section{Relational Interpreter for Quines, Translated from OCanren}
\label{appendix:synQuines}

An implementation of relational interpreter, translated from \OCanren{}. Extra tagging is represented with \textbf{\underbar{underlined bold}} style.
\todo{TODO Link}

\lstinputlisting[language=miniKanren, numbers=left, ndkeywords={symb,seq,val}]{interpreter_syn.scm}

\section{Default Quines Implementation}
\label{appendix:defaultQuines}
\todo{TODO Link}

\lstinputlisting[language=miniKanren,numbers=left]{default_quines.scm}

\end{document}

