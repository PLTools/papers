% This is samplepaper.tex, a sample chapter demonstrating the
% LLNCS macro package for Springer Computer Science proceedings;
% Version 2.20 of 2017/10/04
%
\documentclass[runningheadsm, envcountsect]{llncs}
\pagestyle{headings}

%
% Used for displaying a sample figure. If possible, figure files should
% be included in EPS format.
%
% If you use the hyperref package, please uncomment the following line
% to display URLs in blue roman font according to Springer's eBook style:
% \renewcommand\UrlFont{\color{blue}\rmfamily}

\usepackage{amsmath}
\usepackage{amssymb}
\usepackage{todonotes}
\usepackage{stmaryrd}
\usepackage{xspace}
\usepackage{listings}
\usepackage{comment}
\usepackage{mathtools}
\usepackage{environ}
\usepackage{enumitem}
\usepackage{tikz}
\usetikzlibrary{trees}
\usepackage{multirow}
\usepackage{array}
\usepackage{graphics}
\usepackage[cmtip,all]{xy}
\usepackage{units}
\usepackage{placeins}
\usepackage{lineno}
\usepackage{cite}
\usepackage{wrapfig}

%\xymatrix@C-=0.5cm

\newcommand{\longsquiggly}{\xymatrix@C-=0.5cm{{}\ar@{~>}[r]&{}}}

\newcommand{\todonote}[1]{\todo[inline, color=blue!20]{\textbf{TODO:} #1}}
\newcommand{\sectionword}{Section}
\newcommand{\appendixword}{Appendix}
\newcommand{\figureword}{Fig.}
\newcommand{\lemmaword}{Lem.}
\newcommand{\eqdef}{\overset{\mathrm{def}}{=}}
\newcommand{\ruleno}[1]{\mbox{[\textsc{#1}]}}
\newcommand{\bigslant}[2]{{\raisebox{.2em}{$#1$}\left/\raisebox{-.2em}{$#2$}\right.}}
\newcommand{\lowupbound}[2]{\;\; \overset{\makebox[0pt]{\mbox{\normalfont\tiny {C = $#2$}}}}{ \underset{\makebox[0pt]{\mbox{\normalfont\tiny {C = $#1$}}}}{\gtrless}} \;\;}
\newcommand{\argmax}[1]{\underset{\makebox[0pt]{\mbox{\normalfont\tiny {$#1$}}}}{argmax} \;\;}
\newcommand{\maxd}{\max^{\bullet}}
\newcommand{\ttt}[1]{\mbox{\texttt{#1}}}
\let\emptyset\varnothing

\newcommand{\inbr}[1]{\left<{#1}\right>}
\newcommand{\fv}[1]{\mathcal{FV}\,({#1})}

\newcommand{\grterms}{\mathcal{T}_{\emptyset}}
\newcommand{\Dom}{\mathcal{D}om}
\newcommand{\VRan}{\mathcal{VR}an}

\newcommand{\substitute}[3]{#1\,[\bigslant{#2}{#3}]}
\newcommand{\substupd}[3]{#1[#2 \mapsto #3]}

\newcommand{\taskst}[2]{\langle #1 ,\, #2 \rangle}
\newcommand{\mkenv}[2]{(#1 ,\, #2)}
\newcommand{\unigoal}[2]{#1 \equiv #2}
\newcommand{\conjgoal}[2]{#1 \land #2}
\newcommand{\disjgoal}[2]{#1 \lor #2}
\newcommand{\freshgoal}[2]{\mbox{\lstinline|fresh|} \, #1\;.\; #2}
\newcommand{\invokegoal}[3]{#1\,(#2, \, \dots, \, #3)}
\newcommand{\einit}{e_{init}}
\newcommand{\ninit}{n_{init}}

\newcommand{\schemetrans}[6]{\withenv{#2,\,#3,\,#4,\,#5}{#1} \longsquiggly #6}
\newcommand{\schemewithvset}[2]{{#1}^{#2}}

\begin{comment}
\newcommand{\schemenode}[1]{
%\begin{tikzpicture}[anchor=base,baseline,sibling distance=3cm,edge from parent/.style={draw,-latex}]
 \tikz[anchor=base,baseline,sibling distance=3cm,edge from parent/.style={draw,-latex}] \node[rectangle,draw]{#1};
%\end{tikzpicture}
 }

\newcommand{\schemesarrow}[3]{
%\begin{tikzpicture}[sibling distance=3cm,edge from parent/.style={draw,-latex}]
  \tikz[anchor=base,baseline,sibling distance=3cm,edge from parent/.style={draw,-latex}]\node[rectangle,draw]{#1}child {node[circle,draw]{#3}edge from parent node[right]{\tiny{#2}}};
%\end{tikzpicture}}

\newcommand{\schemedarrow}[3]{
%\begin{tikzpicture}[sibling distance=3cm,edge from parent/.style={draw,-latex}]
 \tikz[anchor=base,baseline,sibling distance=3cm,edge from parent/.style={draw,-latex}] \node[rectangle,draw]{#1}child {node[circle,draw]{#3}edge from parent node[right]{\tiny{#2}}};
%\end{tikzpicture}}

\newcommand{\schemefork}[2]{
  \begin{tikzpicture}[edge from parent/.style={draw,-latex}]
   \coordinate   
      child {node[circle,draw]{#1}}
      child {node[circle,draw]{#2}} ;
\end{tikzpicture}}
\end{comment}

\newcommand{\schemenode}[1]{
%\begin{tikzpicture}[sibling distance=3cm,edge from parent/.style={draw,-latex}]
\tikz[anchor=base,baseline,sibling distance=3cm,edge from parent/.style={draw,-latex}]  \node{#1};
%\end{tikzpicture}
 }

\newcommand{\schemesarrow}[3]{
\begin{tikzpicture}[level distance=30pt,sibling distance=3cm,edge from parent/.style={draw,-latex}]
  \node{#1}child {node{#3}edge from parent node[right]{\tiny{#2}}};
\end{tikzpicture}}

\newcommand{\schemedarrow}[3]{
\begin{tikzpicture}[level distance=30pt,sibling distance=3cm,edge from parent/.style={draw,-latex}]
  \node{#1}child {node{#3}edge from parent node[right]{\tiny{#2}}};
\end{tikzpicture}}

\newcommand{\schemefork}[2]{
  \begin{tikzpicture}[level distance=20pt,edge from parent/.style={draw,-latex}]
   \coordinate   
      child {node{#1}}
      child {node{#2}} ;
\end{tikzpicture}}

\newcommand{\upd}[2]{\mbox{\textbf{upd}}\,(#1,\, #2)}
\newcommand{\constr}[2]{\mbox{\textbf{constr}}\,(#1,\, #2)}

\newcommand{\withenv}[2]{\left< #1 \right> \;\vdash\; #2}
\newcommand{\onepremrule}[2]{\dfrac{#1}{#2}}
\newcommand{\twopremrule}[3]{\dfrac{#1,\; #2}{#3}}
\newcommand{\threepremrule}[4]{\dfrac{#1,\;  #2,\;  #3}{#4}}

\newcommand{\lazystream}[1]{\texttt{Lazy [{#1}]}}
\newcommand{\consstream}[2]{\texttt{Cons #1 [{#2}]}}

\newcommand{\sembr}[1]{\llbracket #1 \rrbracket}
\newcommand{\tra}[1]{\mathcal{T}r^{ans}(#1)}
\newcommand{\trs}[1]{\mathcal{T}r^{st}(#1)}

\newcommand{\mK}{\textsc{miniKanren}\xspace}

\newcommand{\costdisj}[2]{cost_{\oplus}(#1 \oplus #2)}
\newcommand{\costconj}[2]{cost_{\otimes}(#1 \otimes #2)}
\newcommand{\lookuptime}[1]{\texttt{lookup}\,(#1)}
\newcommand{\addtime}[1]{\texttt{add}\,(#1)}
\renewcommand{\O}{\mathcal{O}}

\renewcommand{\labelenumii}{\arabic{enumi}.\arabic{enumii}}
\renewcommand{\labelenumiii}{\arabic{enumi}.\arabic{enumii}.\arabic{enumiii}}
\renewcommand{\labelenumiv}{\arabic{enumi}.\arabic{enumii}.\arabic{enumiii}.\arabic{enumiv}}

%\renewcommand\thelinenumber{\textcolor{red}{linenumber}}
\let\lemma\relax
\spnewtheorem{lemma}{Lemma}[section]{\bfseries}{\itshape}

\newcommand{\repeatlemma}[1]{%
  \begingroup
  \renewcommand{\thelemma}{\ref{#1}}%
  \expandafter\expandafter\expandafter\lemma
  \csname replemma@#1\endcsname
  \endlemma
  \endgroup
}

\newcommand{\repeattheorem}[1]{%
  \begingroup
  \renewcommand{\thetheorem}{\ref{#1}}%
  \expandafter\expandafter\expandafter\theorem
  \csname reptheorem@#1\endcsname
  \endtheorem
  \endgroup
}

\NewEnviron{replemma}[1]{%
  \global\expandafter\xdef\csname replemma@#1\endcsname{%
    \unexpanded\expandafter{\BODY}%
  }%
  \expandafter\lemma\BODY\label{#1}\endlemma
}

\NewEnviron{reptheorem}[1]{%
  \global\expandafter\xdef\csname reptheorem@#1\endcsname{%
    \unexpanded\expandafter{\BODY}%
  }%
  \expandafter\theorem\BODY\label{#1}\endtheorem
}


%% Listings

\lstdefinelanguage{minikanren}{
keywords={fresh},
sensitive=true,
commentstyle=\small\itshape\ttfamily,
keywordstyle=\textbf,
identifierstyle=\ttfamily,
basewidth={0.5em,0.5em},
columns=fixed,
fontadjust=true,
literate={fun}{{$\lambda\;\;$}}1 {->}{{$\to$}}3 {===}{{$\,\equiv\,$}}1 {=/=}{{$\not\equiv$}}1 {|>}{{$\triangleright$}}3 {/\\}{{$\wedge$}}2 {\\/}{{$\vee$}}2,
morecomment=[s]{(*}{*)}
}

\lstset{
mathescape=true,
language=minikanren
}

\usepackage{letltxmacro}
\newcommand*{\SavedLstInline}{}
\LetLtxMacro\SavedLstInline\lstinline
\DeclareRobustCommand*{\lstinline}{%
  \ifmmode
    \let\SavedBGroup\bgroup
    \def\bgroup{%
      \let\bgroup\SavedBGroup
      \hbox\bgroup
    }%
  \fi
  \SavedLstInline
}

%%
%% end of the preamble, start of the body of the document source.
\begin{document}

%\linenumbers

\makeatletter
\let\origsection\section
\renewcommand\section{\@ifstar{\starsection}{\nostarsection}}

\newcommand\nostarsection[1]
{\sectionprelude\origsection{#1}\sectionpostlude}

\newcommand\starsection[1]
{\sectionprelude\origsection*{#1}\sectionpostlude}

\newcommand\sectionprelude{%
  \vspace{-2mm}
}

\newcommand\sectionpostlude{%
  \vspace{-2mm}
}
\makeatother

\setlength{\abovecaptionskip}{-5pt plus 3pt minus 2pt}
\setlength{\belowcaptionskip}{-20pt plus 3pt minus 2pt}

\abovedisplayskip-1mm
\belowdisplayskip0mm
\abovedisplayshortskip-3mm
\belowdisplayshortskip0mm

\setlength{\topsep}{0pt}
\setlength{\partopsep}{0pt plus 0pt minus 0pt}
\setlength{\parskip}{0pt}
%\setlength{\parindent}{0pt}

%%
%% The "title" command has an optional parameter,
%% allowing the author to define a "short title" to be used in page headers.
\title{Scheduling Complexity of Interleaving Search}

\author{Dmitry Rozplokhas\orcidID{0000-0001-7882-4497} \and \\
Dmitry Boulytchev\orcidID{0000-0001-8363-7143}}
%
\authorrunning{D. Rozplokhas and D. Boulytchev}
% First names are abbreviated in the running head.
% If there are more than two authors, 'et al.' is used.
%
\institute{St Petersburg University and JetBrains Research, Russia \\ \email{rozplokhas@gmail.com}, \email{dboulytchev@math.spbu.ru}}

%%
%% This command processes the author and affiliation and title
%% information and builds the first part of the formatted document.
\maketitle

%%
%% The abstract is a short summary of the work to be presented in the
%% article.
\begin{abstract}
  \mK is a lightweight embedded language for logic and relational programming. Many of its useful features come from
  a distinctive search strategy, called \emph{interleaving search}. However, with interleaving search conventional  
  ways of reasoning about the complexity and performance of logical programs become irrelevant. We identify an important
  key component~--- \emph{scheduling}~--- which makes the reasoning for \mK so different, and present a semi-automatic
  technique to estimate the scheduling impact via symbolic execution for a reasonably wide class of programs.
  %We evaluate the
  %proposed technique by analyzing time complexity for a number of programs from the literature and verify the results
  %experimentally.
\keywords{miniKanren, interleaving search, time complexity, symbolic execution}
\end{abstract}


% !TEX TS-program = pdflatex
% !TeX spellcheck = en_US
% !TEX root = main.tex


\section{Introduction}
\label{sec:intro}


One of distinguishable features of \miniKanren{} is the fact that it is a family of languages:
many languages may host different \miniKanren{} implementations.
For example, \faster{}\footnote{\url{https://github.com/michaelballantyne/faster-minikanren} (access date: \DTMdate{2024-06-06})} for \Scheme{} and \Racket{}, \CoreLogic{} for \textsc{Closure}, \OCanren{}~\cite{OCanren} for \OCaml{}, \Klogic{}~\cite{Klogic2023} for \Kotlin{} and others.
The users of these DSLs may want to compare expressive power of various flavor of miniKanren, specifics due to host language, and performance implications of choosing a different host language.


The straightforward solution is to rewrite a number of significant benchmarks for many implementation, as it done for other languages\footnote{\url{https://benchmarksgame-team.pages.debian.net/benchmarksgame/index.html} (access date: \DTMdate{2024-06-06})}.
Doing it manually is time consuming and error prone.
Due to low-level nature of relational programs, it is easy to make spelling mistakes, for example accidentally write wrong identifier in unification arguments.
(We did many of mistakes of this kind while porting programs from \OCanren{} to \Klogic{}.) Moreover, \miniKanren{} doesn't pardon relational programs that solve the same task: it was reported, that the order of conjuncts significantly affects~\cite{scheduling2022} performance even if the search does the same unifications.

The differences between host languages also complicate porting relational code.
For example, \Kotlin{} doesn't support currying and partial applications comparatively to \OCaml{}, and sometimes full $\eta$-expansion is needed.
Also, porting from dynamically typed languages like \Scheme{} to statically typed ones like \OCaml{} could be uneasy for newcomers to statically typed languages.
This porting could be not straightforward:
basic data representations in \OCaml{}/\Klogic{} (algebraic data types and classes with subclasses~--- sum types) is different from \Scheme{} (lists, i.e. arbitrary length tuples~--- product types).
This fact in some cases requires special constraints~\cite{Wildcards2023} to level the expressivity, and in other cases (like relational interpreters) allows to get rid of \emph{absento/symbolo} constraints.

Things could get even more complicated where we want to port larger projects which are using functional/relational approach where relational parts are intermixed with straightforward programming.
The developer is obliged to know relational approach, the original host general purpose language and to have experience  with a new host language.

In this paper we describe current status of our converter from relational \OCanren{} to \Klogic{} and \miniKanren{} in \Scheme{}.
At the moment only relational subset of \OCanren{} is supported, we don't support whole \OCaml{} language.
In next section we describe technical aspects of our approach and currently supported features.
In section \ref{sec:interpreter} we discuss transformation in relational interpreter~\cite{Untagged} from \OCanren{} to \Scheme{} and peculiarities of autogenerated implementation.



\section{Background: Syntax and Semantics of \mK}
\label{sec:background}

In this section, we recollect some known formal descriptions for \mK language that will be used as a basis for our development.
The descriptions here are taken from~\cite{CertifiedSemantics} (with a few non-essential adjustments for presentation purposes) to make
the paper self-contained, more details and explanations can be found there.

The syntax of core \mK is shown in Fig.~\ref{fig:syntax}. 
All data is presented using terms $\mathcal{T}_X$ built from a fixed set of constructors $\mathcal{C}$ with known arities and variables
from a given set $X$.
We parameterize the terms with an alphabet of variables since in the semantic description we will need \emph{two} kinds of variables:
\emph{syntactic} variables $\mathcal{X}$, used for bindings in the definitions, and \emph{logic} variables $\mathcal{A}$, which are
introduced and unified during the evaluation. We assume the set $\mathcal{A}$ ordered and use the notation $\alpha_i$ 
to specify a position of a logical variable w.r.t. this order.

There are five types of goals: unification of two terms, conjunction and disjunction of goals,
%(the ``\lstinline|conde|'' operator from the canonical versions of \mK, split in the two for simplicity),
fresh logic variable introduction, and invocation of some relational definition. For the sake of brevity, in code snippets, we abbreviate
immediately nested ``\lstinline|fresh|'' constructs into the one, writing ``\lstinline|fresh x y $\dots$ . $g$|'' instead of
``\lstinline|fresh x . fresh y . $\dots$ $g$|''. The \emph{specification} $\mathcal{S}$ consists of a set of relational definitions and a top-level goal.
A top-level goal represents a search procedure that returns a stream of substitutions for the free variables of the goal.
%The language we consider is first-order, as goals can not be passed as parameters, returned, or constructed at runtime.


\begin{figure}[t]
\centering
\[
\begin{array}{ccll}
  \mathcal{C} & = & \{C_i^{k_i}\} & \mbox{constructors} \\
  \mathcal{T}_X & = & X \cup \{C_i^{k_i} (t_1, \dots, t_{k_i}) \mid t_j\in\mathcal{T}_X\} & \mbox{terms over the set of variables $X$} \\
  \mathcal{D} & = & \mathcal{T}_\emptyset & \mbox{ground terms}\\
  \mathcal{X} & = & \{ \ttt{x}, \ttt{y}, \ttt{z}, \dots \} & \mbox{syntactic variables} \\
  \mathcal{A} & = & \{ x, y, z \dots \} & \mbox{logic variables} \\
  \mathcal{R} & = & \{ R_i^{k_i}\} &\mbox{relational symbols with arities} \\[2mm]
  \mathcal{G} & = & \mathcal{T_X}\equiv\mathcal{T_X}   &  \mbox{equality} \\
              &   & \mathcal{G}\wedge\mathcal{G}     & \mbox{conjunction} \\
              &   & \mathcal{G}\vee\mathcal{G}       &\mbox{disjunction} \\
              &   & \mbox{\lstinline|fresh|}\;\mathcal{X}\;.\;\mathcal{G} & \mbox{fresh variable introduction} \\
              &   & R_i^{k_i} (t_1,\dots,t_{k_i}),\;t_j\in\mathcal{T_X} & \mbox{relational symbol invocation} \\[2mm]
  \mathcal{S} & = & \{R_i^{k_i} = \lambda\;\ttt{x}_1^i\dots \ttt{x}_{k_i}^i\,.\, g_i;\}\; g, \; \textcolor{blue}{g_i, g\in\mathcal{G}}\phantom{XXX} & \mbox{specification}
\end{array}
\]
\caption{The syntax of \mK}
\label{fig:syntax}
\end{figure}

During the evaluation of \mK program an environment, consisting of a substitution for logic variables and a counter of allocated logic
variables, is threaded through the computation and updated in every unification and fresh variable introduction.
The substitution in the environment at a given point and given branch of evaluation contains all the information about relations between
the logical variables at this point.
%Hereafter we refer to the substitution in the environment at a given point as ``current substitution''
%in informal explanations.
Different branches are combined via \emph{interleaving search} procedure~\cite{Transformers}.
The answers for a given goal are extracted from the final environments.

% \begin{wrapfigure}{r}{0.5\textwidth}
% \setlength{\abovecaptionskip}{0pt plus 3pt minus 2pt}
\begin{figure}[t]
\centering
\[
\begin{array}{ccllcccll}
  \Sigma & = & \mathcal{A} \to \mathcal{T}_\mathcal{A} & \mbox{substitutions} &\qquad\qquad&        S & = & \taskst{\mathcal{G}}{E} & \mbox{task} \\
       E & = & \Sigma \times \mathbb{N}              & \mbox{environments}  &\qquad\qquad&          &   & S \oplus S              & \mbox{sum} \\
         &   &                                       &                      &\qquad\qquad&          &   & S \otimes \mathcal{G}   & \mbox{product} \\
%         &   &                                       &                      &\qquad\qquad&  \hat{S} & = & \diamond \; \mid \; S   & \mbox{states} \\
       L & = & \circ \; \mid \; E      & \mbox{labels}        &\qquad\qquad&  \hat{S} & = & \diamond \; \mid \; S   & \mbox{states} \\
%         &   &                                       &                      &\qquad\qquad&        L & = & \circ \; \mid \; E      & \mbox{labels} 
\end{array}
\]
\caption{States and labels in the LTS for \mK}
\label{fig:operanional_semantics_states_labels}
%\end{wrapfigure}
\end{figure}

This search procedure is formally described by operational semantics in the form of a labeled transition system.
This semantics corresponds to the canonical implementation of interleaving search. 

The form of states and labels in the transition system is defined in \figureword~\ref{fig:operanional_semantics_states_labels}.
Non-terminal states $S$ have a tree-like structure with intermediate nodes corresponding to partially evaluated conjunctions
(``$\otimes$'') or disjunctions (``$\oplus$'').
A leaf in the form $\taskst{g}{e}$ determines a task to evaluate a goal $g$ in an environment $e$. For a conjunction node, its right child
is always a goal since it cannot be evaluated unless some result is provided by the left conjunct.
We also need a terminal state $\diamond$ to represent the end of the evaluation.
The label ``$\circ$'' is used to mark those steps which do not provide an answer; otherwise, a transition is labeled by an updated
environment.

\begin{figure*}[t]
  \renewcommand{\arraystretch}{1.6}
  \[
  \begin{array}{crcr}
    \multicolumn{3}{c}{\taskst{t_1 \equiv t_2}{(\sigma, n)} \xrightarrow{\circ} \Diamond , \, \, \nexists\; mgu\,(t_1 \sigma, t_2 \sigma)} &\ruleno{UnifyFail} \\
    \multicolumn{3}{c}{\taskst{t_1 \equiv t_2}{(\sigma, n)} \xrightarrow{(mgu\,(t_1 \sigma, t_2 \sigma) \circ \sigma),\, n)} \Diamond} & \ruleno{UnifySuccess} \\
    \multicolumn{3}{c}{\taskst{\mbox{\lstinline|fresh|}\, \ttt{x}\, .\, g}{(\sigma, n)} \xrightarrow{\circ} \taskst{g\,[\bigslant{\alpha_{n + 1}}{\ttt{x}}]}{( \sigma, n + 1)}} & \ruleno{Fresh} \\
    \multicolumn{3}{c}{\dfrac{R_i^{k_i}=\lambda\,\ttt{x}_1\dots \ttt{x}_{k_i}\,.\,g}{\taskst{R_i^{k_i}\,(t_1,\dots,t_{k_i})}{e} \xrightarrow{\circ} \taskst{g\,[\bigslant{t_1}{\ttt{x}_1}\dots\bigslant{t_{k_i}}{\ttt{x}_{k_i}}]}{e}}} & \ruleno{Invoke}\\[3mm]
    \taskst{g_1 \lor g_2}{e} \xrightarrow{\circ} \taskst{g_1}{e} \oplus \taskst{g_2}{e} & \ruleno{Disj} &
    \taskst{g_1 \land g_2}{e} \xrightarrow{\circ} \taskst{g_1}{e} \otimes g_2 & \ruleno{Conj} \\    
    \dfrac{s_1 \xrightarrow{l} \Diamond}{(s_1 \oplus s_2) \xrightarrow{l} s_2} & \ruleno{DisjStop} &
    \dfrac{s_1 \xrightarrow{l} s'_1}{(s_1 \oplus s_2) \xrightarrow{l} (s_2 \oplus s'_1)} &\ruleno{DisjStep} \\
    \dfrac{s \xrightarrow{\circ} \Diamond}{(s \otimes g) \xrightarrow{\circ} \Diamond} &\ruleno{ConjStop} &
    \dfrac{s \xrightarrow{e} \Diamond}{(s \otimes g) \xrightarrow{\circ} \taskst{g}{e}}  & \ruleno{ConjStopAns}\\
    \dfrac{s \xrightarrow{\circ} s'}{(s \otimes g) \xrightarrow{\circ} (s' \otimes g)} &\ruleno{ConjStep} &
    \dfrac{s \xrightarrow{e} s'}{(s \otimes g) \xrightarrow{\circ} (\taskst{g}{e} \oplus (s' \otimes g))} & \ruleno{ConjStepAns} 
  \end{array}
  \]
  \caption{Operational semantics of interleaving search}
  \label{fig:operanional_semantics_rules}
\end{figure*}

The transition rules are shown in Fig.~\ref{fig:operanional_semantics_rules}.
\textcolor{blue}{
The first six rules define evaluation of leaf states.
For the disjunction and conjunction the corresponding node states are constructed.
For other types of goals the environment and the evaluated goal are updated in accordance with the task given by the goal: for an equality the most general unifier of the terms is incorporated into the substitution (or execution halts if the terms are non-unifiable); for a fresh construction a new variable is introduced and the counter of allocated variables is incremented; for a relational call the body of the relation is taken as the next goal.
The rest of the rules define composition of evaluation of substates for partial disjunctions and conjunctions.
For a partial disjunction the first constituent is evaluated for one step, then the constituents are swapped (which constutes the \emph{interleaving}), and the label is propagated.
When the evalution of the first constituent of partial disjunction halts, the evaluation proceedes with the second constituent.
For a partial conjunction the first consituent is evaluated until the answer is obtained, then the evaluation of the second constituent with this answer as the environment is scheduled for evaluation together with the remaining partial conjunction (via partial disjunction node).
When the evalution of the first constituent of partial conjunction halts, the evaluation of the conjunction halts, too.
}

The introduced transition system is completely deterministic,
therefore a derivation sequence for a state $s$ determines a certain \emph{trace}~--- a sequence of states and labeled transitions between
them. It may be either finite (ending with the terminal state $\Diamond$) or infinite. We will denote by $\trs{s}$ the sequence of states in
the trace for initial state $s$ and by $\tra{s}$ the sequence of answers in the trace for initial state $s$. The sequence $\tra{s}$ corresponds
to the stream of answers in the reference \mK implementations.

In the following we rely on the following property of leaf states:

\begin{definition}
  A leaf state $\taskst{g}{\mkenv{\sigma}{n}}$ is well-formed iff $\fv{g}\cup\Dom\,(\sigma)\cup\VRan\,(\sigma)\subseteq\{\alpha_1,\dots,\alpha_n\}$, where
  $\fv{g}$ denotes the set of free variables in a goal $g$, $\Dom\,(\sigma)$ and $\VRan\,(\sigma)$~--- the domain of a substitution $\sigma$ and
  a set of all free variables in its image respectively.
\end{definition}

Informally, in a well-formed leaf state all free variables in goals and substitution respect the counter of free logical variables.
This definition is in fact an instance of a more general definition of well-formedness for all states, introduced in~\cite{CertifiedSemantics}, where it is
proven that the initial state is well-formed and any transition from a well-formed state results in a well-formed one.

Besides operational semantics, we will make use of a denotational one analogous to the least Herbrand model. For a relation $R^k$, its denotational semantics $\sembr{R^k}$ is
treated as a $k$-ary relation on the set of all ground terms, where each ``dimension'' corresponds to a certain argument of $R^k$. For example,
$\sembr{\lstinline|append$^o$|}$ is a set of all triplets of ground lists, in which the third component is a
concatenation of the first two. The concrete description of the denotational semantics is given in~\cite{CertifiedSemantics} as well as the proof of
the soundness and completeness of the operational semantics w.r.t. to the denotational one.

\begin{figure}[t]
\centering
\[
\begin{array}{ccll}
B_{nf} & = &  \unigoal{\mathcal{T}_\mathcal{X}}{\mathcal{T}_\mathcal{X}} \; \mid \;
                     \invokegoal{R^k}{\mathcal{T}_\mathcal{X}}{\mathcal{T}_\mathcal{X}} \\
C_{nf} & = & B_{nf} \; \mid \; \conjgoal{C_{nf}}{B_{nf}} \\
F_{nf} & = & C_{nf} \; \mid \; \freshgoal{X}{F_{nf} } \\
D_{nf} & = & F_{nf} \; \mid \; \disjgoal{D_{nf}}{F_{nf}}
\end{array}
\]
\caption{Disjunctive Normal Form for goals}
\label{fig:dnf}
\end{figure}

Finally, we explicitly enumerate all the restrictions required by our method to work:

\begin{itemize}
\item All relations have to be in DNF \textcolor{blue}{(set $D_{nf}$ in \figureword\ref{fig:dnf})}. 
\item We only consider goals which converge with a finite number of answers.
\item All answers have to be ground (\emph{groundness} condition) for all relation invocations encountered
  during the evaluation.
\item All answers have to be unique (\emph{answer uniqueness} condition) for all relation invocations encountered
  during the evaluation.
\end{itemize}


\section{Scheduling Complexity}
\label{sec:scheduling}

%In this section we define a specific value to estimate the scheduling time and derive some equations to calculate this value for a given \emph{semantic
%state}. However, our ultimate goal is to provide a complexity estimation for a given goal. This problem will be addressed in Section~\ref{sec:symbolic},
%in which we will make use of recurrent equations presented here.

We may notice that the operational semantics described in the previous section can be used to calculate the exact number of elementary scheduling steps.
Our first idea is to take the number of states $d\,(s)$ in the finite trace for a given state $s$:

\[ d\,(s) \; \eqdef \; | \trs{s} |  \]

However, it turns out that this value alone does not provide an accurate scheduling complexity estimation. The reason is that some
elementary steps in the semantics are not elementary in existing implementations. Namely, a careful analysis discovers that
each semantic step involves navigation to the leftmost leaf of the state which in implementations corresponds to multiple elementary actions,
whose number is proportional to the height of the leftmost branch of the state in question. Here we provide an \emph{ad-hoc} definition for this value, $t\,(s)$,
which we call the \emph{scheduling factor}:

\[
t\,(s) \eqdef \sum\limits_{s_i \in \trs{s}} lh\,(s_i) 
\]

where $lh\,(s_i)$ is the height of the leftmost branch of the state. 

\begin{comment}
\[
\begin{array}{rcl}
 lh\,(\taskst{g}{e})  &\eqdef& 1 \\
 lh\,(s_1 \oplus s_2) &\eqdef& lh\,(s_1) + 1 \\
 lh\,(s \otimes g)    &\eqdef& lh\,(s) + 1 \\
\end{array}
\]
\end{comment}


\begin{comment}
In the rest of the section, we state a number of lemmas providing estimations for these two values. The proofs for the lemmas (when omitted) can be
found in Appendix~\ref{sec:appendix}.

The first lemma provides a fundamental relation between these two~--- $d$ and $t$,~--- estimations for the scheduling complexity:

\begin{lemma}
\label{lem:d_t_relation}
 $d\,(s) \le t\,(s) \le d^2\,(s)$ for any state $s$.
\end{lemma} }
{ \color{red}
\begin{proof}
  Follows immediately from the definitions of the estimated values and the fact that the height of a state increases by at most $1$ at each step.\qed
\end{proof}
}
\end{comment}

In the rest of the section, we derive recurrent equations which would relate the scheduling complexity for states to the scheduling complexity for their
(immediate) substates. It turns out that to come up with such equations both $t$ and $d$ values have to be estimated simultaneously.  


% We take scheduling factor $t\,(s)$ as a value that determines the scheduling complexity $T_s$, but we will also need to calculate $d\,(s)$ as it will be used in the equations for $t\,(s)$. 

%In $\oplus$-states the substates are evaluated separately, one step at a time for each substate,
%so the total number of semantic steps is just a sum.
%However, for the scheduling factor, there is an extra summand since the heights of the states in
%the trace becomes bigger (by one additional $\oplus$-node on the top).
%This additional node exists in the trace until one of the substates is evaluated completely, so the
%scheduling factor is increased by the number of steps before such an event.
%So we have the following lemma.

The next lemma provides the equations for $\oplus$-states:

\begin{replemma}{lem:sum_measure_equations}
For any two states $s_1$ and $s_2$

\[
\begin{array}{rcl}
  d\,(s_1 \oplus s_2) &=& d\,(s_1) + d\,(s_2) \\
    t\,(s_1 \oplus s_2) &=& t\,(s_1) + t\,(s_2) + \costdisj{s_1}{s_2}
\end{array}
\]

where $\costdisj{s_1}{s_2} = \min\,\{2\cdot d\,(s_1) - 1, 2\cdot d\,(s_2)\}$
\end{replemma}

Informally, for a state in the form $s_1 \oplus s_2$ the substates are evaluated separately, one step at a time for
each substate, so the total number of semantic steps is the sum of those for the substates. However, for the scheduling factor, 
there is an extra summand $\costdisj{s_1}{s_2}$ since the ``leftmost heights'' of the states in the trace are one node greater than those for the
original substates due to the introduction of one additional $\oplus$-node on the top. This additional node persists in the trace until the evaluation
of one of the substates comes to an end, so the scheduling factor is increased by the number of steps until that.

The next lemma provides the equations for $\otimes$-states:\footnote{We assume $\Diamond\otimes g = \Diamond$}

\begin{replemma}{lem:times_measure_equations}
  For any state $s$ and any goal $g$
  
\[
\begin{array}{rclr}
d\,(s \otimes g)  &=&  d\,(s) + \smashoperator[lr]{\sum\limits_{a_i \in \tra{s}}} d\,(\taskst{g}{a_i})& \qquad(\star) \\

 t\,(s \otimes g)  &=&  t\,(s) + \costconj{s}{g} + \smashoperator[lr]{\sum\limits_{a_i \in \tra{s}}} (t\,(\taskst{g}{a_i}) + \costdisj{\taskst{g}{a_i}}{(s'_i \otimes g)})&\qquad(\dagger)
\end{array}
\]

where 
\[
\begin{array}{rcl}
\costconj{s}{g} & = & d\,(s) \\
s'_i & = & \mbox{the first state in the trace for $s$ after} \\
 & & \mbox{a transition delivering the answer $a_i$} \\
\end{array}
\]
\end{replemma}

For the states of the form $s \otimes g$ the reasoning is the same, but the resulting equations are more complicated.
In an $\otimes$-state the left substate is evaluated until an answer is found, which is then taken as
\emph{an environment} for the evaluation of the right subgoal.
Thus, in the equations for $\otimes$-states the evaluation times of the second goal \emph{for all
the answers} generated for the first substate are summed up. The evaluation of the right subgoal
in different environments is added to the evaluation of the left substate via creation of
an $\oplus$-state, so for the scheduling factor there is
an additional summand $\costdisj{\taskst{g}{a_i}}{s'_i}$ for each answer with $s'_i$ being the state
after discovering the answer.
There is also an extra summand $\costconj{s}{g}$ for the scheduling factor because of the
$\otimes$-node that increases the height in the trace, analogous to the one caused by
$\oplus$-nodes.
Note, a $\otimes$-node is always placed immediately over the left substate so this
addition is exactly the number of steps for the left substate.

Unfolding costs definitions in $(\dagger)$ gives us a cumbersome formula that 
includes some intermediate states $s'_i$ encountered during the evaluation. However, as ultimately
we are interested in asymptotic estimations, we can approximate these costs up to a multiplicative constant. We can notice that the value $d\,(s'_i \otimes g)$ occurring in the second argument of $cost_{\oplus}$ includes values $d\,(\taskst{g}{a_j})$ (like in the first argument) for all answers $a_j$ after this intermediate state. So in the sum of all $cost_{\oplus}$ values $d\,(\taskst{g}{a_i})$ may be excluded for at most one answer, and in fact, if we take the maximal one of these values we will get a rather precise approximation. Specifically, we can state the following approximation\footnote{We assume the following definition for
$f\,(x) = g\,(x) + \Theta\,(h\,(x))$: \[\exists C_1, C_2 \in \mathcal{R^{+}}, \, \forall x : g\,(x) + C_1 \cdot h\,(x) \le f\,(x) \le g\,(x) + C_2 \cdot h\,(x) \]}
for $t\,(s \otimes g)$.

\begin{replemma}{lem:otimes_t_approximation}
\[
 t\,(s \otimes g)  =  t\,(s) + \left({\sum\limits_{a_i \in \tra{s}}} t\,(\taskst{g}{a_i})\right) +
 \Theta\,(d\,(s) + \smashoperator[lr]{\sum\limits_{a_i \in \tra{s}}} d\,(\taskst{g}{a_i}) - \smashoperator{\maxd\limits_{a_i \in \tra{s}}} d\,(\taskst{g}{a_i}))	
\]
\end{replemma}

Hereafter we use the following notation: $\displaystyle{\maxd}\, S = \max\,(S\cup\{0\})$.
We can see that the part under $\Theta$ is very similar to the $(\star)$ except that here we exclude $d$ value for one of the answers from the sum.
This difference is essential and, as we will see later, it is in fact responsible for the difference in complexities for our motivating example.
 


\section{Complexity Analysis via Symbolic Execution}
\label{sec:symbolic}

Our approach to complexity analysis is based on a semi-automatic procedure involving symbolic execution.
In the previous section, we presented formulae to compositionally estimate the complexity factors for
\emph{non-leaf states} of operational semantics under the assumption that corresponding estimations for
\emph{leaf states} are given. In order to obtain corresponding estimations for relations as a whole, we
would need to take into account the effects of relational invocations, including the recursive ones.

Another observation is that as a rule we are interested in complexity estimations in terms of some \emph{metatheory}. For
example, dealing with relations on lists we would be interested in estimations in terms of list lengths,
with trees~--- in terms of depth or number of nodes, with numbers~--- in terms of their values, etc. It is
unlikely that a generic term-based framework would provide such specific information automatically. Thus,
a viable approach would be to extract some inequalities involving the complexity factors of certain relational
calls automatically and then let a human being solve these inequalities in terms of a relevant metatheory.

For the sake of clarity we will provide a demonstration of complexity analysis for a specific example~---
\lstinline|append$^o$| relation from the introduction~--- throughout the section.

The extraction procedure utilizes a symbolic execution technique and is completely automatic.
It turns out that the semantics we have is already abstract enough to be used for symbolic
execution with minor adjustments.
In this symbolic procedure, we mark some of the logic variables as ``grounded'' and at certain
moments substitute them with ground terms.
Informally, for some goal with some free logic variables we consider the complexity of a search
procedure which finds the bindings for all non-grounded variables based on the ground values
substituted for the grounded ones.
This search procedure is defined precisely by the operational semantics; however, as the concrete
values of grounded variables are unknown (only the fact of their \emph{groundness}), the whole
procedure becomes symbolic.
In particular, in unification the groundness can be propagated to some non-grounded free variables.
Thus, the symbolic execution is determined by a set of grounded variables (hereafter denoted
as $V\subset\mathcal A$). The initial choice of $V$ determines the problem we analyze.

In our example the objective is to study the execution when we specialize the first two arguments with
ground values and leave the last argument free. Thus, we start with the goal \lstinline|append$^o$ $a$ $b$ $ab$|
(where $a$, $b$ and $ab$ are distinct free logic variables) and set the initial $V = \{ a, b \}$.

We can make an important observation that both complexity factors ($d$ and $t$) are stable w.r.t. the renaming of
free variables; moreover, they are also stable w.r.t. the change of the fresh variables counter as long as it stays
adequate, and change of current substitution, as long as it gives the same terms after application.
Formally, the following lemma holds.

\begin{replemma}{lem:measures_changing_env}
Let $s = \taskst{g}{\mkenv{\sigma}{n}}$ and $s' = \taskst{g^\prime}{\mkenv{\sigma^\prime}{n^\prime}}$ be two well-formed states.
If  there exists a bijective substitution $\pi \colon FV\,(g \sigma) \to FV\,(g^\prime \sigma^\prime)$ such that
$g \sigma \pi = g^\prime \sigma^\prime $, then $d\,(s) = d\,(s^\prime)$ and $t\,(s) = t\,(s^\prime)$.
\end{replemma}

The lemma shows that the set of states for which a call to relation has to be analyzed can
be narrowed down to a certain family of states. 

\begin{definition} Let $g$ be a goal. An initial state for $g$ is $init\,(g)=\taskst{g}{(\varepsilon, \ninit\,(g))} $
with $ n_{init}\,(g) = \min\, \{ n \mid FV\,(g) \subseteq \{ \alpha_1\dots\alpha_n \} \} $
\end{definition}

Due to the Lemma~\ref{lem:measures_changing_env} it is sufficient for analysis of a relational call to consider only the family of initial states since an arbitrary call state
encountered throughout the execution can be transformed into an initial one while preserving both
complexity factors. Thus, the analysis can be performed in a compositional manner where each call can be analyzed separately.
For our example the family of initial states is $q^{app}(\mathbf{a}, \mathbf{b}) = init\,(\lstinline|append$^o$ $\mathbf{a}$ $\mathbf{b}$ $ab$|)$ for arbitrary ground
terms $\mathbf{a}$ and $\mathbf{b}$.

As we are aiming at the complexity estimation depending on specific ground values substituted for grounded variables, in general case extracted
inequalities have to be parameterized by \emph{valuations}~--- mappings from the set of grounded variables to ground terms. As the new variables
are added to this set during the execution, the valuations need to be extended for these new variables. The following definition introduces this notion.

\begin{definition}
  Let $ V \subset U \subset \mathcal{A} $ and $ \rho \colon V \to \mathcal{T}_{\emptyset} $ and $ \rho^\prime \colon U \to \mathcal{T}_{\emptyset} $ be two valuations. We say that $\rho^\prime$ extends $\rho$ (denotation: $ \rho^\prime \succ \rho$) if $\rho^\prime\,(x) = \rho\,(x)$ for all $x \in V$.
\end{definition}

The main objective of the symbolic execution in our case is to find constraints on valuations for every leaf goal in the body of a relation that determine
whether the execution will continue and how a valuation changes after this goal. For internal relational calls, we describe constraints in terms of denotational
semantics (to be given some meaning in terms of metatheory later). We can do it because of a precise equivalence between the answers found by operational semantics
and values described by denotational semantics thanks to soundness and completeness as well as our requirements of grounding and uniqueness of answers.
Our symbolic treatment of equalities relies on the fact that substitutions of ground terms commute, in a certain sense, with unifications. More specifically,
we can use the most general unifier for two terms to see how unification goes for two terms with some free variables substituted with ground terms.
The most general unifier may contain bindings for both grounded and non-grounded variables. A potential most general unifier for terms after substitution
contains the same bindings for non-grounding terms (with valuation applied to their rhs), while bindings for grounding variables turn into equations that
should be satisfied by the unifier with ground value on the left and bound term on the right. In particular, this means that all variables in bindings for
grounded variables become grounded, too. We can use this observation to define an iterative process that determines the updated set of grounded
variables $\upd{U}{\delta}$ for a current set $U$ and a most general unifier $\delta$ and a set of equations $\constr{\delta}{U}$ that should be
respected by the valuation.

\[
\begin{array}{rcl}
\upd{U}{\delta} &=& \begin{cases}
                           U & \quad\forall x \in U : FV\,(\delta\,(x)) \subset U \\
                           \upd{U \cup \displaystyle\bigcup\limits_{x \in U} FV\,(\delta\,(x))}{\delta} & \quad\mbox{otherwise}
                          \end{cases}\\
\constr{\delta}{U} &=& \{ x = \delta\,(x) \mid x \in U \cap \mathcal{D}om\,(\delta) \}
\end{array}
\]

Using these definitions we can describe symbolic unification by the following lemma.

\begin{replemma}{lem:symbolic_unification_soundness}
Let $t_1$, $t_2$ be terms,  $V \subset \mathcal{A}$ and $\rho \colon V \to \grterms$ be a valuation. If $mgu\,(t_1, t_2) = \delta$ and $U = \upd{V}{\delta} $  then $t_1 \rho$ and $t_2 \rho$ are unifiable iff there is some $\rho' \colon U \to \grterms$ such that $\rho' \succ \rho$ and $\forall (y = t) \in \constr{\delta}{U}\,:\, \rho'(y) = t \rho'$.
In such case $\rho'$ is unique and $ \rho \circ mgu\,(t_1 \rho, t_2 \rho) = \delta\circ\rho' $ up to alpha-equivalence (e.g. there exists a bijective substitution $\pi : FV(t_1) \to FV(t_2)$, s.t. $ \rho \circ mgu\,(t_1 \rho, t_2 \rho) = \delta \circ\rho'\circ \pi$).
\end{replemma}

%\begin{wrapfigure}{r}{0.4\textwidth}
%\setlength{\belowcaptionskip}{-15pt plus 3pt minus 2pt}
\begin{figure}[t]
\centering
\[\renewcommand{\arraystretch}{1}
\begin{array}{cccp{1cm}c}
  \schemewithvset{\mathfrak{S}}{\Upsilon} & = &\schemenode{$\unigoal{\mathcal{T}_\mathcal{A}}{\mathcal{T}_\mathcal{A}}$} & &\schemenode{$\invokegoal{R^k}{\mathcal{T}_\mathcal{A}}{\mathcal{T}_\mathcal{A}}$} \\[6mm] 
                                          &   &\schemefork{$\schemewithvset{\mathfrak{S}}{\Upsilon}$}{$\schemewithvset{\mathfrak{S}}{\Upsilon}$} &&
                                               \schemesarrow{$\unigoal{\mathcal{T}_\mathcal{A}}{\mathcal{T}_\mathcal{A}}$}{$\{\mathcal{A}=\mathcal{T}_\mathcal{A}\}$}{$\schemewithvset{\mathfrak{S}}{\Upsilon}$}\\[6mm]  
                                          &   &\multicolumn{3}{c}{\schemedarrow{$\invokegoal{R^k}{\mathcal{T}_\mathcal{A}}{\mathcal{T}_\mathcal{A}}$}{$(\mathcal{T}_\mathcal{A}, \dots, \mathcal{T}_\mathcal{A}) \in \sembr{R^k}$}{$\schemewithvset{\mathfrak{S}}{\Upsilon}$}}
\end{array}\]
\caption{Symbolic Scheme Forms}
\label{fig:scheme_fragments}
%\end{wrapfigure}
\end{figure}

\textcolor{red}{Reviewer about \figureword~\ref{fig:scheme_fragments} and \figureword~\ref{fig:scheme_formation}: typesetting makes the figure hard to read, missing explanations, comments and definitions do not help}

In our description of the extraction process, we use a visual representation of symbolic execution of a relation body for a given set of grounded variables in a form of a \emph{symbolic scheme}.
A symbolic scheme is a tree-like structure with different branches corresponding to execution of different disjuncts and nodes corresponding to equalities and relational calls in the body
augmented with subsets of grounded variables at the point of execution.\footnote{Note the difference with conventional symbolic execution graphs
with different branches representing mutually exclusive paths of evaluation, not the different parts within one evaluation.} Constraints for substituted
grounded variables that determine whether the execution continues are presented as labels on the edges of a scheme.



Each scheme is built as a composition of the five patterns, shown in \figureword~\ref{fig:scheme_fragments} (all schemes are indexed by subsets of grounded variables with $\Upsilon = 2^{\mathcal{A}}$ denoting such subsets).

Note, the constraints after nodes of different types differ: unification puts a constraint in a form of a set of equations on substituted ground values that should be respected while
relational call puts a constraint in a form of a tuple of ground terms that should belong to the denotational semantics of a relation.

The construction of a scheme for a given goal (initially, the body of a relation) mimics a regular execution of a relational program. The derivation rules for scheme
formation have the following form $\schemetrans{g}{\Gamma}{\sigma}{n}{V}{\schemewithvset{\mathfrak{S}}{V}}$. Here $g$ is a goal, $\Gamma$ is a list of \emph{deferred}
goals (these goals have to be executed after the execution of $g$ in every branch in the same order,
initially this list is empty; this resembles continuations, but the analogy is not complete), $\sigma$ and $n$ are substitution and counter from the current
environment respectively, $V$ is a set of grounded variables at the moment.

\begin{figure}[t]
%\setlength{\belowcaptionskip}{-30pt plus 3pt minus 2pt}
\renewcommand{\arraystretch}{2}
  \[
\begin{array}{cr}
  \onepremrule
		{  \schemetrans{g_1}{g_2 : \Gamma}{\sigma}{n}{V}{\schemewithvset{\mathfrak{S}}{V}}  } 
		{  \schemetrans{\conjgoal{g_1}{g_2}}{\Gamma}{\sigma}{n}{V}{ \schemewithvset{\mathfrak{S}}{V} }  } & \ruleno{Conj$_\mathfrak S$}
		\\[5mm]
                
  % \multicolumn{3}{c}{
  \twopremrule
		{  \schemetrans{g_1}{\Gamma}{\sigma}{n}{V}{\schemewithvset{\mathfrak{S_1}}{V}}  }
		{  \schemetrans{g_2}{\Gamma}{\sigma}{n}{V}{\schemewithvset{\mathfrak{S_2}}{V}}  }
		{  \schemetrans{\disjgoal{g_1}{g_2}}{\Gamma}{\sigma}{n}{V}{\parbox[m]{2cm}{ \schemefork{$\schemewithvset{\mathfrak{S_1}}{V}$}{$\schemewithvset{\mathfrak{S_2}}{V}$}} }  } & \ruleno{Disj$_\mathfrak S$}\\[10mm] 
		
		
 \onepremrule
		{  \schemetrans{\substitute{g}{\alpha_n}{\ttt{x}}}{\Gamma}{\sigma}{n + 1}{V}{\schemewithvset{\mathfrak{S}}{V}}  }
		{  \schemetrans{\freshgoal{\ttt{x}}{g}}{\Gamma}{\sigma}{n}{V}{ \schemewithvset{\mathfrak{S}}{V} }  } & \ruleno{Fresh$_\mathfrak S$}\\[5mm]

 %\multicolumn{3}{c}{
  \schemetrans{\unigoal{t_1}{t_2}}{\epsilon}{\sigma}{n}{V}{\parbox[m]{2cm}{\schemenode{$\unigoal{t_1 \sigma}{t_2 \sigma}$}}}& \ruleno{UnifyLeaf$_\mathfrak S$}\\[5mm]

 %\multicolumn{3}{c}{
 \schemetrans{\invokegoal{R^k}{t_1}{t_k}}{\epsilon}{\sigma}{n}{V}{\parbox[m]{2cm}{\schemenode{$\invokegoal{R^k}{t_1 \sigma}{t_k \sigma}$}}}& \ruleno{InvokeLeaf$_\mathfrak S$}\\[5mm] 
		
 %\multicolumn{3}{c}{
  \onepremrule
		{  \nexists mgu\,(t_1 \sigma, t_2 \sigma)  }
		{  \schemetrans{\unigoal{t_1}{t_2}}{g : \Gamma}{\sigma}{n}{V}{\parbox[m]{2cm}{\schemenode{$\unigoal{t_1 \sigma}{t_2 \sigma}$}}} }& \ruleno{UnifyFail$_\mathfrak S$}\\[5mm]

 %\multicolumn{3}{c}{
  \threepremrule
		{  mgu\,(t_1 \sigma, t_2 \sigma) = \delta  }
		{  U = \upd{V}{\delta}  }
		{  \schemetrans{g}{\Gamma}{\sigma \delta}{n}{U}{\schemewithvset{\mathfrak{S}}{U}}  }
		{  \schemetrans{\unigoal{t_1}{t_2}}{g : \Gamma}{\sigma}{n}{V}{\parbox[m]{2cm}{\schemesarrow{$\unigoal{t_1 \sigma}{t_2 \sigma}$}{$\constr{\delta}{U}$}{$\schemewithvset{\mathfrak{S}}{U}$}} }   } & \ruleno{UnifySuccess$_\mathfrak S$}\\[15mm]
		
 %\multicolumn{3}{c}{
  \twopremrule
		{  \mbox{\phantom{XXXXXX}} U =  V \cup \displaystyle\bigcup\limits_{i} FV\,(t_i \sigma) }
		{  \schemetrans{g}{\Gamma}{\sigma}{n}{U}{\schemewithvset{\mathfrak{S}}{U}} \mbox{\phantom{XXXXXX}} }
		{  \schemetrans{\invokegoal{R^k}{t_1}{t_k}}{g : \Gamma}{\sigma}{n}{V}{ \parbox[m]{2cm}{\schemedarrow{$\invokegoal{R^k}{t_1 \sigma}{t_k \sigma}$}{$ (t_1 \sigma, \dots, t_k \sigma) \in \sembr{R^k} $}{$\schemewithvset{\mathfrak{S}}{U}$}} }   } & \ruleno{Invoke$_\mathfrak S$}
 \end{array}
\]
\caption{Scheme Formation Rules}
\label{fig:scheme_formation}
\end{figure}

The rules are shown in \figureword~\ref{fig:scheme_formation}. \ruleno{Conj$_\mathfrak S$} and \ruleno{Disj$_\mathfrak S$} are structural rules: when investigating conjunctions we defer
the second conjunct by adding it to $\Gamma$ and continue with the first conjunct; disjunctions simply result in forks. \ruleno{Fresh$_\mathfrak S$} introduces a fresh logic
variable (not grounded) and updates the counter of occupied variables. When the investigated goal is equality or relational call it is added as a node to the scheme. If there are
no deferred goals, then this node is a leaf (rules \ruleno{UnifyLeaf$_\mathfrak S$} and \ruleno{InvokeLeaf$_\mathfrak S$}). Equality is also added as a leaf if there are some deferred goals,
but the terms are non-unifiable and so the execution stops (rule \ruleno{UnifyFail$_\mathfrak S$}). If the terms in the equality are unifiable and there are deferred goals
(rule \ruleno{UnifySuccess$_\mathfrak S$}), the equality is added as a node and the execution continues for the deferred goals, starting from the leftmost one; also the set of grounded variables
is updated and constraint labels are added for the edge in accordance with \lemmaword~\ref{lem:symbolic_unification_soundness}. For relational calls the proccess is similar: if there are some deferred goals
(rule \ruleno{Invoke$_\mathfrak S$}), all variables occurring in a call become grounded (due to the grounding condition we imposed) and should satisfy the denotational semantics
of the invoked relation.

The scheme constructed by these rules for our \lstinline|append$^o$| example is shown in \figureword~\ref{fig:example_scheme}. For simplicity, we do not show the set of grounded
variables for each node, but instead overline grounded variables in-place. Note, all variables that occur in constraints on the edges are grounded after parent node execution.

Now, we can use schemes to see how the information for leaf goals in relation body is combined with conjunctions and disjunctions. Then we can apply formulae
from \sectionword~\ref{sec:scheduling} to get recursive inequalities (providing lower and upper bounds simultaneously) for both complexity factors.

In these inequalities we need to sum up the values of $d$ and $t$-factor for all leaf goals of a body and for all environments these goals are evaluated for.
The leaf goals are
the nodes of the scheme and evaluated environments can be derived from the constraints attached to the edges. So, for this summation we introduce the following notions: $\mathcal{D}$
is the sum of $d$-factor values and $\mathcal{T}$ is the sum of $t$-factor values for the execution of the body with specific valuation $\rho$.

\begin{wrapfigure}{r}{0.5\textwidth}
\begin{center}
\begin{tikzpicture}[level distance=30pt, sibling distance=10em, edge from parent/.style={draw,-latex}]
   \coordinate   
      child { node {$\unigoal{\overline{a}}{\texttt{Nil}}$}
        child { node {$\unigoal{ab}{\overline{b}}$}
                  edge from parent node[right]{\tiny{${a} = \texttt{Nil}$}} } }
      child { node {$\unigoal{\overline{a}}{\texttt{Cons($h$, $t$)}}$} 
      	child { node {\texttt{append$^o$ $\overline{t}$ $\overline{b}$ $tb$}}
      	   child { node {$\unigoal{ab}{\texttt{Cons($\overline{h}$, $\overline{tb}$)}}$}
      	             edge from parent node[right]{\tiny{$({t}, {b}, {tb}) \in \llbracket \texttt{append$^o$} \rrbracket$}}  }
      	   edge from parent node[right]{\tiny{${a} = \texttt{Cons(${h}$, ${t}$)}$}}  } } ;
\end{tikzpicture}
\end{center}
\caption{Symbolic execution scheme for the goal  \lstinline|append$^o$ $\,a\;$ $b\;$ $ab$|  with initial set of grounded variables $V = \{ a, b \}$. For each node, variables that
  are grounded at the point of execution of this node are overlined. }
\label{fig:example_scheme}
\end{wrapfigure}

 Their definitions are shown in
\figureword~\ref{fig:scheduling_extraction_d_t} (both formulas are given in the same figure as the definitions coincide modulo factor denotations). For nodes, we take
the corresponding value (for equality it always equals $1$). When going through an equality we sum up the rest with an updated valuation (by \lemmaword~\ref{lem:symbolic_unification_soundness}
this sum always has one or zero summands depending on whether the unification succeeds or not). When going through a relational call we take a sum of all valuations that satisfy the
denotational semantics (these valuations will correspond exactly to the set of all answers produced by the call since operational semantics is sound and complete w.r.t. the denotational
one and because we require all answers to be unique). For disjunctions, we take the sum of both branches.

\begin{figure}[t]
\[
\begin{array}{rclcl}
 \mathcal{\nicefrac{D}{T}}\,(&\parbox[m]{1.3cm}{\schemenode{$\unigoal{t_2}{t_2}$}}&)(\rho) &=& 1  \\

 \mathcal{\nicefrac{D}{T}}\,(&\parbox[m]{2.5cm}{\schemenode{$\invokegoal{R^k}{t_1}{t_k}$}}&)(\rho) &=& \nicefrac{d}{t}\,(init\,(\invokegoal{R^k}{t_1 \rho}{t_k \rho})) \\

 \mathcal{\nicefrac{D}{T}}\,(&\parbox[m]{2cm}{\schemesarrow{$\unigoal{t_1}{t_2}$}{$Cs$}{$\schemewithvset{\mathfrak{S}}{U}$}} &)(\rho) &=& 1 +
      \smashoperator{\sum\limits_{\substack{ \rho' \colon V \to \grterms \\
                                      \rho' \succ \rho \\
                                      \forall (y, t) \in Cs\,:\, \rho'\,(y) = t\, \rho'  }}}
           \mathcal{\nicefrac{D}{T}}\,(\schemewithvset{\mathfrak{S}}{U})(\rho')  \\

 \mathcal{\nicefrac{D}{T}}\,(& \parbox[m]{4cm}{\schemedarrow{$\invokegoal{R^k}{t_1}{t_k}$}{ $(t_1, \dots, t_k) \in \sembr{R^k}  $}{$\schemewithvset{\mathfrak{S}}{U}$}} &)(\rho) &=&
      \nicefrac{d}{t}\,(init\,(\invokegoal{R^k}{t_1 \rho}{t_k \rho})) +
      \smashoperator{\sum\limits_{\substack{ \rho' \colon V \to \grterms \\
                                      \rho' \succ \rho \\
                                      (t_1 \rho', \dots, t_k \rho') \in \sembr{R^k}  }}}
           \mathcal{\nicefrac{D}{T}}\,(\schemewithvset{\mathfrak{S}}{U})(\rho')  \\

 \mathcal{\nicefrac{D}{T}}\,(&\parbox[m]{2.5cm}{\schemefork{$\schemewithvset{\mathfrak{S}_1}{V}$}{$\schemewithvset{\mathfrak{S}_2}{V}$}}&)(\rho) &=&
 \mathcal{\nicefrac{D}{T}}\,(\schemewithvset{\mathfrak{S}_1}{V})(\rho) + \mathcal{\nicefrac{D}{T}}\,(\schemewithvset{\mathfrak{S}_2}{V})(\rho)
\end{array}
\]
\caption{Complexity Factors Extraction: $\mathcal D$ and $\mathcal T$}
\label{fig:scheduling_extraction_d_t}
\end{figure}

\begin{figure}[t]
\[
\begin{array}{rclcl}
 \mathcal{L}\,(&\parbox[m]{1.3cm}{\schemenode{$\unigoal{t_2}{t_2}$}}&)(\rho) &=& \{init\,(\unigoal{t_2}{t_2})\} \\
 \mathcal{L}\,(&\parbox[m]{2.5cm}{\schemenode{$\invokegoal{R^k}{t_1}{t_k}$}}&)(\rho) &=& \{init\,(\invokegoal{R^k}{t_1 \rho}{t_k \rho})\} \\
 \mathcal{L}\,(&\parbox[m]{2cm}{\schemesarrow{$\unigoal{t_1}{t_2}$}{$Cs$}{$\schemewithvset{\mathfrak{S}}{U}$}} &)(\rho) &=&  \smashoperator{\bigcup\limits_{\substack{ \rho' \colon V \to \grterms \\
                                      \rho' \succ \rho \\
                                      \forall (y, t) \in Cs\,:\, \rho'\,(y) = t\, \rho'  }}}
           \mathcal{L}\,(\schemewithvset{\mathfrak{S}}{U})(\rho')  \\
 \mathcal{L}\,(& \parbox[m]{4cm}{\schemedarrow{$\invokegoal{R^k}{t_1}{t_k}$}{ $(t_1, \dots, t_k) \in \sembr{R^k}  $}{$\schemewithvset{\mathfrak{S}}{U}$}} &)(\rho) &=&
      \smashoperator{\bigcup\limits_{\substack{ \rho' \colon V \to \grterms \\
                                      \rho' \succ \rho \\
                                      (t_1 \rho', \dots, t_k \rho') \in \sembr{R^k}  }}}
           \mathcal{L}\,(\schemewithvset{\mathfrak{S}}{U})(\rho')  \\
 \mathcal{L}\,(&\parbox[m]{2.5cm}{\schemefork{$\schemewithvset{\mathfrak{S}_1}{V}$}{$\schemewithvset{\mathfrak{S}_2}{V}$}}&)(\rho) &=&
 \mathcal{L}\,(\schemewithvset{\mathfrak{S}_1}{V})(\rho) \cup \mathcal{L}\,(\schemewithvset{\mathfrak{S}_2}{V})(\rho)
\end{array}
\]
\caption{Complexity Factors Extraction: $\mathcal L$}
\label{fig:scheduling_extraction_l}
\end{figure}

As we saw in \sectionword~\ref{sec:scheduling} when computing the scheduling factors we need to exclude from the additional cost the value of $d$-factor
for one of the environments (the largest one). This is true for the generalized formula for a whole scheme, too. This time we need to take all executed
environments for all the leaves of a scheme and exclude the $d$-factor value for a maximal one (the formula for conjunction ensures that we make the exclusion
for the leaf, and the formula for disjunction ensures that we make it for only one of the leaves). So, we will need additional notion $\mathcal{L}$,
similar to $\mathcal{D}$ and $\mathcal{T}$ that will collect all the goals of the form $init\,(g_i \rho)$, where $g_i$ is a leaf goal and $\rho$ is a
valuation corresponding to one of the environments this leaf is evaluated for. \textcolor{blue}{The definition of  $\mathcal{L}$ is shown in \figureword~\ref{fig:scheduling_extraction_l}}.

Now we can formulate the following main theorem that provides the principal recursive inequalities, extracted from the scheme for a given goal.

\begin{reptheorem}{extracted_approximations}
Let $g$ be a goal, and let $\schemetrans{g}{\epsilon}{\varepsilon}{n_{init}(g)}{V}{\schemewithvset{\mathfrak{S}}{V}}$. Then

\[
\begin{array}{rcl}
    d\,(init\,(g\,\rho)) &=& \mathcal{D}\,(\schemewithvset{\mathfrak{S}}{V})(\rho) + \Theta\,(1) \\
   t\,(init\,(g\,\rho)) &=& \mathcal{T}\,(\schemewithvset{\mathfrak{S}}{V})(\rho) + \Theta\,(\mathcal{D}\,(\schemewithvset{\mathfrak{S}}{V})(\rho)
   - \smashoperator{\maxd\limits_{\taskst{g_i}{e_i} \in \mathcal{L}(\schemewithvset{\mathfrak{S}}{V})(\rho)}} d\,(\taskst{g_i}{e_i}) + 1)
\end{array}
\]

\noindent being considered as functions on $\rho \colon V \to T_{\emptyset}$
\end{reptheorem}

The theorem allows us to extract two inequalities (upper and lower bounds) for both factors with a multiplicative constant that is the same for all valuations.

For our example, we can extract the following recursive inequalities from the scheme in \figureword~\ref{fig:example_scheme}. For presentation purposes, we will
not show valuation in inequalities explicitly, but instead show the ground values of grounded variables (using variables in bold font) that determine each valuation.
We can do such a simplification for any concrete relation.

\[
d\,(q^{app}\,(\mathbf{a}, \mathbf{b}))  = (1 + \sum\limits_{\mathbf{a} = \texttt{Nil}} 1) + (1 + \smashoperator{\sum\limits_{\mathbf{h}, \mathbf{t}: \mathbf{a} = \texttt{Cons($\mathbf{h}$, $\mathbf{t}$)}}} (d\,(q^{app}\,(\mathbf{t}, \mathbf{b})) + \smashoperator{\sum\limits_{\mathbf{tb} : (\mathbf{t}, \mathbf{b}, \mathbf{tb}) \in \llbracket \texttt{append$^o$} \rrbracket}} 1)) + \Theta\,(1)
\]\\[0.8mm]
\[
\begin{array}{lclc}
t\,(q^{app}\,(\mathbf{a}, \mathbf{b})) & = & (1 + \sum\limits_{\mathbf{a} = \texttt{Nil}} 1) + (1 + \smashoperator{\sum\limits_{\mathbf{h}, \mathbf{t}: \mathbf{a} = \texttt{Cons($\mathbf{h}$, $\mathbf{t}$)}}} (t\,(q^{app}\,(\mathbf{t}, \mathbf{b})) + \smashoperator{\sum\limits_{\mathbf{tb} : (\mathbf{t}, \mathbf{b}, \mathbf{tb}) \in \llbracket \texttt{append$^o$} \rrbracket}} 1)) & + \\
 & &\Theta((1 + \sum\limits_{\mathbf{a} = \texttt{Nil}} 1) + (1 + \smashoperator{\sum\limits_{\mathbf{h}, \mathbf{t}: \mathbf{a} = \texttt{Cons($\mathbf{h}$, $\mathbf{t}$)}}} (d\,(q^{app}\,(\mathbf{t}, \mathbf{b})) + \smashoperator{\sum\limits_{\mathbf{tb} : (\mathbf{t}, \mathbf{b}, \mathbf{tb}) \in \llbracket \texttt{append$^o$} \rrbracket}} 1)) & - \\
& & \smashoperator{\maxd\limits_{\substack{
                                   \mathbf{h}, \mathbf{t}, \mathbf{tb}: \mathbf{a} = \texttt{Cons($\mathbf{h}$, $\mathbf{t}$)} \land \\
                                   (\mathbf{t}, \mathbf{b}, \mathbf{tb}) \in \llbracket \texttt{append$^o$} \rrbracket
                                 }
                                }} \{ d\,(init\,(\unigoal{ab}{\mathbf{b}})), d\,(init\,(\unigoal{ab}{\texttt{Cons($\mathbf{h}$, $\mathbf{tb}$)}})) \} & + 1) 
\end{array}
\]


Automatically extracted recursive inequalities, as a rule, are cumbersome, but they contain all the information on how scheduling affects the complexity.
Often they can be drastically simplified by using metatheory-level reasoning.

For our example, we are only interested in the case when substituted values represent some lists. We thus perform the usual for lists case analysis
considering the first list empty or non-empty. We can also notice that the excluded summand equals one. So we can rewrite the inequalities in the following way:

\[
\begin{array}{lcl}
d\,(q^{app}\,(\texttt{Nil}, \mathbf{b})) & = & \Theta\,(1) \\
d\,(q^{app}\,(\texttt{Cons($\mathbf{h}$, $\mathbf{t}$)}, \mathbf{b})) & = & d\,(q^{app}\,(\mathbf{t}, \mathbf{b})) + \Theta\,(1) \\
t\,(q^{app}\,(\texttt{Nil}, \mathbf{b})) & = & \Theta\,(1) \\
t\,(q^{app}\,(\texttt{Cons($\mathbf{h}$, $\mathbf{t}$)}, \mathbf{b})) & = & t\,(q^{app}\,(\mathbf{t}, \mathbf{b})) + \Theta\,(d\,(q^{app}\,(\mathbf{t}, \mathbf{b}))) \\
\end{array}
 \]
 
These trivial linear inequalities can be easily solved:

\[
\begin{array}{lcl}
d\,(q^{app}\,(\mathbf{a}, \mathbf{b})) & = & \Theta\,(len\,(\mathbf{a})) \\
t\,(q^{app}\,(\mathbf{a}, \mathbf{b})) & = & \Theta\,(len^2\,(\mathbf{a})) \\
\end{array}
 \]
 
In this case, scheduling makes a big difference and changes the asymptotics. Note, we expressed the result using notions from metatheory
($len$ for the length of the list represented by a term).

In contrast, if we consider the optimal definition \lstinline|append$_{opt}^o$| the analysis of the call
$q^{app\text{-}opt}\,(\mathbf{a}, \mathbf{b}) = init\,(\texttt{append$_{opt}^o$} \, \mathbf{a} \, \mathbf{b} \, ab)$ is analogous,
but among the candidates for exclusion there is the value $d\,(q^{app\text{-}opt}\,(\mathbf{t}, \mathbf{b}))$ since the recursive
call is placed in a leaf. So the last simplified recursive approximation is the following (the rest is
the same as in our main example):

\[t\,(q^{app\text{-}opt}\,(\texttt{Cons\,($\mathbf{h}$, $\mathbf{t}$)}, \mathbf{b})) = t\,(q^{app\text{-}opt}\,(\mathbf{t}, \mathbf{b})) + \Theta\,(1) \]

So in this case the complexity of both factors is linear on $len\,(\mathbf{a})$.


\section{Evaluation}

\label{sec:evaluation}

In this section, we present an evaluation of 
implemented constructive negation on a series of examples.

\subsection{If-then-else}

Using relational if-then-else operator, 
presented in section~\ref{sec:ifte},
we have implemented several 
higher-order relations over lists, namely 
\lstinline{find} (Listing~\ref{lst:eval-find}), 
\lstinline{remove}\footnote{Note, this implementation 
differs from the one in Section~\ref{sec:intro}, but 
it is easy to see that these two are semantically equivalent.} (Listing~\ref{lst:eval-remove}) 
and \lstinline{filter} (Listing~\ref{lst:eval-filter}).
These relations are almost identical (syntactically) to their
functional implementations.
We have tested that these relations can be run
in various directions and produce the expected results.
For example, the goal \lstinline{filter p q q}
with the predicate \lstinline{p} equal to

\begin{lstlisting}
  fun l -> fresh (x) (l === [x])
\end{lstlisting}

stating that the given list should be a singleton list,
starts to generate all singleton lists.
Vice versa, the goal \lstinline{filter p q []} 
with that same \lstinline{p} generates 
all lists, constrained to be not a singleton list.

Listings~\ref{lst:eval-p}-\ref{lst:eval-filter-queries} give 
more concrete examples of queries to these relations.
In the listing the syntax \lstinline{run n q g}
means running a goal \lstinline{g} with 
the free variable \lstinline{q}
taking the first \lstinline{n} answers (``\lstinline{*}'' denotes all answers).
After the sign $\leadsto$ the result of the query is given.
The result \lstinline{fail} means that the query has failed.
The result \lstinline[mathescape]|succ {{a$_1$}; ... {a$_n$}} |
means that the query has succeeded delivering $n$ answers.
Each answer represents a set of constraint on free variables.
Constraints are of two forms: equality constraints, e.g. \lstinline{q = (1, _.$_0$)}, 
or disequality constraints, e.g. \lstinline{q $\neq$ (1, _.$_0$)}.
The terms of the form \lstinline{_.$_i$} in the answer
denote some universally quantified variables.

\begin{minipage}[thb]{.3\textwidth}
\begin{lstlisting}[
  caption={A definition of \code{find} relation},
  label={lst:eval-find}
]
let find p e xs =
  fresh (x xs' ys') (
    xs === x::xs' /\
    ifte (p x)
      (e === x)
      (find p e xs')
  )
\end{lstlisting}
\end{minipage}\hfill
\begin{minipage}[thb]{.3\textwidth}
\begin{lstlisting}[
  caption={A definition of \code{remove} relation},
  label={lst:eval-remove}
]
let remove p xs ys =
  (xs === [] /\ ys === [])
  \/
  fresh (x xs' ys') (
    xs === x::xs' /\
    ifte (p x)
      (ys === xs')
      (ys === x::ys' /\ 
       remove p xs' ys')
  )
\end{lstlisting}
\end{minipage}\hfill
\begin{minipage}[thb]{.3\textwidth}
\begin{lstlisting}[
  caption={A definition of \code{filter} relation},
  label={lst:eval-filter}
]
let filter p xs ys =
  (xs === [] /\ ys === [])
  \/
  fresh (x xs' ys') (
    xs === x::xs' /\
    (ifte (p x)
      (ys === x :: ys')
      (ys === ys')) /\
    filter p xs' ys'
  )
\end{lstlisting}
\end{minipage}

% \vspace{3cm}

\begin{minipage}[thb]{0.4\textwidth}
\begin{lstlisting}[
  caption={Definition of the predicate \lstinline{p}},
  label={lst:eval-p}
]
let p l = fresh (x) (l === [x])
\end{lstlisting}
\begin{lstlisting}[
  caption={Example of queries to \lstinline{find}},
  label={lst:eval-find-queries}
]
run 3 q (fresh (e) find p e q) 
$\leadsto$ succ {
     { q = [_.$_0$] :: _.$_1$ }
     { q = _.$_0$ :: [_.$_1$] :: _.$_2$; 
         _.$_0$ $\neq$ [_.$_3$] }
     { q = _.$_0$ :: _.$_1$ :: [_.$_2$] :: _.$_3$; 
         _.$_0$ $\neq$ [_.$_4$]; _.$_1$ $\neq$ [_.$_5$] }
   }
\end{lstlisting}
\end{minipage}\hfill
\begin{minipage}[thb]{0.4\textwidth}
\begin{lstlisting}[
  caption={Example of queries to \lstinline{remove}},
  label={lst:eval-remove-queries}
]
run * q (fresh (e) remove p q [[ ]]) 
$\leadsto$ succ {
     { q = [[_.$_0$]; [ ]] }
     { q = [[ ]] }
     { q = [[ ]; [_.$_0$]] }
   }

run 3 q (fresh (e) remove p q q) 
$\leadsto$ succ {
     { q = [] }
     { q = [_.$_0$], _.$_0$ $\neq$ [_.$_1$] }
     { q = [_.$_0$; _.$_1$]; 
         _.$_0$ $\neq$ [_.$_2$]; _.$_1$ $\neq$ [_.$_3$] }
   }
\end{lstlisting}
\end{minipage}

\begin{minipage}[thb]{0.4\textwidth}
\begin{lstlisting}[
  caption={Example of queries to \lstinline{filter}},
  label={lst:eval-filter-queries}
]
run 3 q (filter p q q) 
$\leadsto$ succ {
     { q = [ ] }
     { q = [_.$_0$] }
     { q = [_.$_0$; _.$_1$] }
   }

run 3 q (filter p q [1]) 
$\leadsto$ succ {
     { q = [[1]] }
     { q = [_.$_0$; [1]]; _.$_0$ $\neq$ [_.$_1$] }
     { q = [[1]; _.$_0$]; _.$_0$ $\neq$ [_.$_1$] }
   }

run 3 q (filter p q [ ]) 
$\leadsto$ succ {
     { q = [] }
     { q = [_.$_0$]; _.$_0$ $\neq$ [_.$_1$] }
     { q = [_.$_0$; _.$_1$]; 
            _.$_0$ $\neq$ [_.$_2$]; _.$_1$ $\neq$ [_.$_3$] }
   }
\end{lstlisting}
\end{minipage}

\subsection{Universal quantification}

In the Section~\ref{sec:impl-univ} we presented 
the \lstinline{forall} goal constructor 
which is implemented through the double negation.
We have observed, that although \lstinline{forall g}
does not terminate when the goal \lstinline{g x} 
has an infinite number of answers 
(assuming \lstinline{x} is a fresh variable),
it does terminate in the case when \lstinline{g x} has 
a finite number of answers.
The behavior of \lstinline{forall} in this case is sound
even in the presence of disequality constraints or nested quantifiers. 

The Table~\ref{tab:univ} gives some concrete examples.
The left column contains the tested goals\footnote{
We typeset the goals in terms of first-order logic syntax 
instead of \textsc{OCanren} syntax for brevity and clarity.} 
and the right column gives the obtained results.
For the results we use the same notation 
as in the previous section.

\begin{table}[th]
  \centering
  \def\arraystretch{1.5}
  \begin{tabularx}{\textwidth}{|X|X|}
    \hline

    $\forall x\ldotp x = q$ & 
      \texttt{fail} \\
    \hline

    $\forall x\ldotp \exists y\ldotp x = y$ & 
      \texttt{succ \{[q = \_.$_0$]\}} \\
    \hline

    $\forall x\ldotp \exists y\ldotp x = y \wedge y = q$ &
      \texttt{fail} \\
    \hline

    $\forall x\ldotp q = (1, x)$ & 
      \texttt{fail} \\
    \hline

    $\forall x\ldotp \exists y\ldotp y = (1, x)$ & 
      \texttt{succ \{[q = \_.$_0$]\}} \\
    \hline

    $\forall x\ldotp \exists y\ldotp x = (1, y)$ &
      \texttt{fail} \\
    \hline

    $\forall x\ldotp x \neq q$ & \texttt{fail} \\
    \hline

    $\forall x\ldotp \exists y\ldotp x \neq y$ & 
      \texttt{succ \{[q = \_.$_0$]\}} \\
    \hline

    $\forall x\ldotp \exists y\ldotp x \neq y \wedge y = q$ & 
      \texttt{fail} \\
    \hline

    $\forall x\ldotp q \neq (1, x)$ & 
      \texttt{succ \{[q $\neq$ (1, \_.$_0$)]\}} \\
    \hline

    $(\exists x\ldotp q = (1, x)) \wedge (\forall x\ldotp q \neq (1, x))$ & 
      \texttt{fail} \\
    \hline

    $\forall x\ldotp (x, x) \neq (0, 1)$ & 
      \texttt{succ \{[q = \_.$_0$]\}} \\
    \hline

    $\forall x\ldotp (x, x) \neq (1, 1)$ & 
      \texttt{fail} \\
    \hline

    $\forall x\ldotp (x, x) \neq (q, 1)$ & 
      \texttt{succ \{[q $\neq$ 1]\}} \\
    \hline

    $\exists a~ b\ldotp q = (a, b) \wedge \forall x\ldotp (x, x) \neq (a, b)$ & 
      \texttt{succ \{[q = (\_.$_0$, \_.$_1$); \_.$_0$ $\neq$ \_.$_1$]\}} \\
    \hline

  \end{tabularx}
  \caption{\lstinline{forall} evaluation}
  \label{tab:univ}
\end{table}

% !TEX TS-program = pdflatex
% !TeX spellcheck = en_US
% !TEX root = main.tex

\section{Related Work}
\label{related}

GUI design and implementation has been a hot topic for decades. Thus, to no surprise there is a lot of frameworks, approaches, papers
and reports on the subject. A fair share of them (if not all) present declarative and automatic solutions. A careful study, however,
discovers that this ``declarativeness'' and ``automation'' is understood differently then in our case.

First of all, we need to mention some software frameworks and tools for design and implementation of GUI and visualization of data, for example, \textsc{React}~\cite{react},
\textsc{Jetpack Compose}~\cite{Jetpack}, \textsc{SwiftUI}~\cite{SwiftUI}, \textsc{Streamlit}~\cite{Streamlit}, \textsc{D3}~\cite{D3} and others.
These frameworks provide a number of layout primitives which end-users can employ in order to render their data or UI. For example, \textsc{Streamlit}
provides a number of builtin layout primitives like ``columns'', ``container'', ``modal dialog'', etc.~\cite{StreamlitLayout} and an endless
variety of third-party external components. These primitives allow end-users to abstract away of concrete controls coordinate calculation and their
relative alignment; they also prescribe a reasonable behavior on enclosing pane resizing. However, which layout primitives to use is decided by
end-users, not the system. If due to any reason the layout needs to be changed these changes have to be implemented manually. In our case
end-users do not specify concrete layouts, only the logical structure of the UI. The guideline takes care of concrete layout, depending on
external constraints such as enclosing pane size, screen resolution or even regional settings (for example, right-to-left writing system). As long
as the logical structure remains unchanged no interference from end-users is required for laying out the UI in different settings. On the
other hand these frameworks can be used as back-ends in our approach since they provide a similar set of layout primitives.

Constraint programming has already been used for deciding the placement of GUI controls. One of the examples are constraint reactive programming
language \textsc{Wallingford}~\cite{Wallingford2016} and the \textsc{Cassowary} system~\cite{Cassowary2001}. \textsc{Wallingford} allows to attach
constraints of various strength to different values in the program. The system reacts to the time changes and updates these values without violation
of the constraints. For example, one could calculate a width of a GUI control as the sine of current time. The \textsc{Cassowary} system and its
descendants allow to calculate the sizes and positions of controls dynamically, for example at the moment of canvas resize.
%The background theory is linear arithmetic.
It supports many different constraints, for example, Z-ordering, arithmetic operators (for example, a control's width can be the half
of another one's height), overlapping views, etc.  These systems are targeted for the tasks of dynamic adaptation the sizes of controls on resize.
Also, they don't provide support of expressing general rules about correct control placing, the constraints are added for values of concrete structure.
In our work we make a strict separation between GUI structure and rules of correct positioning of controls. We have doubts about expressing non-deterministic
layouts in these systems, for example, if vertical or horizontal placement of controls depend on their sizes. On the other hand, the number of various
constraint types is larger than in our approach. For example, they allow to express overlapping views, but based on our experience we initially ruled out
this option, and our current implementation doesn't generate such layouts at all.

Finally, in recent years methods and approaches from AI in ML/data-science sense start to percolate into the area. Some of the works address much more
ambitious objectives, than ours.

First of all, there is a direction of research on UI code generation from images~\cite{Cai2023}. Given a designer-drawn form, a code
generator recognizes UI controls and their relative placements and generates implementation code for one of GUI frameworks. This approach
is completely orthogonal and incompatible with what we suggest. Indeed, it requires an interaction with a designer when implementing every piece of
an interface while in our case a designer is only involved when guidelines are developed; then our system produces guideline-compatible
interfaces \emph{en masse} automatically. In~\cite{Robust} a slightly different task is addressed: given a picture of an interface synthesize its
implementable layout in terms of Android GUI primitives which would be scalable across various devices while avoiding a typical layout errors.
To achieve this goal, after recognizing UI controls and their locations a certain set of relational layout constraints is extracted. This
set of constraints resembles our layout primitives but is aligned with Android's \texttt{ConstraintLayout} widget~\cite{ConstraintLayout} semantics.
Given these constraints and a set of \emph{robustness properties} developed by the authors an implementation code is generated with
the aid of a probabilistic model trained on a large set of existing Android interfaces. This implementation is more stable under screen size
or resolution change, than that provided by image recognition only. Interestingly, one of the motivations for the work, as the authors
specify explicitly, is that ``the same layout needs to be rendered potentially on more than 15 000 Android devices with $\approx$100 different density
independent screen sizes. Requiring the user to provide and maintain input specifications for all of them is infeasible yet highly desirable.''
That is exactly what our system is capable of doing within a few minutes and with 100\% accuracy, so we consider our approach
much more general.

Automatic design of consistent (uniform throughout an application) GUI is addressed in~\cite{LearningGUI}. The approach is based on the idea of completion:
assuming there is already a set of consistent layouts a problem of adding yet another element is addressed; the addition should be consistent with the previous
designs. While this problem is, indeed, related to that we address (indeed, consider an existing set of designs as an implicitly specified guideline),
we can identify a number of potential weaknesses. First, only addition of a component is considered, but not removal; second, the addition/subtraction
of components does not necessarily lead to a ``monotonic'' change of the layout (adding yet another text field may result is a drastically different placement);
finally, the initial set of designs to be consistent with rarely comes out of a thin air; most likely it is a result of following some
existing (and perhaps implicit) guidelines, which should be specified and followed explicitly.

In~\cite{Grid} a much more ambitious problem of synthesizing a layout with no guidelines, based only on aesthetic, ergonomic, etc., metrics is considered.
Genetic algorithm is employed for synthesis, and users' feedback is used as a way to measure the quality of the synthesis; the synthesis itself can
sometimes take hours. While the approach presumably allows to synthesize aesthetically convincing layouts with no designer input, the problem of
layout consistency for sufficiently different structures is left unaddressed.

An interesting problem of layout exploration is considered in~\cite{Scout}. The objective is to help the \emph{designers} to develop convincing and
diverse layouts. A set of constraints is introduced which designers can use in order to specify some requirements for the design. Interestingly, in our terms the
set of constraints is a mixture of layout and structural ones: for example, both order and alignment constraints are present. Given
a number of constraints the system generates a set of layouts consistent with this constraints; by updating the constraints a designer can
continue the exploration. As a constraint solver modifier branch-and-bound algorithm is used; as the number of feasible solutions
turned out to be enormous a set of heuristic metrics was developed to rule aesthetically unfeasible designs out. While this work
shares some similarities with ours, being targeted on designers it can be considered as a mean to \emph{develop guidelines}, not to synthesize
guideline-consistent designs.

% !TEX TS-program = pdflatex
% !TeX spellcheck = en_US
% !TEX root = main.tex

\section{Discussion and Future Work}

We presented the results of our work on guideline-based synthesis of GUI layouts. The approach we take makes it possible to
automatically build GUI layouts which by construction comply with a designer-specified set of rules. The prototype implementation
we developed allows to synthesize the layouts for the real-word industrial GUI components w.r.t. the real-world guidelines in
appropriate time.

One interesting question which may arise is if the application of relational programming is essential for this problem to be solved. Indeed, the set of guidelines
describes a matching (or rewriting) system which, in principle, can be directly implemented without any use of relational techniques. We argue, however,
that in this case a whole piece of work on justification of the correctness of the solution would have been repeated anew. In our
case, the justification trivially follows from the completeness of the \textsc{miniKanren} search and refutational completeness of our solution. We also speculate
that such a solution would require reinventing of some implementation techniques to support nondeterminism and backtracking, which are already native to relational
programming. Finally, the duality between patterns over structure and relational goals (initially unexpected for us), to our opinion, witnesses, that relational
programming is a truly relevant technique for this problem. %, and we are not stretching an owl over a globe.

%We can also consider the task of getting rid of extra solver (\textsc{Z3}) as relevant; this would not only simplify the infrastructure of the system, but also allow to
%integrate the constraint resolution phase into the constraint synthesis, improving the performance of the whole system.
%, and finally provide the completeness
%of the synthesizer, which we strictly speaking do not yet have. To do this, we plan to develop a relational integer inequality solver with a binary representation
%of fixed-size numbers. Note that for satisfactory performance it will also be necessary to extend \OCanren{} with inequality operations for such numbers in the
%form of a new type of constraint.


% \section{Unification and Reification Complexity}
\label{sec:uni-rei}

Syntactic unification of terms is a core operation for logic programming in whole and relational programming in particular.
However, the performance characteristics of conventional unification algorithms are rather hard to assess.
The known worst-case estimations say very little about the behavior of unification in \emph{practically important cases}, and, in
general, the very notion of ``practical importance'' is hard to formalize (which constitutes a generic problem for applied complexity as well).

The practical observations witness, that while the worst-case complexity for the conventional unification algorithm is exponential, in the majority of
cases met in practical logic programming the time for each unification problem instance throughout the program execution is linear or even constant on the size of the input.

%So the inner workings of unification are often neglected when estimating the performance of programs.

\mK has a distinctive way of implementing unification fitting in accordance with its ideology. First, since \mK aims at the purely functional implementation of an embedded logical
language it uses a triangular form of substitution~\cite{UnificationTheory} which allows a simple extension in a non-mutable fashion. Such substitutions are lazy in the sense that
they hold a partially substituted value for each variable, so to obtain a fully substituted value it may be necessary to apply a substitution repeatedly. In particular, a full
cycle of substitution application is needed at the end of the search to get the result for a queried variable. This process is called \emph{reification}. \mK uses the conventional Robinson's
algorithm for unification~\cite{RobinsonsAlgorithm}, adjusted for triangular substitutions~\cite{TRS}. Second, since \mK commits to adhere to logical consistency, by default it always
performs \emph{occurs checks} during the unification. Occurs check ensures that a binding being added into the substitution does not introduce any circularity, which is crucial for
establishing the soundness of unification results. However, being rarely violated, occurs check introduces a significant performance penalty, so some logical languages (such as \textsc{Prolog})
omit it.

In this section, we show how the complexity of unification can be assessed for many practical cases. Specifically, we present two dynamic criteria
for relational programs, under which every unification (omitting occurs check) in the program performs a constant number of basic operations. At the same
time, the occurs check, which complexity can be estimated separately, adds a significant overhead to the execution time and often increases the asymptotic complexity.
A number of programs satisfying given tests and showing the impact of occurs check are listed in \sectionword~\ref{sec:evaluation}.

The actual time of unification depends on a concrete choice for a data structure to represent triangular substitutions (which are, abstractly, maps from integers to terms).
Therefore we parameterize our estimations by two values~--- $\lookuptime{\sigma}$ and $\addtime{\sigma}$,~--- which represent, respectively, the
worst-case asymptotic complexity for lookup and add operations w.r.t. to a substitution $\sigma$. The obvious candidate data structure is standard library maps
for a host language (and many implementations like \textsc{miniKanren}-with-symbolic-constraints and \textsc{OCanren} follow this recipe). 
For this data structure both operations have logarithmic complexity,
so we expect this multiplier to be negligible. However, some implementations like \textsc{microKanren} use associative lists for simplicity of presentation (which have linear-time
lookup and constant-time addition complexities) or more complex data structures like random-access lists (which have a log-time lookup and average constant-time addition complexities),
so we keep this parameterization for the general case. The review of the performance of different date structures for triangular substitutions is given in~\cite{SubstDataStructs}.

The basic building block for the unification with triangular substitution is a \emph{walk} operation. This operation checks whether a given variable is mapped by a given substitution to a
term with a constructor at the top level. ``Walk'' continually looks up the substitution until a binding to a non-variable is found or until there is no binding at all. This
process can diverge only when there is a circular binding in the substitution, which, in turn, is excluded by the occurs check, so the substitutions are always consistent
in this sense~\cite{NominalUnificationWithTriangularSubstitutions}. Nevertheless ``walk'' can require a linear number (on the size of substitution) of lookups.
However, the variable-to-variable bindings occur rarely in practice, and usually ``walk'' finds the required binding in one step. We can take the absence of variable-to-variable bindings
as our first criterion: for \emph{flat} substitutions (substitutions without such bindings) ``walk'' always makes only one lookup. We can relax this requirement by allowing a
constant number, independent of the input parameters of the topmost goal, of variable-to-variable bindings.

\begin{definition}
A substitution $\sigma$ is called \emph{constant-factor flat} if the number variable-to-variable bindings in $\sigma$ does not depend on the input parameters of the topmost goal.
\end{definition}

\begin{lemma}
If during the evaluation of a goal all substitutions are constant-factor flat, then the time of any walk during that evaluation on substitution $\sigma$ is $\lookuptime{\sigma}$.
\end{lemma}

The unification of two terms goes by the standard recursive descent. Each time a variable is encountered in a term being unified a ``walk'' step is performed. If it ends up with
an unbound variable the occurs check is performed and, if it succeeded, the substitution is extended. As the substitution grows during the process  the unified terms grow,
too, and the descent can go beyond the size of initial terms. But we argue that this happens not that often. For example, for a linear case (when any variable occurs in unified terms at
most once) the extensions of the substitution during the unification do not affect the unification in other branches, so the recursion will stop at the minimal height of the terms. 

\begin{lemma}
  Given two terms $t_1$ and $t_2$ and a current constant-factor flat substitution $\sigma$, if any variable occurs at most once in at most one of the terms $\{t_1 \sigma, t_2 \sigma\}$,
  then the time this unification takes, excluding occurs checks, is $\O\,(\min\,\{|t_1 \sigma|, |t_2 \sigma|\}) \cdot (\lookuptime{\sigma} + \addtime{\sigma})$.
\end{lemma}

In particular, if the size of one of those two terms does not depend on the input parameters (which is usually the case) the unification performs a constant number of basic operations.
This is our second criterion: linearity and constant size of one of the terms in every unification.

In the presence of occurs checks, however, we need to also go through every term we add in the substitution. This changes ``$\min$'' in the estimation above to ``$\max$'', making a huge
difference. Roughly speaking, in average the number of basic operations for every unification with occurs checks is approximately an average size of all terms unified in the program
(which is usually linear of the input). Therefore occurs checks add a huge time overhead for program execution in \mK, both for asymptotics (see \sectionword~\ref{sec:evaluation})
and for observable time~\cite{WillThesis}. This fact calls for an investigation into ways of going around occurs checks in \mK. Although simply omitting them is not an option for \mK,
there are other known approaches (mostly explored for \textsc{Prolog}), for example, static tests ensuring that occurs checks for a given program will never be violated~\cite{OccursCheckStaticTest}.
As far as we know, for now, there are no such solutions adopted for \mK.

For now, as we estimate the time of every individual unification it might be not clear how it relates to the estimations for the scheduling time. But since we consider cases in which unification
is relatively fast (constant number of basic operations), the number of unifications during an execution plays a more important role. And the number of unifications can be simply limited by the number of semantic
steps $d\,(s_{init}\,(input))$ (because every unification is a separate step in the semantics). Similarly, although the time of basic operations depends on the size of substitution in different points
of execution, logical variables for these bindings come either from the input (where there is usually a constant number of them) or from fresh variable allocations during the evaluation
(each of which is a separate step in the semantics). So the number of allocated logic variables and, therefore, the maximum possible number of bindings are limited by $FV\,(input) + d\,(s_{init}\,(input))$.

So, for example, for a usual case, when our two criteria are satisfied and the input contains a constant number of logic variables, for a standard implementation and without occurs checks
the total time of unifications $T_{uni}$ is $\O\,(d\,(s_{init}\,(input))\cdot \log d\,(s_{init}\,(input)))$

The time of reification $T_r$ can be estimated in the same way, since reification simply goes through the resulting term similarly to occur check. So in the case when the resulting
substitution is constant-factor flat, the number of basic operations for the reification is limited by the size of the output (multiplied by a constant). This time is usually dominated by the time of unification and scheduling, but not always (see examples in \sectionword~\ref{sec:evaluation}).

% \section{Complexity Analysis via Symbolic Execution Schemes}
\label{sec:symbolic}

In the previous sections, we presented some methods to estimate the time complexity for scheduling and unification/reification (for the latter two only for some practical cases) in \mK, but they work
only for relational search in general, not for a specific relational program. In this section, we show how the latter task can be formulated and how those methods can be combined to solve
it using a notion of \emph{symbolic execution}.
Specifically, we add symbolic variables to \mK and use \emph{symbolic execution schemes} to build recursive inequalities for all components of our performance model.
These inequalities then can be solved to provide a symbolic representation for asymptotic estimations.

The application of symbolic execution for time complexity analysis is well-known and was explored for logic programming in particular~\cite{SymbolicExecutionForAnalysis}.
Usually, symbolic execution graphs are used to capture all the details of program execution which are significant for performance, and then the standard techniques for time (or other)
analysis of rewriting systems are applied. In contrast, we need symbolic execution graphs only as a neat representation of a general scheme of a relational search for a
given program and then bring in performance details using \emph{ad hoc} methods described in the previous sections. So we use a restricted version of a graph that corresponds
precisely to a body of a relation, not unfolding any relational calls. For this reason, we refer to them as ``symbolic execution schemes'' rather than ``symbolic execution graphs''.
This also means that we suppose that we know what answers any relational call in the program produces before we start the time complexity analysis.

\begin{figure}[t]
\begin{tabular}{p{6cm}p{6cm}}
\begin{lstlisting}[basicstyle=\small]
   prefix$^o$ = fun n p ->
     (p === Nil) \/
     (fresh (n' pt pti)
        (n === S(n')) /\
        (prefix$^o$ n' pt) /\
        (incr-list$^o$ pt pti) /\
        (p === Cons(S(O), pti))
     )
\end{lstlisting} &
\begin{lstlisting}[basicstyle=\small]
   incr_list$^o$ = fun a r ->
     ((a === Nil) /\ (r === Nil)) \/
     (fresh (h t tr)
        (a === Cons(h, t)) /\
        (r === Cons(S(h), tr)) /\
        (incr_list$^o$ t tr)
     )
\end{lstlisting}
\end{tabular}

\caption{Relational Prefixes Example}
\label{fig:prefixo_definition}
\end{figure}

To present the whole process in a clearer way we will go through it with a specific artificial example, in which almost all important details are presented. The example is a relational program
for generating all prefixes of a list \lstinline|$[1,\dots,n]$| (with numbers represented in Peano encoding with constructors \lstinline|O| and \lstinline|S|). Consider the
following creative recursive solution: we either take an empty prefix or take any prefix for the same task for $n - 1$ (if $n > 0$), increment all the elements and add $1$ at the beginning.
The relation \lstinline|prefix$^o$| in \figureword~\ref{fig:prefixo_definition}, relating a number $n$ to some prefix $p$, follows this description directly. It uses a straightforwardly implemented relation \lstinline|incr-list$^o$|
that increments all numbers in a given list. This relation provides the required results: if we instantiate $n$ with some
Peano number and put a free logical variable for $p$ then $p$ will be bound to every prefix exactly once. It is an inefficient solution in many ways, but it is nice for presentation.

Now we want to estimate the time the search will take depending on a number we put as an input. To make our reasoning more precise we introduce the notion of \emph{symbolic variables}, which we will
denote with an overline ($\overline{a}, \overline{b}, \dots$) as opposed to the usual logic variables, which we will denote using a question mark ($a^{?}, b^{?}, \dots$). The symbolic
variables can be considered on two levels. At the level of symbolic execution, each symbolic variable in \mK stands for some ground term (a term without logic variables inside), but we do not know which
term exactly. At the metalevel, where we reason about the complexity of a program, a symbolic variable $\overline{x}$ stands for a representation of some object $x$ from metatheory (it can be a number, a
string, or a graph, for example) as a ground term, and we analyze how the program behaves depending on this object or some of its parameters. We will distinguish between these two levels throughout the
whole process of complexity analysis. For our example we consider the parametric query \lstinline{prefix$^o$ $\overline{k}$ $a^{?}$} with the first parameter instantiated by some number $k$
represented as a ground Peano term and second parameter left as a free logic variable, and ask how much time the search and its different stages will take depending on the value of $k$.

Our approach estimates the time complexity for some specific relational call with symbolic variables as arguments, not for a relation in general. We name every call we encounter to use these names in our
notations throughout the analysis (for example $pref = $ \lstinline{prefix$^o$ $\overline{k}$ $a^?$}). During the analysis we separately compute a number of factors for
the query that correspond to components of the overall time of the search in our model: the number of semantic steps $d^{pref}\,(k)$ and the scheduling factor $t_s^{pref}\,(k)$, which correspond to the number of semantic steps and the scheduling factor defined in \sectionword~\ref{sec:scheduling}, $t_{uni}^{pref}\,(k)$, which is the number of basic
operations performed during unifications in the execution of the call, excluding basic operations in occurs checks, $t_{occ}^{pref}(k)$, which is the number of basic operations performed during occurs checks, and $t_r^{pref}(k)$, which is the number of basic operations performed during the reification.

To achieve this, we build a symbolic execution scheme, mirroring the body of the examined relation, identify recursive calls, and reconstructing recursive inequalities for all the aforementioned factors
by using the estimations described in the previous sections. 
We put a number of restrictions for the examined relational call for our approach (however, as can be seen from the
\sectionword~\ref{sec:evaluation}, the huge variety of real examples satisfy them): the two criteria from \sectionword~\ref{sec:uni-rei} should be satisfied (which we can check using the symbolic execution, too), relations should be in disjunctive normal form.

We need to know also two extra pieces of information to perform the analysis for a given call. First, to know how to proceed after recursive calls we need to know the answers that the calls produce.
We describe them by sets of substitutions binding all the logical variables in the query to terms, possibly containing fresh logic variables and symbolic variables, the latter we then specify in the metatheory (for example, $ANS^{pref}\,(k) = \{[a^? \gets \overline{p}] \mid \textit{$p$ is a prefix of the list \texttt{\lstinline|[$1$, .., $k$]|}} \} $). Second, we need to know all information for non-recursive relational calls
in the scheme (the values of all the complexity factors, produced answers, whether the requirements are satisfied). So we need to go and analyze these calls using the same approach
before we can examine the given call or reuse the information if we have already analyzed relevant calls before. For this reason, we require the absence of mutual recursion in the examined calls
(it should be eliminated using standard techniques) and analyze them in the order of topological sorting. For $pref$ call we will need this information for the internal call $incr = $ \lstinline{incr-list$^o$ $\overline{l}$ $r^?$}, so we will analyze it first along the way.

% Here we just give it without details of the analysis (the analysis is much simpler than that for the $pref$ call): the requirements are satisfied,
% the answers are $ANS^{incr}(l) = \{[r^? \gets \overline{l'}] \; \mid \; length(l) = length(l') \land \forall i, \; l'[i] = l[i] + 1 \}$,
% and the complexity factors are as follows:

% \[
% \begin{array}{rcl}
%  d^{incr}\,(l) &\in& \O\,(len\,(l)) \\
%  t_s^{incr}\,(l) &\in& \O\,(len\,(l)) \\
%  t_{uni}^{incr}\,(l) &\in& \O\,(len\,(l)) \\
%  t_{occ}^{incr}\,(l) &\in& \O\,(len\,(l) \,\cdot\, size\,(l)) \\
%  t_{r}^{incr}\,(l) &\in& \O\,(size\,(l)) \\
%  \textit{where} && \\
%  size\,(l) &=& \sum_{x \in l} |\overline{x}| 
% \end{array} \]

\begin{figure}[t]
\begin{center}
\xymatrix{
     & \texttt{incr-list$^o$ $\overline{l}$ $r^?$} \ar[dl] \ar[dr] & \\
     \overline{l} \equiv \texttt{Nil} \ar[d]^{\{ [] \; \mid \; \overline{l} \;=\; \texttt{Nil} \}} & & \overline{l} \equiv \texttt{Cons($h^?$, $t^?$)} \ar[d]^{\{ [h^? \gets \overline{x}, t^? \gets \overline{l'}] \; \mid \; \overline{l} \;=\; \texttt{Cons($\overline{x}$, $\overline{l'}$)} \}} \\
     r^? \equiv \texttt{Nil} \ar[d]^{\{ [r^? \gets \texttt{Nil}] \}} & & r^? \equiv \texttt{Cons(S($\overline{x}$), $tr^?$)} \ar[d]^{\{ [r^? \gets \texttt{Cons(S($\overline{x}$), $tr^?$)}] \}} \\
     & & \texttt{incr-list$^o$ $\overline{l'}$ $tr^?$} \ar[d]^{ \{[tr^? \gets \overline{sl'}] \; \mid \; length(sl') = length(l') \land \forall i, \; sl'[i] = l'[i] + 1 \} } \\
     & & \\
}
\end{center}

\caption{Symbolic execution scheme for the $incr$ call}
\label{fig:incr_scheme}
\end{figure}

\begin{figure}[t]
\begin{center}
\xymatrix{
     & \texttt{prefix$^o$ $\overline{k}$ $p^?$} \ar[dl] \ar[dr] & \\
     p^? \equiv \texttt{Nil} \ar[d]^{\{ [p^? \gets \texttt{Nil}] \}} & & \overline{k} \equiv \texttt{S $n^?$} \ar[d]^{\{ [n^? \gets \overline{k'}] \; \mid \; \overline{k} \;=\; \texttt{S $\overline{k'}$} \}} \\
     & & \texttt{prefix$^o$ $\overline{k'}$ $pt^?$} \ar@2[d]^{ \{[pt^? \gets \overline{l}] \; \mid \; \textit{$l$ is a prefix of the list $[1..k']$} \} } \\
     & & \texttt{incr-list$^o$ $\overline{l}$ $pti^?$} \ar[d]^{ \{[pti^? \gets \overline{l'}] \; \mid \; length(l) = length(l') \land \forall i, \; l'[i] = l[i] + 1 \} } \\
     & & p^? \equiv \texttt{Cons (S(O), $\overline{l'}$)} \ar[d]^{ \{[p^? \gets \texttt{Cons (S(O), $\overline{l'}$)}] \} } \\
     & & \\
}
\end{center}

\caption{Symbolic execution scheme for the $pref$ call}
\label{fig:pref_scheme}
\end{figure}


The symbolic execution scheme for the $pref$ call is shown in \figureword~\ref{fig:pref_scheme} and the scheme for the internal call $incr$ is shown in \figureword~\ref{fig:incr_scheme}. A symbolic execution scheme shows unifications and internal relational calls evaluated during the search for the initial call and the answers that are threaded through the search. The initial call is at the top.
For simplicity we work only with the relations in disjunctive normal form, each disjunct is represented as a separate column on the scheme. The nodes of the column are
the unifications and relational calls in the given conjunct, they are written down sequentially in the same order as in the relation and connected by arrows. Arrows are
labeled with the description of a set of answers, produced by the previous node. This description is represented as a set of lists of bindings for logical variables by which the
substitution is extended, the generator of the set (the condition after the `$\mid$' symbol) is described in terms of metatheory. For the analysis we need to distinguish
cases when multiple answers are produced so we denote by a single arrow $\downarrow$ the sets that we know to have no more than one answer, and put a double arrow $\Downarrow$
in other cases. The answers produced by internal relational calls are given as a prerequisite for the analysis. The unifications may produce new substitutions for
both logic variables and symbolic variables. The definition of unification with both logic and symbolic variables is shown on \figureword~\ref{fig:symbolic_unification}.
Bindings for logical variables in the result are extensions of the substitution in the environment after this unification, and bindings for symbolic variables are conditions
for the objects in metatheory represented by these symbolic variables, under which we continue to execute the current branch. For example, the unification for $\overline{x} \equiv f\,(t_1, \dots, t_k)$ will succeed only for object $x$ such that its representation is $f\,(\overline{x_1}, \dots, \overline{x_k})$, where $\overline{x_i}$ are the representations which are the terms
unifiable with $t_i$. So we add bindings for symbolic variables to the generator of the set in the form of equalities. We apply bindings for both logic and symbolic variables in
all nodes after we get them to show the fully substituted values of terms.

%\begin{figure}[t]
%  \small
%\[
%\begin{array}{lll}
%  U(w^?, w^?) &= \epsilon & \\
%  U(w^?, t) &= \bot & \textit{if $w^? \in FV(t)$} \\
%  U(w^?, t) &= [w^? \gets t] & \textit{if $t \ne w^? \land w^? \not\in FV(t)$} \\
%  U(\overline{x}, w^?) &= [w^? \gets \overline{x}] &  \\
%  U(\overline{x}, \overline{y}) &= [\overline{x} \gets \overline{y}] &  \\
%  U(\overline{x}, f(t_1, \dots, t_k)) &= [\overline{x} \gets f(\overline{x_1}, \dots, \overline{x_k})] \circ U(\overline{x_1}, t_1) \circ \dots \circ  U(\overline{x_k}, t_k)  & \textit{where $\overline{x_i}$ are fresh}  \\
%  U(f(t_1, \dots, t_k), w^?) &= \bot & \textit{if $w^? \in FV(f(t_1, \dots, t_k))$} \\
%  U(f(t_1, \dots, t_k), w^?) &= [w^? \gets f(t_1, \dots, t_k)] & \textit{if $w^? \not\in FV(f(t_1, \dots, t_k))$} \\
%  U(f(t_1, \dots, t_k), \overline{x}) &= [\overline{x} \gets f(\overline{x_1}, \dots, \overline{x_k})] \circ U(t_1, \overline{x_1}) \circ \dots \circ U(t_k, \overline{x_k})  & \textit{where $\overline{x_i}$ are fresh}  \\
 % U(f(t_1, \dots, t_k), f(t'_1, \dots, t'_k)) &= U(t_1, t'_1) \circ \dots \circ U(t_k, t'_k)  & \\
 % U(f(t_1, \dots, t_k), g(t'_1, \dots, t'_{k'})) &= \bot  & \textit{if $f \ne g$} \\
%  
%\end{array}
%\]
%  \caption{Unification for terms with logic and symbolic variables; $t$ stands for an arbitrary term.}
%  \label{fig:symbolic_unification}
%\end{figure}

\begin{figure}[t]
  \small
\[
\begin{array}{lll}
  U(w^?, w^?) &= \epsilon & \\
  U(w^?, t) &= \bot & \textit{if $w^? \in FV(t)$} \\
  U(w^?, t) &= [w^? \gets t] & \textit{if $t \ne w^? \land w^? \not\in FV(t)$} \\
  U(t, w^?) &= U(w^?, t) & \textit{if $t$ is not a logic variable} \\
  U(\overline{x}, \overline{y}) &= [\overline{x} \gets \overline{y}] &  \\
  U(\overline{x}, f(t_1, \dots, t_k)) &= [\overline{x} \gets f(\overline{x_1}, \dots, \overline{x_k})] \circ U(\overline{x_1}, t_1) \circ \dots \circ U(\overline{x_k}, t_k)  & \textit{where $\overline{x_i}$ are fresh}  \\
  U(f(t_1, \dots, t_k), \overline{x}) &= U(\overline{x}, f(t_1, \dots, t_k))  \\
  U(f(t_1, \dots, t_k), f(t'_1, \dots, t'_k)) &= U(t_1, t'_1) \circ \dots \circ U(t_k, t'_k)  & \\
  U(f(t_1, \dots, t_k), g(t'_1, \dots, t'_{k'})) &= \bot  & \textit{if $f \ne g$} \\
  
\end{array}
\]
  \caption{Unification for terms with logic and symbolic variables; $t$ stands for an arbitrary term.}
  \label{fig:symbolic_unification}
\end{figure}

This scheme presents all the information we need to check the criteria and calculate the complexity of all the factors using the results from the previous sections.

\begin{enumerate}
\item To check that all substitutions are flat during the evaluation we need to know that all non-recursive internal calls satisfy this condition and to check that no variable-to-variable bindings are added during the evaluation of the body of the relation. To check this we can simply check that rhs of all bindings on arrows after unifications are not logical variables (then the value in the binding
necessarily has a constructor on the top-level).

If there are no recursive calls in the scheme, we can allow variable-to-variable bindings after substitutions, since there will be at most a constant number of them and substitutions will always be constant-factor flat.

The second criterion (linearity and constant size of one of the terms for every unification) we can easily check directly: for every unification on the scheme each logical variable should occur at most once and one of the terms should have no symbolic variables. We also need the criterion to be satisfied for all the internal calls.

For the calls $incr$ and $pref$ both criteria are satisfied.

\item To estimate $d^{incr}\,(l)$ and $d^{pref}\,(k)$ we use lemmas \ref{lem:conjunction_metrics_calc} and \ref{lem:disjunction_metrics_calc}. Specifically, we just add the corresponding value for every internal call summed up for all the answers for which the call is executed and also add a constant to handle the rest (unifications and fresh variable introductions). We know the value of $d^q\,(\dots)$ for every internal call $q$: for non-recursive internal calls we have the estimation up to a multiplicative constant from the previous analysis, for recursive internal calls it's just the value of the same function with a different argument.
  
So, for the $incr$ call we have:

\[ d^{incr}\,(l) \le C + \displaystyle\sum\limits_{\overline{l} \;=\; \texttt{Cons($\overline{x}$, $\overline{l'}$)}} d^{incr}\,(l') \]

Considering two cases when $l$ is an empty and a non-empty list we can simplify the inequality above into the following two:

\[
\begin{array}{lcl}
d^{incr}\,([]) &\le& C \\
d^{incr}\,(x : l') &\le& C + d^{incr}\,(l')
\end{array} \]

which we can easily solve and get $d^{incr}\,(l) \in \O\,(len\,(l))$.

And for the $pref$ call we have:

\[ d^{pref}\,(k) \le C + \displaystyle\sum\limits_{\overline{k} \;=\; \texttt{S $\overline{k'}$}} (d^{pref}\,(k') + \displaystyle\sum\limits_{\textit{$l$ is a prefix of the list $[1..k']$}} d^{incr}\,(l)) \]

Which we can rewrite and simplify again by considering two cases and substituting calculated complexity for $d^{incr}\,(l)$:

\[
\begin{array}{lcl}
d^{pref}\,(0) &\le& C \\
d^{pref}\,(k' + 1) &\le& C + d^{pref}\,(k') + \displaystyle\sum\limits_{i \in [0..k']} C \cdot i \\
            &\le& d^{pref}\,(k') + C \cdot k'^2 
\end{array} \]

From which we get $d^{pref}\,(k) \in \O\,(k^3)$.

\item For $t_s^{incr}\,(l)$ and $t_s^{pref}\,(k)$ we do basically the same using the same lemmas \ref{lem:conjunction_metrics_calc} and \ref{lem:disjunction_metrics_calc}. The difference is that for every internal call $q$ along with $t_s^q\,(\dots)$ we have to add $d^q\,(\dots)$
  multiplied by a constant (for recursive calls we use the complexity calculated at the previous step). There is a possible exception, however (identified in the lemmas): for one column that has only single arrows, we can omit additional $d^q\,(\dots)$ for the last call in the column (if the column ends with a call).
  By lemma~\ref{lem:disjunction_metrics_calc} we can pick any column, it might make difference only when this value $d^q\,(\dots)$ dominates all the other values. In particular, in the case when a relation has one recurisve call, if it is in the end of a conjunction we omit additional value $d^q\,(\dots)$ for this call, and when it is not in the end we can not omit this additional value (and can omit the additional value $d^q\,(\dots)$ for the last non-recursive call instead, which will be dominated by the value $t^q\,(\dots)$ for this call with a multiplicative factor anyway). This omission is precisely the reason for the change in complexity when the recursive call is moved to the end, like in the initial example of \lstinline|length$^o$| and \lstinline|length$_d^o$| relations in \sectionword~\ref{sec:intro}. Likewise, for $incr$ call we omit additional value $d^{incr}\,(l')$ for the recursive call and get the same inequality as the one for the number of steps:

\[ t_s^{incr}\,(l) \le C + \displaystyle\sum\limits_{\overline{l} \;=\; \texttt{Cons($\overline{x}$, $\overline{l'}$)}} t_s^{incr}\,(l') \]
 
So, $t_s^{incr}\,(l)$ is also in $\O\,(len\,(l))$.
 
In contrast, in $pref$ call the recursive call is not in the end, so we have additional value $d^{pref}(k')$ for it, which affects the resulting complexity:

  \[
\begin{array}{rclc}
  t_s^{pref}\,(k) & \le C + \displaystyle\sum\limits_{\overline{k} \;=\; \texttt{S $\overline{k'}$}} (& t_s^{pref}\,(k') & + \\
                &     & C \cdot d^{pref}(k') & + \\
                &     & \displaystyle\sum\limits_{\textit{$l$ is a prefix of the list $[1..k']$}} (t_s^{incr}\,(l) + C \cdot d^{incr}\,(l))) & \\
\end{array}
\]

After the simplification, we get the following two inequalities:

\[ \begin{array}{rcl}
t_s^{pref}\,(0) &\le& C \\
t_s^{pref}\,(k' + 1) &\le& C + t_s^{pref}\,(k') + C \cdot k'^3 + \displaystyle\sum\limits_{i \in [0..k']} (C \cdot i + C \cdot i) \\
                  &\le& t_s^{pref}\,(k') + C \cdot k'^3 
\end{array} \]

And after solving them we get $t_s^{pref}\,(k) \in \O\,(k^4)$.

\item To estimate $t_{uni}^{incr}\,(l)$ and $t_{uni}^{pref}\,(k)$ we just do the same summation, counting the number of unifications in the scheme and in the internal calls.

For $incr$ we have the following inequality:

\[ t_{uni}^{incr}\,(l) \le 1 + (\displaystyle\sum\limits_{\overline{l} \;=\; \texttt{Nil}} 1) + 1 + \displaystyle\sum\limits_{\overline{l} \;=\; \texttt{Cons($\overline{x}$, $\overline{l'}$)}} (1 + t_{uni}^{incr}\,(l') ) \]

The simplified version is the following:

\[ \begin{array}{lcl}
t_{uni}^{incr}\,([]) &\le& C \\
t_{uni}^{incr}\,(x : l') &\le& C + t_{uni}^{incr}\,(l') \\
\end{array} \]

And the result is $t_{uni}^{incr}\,(l) \in \O\,(len\,(l))$.

And for $pref$ we have the following inequality:

  \[
    t_{uni}^{pref}\,(k) \le 1 + 1 + \displaystyle\sum\limits_{\overline{k} \;=\; \texttt{S $\overline{k'}$}} (t_{uni}^{pref}\,(k') + \displaystyle\sum\limits_{\textit{$l$ is a prefix of the list $[1..k']$}} (t_{uni}^{incr}\,(l) + \displaystyle\sum\limits_{l': \; length(l) = length(l') \land \forall i, \; l'[i] = l[i] + 1} 1))
    \]

The simplified version is the following:

\[ \begin{array}{rcl}
t_{uni}^{pref}\,(0) &\le& C \\
t_{uni}^{pref}\,(k' + 1) &\le& C + t_{uni}^{pref}\,(k') + \displaystyle\sum\limits_{i \in [0..k']} (C \cdot i + 1) \\
&\le& t_{uni}^{pref}\,(k') + C \cdot k'^2
\end{array} \]

And the result is $t_{uni}^{pref}\,(k) \in \O\,(k^3)$.

\item To estimate $t_{occ}^{incr}\,(l)$ and $t_{occ}^{pref}\,(k)$ we just do the same summation, counting the sizes of rhs in bindings on arrows after every unification on the scheme and the same in the internal calls.

For $incr$ we have the following inequality:

\[ t_{occ}^{incr}\,(l) \le (\displaystyle\sum\limits_{\overline{l} \;=\; \texttt{Nil}} |\texttt{Nil}|) + \displaystyle\sum\limits_{\overline{l} \;=\; \texttt{Cons($\overline{x}$, $\overline{l'}$)}} (|\overline{x}| + size(l') + |\texttt{Cons(S($\overline{x}$), tr)}| + t_{occ}^{incr}\,(l') ), \]

where $size\,(l') = \displaystyle\sum\limits_{y \in l'} |\overline{y}|$.

The simplified version is the following:

\[ \begin{array}{lcl}
t_{uni}^{incr}\,([]) &\le& C \\
t_{uni}^{incr}\,(x : l') &\le& C + 2 |\overline{x}| + size(l') + t_{uni}^{incr}\,(l') \\
\end{array} \]

And the result is $t_{uni}^{incr}\,(l) \in \O\,(len\,(l) \cdot size\,(l))$.

And for $pref$ we have the following inequality:

%!!!!!!!!!!!!

\[ \begin{array}{c} t_{occ}^{pref}\,(k) \le C + \displaystyle\sum\limits_{\overline{k} \;=\; \texttt{S $\overline{k'}$}} (k' + t_{occ}^{pref}\,(k') + \displaystyle\sum\limits_{\textit{$l$ is a prefix of the list $[1..k']$}} (t_{occ}^{incr}\,(l) + \\ \displaystyle\sum\limits_{l': \; length(l) = length(l') \land \forall i, \; l'[i] = l[i] + 1} |\texttt{Cons (S(O), $\overline{l'}$)}|)) \end{array}  \]

The simplified version is the following.

\[ \begin{array}{rcl}
t_{occ}^{pref}\,(0) &\le& C \\
t_{occ}^{pref}\,(k' + 1) &\le& C + k' + t_{occ}^{pref}\,(k') + \displaystyle\sum\limits_{i \in [0..k']} (C \cdot i^3 + C \cdot i^2 + C) \\
\phantom{t_{occ}^{pref}\,(k' + 1)} &\le& t_{occ}^{pref}\,(k') + C \cdot k'^4
\end{array} \]

And the result is $t_{occ}^{pref}\,(k) \in \O\,(k^5)$.

\item Finally, to estimate $t_{r}^{pref}\,(k)$ we just sum the sizes of all answers from $ANS^{pref}\,(k)$.

$t_{r}^{pref}\,(k) = \displaystyle\sum\limits_{\textit{$l$ is a prefix of the list $[1..k]$}} size(l) \le \displaystyle\sum\limits_{i \in [0..k]} C \cdot i^2 \le C \cdot k^3$

So $t_{r}^{pref}\,(k) \in \O\,(k^3)$.

\end{enumerate}

This way we get the complexity for all the components of the search. Now we can combine them to get the complete estimation. To get the time related to the unification we should
multiply $t_{uni}^{pref}$, $t_{occ}^{pref}$, $t_r^{pref}$ by a multiplier $(\lookuptime{|\sigma|} + \addtime{|\sigma|}))$ which we can estimate
by $(\lookuptime{d^{pref}(k)} + \addtime{d^{pref}(k)}))$. So, for example, for an implementation with standard-library maps for substitutions the complete time of the
search for the call \lstinline{prefix$^o$ $\overline{k}$ $a^?$} is $\O\,(k^4 + k^3 \log k + k^5 \log k + k^3 \log k) = \O\,(k^5 \log k)$ with occurs checks and
$\O\,(k^4 + k^3 \log k + k^3 \log k) = \O\,(k^4)$ without.

% \section{Evaluation}

\label{sec:evaluation}

In this section, we present an evaluation of 
implemented constructive negation on a series of examples.

\subsection{If-then-else}

Using relational if-then-else operator, 
presented in section~\ref{sec:ifte},
we have implemented several 
higher-order relations over lists, namely 
\lstinline{find} (Listing~\ref{lst:eval-find}), 
\lstinline{remove}\footnote{Note, this implementation 
differs from the one in Section~\ref{sec:intro}, but 
it is easy to see that these two are semantically equivalent.} (Listing~\ref{lst:eval-remove}) 
and \lstinline{filter} (Listing~\ref{lst:eval-filter}).
These relations are almost identical (syntactically) to their
functional implementations.
We have tested that these relations can be run
in various directions and produce the expected results.
For example, the goal \lstinline{filter p q q}
with the predicate \lstinline{p} equal to

\begin{lstlisting}
  fun l -> fresh (x) (l === [x])
\end{lstlisting}

stating that the given list should be a singleton list,
starts to generate all singleton lists.
Vice versa, the goal \lstinline{filter p q []} 
with that same \lstinline{p} generates 
all lists, constrained to be not a singleton list.

Listings~\ref{lst:eval-p}-\ref{lst:eval-filter-queries} give 
more concrete examples of queries to these relations.
In the listing the syntax \lstinline{run n q g}
means running a goal \lstinline{g} with 
the free variable \lstinline{q}
taking the first \lstinline{n} answers (``\lstinline{*}'' denotes all answers).
After the sign $\leadsto$ the result of the query is given.
The result \lstinline{fail} means that the query has failed.
The result \lstinline[mathescape]|succ {{a$_1$}; ... {a$_n$}} |
means that the query has succeeded delivering $n$ answers.
Each answer represents a set of constraint on free variables.
Constraints are of two forms: equality constraints, e.g. \lstinline{q = (1, _.$_0$)}, 
or disequality constraints, e.g. \lstinline{q $\neq$ (1, _.$_0$)}.
The terms of the form \lstinline{_.$_i$} in the answer
denote some universally quantified variables.

\begin{minipage}[thb]{.3\textwidth}
\begin{lstlisting}[
  caption={A definition of \code{find} relation},
  label={lst:eval-find}
]
let find p e xs =
  fresh (x xs' ys') (
    xs === x::xs' /\
    ifte (p x)
      (e === x)
      (find p e xs')
  )
\end{lstlisting}
\end{minipage}\hfill
\begin{minipage}[thb]{.3\textwidth}
\begin{lstlisting}[
  caption={A definition of \code{remove} relation},
  label={lst:eval-remove}
]
let remove p xs ys =
  (xs === [] /\ ys === [])
  \/
  fresh (x xs' ys') (
    xs === x::xs' /\
    ifte (p x)
      (ys === xs')
      (ys === x::ys' /\ 
       remove p xs' ys')
  )
\end{lstlisting}
\end{minipage}\hfill
\begin{minipage}[thb]{.3\textwidth}
\begin{lstlisting}[
  caption={A definition of \code{filter} relation},
  label={lst:eval-filter}
]
let filter p xs ys =
  (xs === [] /\ ys === [])
  \/
  fresh (x xs' ys') (
    xs === x::xs' /\
    (ifte (p x)
      (ys === x :: ys')
      (ys === ys')) /\
    filter p xs' ys'
  )
\end{lstlisting}
\end{minipage}

% \vspace{3cm}

\begin{minipage}[thb]{0.4\textwidth}
\begin{lstlisting}[
  caption={Definition of the predicate \lstinline{p}},
  label={lst:eval-p}
]
let p l = fresh (x) (l === [x])
\end{lstlisting}
\begin{lstlisting}[
  caption={Example of queries to \lstinline{find}},
  label={lst:eval-find-queries}
]
run 3 q (fresh (e) find p e q) 
$\leadsto$ succ {
     { q = [_.$_0$] :: _.$_1$ }
     { q = _.$_0$ :: [_.$_1$] :: _.$_2$; 
         _.$_0$ $\neq$ [_.$_3$] }
     { q = _.$_0$ :: _.$_1$ :: [_.$_2$] :: _.$_3$; 
         _.$_0$ $\neq$ [_.$_4$]; _.$_1$ $\neq$ [_.$_5$] }
   }
\end{lstlisting}
\end{minipage}\hfill
\begin{minipage}[thb]{0.4\textwidth}
\begin{lstlisting}[
  caption={Example of queries to \lstinline{remove}},
  label={lst:eval-remove-queries}
]
run * q (fresh (e) remove p q [[ ]]) 
$\leadsto$ succ {
     { q = [[_.$_0$]; [ ]] }
     { q = [[ ]] }
     { q = [[ ]; [_.$_0$]] }
   }

run 3 q (fresh (e) remove p q q) 
$\leadsto$ succ {
     { q = [] }
     { q = [_.$_0$], _.$_0$ $\neq$ [_.$_1$] }
     { q = [_.$_0$; _.$_1$]; 
         _.$_0$ $\neq$ [_.$_2$]; _.$_1$ $\neq$ [_.$_3$] }
   }
\end{lstlisting}
\end{minipage}

\begin{minipage}[thb]{0.4\textwidth}
\begin{lstlisting}[
  caption={Example of queries to \lstinline{filter}},
  label={lst:eval-filter-queries}
]
run 3 q (filter p q q) 
$\leadsto$ succ {
     { q = [ ] }
     { q = [_.$_0$] }
     { q = [_.$_0$; _.$_1$] }
   }

run 3 q (filter p q [1]) 
$\leadsto$ succ {
     { q = [[1]] }
     { q = [_.$_0$; [1]]; _.$_0$ $\neq$ [_.$_1$] }
     { q = [[1]; _.$_0$]; _.$_0$ $\neq$ [_.$_1$] }
   }

run 3 q (filter p q [ ]) 
$\leadsto$ succ {
     { q = [] }
     { q = [_.$_0$]; _.$_0$ $\neq$ [_.$_1$] }
     { q = [_.$_0$; _.$_1$]; 
            _.$_0$ $\neq$ [_.$_2$]; _.$_1$ $\neq$ [_.$_3$] }
   }
\end{lstlisting}
\end{minipage}

\subsection{Universal quantification}

In the Section~\ref{sec:impl-univ} we presented 
the \lstinline{forall} goal constructor 
which is implemented through the double negation.
We have observed, that although \lstinline{forall g}
does not terminate when the goal \lstinline{g x} 
has an infinite number of answers 
(assuming \lstinline{x} is a fresh variable),
it does terminate in the case when \lstinline{g x} has 
a finite number of answers.
The behavior of \lstinline{forall} in this case is sound
even in the presence of disequality constraints or nested quantifiers. 

The Table~\ref{tab:univ} gives some concrete examples.
The left column contains the tested goals\footnote{
We typeset the goals in terms of first-order logic syntax 
instead of \textsc{OCanren} syntax for brevity and clarity.} 
and the right column gives the obtained results.
For the results we use the same notation 
as in the previous section.

\begin{table}[th]
  \centering
  \def\arraystretch{1.5}
  \begin{tabularx}{\textwidth}{|X|X|}
    \hline

    $\forall x\ldotp x = q$ & 
      \texttt{fail} \\
    \hline

    $\forall x\ldotp \exists y\ldotp x = y$ & 
      \texttt{succ \{[q = \_.$_0$]\}} \\
    \hline

    $\forall x\ldotp \exists y\ldotp x = y \wedge y = q$ &
      \texttt{fail} \\
    \hline

    $\forall x\ldotp q = (1, x)$ & 
      \texttt{fail} \\
    \hline

    $\forall x\ldotp \exists y\ldotp y = (1, x)$ & 
      \texttt{succ \{[q = \_.$_0$]\}} \\
    \hline

    $\forall x\ldotp \exists y\ldotp x = (1, y)$ &
      \texttt{fail} \\
    \hline

    $\forall x\ldotp x \neq q$ & \texttt{fail} \\
    \hline

    $\forall x\ldotp \exists y\ldotp x \neq y$ & 
      \texttt{succ \{[q = \_.$_0$]\}} \\
    \hline

    $\forall x\ldotp \exists y\ldotp x \neq y \wedge y = q$ & 
      \texttt{fail} \\
    \hline

    $\forall x\ldotp q \neq (1, x)$ & 
      \texttt{succ \{[q $\neq$ (1, \_.$_0$)]\}} \\
    \hline

    $(\exists x\ldotp q = (1, x)) \wedge (\forall x\ldotp q \neq (1, x))$ & 
      \texttt{fail} \\
    \hline

    $\forall x\ldotp (x, x) \neq (0, 1)$ & 
      \texttt{succ \{[q = \_.$_0$]\}} \\
    \hline

    $\forall x\ldotp (x, x) \neq (1, 1)$ & 
      \texttt{fail} \\
    \hline

    $\forall x\ldotp (x, x) \neq (q, 1)$ & 
      \texttt{succ \{[q $\neq$ 1]\}} \\
    \hline

    $\exists a~ b\ldotp q = (a, b) \wedge \forall x\ldotp (x, x) \neq (a, b)$ & 
      \texttt{succ \{[q = (\_.$_0$, \_.$_1$); \_.$_0$ $\neq$ \_.$_1$]\}} \\
    \hline

  \end{tabularx}
  \caption{\lstinline{forall} evaluation}
  \label{tab:univ}
\end{table}

% \section{Conclusion and future work}

We presented an approach for pattern matching implementation synthesis using relational programming. Currently, it demonstrates a good performance only
on a very small problems. The performance can be improved by searching for new ways to prune the search space and by speeding up the implementation of
relations and structural constraints. Also it could be interesting to integrate structural constraints more closely into \textsc{OCanren}'s core.
Discovering an optimal order of samples and reducing the complete set of samples is another direction for research.

The language of intermediate representation can be altered, too. It is interesting to add to an intermediate language so-called \emph{exit nodes}
described in~\cite{maranget2001}. The straightforward implementation of them might require nominal unification, but we are not aware of any
\textsc{miniKanren} implementation in which both disequality constraints and nominal unification~\cite{alphaKanren} coexist nicely.

At the moment we support only simple pattern matching without any extensions. It looks technically easy to extend our approach with
non-linear and disjunctive patterns. It will, however, increase the search space and might require more optimizations.





%%
%% The acknowledgments section is defined using the "acks" environment
%% (and NOT an unnumbered section). This ensures the proper
%% identification of the section in the article metadata, and the
%% consistent spelling of the heading.
%%\begin{acks}

%%\end{acks}

%%
%% The next two lines define the bibliography style to be used, and
%% the bibliography file.
\bibliographystyle{splncs04}
\bibliography{main}

%%
%% If your work has an appendix, this is the place to put it.
\appendix

% \setlength{\abovecaptionskip}{0pt plus 3pt minus 2pt}
\setlength{\belowcaptionskip}{0pt plus 3pt minus 2pt}

\abovedisplayskip1mm
\belowdisplayskip1mm
\abovedisplayshortskip1mm
\belowdisplayshortskip1mm

\setlength{\topsep}{5pt}
\setlength{\partopsep}{5pt plus 0pt minus 0pt}
\setlength{\parskip}{5pt}
\setlength{\parindent}{10pt}

\clearpage
\section{Proofs}
\label{sec:proofs_appendix}

\begin{lemma}
\label{lem:theta_constant}
Let $C_f$ and $C_g$ be real positive constants. If

\[ h\,(x) = f\,(x) + C_f + \Theta\,(g\,(x) + C_g) \]

and value $g\,(x)$ is positive for arbitrary $x$ then

\[ h\,(x) = f\,(x) + \Theta\,(g\,(x)) \]

\end{lemma}
\begin{proof}
\begin{enumerate}
\item For some real postive $C_1$: $h\,(x) \ge f\,(x) + C_f + C_1 \cdot (g\,(x) + C_g)$, thus $h\,(x) \ge f\,(x) + C_1 \cdot g\,(x)$
\item For some real postive $C_2$: $h\,(x) \le f\,(x) + C_f + C_2 \cdot (g\,(x) + C_g)$, thus $h\,(x) \ge f\,(x) + (C_f + C_2 + C_2 \cdot C_g) \cdot g\,(x)$
\qed
\end{enumerate}
\end{proof}

\begin{lemma}
\label{lem:theta_sum}
If

\[ h_1\,(x) = f_1\,(x) + \Theta\,(g_1\,(x)) \]

and

\[ h_2\,(x) = f_2\,(x) + \Theta\,(g_2\,(x)) \]

then

\[ (h_1\,(x) + h_2\,(x)) = (f_1\,(x) + f_2\,(x)) + \Theta\,(g_1\,(x) + g_2\,(x)) \]

\end{lemma}
\begin{proof}
Both lower- and upper-bound constants are the sums of the two corresponding constants for the given approximations.
\qed
\end{proof}

\begin{lemma}
\label{lem:theta_absorb}
Let $g_1\,(x) \ge g_2\,(x) > 0$ for all $x$. If

\[ h\,(x) = f\,(x) + \Theta\,(g_1\,(x) + g_2\,(x)) \]

then

\[ h\,(x) = f\,(x) + \Theta\,(g_1\,(x)) \]

\end{lemma}
\begin{proof}
\begin{enumerate}
\item For some real  postive $C_1$: \[ h\,(x) \ge f\,(x) + C_1 \cdot (g_1\,(x) + g_2\,(x)) \ge f\,(x) + C_1 \cdot g_1\,(x) \]
\item For some real  postive $C_2$: \[ h\,(x) \le f\,(x) + C_2 \cdot (g_1\,(x) + g_2\,(x)) \le f\,(x) + 2 C_2 \cdot g_1\,(x) \]
\qed
\end{enumerate}
\end{proof}

\begin{lemma}
\label{lem:task_measure_equations}
  If

  \[\taskst{g}{e} \xrightarrow{l} s^\prime\]

  then

  \[
    \begin{array}{rcl}
    d\,(\taskst{g}{e}) &=& d\,(s^\prime) + 1\\
    t\,(\taskst{g}{e}) &=& t\,(s^\prime) + 1
    \end{array}
  \]
\end{lemma}
\begin{proof}
    Immediately from the definitions of the estimated values.
    \qed
\end{proof}


\repeatlemma{lem:sum_measure_equations}
\begin{proof}
Induction on the sum of values $d\,(s_1) + d\,(s_2)$. Both equations easily hold for both the rules \ruleno{DisjStop} and \ruleno{DisjStep} if we apply
the inductive hypothesis for the next state.
\qed
\end{proof}

\repeatlemma{lem:times_measure_equations}
\begin{proof}
Induction on the value $d\,(s \times g)$. Both equations easily hold for all the rules \ruleno{ConjStop}, \ruleno{ConjStopAns}, \ruleno{ConjStep}, \ruleno{ConjStepAns}
if we apply the inductive hypothesis for the next state.
\qed
\end{proof}

\repeatlemma{lem:otimes_t_approximation}
\begin{proof}
After unfolding the defintions and throwing out the identical parts we have the following statement:

\[ \begin{array}{l}
d\,(s) + \smashoperator[lr]{\sum\limits_{a_i \in \tra{s}}} \min\,\{2\cdot d\,(\taskst{g}{a_i}) - 1, 2\cdot d\,(s'_i \otimes g)\}  = \\
= \Theta\,(d\,(s) + \smashoperator[lr]{\sum\limits_{a_i \in \tra{s}}} d\,(\taskst{g}{a_i}) - \smashoperator{\maxd\limits_{a_i \in \tra{s}}} d\,(\taskst{g}{a_i})) \\
\end{array} \]

If $\tra{s}$ is empty, the statement is trivial.

If $\tra{s}$ is not empty, $\displaystyle{\maxd}$ turns into a simple $\max$. We establish lower and upper bounds separately.

\begin{enumerate}
  \item
  Lower bound. Let's show that 
  
  \[ \begin{array}{l}
  d\,(s) + \smashoperator[lr]{\sum\limits_{a_i \in \tra{s}}} \min\,\{2\cdot d\,(\taskst{g}{a_i}) - 1, 2\cdot d\,(s'_i \otimes g)\}  \ge \\
  \ge d\,(s) + \smashoperator[lr]{\sum\limits_{a_i \in \tra{s}}} d\,(\taskst{g}{a_i}) - \smashoperator{\max\limits_{a_i \in \tra{s}}} d\,(\taskst{g}{a_i}) \\
  \end{array} \]

  First, we can decrease both arguments of $\min$:

  \[ \begin{array}{l}
  d\,(s) + \smashoperator[lr]{\sum\limits_{a_i \in \tra{s}}} \min\,\{2\cdot d\,(\taskst{g}{a_i}) - 1, 2\cdot d\,(s'_i \otimes g)\}  \ge \\
  \ge d\,(s) + \smashoperator[lr]{\sum\limits_{a_i \in \tra{s}}} \min\,\{ d\,(\taskst{g}{a_i}), d\,(s'_i \otimes g)\} \\
  \end{array} \]

  Let's consider two cases.

  \begin{enumerate}
    \item The minimum in this expression is always reached on the first argument. Then
    
    \[ \begin{array}{l}
        d\,(s) + \smashoperator[lr]{\sum\limits_{a_i \in \tra{s}}} \min\,\{ d\,(\taskst{g}{a_i}), d\,(s'_i \otimes g)\} = \\
        = d\,(s) + \smashoperator[lr]{\sum\limits_{a_i \in \tra{s}}} d\,(\taskst{g}{a_i}) \ge \\
        \ge d\,(s) + \smashoperator[lr]{\sum\limits_{a_i \in \tra{s}}} d\,(\taskst{g}{a_i}) - \smashoperator{\max\limits_{a_i \in \tra{s}}} d\,(\taskst{g}{a_i}) \\
  \end{array} \]
  
    \item $a_m$ is the first answer, such that the minimum for $a_m$ is reached on the second argument. Then the answers up to $a_m$ are sufficient to prove the bound. 
    
    \[ \begin{array}{l}
        d\,(s) + \smashoperator[lr]{\sum\limits_{a_i \in \tra{s}}} \min\,\{ d\,(\taskst{g}{a_i}), d\,(s'_i \otimes g)\} \ge \\
        \ge d\,(s) + \smashoperator[lr]{\sum\limits_{a_i \in \{ a_1, \dots, a_m \}}} \min\,\{ d\,(\taskst{g}{a_i}), d\,(s'_i \otimes g)\} = \\
        = d\,(s) + \smashoperator[lr]{\sum\limits_{a_i \in \{ a_1, \dots, a_{m - 1} \}}} d\,(\taskst{g}{a_i}) + d\,(s'_m \otimes g) = \\
        = d\,(s) + \smashoperator[lr]{\sum\limits_{a_i \in \{ a_1, \dots, a_{m - 1} \}}} d\,(\taskst{g}{a_i}) + d\,(s'_m) + \smashoperator[lr]{\sum\limits_{a_i \in (\tra{s} \setminus \{ a_{1}, \dots, a_{m} \})}} d\,(\taskst{g}{a_i})  \ge \\
        \ge d\,(s) + \smashoperator[lr]{\sum\limits_{a_i \in \tra{s}}} d\,(\taskst{g}{a_i}) -  d\,(\taskst{g}{a_m}) \ge \\
        \ge d\,(s) + \smashoperator[lr]{\sum\limits_{a_i \in \tra{s}}} d\,(\taskst{g}{a_i}) -  \smashoperator{\max\limits_{a_i \in \tra{s}}} d\,(\taskst{g}{a_i}) \\
  \end{array} \]
  \end{enumerate}

  \item 
  Upper bound. Let's show that 
  
  \[ \begin{array}{l}
  d\,(s) + \smashoperator[lr]{\sum\limits_{a_i \in \tra{s}}} \min\,\{2\cdot d\,(\taskst{g}{a_i}) - 1, 2\cdot d\,(s'_i \otimes g)\}  \le \\
  \le 4 \cdot (d\,(s) + \smashoperator[lr]{\sum\limits_{a_i \in \tra{s}}} d\,(\taskst{g}{a_i}) - \smashoperator{\max\limits_{a_i \in \tra{s}}} d\,(\taskst{g}{a_i})) \\
  \end{array} \]
  
  For the upper bound we replace every $\min$ with any of its arguments (the result can only increase). Let $a_m = \argmax{a_i \in \tra{s}} d\,(\taskst{g}{a_i})$. Then for the upper bound let's
  go with the second argument of $\min$ for $a_m$ and with the first argument for all others.
  
   \[ \begin{array}{l}
  d\,(s) + \smashoperator[lr]{\sum\limits_{a_i \in \tra{s}}} \min\,\{2\cdot d\,(\taskst{g}{a_i}) - 1, 2\cdot d\,(s'_i \otimes g)\}  \le \\
  \le d\,(s) + 2 \cdot d\,(s'_m \otimes g) + \smashoperator[lr]{\sum\limits_{a_i \in (\tra{s} \setminus \{a_m\})}} (2\cdot d\,(\taskst{g}{a_i})) \le \\
  \le d\,(s) + 2 \cdot d\,(s'_m) + 2\cdot\smashoperator[lr]{\sum\limits_{a_i \in (\begin{array}{l} \tra{s} \setminus \\ \{a_1, \dots, a_m\}) \end{array}}} d\,(\taskst{g}{a_i}) + 2\cdot\smashoperator[lr]{\sum\limits_{a_i \in (\tra{s} \setminus \{a_m\})}} d\,(\taskst{g}{a_i}) \le \\
  \le d\,(s) + 2 \cdot d\,(s) + 2\cdot\smashoperator[lr]{\sum\limits_{a_i \in (\tra{s} \setminus \{a_m\})}} d\,(\taskst{g}{a_i}) + 2\cdot\smashoperator[lr]{\sum\limits_{a_i \in (\tra{s} \setminus \{a_m\})}} d\,(\taskst{g}{a_i}) \le \\
  \le 4 \cdot d\,(s) + 4\cdot\smashoperator[lr]{\sum\limits_{a_i \in (\tra{s} \setminus \{a_m\})}} d\,(\taskst{g}{a_i}) \\
  \end{array} \] $\qed$

\end{enumerate}

\end{proof}

\begin{lemma}
\label{lem:times_gen_measure_approximations}

Let $s = ((s_0 \otimes g_1) \dots \otimes g_k)$ and let $A_i$ be a set of all answers that are passed to $g_i$, i.e.

\[
\begin{array}{rcl}
A_1 &=& \tra{s_0} \\
A_{i + 1} & = & \bigcup\limits_{a \in A_i} \tra{\taskst{g_i}{a}} 
\end{array}
\]

Then

\[
\begin{array}{rcrl}
d\,(s) &=& & d\,(s_0) + \sum\limits_{1 \le i \le k} \displaystyle\sum\limits_{a \in A_i} d\,(\taskst{g_i}{a}) \\
\\
t\,(s) &=& & t\,(s_0) + \sum\limits_{1 \le i \le k} \displaystyle\sum\limits_{a \in A_i} t\,(\taskst{g_i}{a}) + \\
& & + \Theta\,(& d\,(s_0) + \sum\limits_{1 \le i \le k} \displaystyle\sum\limits_{a \in A_i} d\,(\taskst{g_i}{a}) - \maxd\limits_{a \in A_k}  d\,(\taskst{g_k}{a})) \\
\end{array}
\]

\end{lemma}
\begin{proof}
First we show that for all $i$, $A_{i + 1} = \tra{((s_0 \otimes g_1) \dots \otimes g_i)}$ (it's a simple induction on $i$ and then on the trace).

Then we can prove the statement by induction on $k$.
We unfold $d\,(s \otimes g_{k+1})$ and $t\,(s \otimes g_{k+1})$ using equations in \lemmaword~\ref{lem:times_measure_equations}.
Then we rewrite $d\,(s)$ and $t\,(s)$ in these equations with inductive hypothesis.
For $d\,(s \otimes g_{k+1})$ we get exactly the equation we need.
For $t\,(s \otimes g_{k+1})$ we have a sum of the following parts (first two from the unfolding of $t\,(s)$, last two from the rest of the equation for $t\,(s \otimes g_{k+1})$):
\begin{enumerate}
\item $t\,(s_0) + \displaystyle{\sum\limits_{1 \le i \le k}} \displaystyle\sum\limits_{a \in A_i} t\,(\taskst{g_i}{a})$
\item $ \Theta\,(d\,(s_0) +  \displaystyle{\sum\limits_{1 \le i \le k}} \displaystyle\sum\limits_{a \in A_i} d\,(\taskst{g_i}{a}) - \maxd\limits_{a \in A_k}  d\,(\taskst{g_k}{a}))$
\item $\displaystyle\sum\limits_{a \in A_{k+1}} t\,(\taskst{g_{k+1}}{a})$
\item $ \Theta\,(d\,(s_0) +  \displaystyle{\sum\limits_{1 \le i \le k}} \displaystyle\sum\limits_{a \in A_i} d\,(\taskst{g_i}{a}) + \displaystyle\sum\limits_{a \in A_k} d\,(\taskst{g_{k+1}}{a}) - \maxd\limits_{a \in A_{k + 1}}  d\,(\taskst{g_k}{a}))$
\end{enumerate}

We can see that the second part is subsumed by the last part (by \lemmaword~\ref{lem:theta_absorb}). The rest gives exactly the equation we need.
\qed
\end{proof}

Here is the general definition of well-formedness of states from~\cite{CertifiedSemantics}.

\begin{definition}
  Well-formedness condition for extended states:

  \begin{itemize}
  \item $\diamond$ is well-formed;
  \item $\inbr{g, \sigma, n}$ is well-formed iff $\fv{g}\cup\Dom\,(\sigma)\cup\VRan\,(\sigma)\subseteq\{\alpha_1,\dots,\alpha_n\}$;
  \item $s_1\oplus s_2$ is well-formed iff $s_1$ and $s_2$ are well-formed;
  \item $s\otimes g$ is well-formed iff $s$ is well-formed and for all leaf triplets $\inbr{\_,\_,n}$ in $s$ it is true that $\fv{g}\subseteq\{\alpha_1,\dots,\alpha_n\}$.
  \end{itemize}

\end{definition}

We will need \lemmaword~\ref{lem:measures_changing_env} in the following generalized form.

\begin{lemma}
\label{lem:gen_measures_changing_env}
Let $\pi \colon \{ \alpha_1, \dots, \alpha_N \} \to \{ \alpha_1, \dots, \alpha_{N'} \}$ be an injective function on variables and let $R_{\pi}$ be the following inductively defined relation on states:
\[ \begin{array}{lcl}
\Diamond R_{\pi} \Diamond & & \\
\taskst{g}{\mkenv{\sigma}{n}} R_{\pi} \taskst{g'}{\mkenv{\sigma'}{n'}} & \textit{iff} & g \sigma \pi = g' \sigma' \\
(s_1 \oplus s_2) R_{\pi} (s'_1 \oplus s'_2) & \textit{iff} & s_1 R_{\pi} s'_1 \land s_2 R_{\pi} s'_2  \\
(s \otimes g) R_{\pi} (s' \otimes g') & \textit{iff} & s R_{\pi} s' \land \\
\multicolumn{3}{l}{\textit{\quad for all substates $\taskst{g_i}{\mkenv{\sigma_i}{n_i}}$ in $s$}}\\
\multicolumn{3}{l}{\textit{\quad and corresponding substates $\taskst{g'_i}{\mkenv{\sigma'_i}{n'_i}}$ in $s'$,}} \\
\multicolumn{3}{l}{\quad g \sigma_i \pi = g' \sigma'_i} \\
\end{array} \]

Then for any two well-formed states $s$ and $s'$, such that all counters occuring in $s$ are less or equal than some $n$ and all counters occuring in $s'$ are less or equal than some $n'$ and $s R_{\pi} s'$ for some injective function $\pi \colon \{ \alpha_1, \dots, \alpha_n \} \to \{ \alpha_1, \dots, \alpha_n' \}$,

\[ d\,(s) = d\,(s') \]
and

\[t\,(s) = t\,(s') \]

and there is a bijection $b$ between sets of answers $\tra{s}$ and $\tra{s'}$ such that for any answer $a = \mkenv{\sigma_r}{n_r} \in \tra{s}$ there is a corresponding answer $b(a) = \mkenv{\sigma'_r}{n'_r} \in \tra{s'}$, s.t. $\sigma_r = \sigma \delta$ for some $\sigma$ that is a subtitution in some leaf substate of $s$ and $\sigma'_r = \sigma' \delta'$ for $\sigma'$ that is the substitutution of the corresponding leaf substate of $s'$ and there is an injective function $\pi_r \colon \{ \alpha_1, \dots, \alpha_{n_r} \} \to \{ \alpha_1, \dots, \alpha_{n'_r} \}$ such that $\pi_r \succ \pi$ and $\pi \delta' = \delta \pi_r$.
\end{lemma}
\begin{proof}
We prove it by induction on the length of the trace for $s$; simultaneously we prove that the next states in the traces for $s$ and $s'$ also satisfy the relation $R_{\pi_r}$ for some $\pi_r$ (s.t. $\pi_r \succ \pi$). The equalities $d\,(s) = d\,(s')$ and $t\,(s) = t\,(s')$ are obvious in this induction, because states in the relation $R_{\pi}$ always have the same form (and therefore the same left height). To prove the fact about the bijection between answers and the fact that the relation holds for the next states we conduct an internal induction on the relation of operational semantics step. When we move through the introduction of the fresh variable we extend the injective function changing the variable by a binding between new fresh variables. In the base case of unification, when we extend the substitutions by the most general unifiers, we have the fact about the bijection between the sets of answers (singleton in this case) for the same injective renaming function by definition of the unification algorithm (we may change the names of variables before the unification and the result will be the same as if we do it after the unification):

\[ \pi \, mgu\,(t_1 \sigma \pi, t_2 \sigma \pi) = mgu\,(t_1 \sigma, t_2 \sigma) \, \pi  \]

For the case when we incorporate the answer obtained at this step in the next state (in rules $\ruleno{ConjStopAns}$ and $\ruleno{ConjStepAns}$) we use the statement about the bijection between the sets of answers from the inductive hypothesis to prove that the next states satisfy the relation:

\[ g \sigma \delta \pi_r = g \sigma \pi \delta' = g' \sigma' \delta'  \] 

All other cases naturally follow from inductive hypotheses.\qed

\end{proof}

\repeatlemma{lem:measures_changing_env}
\begin{proof}
It is a special case of \lemmaword~\ref{lem:gen_measures_changing_env}.\qed
\end{proof}

\begin{lemma}
\label{lem:update_substitutions_FV}

Let $\taskst{g_0}{\mkenv{\sigma_0}{n_0}}$ be a leaf state. Then for every answer $\mkenv{\sigma'}{n'}$ in $\tra{\taskst{g_0}{\mkenv{\sigma_0}{n_0}}}$, $\sigma' = \sigma_0 \delta$ for some substitution $\delta$, such that $\Dom\,(\delta) \cup \VRan\,(\delta) \subset FV\,(g_0 \sigma_0) \cup \{ \alpha_i \mid i > n_0 \}$.

\end{lemma}
\begin{proof}

First, we need some notions to generalize the statement.

Let $ENV\,(s)$ be the set of environments that occur in the given state.

\[ \begin{array}{lcl}
ENV\,(\Diamond) & = & \emptyset \\
ENV\,(\taskst{g}{e}) & = & \{ e \} \\
ENV\,(s_1 \oplus s_2) & = & ENV\,(s_1) \cup ENV\,(s_2) \\
ENV\,(s \otimes g) & = & ENV\,(s) 
\end{array} \]

Now, let's generalize the set of variables updated by answers from the statement to an arbitrary state.

\[ \begin{array}{lcl}
\Delta\,(\Diamond) & = & \emptyset \\
\Delta\,(\taskst{g}{\mkenv{\sigma}{n}}) & = & FV\,(g \sigma) \cup \{ \alpha_i \mid i > n \} \\
\Delta\,(s_1 \oplus s_2) & = & \Delta\,(s_1) \cup \Delta\,(s_2) \\
\Delta\,(s \otimes g) & = & \Delta\,(s_1) \cup \bigcup\limits_{\mkenv{\sigma}{n} \in ENV\,(s)} FV\,(g \sigma)
\end{array} \]

Now we can generalize the statement but for one semantical step only: if $s \xrightarrow{l} s'$, then the following three conditions hold:

\begin{enumerate}
\item $\Delta\,(s) \supset \Delta\,(s')$
\item If $l = \mkenv{\sigma'}{n'}$ then there exists $\mkenv{\sigma}{n} \in ENV\,(s)$ and substitution $\delta$, such that $\sigma' = \sigma \delta$ and $n' = n$ and $\Dom\,(\delta) \cup \VRan\,(\delta) \subset \Delta\,(s)$
\item For any $\mkenv{\sigma'}{n'} \in ENV\,(s')$ there exists $\mkenv{\sigma}{n} \in ENV\,(s)$ and substitution $\delta$, such that $\sigma' = \sigma \delta$ and $n' \ge n$ and $\Dom\,(\delta) \cup \VRan\,(\delta) \subset \Delta\,(s)$
\end{enumerate}

We prove it by the induction on semantical step relation (we have to prove all three conditions simultaneously).

\begin{enumerate}
\item The first condition is simple for the steps from leaf goals: the counters of occupied variables can only increase and the sets of free variables of subgoals can only decrease (except for the case of fresh variable introduction, where a new free variable appears, but it is greater than the counter); in case of invocation we use the fact that the body of any relation is closed (there are no free variables except for the arguments). For the $\ruleno{ConjStopAns}$ rule we use the second condition: the next step has the environment that updates one of the environments in $s$, so it does not introduce new variables in $\Delta$ and the counter also may only increase. For the $\ruleno{ConjStep}$ rule we use the third condition: all substitutions from environments of updated state after application to a goal do not introduce new variables in $\Delta$, the rest is handled by the inductive hypothesis. For the $\ruleno{ConjStepAns}$ rule we combine two previous arguments and for other cases the first condition is obvious from the inductive hypothesis.

\item The second condition needs to be proven only for the $\ruleno{UnifySuccess}$ rule (where it follows from the properties of the unification algorithm) and for the rules $\ruleno{DisjStop}$ and $\ruleno{DisjStep}$ (where it is obvious from the inductive hypothesis).

\item The third condition follows simply in all cases from the inductive hypothesis and from the second condition (for the rules $\ruleno{ConjStep}$ and $\ruleno{ConjStepAns}$ where the answer is incorporated in the next state).
\end{enumerate}

Now, the statement of the lemma follows from the generalized statement for one step: at each step substitutions in the answers and in the next step are composed with some additional substitutions that manipulate with only variables from the set $\Delta$ for this step, which is a subset of the set $\Delta$ in the beginning, which is exactly what we need.
\qed

\end{proof}

\repeatlemma{lem:symbolic_unification_soundness}
\begin{proof} $ $

From the corectness of the Robinson's unification algoithm we know that a substitution unifies the pair of terms $(t_1, t_2)$ iff it unifies all pairs of terms from the set $\{ (x, \delta\,(x)) \mid x \in \Dom\,(\delta) \}$ (because we obtain $\delta$ from the pair $(t_1, t_2)$ by Robinson's algorithm that maintains equivalent unification problem).

First, let's notice that similarly a substitution unifies the pair of terms $(t_1 \rho, t_2 \rho)$ iff it unifies all pairs of terms  $T = \{ (\rho\,(x), \delta\,(x) \rho) \mid x \in \Dom\,(\delta) \}$: $\nu$ is such substitution iff $\rho \nu$ unifies the terms $(t_1, t_2)$. And then also any most general unifier for $(t_1 \rho, t_2 \rho)$ is the most general unifier for $T$ and vice versa (by definition).

So now we need to show that $T$ is unifiable iff the unique $\rho'$ from the statement of the lemma exists (and that the most general unifier for $T$ can be defined with $\rho'$ and $\delta$). In both directions we will use the induction on the construction of the set of variables $U$, so lets consider the following sequence $U_i$: $U_0 = V$ and $U_{i+1} = \{ U_i \cup \bigcup\limits_{x \in U_{i}} FV\,(\delta\,(x)) \}$ (so $U$ is $U_l$ such that $U_l = U_{l + 1}$).

\begin{enumerate}
\item Suppose there is a substitution $\tau$ that unifies all the terms in $T$. Let's show that there is a unique $\rho'$ such that $\rho' \succ \rho$ and $\forall (y, \, t) \in \constr{\delta}{U}, \\ \rho'(y) = t \rho'$.

We know that $\rho\,(x) \tau = \delta\,(x) \rho \tau$ for all $x \in \Dom\,(\delta)$. We need the same condition for $\rho'$ for all $x \in U \cap \Dom\,(\delta)$. We can now show by induction on $i$ that for all variables $x \in U_i$ ($i \ge 1$) the value $\tau\,(x)$ is ground and uniquely defined for a given $\rho$, so they can be taken as values of $\rho'$ on variables from $U \setminus V$ and they are the only possible values. First, look at a pair $(\rho\,(x), \delta\,(x) \rho)$ in $T$ for some $x \in V$. We know that $\rho\,(x) \tau = \delta\,(x) \rho \tau$ and the term on the lhs is ground and uniquely defined by $\rho$. So the values of $\tau$ for all free variables of $\delta\,(x) \rho$ are ground and uniquely defined by $\rho$, too. If we do it for all such pairs in $T$ we will get the statement for $U_1$, then we can repeat this reasoning by induction for all $U_i$.

\item Now suppose there is $\rho'$ such that $\rho' \succ \rho$ and $\forall (y, \, t) \in \constr{\delta}{U}, \rho'(y) = t \rho'$. Let's construct the most general unifier for $T$ using the Robinson's algorithm.

Let's split $T$ on $T_U = \{ (\rho\,(x), \delta\,(x) \rho) \mid x \in U \cap \Dom\,(\delta) \}$ and $T_{-U} = \{ (\rho\,(x), \delta\,(x) \rho) \mid x \in  \Dom\,(\delta) \setminus U \}$. We will be applying rules from the Robinson's algorithm to $T$ for pairs of terms from $T_U$.

First, let's look at some pair $(\rho\,(x), \delta\,(x) \rho)$ for $x \in V \cap \Dom\,(\delta)$. By definition of $\rho'$ we have $\rho\,(x) \rho' = \delta\,(x) \rho \rho'$. The first term in the pair is ground, the second one may contain free variables (then they are variables from $U_1$). If the second term is ground, too, they are equal and we can delete this pair. Otherwise, using decomposition rule we decompose this pair to pairs of terms with second term being variable. After this, we will have pairs $(\rho'(y), y)$ for all $y \in FV\,(\delta\,(x) \rho)$. After that we do it for all such pairs for all variables from $V \cap \Dom\,(\delta)$, this pairs will turn into swapped bindings $(\rho'(y), y)$ for all $y \in U_1$ (maybe with repetitions). We then can discard the duplicates, swap the elements and apply this bindings in the rest of $T$. Now all the pairs $(\rho\,(x), \delta\,(x) \rho)$ for $x \in (U_1 \setminus V) \cap \Dom\,(\delta)$ after substitution have ground terms as the left term and we can repeat the transformation for all these pairs. We can repeat this process by induction on $i$ for $x \in U_i$, until $U_i$ becomes equal to $U$.

After this application of rules all pairs from $T_U$ are decomposed and turned into the substitution (as a set of bindings) $\rho'\restriction_{U \setminus V}$. On the other hand the pairs from $T_{-U}$  are not decomposed, just applied substitutions to them so every pair from this part still has some variable $z$ as the first term (because $z\not\in U$) and term $\delta\,(z) \rho (\rho'\restriction_{U \setminus V})$ as the second term. So we can see that we turned the set of terms $T$ into the substitution ${(\delta\restriction_{\Dom\,(\delta) \setminus U} \rho')\restriction_{\Dom\,(\delta) \setminus V}}$ which equals ${(\delta \rho')\restriction_{\Dom\,(\delta) \setminus V}}$ because $\rho'$ unifies bindings in $\delta$, so this substitution is the most general unifier for $T$ and therefore for terms $t_1 \rho$ and $t_2 \rho$. Then this substitution is alpha-equivalent to $mgu\,(t_1 \rho, t_2 \rho)$ (because most general unifiers are unique up to alpha-equivalence). So if we take $\rho mgu\,(t_1 \rho, t_2 \rho)$, we will get substitution alpha-equivalent to the substitution ${\rho ((\delta \rho')\restriction_{\Dom\,(\delta) \setminus V})}$ which equals to the substitution $\delta \rho'$ (obviously separately for variables from $V$ andfrom outside $V$).
\qed
 

\end{enumerate}
\end{proof}

\repeattheorem{extracted_approximations}
\begin{proof}
$ $\newline
\begin{enumerate}
\item Suppose we have proven this statement for $g \in C_{nf}$. Let's show it holds for $g \in D_{nf}$.
	\begin{enumerate}
	\item First, we prove it for  $g \in F_{nf}$ by induction on the goal.
	After unfolding each fresh constructor we get a goal with the same scheme.	
	Also going through each fresh constuct increases the values of $d\,(\cdot)$ and $t\,(\cdot)$ by $1$ by \lemmaword~\ref{lem:task_measure_equations}, this additional constant can be deleted by \lemmaword~\ref{lem:theta_constant}.
	
	\item Now we prove it for $g \in D_{nf}$ by induction on the goal. Let $g = \disjgoal{g_1}{g_2}$. Let $\schemewithvset{\mathfrak{S}_1}{V}$ be $\schemewithvset{\mathfrak{S}_2}{V}$ be children of the root in $\schemewithvset{\mathfrak{S}}{V}$.
		
	By \lemmaword~\ref{lem:task_measure_equations} and \lemmaword~\ref{lem:sum_measure_equations} and \lemmaword~\ref{lem:measures_changing_env} we have the following equations:

    \[ \begin{array}{lcl}
	d\,(\taskst{\disjgoal{g_1}{g_2}}{e_{init}}) &=& d\,(\taskst{g_1}{e_{init}}) + d\,(\taskst{g_2}{e_{init}}) + 1 \\
	\\
	t\,(\taskst{\disjgoal{g_1}{g_2}}{e_{init}}) &=& t\,(\taskst{g_1}{e_{init}}) + t\,(\taskst{g_2}{e_{init}}) \\
	&& + \min\,(2 d\,(\taskst{g_1}{e_{init}}) - 1, 2 d\,(\taskst{g_2}{e_{init}})) + 1
	\end{array} \]
	
	After rewriting the right part with inductive hypotheses (combining them using \lemmaword~\ref{lem:theta_constant} and \lemmaword~\ref{lem:theta_sum}) we get the following approximations.
	
	 \[ \begin{array}{lcl}
	d\,(\taskst{\disjgoal{g_1 \rho}{g_2 \rho}}{e_{init}}) &=& \mathcal{D}(\schemewithvset{\mathfrak{S}_1}{V})(\rho) + \mathcal{D}(\schemewithvset{\mathfrak{S}_2}{V})(\rho) + \Theta\,(1) \\
	\\
	t\,(\taskst{\disjgoal{g_1 \rho}{g_2 \rho}}{e_{init}}) &=& \mathcal{T}(\schemewithvset{\mathfrak{S}_1}{V})(\rho) + \mathcal{T}(\schemewithvset{\mathfrak{S}_2}{V})(\rho) + \\
	& & \Theta\,(\min\,(\mathcal{D}(\schemewithvset{\mathfrak{S}_1}{V})(\rho), \mathcal{D}(\schemewithvset{\mathfrak{S}_2}{V})(\rho)) + \\
	& & + (\mathcal{D}(\schemewithvset{\mathfrak{S}_1}{V})(\rho) - \smashoperator{\maxd\limits_{\taskst{g_i}{e_i} \in \mathcal{L}(\schemewithvset{\mathfrak{S}_1}{V})(\rho)}} d\,(\taskst{g_i}{e_i})) + \\
	& & + (\mathcal{D}(\schemewithvset{\mathfrak{S}_2}{V})(\rho) - \smashoperator{\maxd\limits_{\taskst{g_i}{e_i} \in \mathcal{L}(\schemewithvset{\mathfrak{S}_2}{V})(\rho)}} d\,(\taskst{g_i}{e_i})) + 1) \\
	\end{array} \]
	
	For $d\,(\cdot)$ it is exactly what we need. 
   
   	W.l.o.g. let's suppose the minimum in the approximation for $t\,(\cdot)$ is achieved at the first argument.
   
   	Let's consider two cases: which of $\smashoperator{\maxd\limits_{\taskst{g_i}{e_i}  \in \mathcal{L}(\schemewithvset{\mathfrak{S}_l}{V})(\rho)}}d\,(\taskst{g_i}{e_i})$ is the maximal leaf for the whole scheme.
   
   		\begin{enumerate}
   		\item Suppose $\smashoperator{\maxd\limits_{\taskst{g_i}{e_i} \in \mathcal{L}(\schemewithvset{\mathfrak{S}_1}{V})(\rho)}} d\,(\taskst{g_i}{e_i}) \le \smashoperator{\maxd\limits_{\taskst{g_i}{e_i} \in \mathcal{L}(\schemewithvset{\mathfrak{S}_2}{V})(\rho)}} d\,(\taskst{g_i}{e_i})$.
   		
   		Then we can absorb the summand $(\mathcal{D}(\schemewithvset{\mathfrak{S}_1}{V})(\rho) - \smashoperator{\maxd\limits_{\taskst{g_i}{e_i} \in \mathcal{L}(\schemewithvset{\mathfrak{S}_1}{V})(\rho)}} d\,(\taskst{g_i}{e_i}))$ under $\Theta$ by the larger summand $\mathcal{D}(\schemewithvset{\mathfrak{S}_1}{V})(\rho)$ (which came from $\min$) by \lemmaword~\ref{lem:theta_absorb}. We get the following approximation which is exactly what we need:
   		
   		\[ \begin{array}{lcl}
		t\,(\taskst{\disjgoal{g_1 \rho}{g_2 \rho}}{e_{init}}) &=& \mathcal{T}(\schemewithvset{\mathfrak{S}_1}{V})(\rho) + \mathcal{T}(\schemewithvset{\mathfrak{S}_2}{V})(\rho) + \\
		& & \Theta\,(\mathcal{D}(\schemewithvset{\mathfrak{S}_1}{V})(\rho) + \\
		& & + (\mathcal{D}(\schemewithvset{\mathfrak{S}_2}{V})(\rho) - \smashoperator{\maxd\limits_{\taskst{g_i}{e_i} \in \mathcal{L}(\schemewithvset{\mathfrak{S}_2}{V})(\rho)}} d\,(\taskst{g_i}{e_i})) + 1) \\
		\end{array} \]
		
		\item Suppose $\smashoperator{\maxd\limits_{\taskst{g_i}{e_i} \in \mathcal{L}(\schemewithvset{\mathfrak{S}_1}{V})(\rho)}} d\,(\taskst{g_i}{e_i}) > \smashoperator{\maxd\limits_{\taskst{g_i}{e_i} \in \mathcal{L}(\schemewithvset{\mathfrak{S}_2}{V})(\rho)}} d\,(\taskst{g_i}{e_i})$.
		
		Then we establish the lower and the upper bounds separately.
		
			\begin{enumerate}
   			\item For the lower bound we again first absorb the summand $(\mathcal{D}(\schemewithvset{\mathfrak{S}_1}{V})(\rho) - \smashoperator{\maxd\limits_{\taskst{g_i}{e_i} \in \mathcal{L}(\schemewithvset{\mathfrak{S}_1}{V})(\rho)}} d\,(\taskst{g_i}{e_i}))$ by \lemmaword~\ref{lem:theta_absorb} to get the following.

			\[ \begin{array}{lcl}
			t\,(\taskst{\disjgoal{g_1 \rho}{g_2 \rho}}{e_{init}}) &\ge& \mathcal{T}(\schemewithvset{\mathfrak{S}_1}{V})(\rho) + \mathcal{T}(\schemewithvset{\mathfrak{S}_2}{V})(\rho) + \\
			& & C_1 \cdot (\mathcal{D}(\schemewithvset{\mathfrak{S}_1}{V})(\rho) + \\
			& & + (\mathcal{D}(\schemewithvset{\mathfrak{S}_2}{V})(\rho) - \smashoperator{\maxd\limits_{\taskst{g_i}{e_i} \in \mathcal{L}(\schemewithvset{\mathfrak{S}_2}{V})(\rho)}} d\,(\taskst{g_i}{e_i})) + 1) \\
			\end{array} \]		
		
			Then replace $(-\smashoperator{\maxd\limits_{\taskst{g_i}{e_i} \in \mathcal{L}(\schemewithvset{\mathfrak{S}_2}{V})(\rho)}} d\,(\taskst{g_i}{e_i}))$ \\ by $(-\smashoperator{\maxd\limits_{\taskst{g_i}{e_i} \in \mathcal{L}(\schemewithvset{\mathfrak{S}_1}{V})(\rho)}} d\,(\taskst{g_i}{e_i}))$ which is smaller by assumption.
		
			\item For the upper bound we first replace $\mathcal{D}(\schemewithvset{\mathfrak{S}_1}{V})(\rho)$ that came form $\min$ by $\mathcal{D}(\schemewithvset{\mathfrak{S}_2}{V})(\rho)$ which is larger by the assumption and then absorb the summand $(\mathcal{D}(\schemewithvset{\mathfrak{S}_2}{V})(\rho) - \smashoperator{\maxd\limits_{\taskst{g_i}{e_i} \in \mathcal{L}(\schemewithvset{\mathfrak{S}_2}{V})(\rho)}} d\,(\taskst{g_i}{e_i}))$ by it.
			
			\end{enumerate}		
   		\end{enumerate}
	\end{enumerate}

\item Now let's prove the statement of the theorem for $g \in C_{nf}$.

Let $g = { \conjgoal{(\conjgoal{(\conjgoal{g_1}{g_2})}{\dots})}{g_k} }$ with $g_i \in B_{nf}$.

First, notice that the state $\taskst{g}{e_{init}}$ is transformed into the state \\ ${ ((\taskst{g_1}{\mkenv{\varepsilon}{n_{init}(g)}} \otimes g_2) \otimes \dots) \otimes g_k }$ after $(k - 1)$ steps of turning conjunctions into $\otimes$-states. All states during this steps except the resulting one add $(k - 1)$ to the value $d\,(\taskst{g}{e_{init}})$ and $\dfrac{(k - 1)(k - 2)}{2}$ to the value $t\,(\taskst{g}{e_{init}})$, we can hide this constants under $\Theta$ by \lemmaword~\ref{lem:theta_absorb}. \lemmaword~\ref{lem:times_gen_measure_approximations} gives us the approximations of the measures for the state ${ ((\taskst{g_1}{\mkenv{\varepsilon}{n_{init}(g)}} \otimes g_2) \otimes \dots) \otimes g_k }$. To put it in a convenient form we will use the following definitions.

\[ \begin{array}{lcl}
D\,(s, \epsilon) & = & d\,(s) \\
D\,(s, g : \Gamma) & = & d\,(s) + \smashoperator{\sum\limits_{a \in \tra{s}}} D\,(\taskst{g}{a}, \Gamma) \\
\\
T\,(s, \epsilon) & = & t\,(s) \\
T\,(s, g : \Gamma) & = & t\,(s) + \smashoperator{\sum\limits_{a \in \tra{s}}} T\,(\taskst{g}{a}, \Gamma) \\
\\
L\,(s, \epsilon) & = & \{ s \} \\
L\,(s, g : \Gamma) & = & \smashoperator{\bigcup\limits_{a \in \tra{s}}} T\,(\taskst{g}{a}, \Gamma) \\
\end{array}
\]

Now,  \lemmaword~\ref{lem:times_gen_measure_approximations} gives us the following approximations if we denote ${ g_2 \rho : \dots g_k \rho}$ by $\Gamma$.

\[ \begin{array}{lcl}
d\,(\taskst{g \rho}{e_{init}}) &=& D\,(\taskst{g_1 \rho}{\mkenv{\varepsilon}{n_{init}(g)}}, \Gamma) + \Theta\,(1) \\
\\
t\,(\taskst{g \rho}{e_{init}}) &=& T\,(\taskst{g_1 \rho}{\mkenv{\varepsilon}{n_{init}(g)}}, \Gamma) + \\ 
& & \Theta\,(D\,(\taskst{g_1 \rho}{\mkenv{\varepsilon}{n_{init}(g)}}, \Gamma) - \smashoperator{\maxd\limits_{s \in L\,(\taskst{g_1 \rho}{\mkenv{\varepsilon}{n_{init}(g)}}, \Gamma)}} d\,(s) + 1) \\
\end{array} \]

It's the approximation in the form required in the statement of the theorem. What remains to be proven are the following equalities:

\[ \begin{array}{lcl}
D\,(\taskst{g_1 \rho}{\mkenv{\varepsilon}{n_{init}(g)}}, \Gamma) &=& \mathcal{D}(\schemewithvset{\mathfrak{S}}{V})(\rho) \\
T\,(\taskst{g_1 \rho}{\mkenv{\varepsilon}{n_{init}(g)}}, \Gamma) &=& \mathcal{T}(\schemewithvset{\mathfrak{S}}{V})(\rho) \\
\{ d\,(s) \mid s \in L\,(\taskst{g_1 \rho}{\mkenv{\varepsilon}{n_{init}(g)}}, \Gamma)\} &=& \{ d\,(s) \mid s \in \mathcal{L}(\schemewithvset{\mathfrak{S}}{V})(\rho) \} \\
\end{array} \]
	
We can prove them by induction, but first we need to generalize the statement. Let $g$ be a goal from $B_{nf}$ and $\Gamma = { g_1 : \dots : g_m : \epsilon }$~--- a sequence of goals from $B_{nf}$, $\sigma$ be a substitution, $n$ be a fresh variable counter such that the state ${ (((\taskst{g}{\mkenv{\sigma}{n}} \otimes g_1) \otimes \dots) \otimes g_m) }$ is well-formed, and $V$ be a subset of variables $\{ \alpha_1, \dots, \alpha_n \}$. Then if \[ \schemetrans{g}{\Gamma}{\sigma}{n}{V}{\schemewithvset{\mathfrak{S}}{V}} \] then the following equalities hold for any $\rho \colon V \to \mathcal{T}_{\emptyset}$.

\[ \begin{array}{lcl}
D\,(\taskst{g}{\mkenv{\sigma \rho}{n}}, \Gamma) &=& \mathcal{D}(\schemewithvset{\mathfrak{S}}{V})(\rho) \\
T\,(\taskst{g}{\mkenv{\sigma \rho}{n}}, \Gamma) &=& \mathcal{T}(\schemewithvset{\mathfrak{S}}{V})(\rho) \\
\{ d\,(s) \mid s \in L\,(\taskst{g}{\mkenv{\sigma \rho}{n}}, \Gamma)\} &=& \{ d\,(s) \mid s \in \mathcal{L}(\schemewithvset{\mathfrak{S}}{V})(\rho) \} \\
\end{array} \]

(to get from this generalization to the equations above we should take $\sigma = \varepsilon$ and apply \lemmaword~\ref{lem:measures_changing_env} to move $\rho$ from environment to the goal).

We prove the generalized statement by the induction on $\Gamma$ and considering cases when $g$ is an equality or a relational call. The reasoning is exactly the same for all three notions $D\,(\cdot)$, $T\,(\cdot)$ and $L\,(\cdot)$, so we demonstrate only the proof of the equality between $D\,(\cdot)$ and $\mathcal{D}(\cdot)(\cdot)$.

	\begin{enumerate}

	\item Let $\Gamma = \epsilon$ and $g = (\unigoal{t_1}{t_2})$.
	
	In this case we have \[ d\,(\taskst{\unigoal{t_1}{t_2}}{\mkenv{\sigma \rho}{n}}) = d\,(\taskst{\unigoal{t_1 \sigma \rho}{t_2 \sigma \rho}}{e_{init}}) = 1 \]
	
	\item Let $\Gamma = \epsilon$ and $g = (\invokegoal{R^k}{t_1}{t_k})$.
	
	In this case we have \[ d\,(\taskst{\invokegoal{R^k}{t_1}{t_k}}{\mkenv{\sigma \rho}{n}}) = d\,(\taskst{\invokegoal{R^k}{t_1 \sigma \rho}{t_k \sigma \rho}}{e_{init}}) \]
	
	Which is true by \lemmaword~\ref{lem:measures_changing_env}.
	
	\item Let $\Gamma = g' : \Gamma'$ and $g = (\unigoal{t_1}{t_2})$.
	
	\begin{enumerate}
	
	    \item If the terms $t_1 \sigma$ and $t_2 \sigma$ are non-unifiable, we have \[ d\,(\taskst{\unigoal{t_1}{t_2}}{\mkenv{\sigma \rho}{n}}) + \smashoperator{\sum\limits_{mgu\,(t_1 \sigma \rho, t_2 \sigma \rho) = \delta'}} D\,(\taskst{g'}{\mkenv{\sigma \rho \delta'}{n}}, \Gamma') = 1 \]
	    
	    And it is obviously true because the sum is empty since the more specific terms there are non-unifiable also.
	
	    \item If they are unifiable, we have \[ d\,(\taskst{\unigoal{t_1}{t_2}}{\mkenv{\sigma \rho}{n}}) + \smashoperator{\sum\limits_{mgu\,(t_1 \sigma \rho, t_2 \sigma \rho) = \delta'}} D\,(\taskst{g'}{\mkenv{\sigma \rho \delta'}{n}}, \Gamma') = \]
	
	\[ = 1 +
      \smashoperator{\sum\limits_{\substack{ \rho' \colon U \to \grterms \\
                                      \rho' \succ \rho \\
                                      \forall (y, t) \in Cs, \rho'(y) = t \rho'  }}}
           \mathcal{D}(\schemewithvset{\mathfrak{S}}{U})(\rho')  \]
           
    		where
    
    \[ \begin{array}{lcl}
    \delta & = & mgu\,(t_1 \sigma, t_2 \sigma) \\
    U & = & \upd{V}{\delta} \\
    Cs & = & \constr{\delta}{U} \\
	\end{array} \]
	
			The left summands are obviously equal. The rest is basically covered by \lemmaword~\ref{lem:symbolic_unification_soundness}. By this lemma there exists a most general unifier $\delta'$ iff the required $\rho'$ exists. So both sums are non-empty under the same conditions and have at most one summand (since $\rho'$ is unique), and if it is the case these summands are equal by \lemmaword~\ref{lem:symbolic_unification_soundness}, the inductive hypothesis and the fact that the value $D$ is stable w.r.t. renaming of variables (it is a generalization of \lemmaword~\ref{lem:measures_changing_env} that follows simply from \lemmaword~\ref{lem:gen_measures_changing_env}):
	
\[ \begin{array}{l}
D\,(\taskst{g'}{\mkenv{\sigma \rho \delta'}{n}}, \Gamma') \stackrel{\text{\lemmaword~\ref{lem:symbolic_unification_soundness}}}{=} D\,(\taskst{g'}{\mkenv{\sigma \delta \rho' \tau}{n}}, \Gamma') = \\
= D\,(\taskst{g'}{\mkenv{\sigma \delta \rho'}{n}}, \Gamma') \stackrel{\text{ind.hyp.}}{=} \mathcal{D}(\schemewithvset{\mathfrak{S}}{U})(\rho')
\end{array} \] 

	\end{enumerate}
	
	\item Let $\Gamma = g' : \Gamma'$ and $g = (\invokegoal{R^k}{t_1}{t_k})$.
	
	In this case we have \[ d\,(\taskst{\invokegoal{R^k}{t_1}{t_k}}{\mkenv{\sigma\rho}{n}}) + \smashoperator{\sum\limits_{a \in \tra{\taskst{\invokegoal{R^k}{t_1}{t_k}}{\mkenv{\sigma\rho}{n}}}}} D\,(\taskst{g'}{a}, \Gamma') = \]
	
	\[ = d\,(\taskst{\invokegoal{R^k}{t_1 \sigma \rho}{t_k \sigma \rho}}{e_{init}}) +
      \smashoperator{\sum\limits_{\substack{ \rho' \colon U \to \grterms \\
                                      \rho' \succ \rho \\
                                      (t_1 \sigma \rho', \dots, t_k \sigma \rho') \in \sembr{R^k}  }}}
           \mathcal{D}(\schemewithvset{\mathfrak{S}}{U})(\rho')  \]
           
    where
    
    \[ \begin{array}{lcl}
    U & = &  V \cup \bigcup_{i} FV\,(t_i \sigma) \\
    \end{array} \]
    
    The left summands are equal by \lemmaword~\ref{lem:measures_changing_env}.
    
    Each $a \in \tra{\taskst{\invokegoal{R^k}{t_1}{t_k}}{\mkenv{\sigma\rho}{n}}}$ has the form $(\sigma \rho \delta, n')$ for some $\delta$ and some $n' > n$ by \lemmaword~\ref{lem:gen_measures_changing_env}. All free variables of terms $t_i$ are mapped into ground terms in each delta, because the relational call is grounding.
    
    There is a bijection between the set of answers under the sum on the lhs and a set of suitable $\rho'$ at the rhs: all ground values for the free variables that are consistent with the denotational semantics are present in some answer (because of completeness of the operational semantics w.r.t. to the denotational one) and in exactly one (because the call is answer-unique).
    
    Now, we need to show that if we take any answer $(\sigma \rho \delta, n')$ and $\rho'$ that corresponds to it by the described bijection then the values for them under the sums are equal. First, we need to replace $D\,(\taskst{g'}{(\sigma \rho \delta, n')}, \Gamma')$ by $D\,(\taskst{g'}{(\sigma \rho \delta\restriction_U, n)}, \Gamma')$, we can do it because all variables from $\Dom\,(\delta) \setminus U$ have indices greater or equal than $n$ (by \lemmaword~\ref{lem:update_substitutions_FV}) and therefore these variables are not free variables of goals from $(g' : \Gamma')$, so the values of $d$ on this goals will be the same for the restricted substitution (it follows simply from \lemmaword~\ref{lem:gen_measures_changing_env}). After this replacement we can directly apply the inductive hypothesis.
    \qed
	
	\end{enumerate}


\end{enumerate}
\end{proof}



% \section{Definitions of the evaluated relations}
\label{sec:examples_definitions}

Here are the definitions of the \mK relations we used for evaluation. Different queries to the same relation may require different orders in conjunctions.

\begin{enumerate}

\item Comparison of Peano numbers

\begin{minipage}{\linewidth}
Definition:

\begin{lstlisting}[basicstyle=\small]
   le$^o$ = fun x y ->
     (x === O) \/
     (fresh (x' y')
        (x === S(x')) /\
        (y === S(y')) /\
        (le$^o$ x' y')
     )
\end{lstlisting}
\end{minipage}

\begin{minipage}{\linewidth}
For queries:

\begin{itemize}
\item[-] \lstinline{le$^o$ $\overline{n}$ $\overline{m}$}

\item[-] \lstinline{le$^o$ $x^?$ $\overline{m}$}

\item[-] \lstinline{le$^o$ $\overline{n}$ $y^?$}
\end{itemize}
\end{minipage}


\item Sum of Peano numbers

\begin{minipage}{\linewidth}
Definition:

\begin{lstlisting}[basicstyle=\small]
   plus$^o$ = fun x y r ->
     ((x === O) /\ (y === r)) \/
     (fresh (x' r')
        (x === S(x')) /\
        (r === S(r')) /\
        (plus$^o$ x' y r')
     )
\end{lstlisting}
\end{minipage}

\begin{minipage}{\linewidth}
For queries:

\begin{itemize}
\item[-] \lstinline{plus$^o$ $\overline{n}$ $\overline{m}$ $r^?$}

\item[-] \lstinline{plus$^o$ $\overline{n}$ $y^?$ $\overline{k}$}

\item[-] \lstinline{plus$^o$ $x^?$ $y^?$ $\overline{k}$}
\end{itemize}
\end{minipage}


\item Product of Peano numbers

\begin{minipage}{\linewidth}
Definition \#1:

\begin{lstlisting}[basicstyle=\small]
   mult$^o$ = fun x y r ->
     ((x === O) /\ (r === O)) \/
     (fresh (x' r')
        (x === S(x')) /\
        (mult$^o$ x' y r') /\
        (plus$^o$ r' y r)
     )
\end{lstlisting}
\end{minipage}

\begin{minipage}{\linewidth}
For query:

\begin{itemize}
\item[-] \lstinline{mult$^o$ $\overline{n}$ $\overline{m}$ $r^?$}
\end{itemize}
\end{minipage}

\begin{minipage}{\linewidth}
Definition \#2:

\begin{lstlisting}[basicstyle=\small]
   mult$^o$ = fun x y r ->
     ((x === O) /\ (r === O)) \/
     (fresh (x' r')
        (x === S(x')) /\
        (plus$^o$ r' y r) /\
        (mult$^o$ x' y r')
     )
\end{lstlisting}
\end{minipage}

\begin{minipage}{\linewidth}
For queries:

\begin{itemize}
\item[-] \lstinline{mult$^o$ $x^?$ $\overline{m + 1}$ $\overline{k}$}

\item[-] \lstinline{mult$^o$ S($x^?$) S($y^?$) $\overline{k}$}
\end{itemize}
\end{minipage}


\item Length of a list

\begin{minipage}{\linewidth}
Definition \#1:

\begin{lstlisting}[basicstyle=\small]
   length$_d^o$ = fun a r ->
     ((a === Nil) /\ (x === O)) \/
     (fresh (h t r')
        (a === Cons(h, t)) /\
        (length$_d^o$ t r') /\
        (r === S(r'))
     )
\end{lstlisting}
\end{minipage}

\begin{minipage}{\linewidth}
For query:

\begin{itemize}
\item[-] \lstinline{length$_d^o$ $\overline{l}$ $r^?$}
\end{itemize}
\end{minipage}

\begin{minipage}{\linewidth}
Definition \#2:

\begin{lstlisting}[basicstyle=\small]
   length$^o$ = fun a r ->
     ((a === Nil) /\ (x === O)) \/
     (fresh (h t r')
        (a === Cons(h, t)) /\
        (r === S(r')) /\
        (length$^o$ t r')
     )
\end{lstlisting}
\end{minipage}

\begin{minipage}{\linewidth}
For queries:

\begin{itemize}
\item[-] \lstinline{length$^o$ $\overline{l}$ $r^?$}

\item[-] \lstinline{length$^o$ $a^?$ $\overline{n}$}
\end{itemize}
\end{minipage}


\item Incrementing all elements in a list

\begin{minipage}{\linewidth}
Definition:

\begin{lstlisting}[basicstyle=\small]
   incr_list$^o$ = fun a r ->
     ((a === Nil) /\ (r === Nil)) \/
     (fresh (h t tr)
        (a === Cons(h, t)) /\
        (r === Cons(S(h), tr)) /\
        (incr_list$^o$ t tr)
     )
\end{lstlisting}
\end{minipage}

\begin{minipage}{\linewidth}
For queries:

\begin{itemize}
\item[-] \lstinline{incr_list$^o$ $\overline{l}$ $r^?$}

\item[-] \lstinline{incr_list$^o$ $a^?$ $\overline{l}$}
\end{itemize}
\end{minipage}



\item Concatination of two lists

\begin{minipage}{\linewidth}
Definition:

\begin{lstlisting}[basicstyle=\small]
   append$^o$ = fun a b r ->
     ((a === Nil) /\ (b === r)) \/
     (fresh (h t tb)
        (a === Cons(h, t)) /\
        (r === Cons(h, tb)) /\
        (append$^o$ t b tb)
     )
\end{lstlisting}
\end{minipage}

\begin{minipage}{\linewidth}
For queries:

\begin{itemize}
\item[-] \lstinline{append$^o$ $\overline{l_1}$ $\overline{l_2}$ $r^?$}

\item[-] \lstinline{append$^o$ $a^?$ $b^?$ $\overline{l}$}
\end{itemize}
\end{minipage}


\item Inversion of a list

\begin{minipage}{\linewidth}
Definition \#1:

\begin{lstlisting}[basicstyle=\small]
   reverse$^o$ = fun a r ->
     ((a === Nil) /\ (r === Nil)) \/
     (fresh (h t tb)
        (a === Cons(h, t)) /\
        (reverse$^o$ t tr) /\
        (append$^o$ tr Cons(h, Nil) r)
     )
\end{lstlisting}
\end{minipage}

\begin{minipage}{\linewidth}
For query:

\begin{itemize}
\item[-] \lstinline{reverse$^o$ $\overline{l}$ $r^?$}
\end{itemize}
\end{minipage}
 
\begin{minipage}{\linewidth}
Definition \#2:

\begin{lstlisting}[basicstyle=\small]
   reverse$^o$ = fun a r ->
     ((a === Nil) /\ (r === Nil)) \/
     (fresh (h t tb)
        (a === Cons(h, t)) /\
        (append$^o$ tr Cons(h, Nil) r) /\
        (reverse$^o$ t tr)
     )
\end{lstlisting}
\end{minipage}

\begin{minipage}{\linewidth}
For query:

\begin{itemize}
\item[-] \lstinline{reverse$^o$ $a^?$ $\overline{l}$}
\end{itemize}
\end{minipage}

\end{enumerate} 

\end{document}
\endinput
%%
