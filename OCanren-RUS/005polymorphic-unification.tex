% !TeX spellcheck = ru_RU
% !TEX root = main.tex

\section{Полиморфная унификация}
\label{sec:unification}

Довольно естественно попытаться реализовать полиморфную унификацию в языке, оснащенном полиморфным сравнением --- удобной, но порою спорной функциональностью\footnote{Например, \url{https://blogs.janestreet.com/the-perils-of-polymorphic-compare} (проверено \DTMDate{2025-01-10})}.
Как и полиморфное сравнение, полиморфная унификация производит обход значений, используя знания о представлении значений во время выполнения программы.
Неоспоримым преимуществом данного подхода является то, что, во-первых, для использования унификации программисту не придется писать стереотипного (англ. boilerplate) кода, а во-вторых, то, что данных подход позволяет получить наиболее эффективную реализацию.
С другой стороны, недостатки полиморфного сравнения также наследуются, в частности унификация будет зависать на циклических структурах данных и не работать со значениями-функциями.
Так как мы не ожидаем никакого разумного поведения в этих случаях, основной проблемой останется то, что компилятор не будет способен предупредить об использовании этих крайних случаев.
Другим недостатком является то, что реализацию придется изменять каждый раз, когда представление данных в компиляторе поменяется.
Также полиморфная унификация реализована в слабо типизированной манере с использование модуля \lstinline|Obj|, и её необходимо аккуратно тестировать.

%We consider it rather natural to employ polymorphic unification in a language already equipped
%with polymorphic comparison~--- a convenient, but somewhat disputable\footnote{See, for example,
%\url{https://blogs.janestreet.com/the-perils-of-polymorphic-compare} (accessed: \DTMDate{2025-01-10})} feature. Like polymorphic comparison,
%polymorphic unification performs a traversal of values, exploiting intrinsic knowledge of their runtime
%representation. The undeniable benefits of this solution are, first, that in order to perform unification
%for user types no ``boilerplate'' code is needed, and, second, that this approach seems to deliver the
%most efficient implementation. On the other hand, all the pitfalls of polymorphic comparison are inherited as
%well; in particular, polymorphic unification loops for cyclic data structures and does not work for functional
%values. Since we generally do not expect any reasonable outcome in these cases, the only remaining problem is that
%the compiler is incapable of providing any assistance in identifying and avoiding these cases. Another drawback is that
%the implementation of polymorphic unification relies on the runtime representation of values and has to be fixed
%every time the representation changes.  Finally, as it is written in an unsafe manner using the \lstinline|Obj| interface,
%it has to be carefully developed and tested.

Важным различием между полиморфным сравнением и унификацией является то, что сравнение только изучает свои операнды,
а унификация постепенно конструирует подстановку-результат.
Отображение переменных в термы из подстановки будет использоваться далее для реификации ответов.
Итого, нам необходимо показать, что в результате не конструируются неправильно типизированные значения.
Может показаться, что это свойство будет соблюдаться естественным образом, так как сутью синтаксической унификации является проверка того, что некоторые объекты считаются равными.
Тем не менее существуют различные системы типов и различные реализации унификации, к тому же иногда \emph{одинаковые значения} могут быть протипизированы \emph{по-разному}, поэтому нижу будет представлено обоснование корректности в абстрактном случае, а его части будут переиспользованы для конкретных.

%An important difference between polymorphic comparison and unification is that the former only inspects its operands,
%while the results of unification are recorded in a substitution (a mapping from logical variables to terms), which
%later is used to reify answers. So, generally speaking, we have to show that no ill-typed
%terms are constructed as a result. Overall, this property seems to be maintained vacuously, since the very
%nature of (syntactic) unification is to detect whether some things can be considered equal. Nevertheless there are
%different type systems and different unification implementations; in addition \emph{equal things} can be
%\emph{differently typed}, so we provide here a correctness justification for a well-defined abstract case, and will
%reuse this conclusion for various concrete cases.

Для начала рассмотрим три алфавита:

\[
\begin{array}{rcl}
  \tau,\dots&\text{---}&\mbox{типы;}\\
  x^\tau,\dots&\text{---}&\mbox{типизированные логические переменные;}\\
  C_k^{\tau_1\times\tau_2\times\dots\times\tau_k\to\tau} (k\ge 0),\dots&\text{---}&\mbox{конструкторы типов.}
\end{array}
\]

Множество всех правильно построенных типов задается индуктивно:
%The set of all well-formed typed terms is defined by mutual induction for all types:

\[
t^\tau=x^\tau\mid C_k^{\tau_1\times\tau_2\times\dots\times\tau_k\to\tau}(t^{\tau_1},\,t^{\tau_2},\,\dots,\,t^{\tau_k})
\]

\noindent Для простоты сократим нотацию  $C_k^{\tau_1\times\tau_2\times\dots\times\tau_k\to\tau}(t^{\tau_1},\,t^{\tau_2},\,\dots,\,t^{\tau_k})$ до
$C_k^\tau(t^{\tau_1},\,t^{\tau_2},\,\dots,\,t^{\tau_k})$, держа в голове, что для любого конструктора и для всех его вхождений его аргументы имеют соответствующие типы.

%\noindent For simplicity from now on we abbreviate the notation $C_k^{\tau_1\times\tau_2\times\dots\times\tau_k\to\tau}(t^{\tau_1},\,t^{\tau_2},\,\dots,\,t^{\tau_k})$ into
%$C_k^\tau(t^{\tau_1},\,t^{\tau_2},\,\dots,\,t^{\tau_k})$, keeping in mind that for any concrete constructor and for all its occurrences
%in arbitrary terms all its subterms in corresponding positions agree in types.

\begin{comment}
We need also to define the notion of a subterm  $t^\tau[p]$ of a term $t^\tau$ at given position $p$:

$$
\begin{array}{rcl}
 p=\epsilon\mid\{1, 2, 3,\dots\}\bullet p&-&\mbox{the set of positions}\\
 t^\tau[\epsilon]=t^\tau&-&\mbox{base case}\\
 C_k^\tau(t_1^{\tau_1},\,t_2^{\tau_2},\dots,\,t_k^{\tau_k})[i\bullet p]=t_i^{\tau_i}[p], 1\le i \le k&-&\mbox{inductive case}
\end{array}
$$
\end{comment}

В данном изложении мы не рассматриваем произвольные структуры над множеством типов, содержащие что-то кроме операции равенства.
Также мы считаем, что всем термам во время исполнения соответствует некоторый тип.
Воспользуется этим свойством чтобы реализовать алгоритм унификации на OCaml, используя некоторые представления для термов и типов:
%In this formulation we do not consider any structure over the set of types besides type equality, and we assume all terms are explicitly
%attributed to their types at runtime. We employ this property to implement a unification algorithm in regular OCaml, using some
%representation for terms and types:

\begin{lstlisting}[mathescape=true]
    val unify : term -> term -> subst option -> subst option,
\end{lstlisting}

\noindent где \enquote{\lstinline=term=} соответствует типу представления типизированных термов, а \enquote{\lstinline=subst=} --- это подставнока (частичное отображение переменных в термы).
Унификация может закончиться неуспешно (поэтому в типе результат присутствует \enquote{\lstinline=option=}),
выполняется в контексте некоторой подстановки (поэтому \enquote{\lstinline=subst=} в третьем аргументе) и может запускать по цепочке (для этого
\enquote{\lstinline=option=} в третьем аргументе).

%\noindent where \enquote{\lstinline=term=} stands for the type representing typed terms, and \enquote{\lstinline=subst=} stands for the type of
%substitution (a partial mapping from logic variables to terms). Unification can fail (hence \enquote{\lstinline=option=} in the result type),
%is performed in the context of existing substitution (hence \enquote{\lstinline=subst=} in the third argument) and can be
%chained (hence \enquote{\lstinline=option=} in the third argument).

Используется тот же самый алгоритм унификации с треугольной подстановкой, что и в минимальной реализации~\cite{MicroKanren}.
Также опущены некоторые не очень важные детали (такие как \enquote{occurs check}), которые нужны в настоящей реализации, а также воздержимся от обсуждения алгоритма самого по себе: превосходное описание~\cite{Kumar} вместе с доказательством корректности можно найти в источниках.

%We use exactly the same unification algorithm with triangular substitution as in the reference implementation~\cite{MicroKanren}. We
%omit here some not-so-important details (like \enquote{occurs check}), which are kept in the actual implementaion, and refrain from discussing
%the nature and properties of the algorithm
%itself (an excellent description, including a certified correctness proof, can be found in~\cite{Kumar}).

Псевдокод ниже демонстрирует схему реализации:
%The following snippet presents the implementation:

\begin{lstlisting}[mathescape=true,numbers=left,numberstyle=\small,stepnumber=1,numbersep=-5pt]
    let rec unify $t_1^\tau$ $t_2^\tau$ $subst$ =
      let rec walk $s$ $t^\tau$ =
        match $t^\tau$ with
        | $x^\tau$ when $x^\tau\in dom(s)$ -> $\;\;$walk $s$ $(s\;\;x^\tau)$
        | _ -> $t^\tau$
      in
      match $subst$ with
      | None -> None
      | Some $s$ ->
          match walk $s$ $t_1^\tau$, walk $s$ $t_2^\tau$ with
          | $x_1^\tau$, $x_2^\tau$ when $x_1^\tau$ = $x_2^\tau$ -> $subst$
          | $x_1^\tau$, $q_2^\tau$ -> Some ($s\;[x_1^\tau \gets q_2^\tau]$)
          | $q_1^\tau$, $x_2^\tau$ -> Some ($s\;[x_2^\tau \gets q_1^\tau]$)
          | $C^\tau(t_1^{\tau_1},\dots,t_k^{\tau_k})$, $C^\tau(p_1^{\tau_1},\dots,p_k^{\tau_k})$ ->
              unify $t_k^{\tau_k}$ $p_k^{\tau_k}$(.. (unify $t_1^{\tau_1}$ $p_1^{\tau_1}$ $subst$)$..$)
          | $\_$, $\_$ -> None
\end{lstlisting}

Напоминаем читателю, что в верхних индексах указаны типы, которые мы считаем частью значений.
Например, в строке 1 имеется в виду, что мы запускаем \lstinline|unify|
от термов $t_1$ и $t_2$, у которых один и тот же тип $\tau$. Предположим, что самый первый запуск унификации происходит от термов одного типа, и что любая подстановка может быть получена только с помощью одной или последовательности унификаций.

%We remind the reader that all superscripts correspond to type attributes, which we consider here as
%parts of values being manipulated. For example, line 1 means that we apply \lstinline|unify|
%to terms $t_1$ and $t_2$, and expect their types to be equal $\tau$.
%We assume that at the top level unification is always applied to some terms of the same type and that any
%substitution can only be acquired from the empty one by a sequence of unifications.

Покажем, что в этих предположения все атрибуты с типами не нужны: они не влияют на исполнение  \lstinline|unify| и могут быть стёрты.
Единственное место, где невозможно предъявить атрибут явно --- это строка 4, где возвращается результат применения подстановки.
\textcolor{red}{walk где описан?}.
Можно показать по индукции, что любая подстановка обладает следующим свойством: если в  подстановке $s$ определено значение переменной $x^\tau$, то это значение имеет тип $\tau$, и, следовательно,  \lstinline|walk $s$ $t^\tau$| всегда возвращает значения типа  $\tau$.

%We are going to show that under these assumptions all type attributes are superfluous~--- they
%do not affect the execution of \lstinline|unify| and can be removed. Note that the only place where we
%were incapable of providing an explicit type attribute was in line 4, where the result of
%substitution application was returned.
%However, we can prove by induction that any substitution respects the following property: if a substitution $s$ is defined for a variable $x^\tau$,
%then $s\;\;x^\tau$ is attributed with the type $\tau$ (and, consequently, \lstinline|walk $s$ $t^\tau$| always returns a term of type $\tau$).

Это свойство очевидно выполняется для пустых подстановок.
Рассмотрим произвольную подстановку $s$, для которой это свойство выполнено. В 10й строке совершаются два вызова ---
\lstinline|walk $s$ $t_1^\tau$| и \lstinline|walk $s$ $t_2^\tau$| --- и разбираются их результаты.
По индукционному предположению у них тип $\tau$ и производить сопоставление с образцом в строке 14 можно.
В сроке 11 подстановка возвращается без изменений, в строках 12 и 13 --- изменяется с сохранением свойства.
Наконец, в 15й строке осуществляется несколько вызовов \lstinline|unify|, но все они запускаются на термах одинакового типа, а значит, что подстановка обладает интересующим нас свойством. Индукция по структуре терма завершает доказательство.

%Indeed, this property vacuously holds for the empty substitution. Let $s$ be some substitution, for which the
%property holds. In the 11 we return an unchanged substitution; in line 10 we perform two calls~---
%\lstinline|walk $s$ $t_1^\tau$| and \lstinline|walk $s$ $t_2^\tau$| and match their results. However,
%by our induction hypothesis these results are again attributed to the type $\tau$, which justifies the
%pattern matching. In line 11 we return the substitution unchanged, in lines 12 and 13 we extend the
%existing substitution but preserve the property of interest. Finally, in line 15 we chain a few
%applications of \lstinline|unify|; note that, again, all these calls are performed for terms of equal
%types, starting from a substitution possessing the property of interest.
%A simple induction on the chain length completes the proof.

Итак, атрибуты с типами не важны --- они никогда не анализируются и не ограничивают сопоставления с образцом, а значит могут быть полностью стёрты.
Также стоит обратить внимание, что представлением термов может служить обычное представление среды исполнения OCaml.
Но это нельзя сделать естественным способом, придется воспользоваться низкоуровневым интерфейсом модуля \lstinline|Obj|.
Также понадобится некоторый способ отличать вхождения логических переменных в терм, в оригинальной реализации эта проблема решалась с помощью договоренности --- роскошь, которую мы не может себе позволить\footnotetext{\textcolor{red}{как-то коряво}}. Мы вернемся к этому вопросу в следующем разделе.

%So, type attributes are inessential~--- they are never analyzed and never restrict pattern matching; hence,
%they can be erased completely.
%We can notice now that for the representation of terms we can use OCaml's native runtime representation.
%It can not be done, however, using regular OCaml~--- we have to utilize the low-level, unsafe interface \lstinline|Obj|.
%Note also, we need some way to identify the occurrences of logical variables inside the terms (in the original \miniKanren{} implementation the ambiguity between variable and non-variable datum representation is resolved by a convention~--- a luxury we cannot afford).
%We postpone the discussion on this subject until the next section.

Реализация унификации называется \emph{полиморфной}, потому что на верхнем уровне можно приписать тип
%We call our implementation \emph{polymorphic}, since at the top level it is defined as

\begin{lstlisting}
   val unify : $\alpha$ -> $\alpha$ -> subst option -> subst option
\end{lstlisting}

Тип подстановки не является полиморфным, что означает, что компилятор полностью теряет типы значений, хранящихся в подстановке.
Они восстанавливаются во время реификации ответов (раздел~\ref{sec:reification}).
Вне унификации компилятор поддерживает типизацию, что означает что все термы, подтермы и логические переменные согласуются по типам во всех контекстах.

%The type of substitution is not polymorphic, which means that the compiler completely loses the track
%of types of values stored in a substitution. These types are recovered later during the reification-of-answers phase (see Section~\ref{sec:reification}).
%Outside the unification the compiler maintains typing, which means that all terms, subterms, and variables agree in their types
%in all contexts.
