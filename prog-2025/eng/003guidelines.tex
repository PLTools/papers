% !TEX TS-program = pdflatex
% !TeX spellcheck = en_US
% !TEX root = main.tex
\section{Layout Synthesis as a Constraint Satisfaction Problem}
\label{sec:guidelines}

%As we said earlier we are interested in a fully automatic approach to lay out GUI controls in a way prescribed
%by design guidelines. In this section we give a formal definition of what design guidelines are
%and how they define layouts, and specify the concrete problem of finding layouts which
%satisfy the constraints induced by both guidelines and the essence of layout primitives.

In our model given in the previous section the logical structure of GUI is described by
a number of relations between a certain set of controls. Let $\mathcal{C}$ be a set of
controls and let a structure $\Sigma=\{r^{n_i}_i\, |\, r^{n_i}_i\subseteq\mathcal{C}^{n_i}\}$ be a set of
relations, where $r^{n_i}_i$ is an \mbox{$n_i$-ary} relation on $\mathcal{C}$; without loss
of generality we do not consider property-like relations like size, text contents, etc.

A \emph{relational pattern} is a syntactic form $r^n\,(X_1,\dots,X_n)$, where $r^n$ is $n$-ary
relation and $X_j$ are variables (not necessarily distinct).

A \emph{layout template} is a syntactic form $l\,(X_1,\dots,X_n)$, where $l$ is a layout primitive and
$X_i$~--- variables.

%\footnote{In the previous section we only defined binary layout primitives; however
%here we use a more general form to avoid making impression that the number of arguments is essential.}.

We denote $s\,[X_i\gets c_i]$ the result of substitution of controls $c_i\in\mathcal{C}$
for variables $X_i$ in a relational pattern or layout template $s$, $\mathcal{FV}\,(s)$~--- the set of all
variables in a relational pattern or layout template $s$, and call \emph{layout instance} a layout
template with no variables.

Let $r^n\,(X_1,\dots,X_n)$ be a relational pattern. If for some controls $c_1,\dots,c_n\in\mathcal{C}$

\[
(c_1,\dots,c_n)\in r^n
\]

\noindent  then we say that the result of substitution

\[
r^n\,(X_1,\dots,X_n)\,[X_i\gets c_i]
\]

\noindent \emph{occurs} in the structure $\Sigma$ (we remind that $r^n$ is an $n$-ary relation on the set of controls).


A \emph{guideline rule} is a syntactic form

\[
p_1,\dots,p_k\mapsto l_1,\dots,l_m
\]


\noindent  where $p_i$~--- relational patterns, $l_j$~--- layout templates. We stipulate that any variable in the right-hand side of a
rule has an occurrence in its left-hand side; no other restrictions are assumed. Informally a guideline rule says that if
some controls are related in a certain manner in the structure then a specific layout primitives for (some of) these
controls can be used.

%For example, a rule

%\[
%\term{describes}\,(X,\,Y)\mapsto\term{hor}\,(X,\,Y)
%\]

%says that if some controls $X$ and $Y$ are in a relation ``$\term{describes}$'' then $X$ and $Y$ can be placed horizontally one after another.


Let us have a structure $\Sigma$ and a set $S$ of layout instances. We say that this set is \emph{confirmed} by a set of
guideline rules \emph{R} iff for each control $l^*\in S$

\begin{enumerate}
\item there is a rule $p_1,\dots,p_n\mapsto l_1,\dots,l_i,\dots, l_m$ in $R$
  such that $l^*=l_i\,[X_j\gets c_j]$  for
  some $\{c_j\}\subseteq\mathcal{C}$, where $\mathcal{FV}\,(l_i)=\{X_j\}$;

\item for all variables $\{Y_1,\dots,Y_t\}=\mathcal{FV}\,(p_1,\dots,p_n)\setminus\mathcal{FV}\,(l_i)$ there exist
  controls $c^\prime_1,\dots,c^\prime_t\in\mathcal{C}$ such that \mbox{$p_j\,[X_r\gets c_r,\dots,Y_s\gets c^\prime_s]$}
  for all $j$ occurs in the structure $\Sigma$;

\item \mbox{$l_j\,[X_r\gets c_r,\dots,Y_s\gets c^\prime_s]\in S$} for all $j$.

\end{enumerate}

Informally speaking a set $S$ is confirmed by $R$ iff each its control $l^*$ can be properly
derived according to $R$. This means that there should be some rule containing
layout template $l_i$ which is turned into $l^*$ by some substitution of its
variables $X_i$. However, in this rule there potentially can be other variables $Y_j$ besides $X_i$.
If there exists a substitution for them which, together with the substitution for $X_i$, makes all
relational patterns in the left-hand side to occur in $\Sigma$, then the rule can be
applied and $l^*$ can be derived. Finally, as there can be other layout templates in the
right-hand side of this rule, they will be derived as well, and thus corresponding
layout instances have to be included in $S$.

Confirmed sets represent all layouts which can be justified according to the guidelines in the sense that they
do not contain ``extra'' layout instances appeared out of thin air. However being confirmed does not
necessarily mean to be a ``good'' layout in the sense a regular designer would imply. Indeed, an
empty set is always confirmed, but can hardly be considered as a valuable layout. Thus, we define
another property: \emph{coverage}.

We say that a set of layout instances $S$ \emph{covers} a structure $\Sigma$ iff for each control $c\in\mathcal{C}$
there is a layout instance in $S$ which binds $c$ (i.e. contains $c$ as a parameter). Coverage
informally means that no control of a structure is left ``unaddressed'' by given layout.

Confirmed sets in some sense correspond to \emph{sound} layouts w.r.t. to guidelines while covering~--- to
\emph{complete}; however we refrain from calling them such since we reserve more conventional meanings of these terms.

The next requirement comes from an observation that guideline rules are expected to be applied, not ignored. We
call a confirmed and covering set of layout instances $S$ \emph{maximal} iff there is no
confirmed and covering set $S^\prime$ such that $S\subset S^\prime$.

Finally, layouts can contain conflicts. For example, if a layout says that a control $A$ has to be put on
the top of control $B$ and visa versa then, obviously, it is unusable since these two primitives introduce
incompatible constraints. Among the conflicts we can identify the following:

\begin{itemize}
\item There should be no loops or branching comprised of layout primitives for vertical and horizontal composition. This
  requirement originates from the observation that the relations introduced by these
  primitives have to be unions of linear orders.
\item If two compositional virtual controls are laid out horizontally then there should not be
  controls within both of them which are laid out vertically; symmetrically for vertically
  laid out virtual controls.
\item No two controls can be laid out vertically and horizontally at the same time.
\end{itemize}

These conflicts originate from the fact that we actually place the controls on a two-dimensional plane.
There are actually other cases which we omit here for space considerations. We call a set of
instances \emph{compatible} iff it does not contain conflicts.

To summarize, we are interested in confirmed, covering, maximal and compatible layouts w.r.t. a given
set of guideline rules. We call such layouts \emph{sound}.

It is worth mentioning the worst-case complexity of the problem in question. First, it is rather
easy to see that finding if there exists a sound layout is $NP$-complete in general case.
Indeed, given an undirected graph we can construct a structure where controls correspond to its nodes
and some binary relation (call it ``$E$'') corresponds to edges. Then we take the following guideline:

\[
\begin{array}{rcl}
  E\,(x,\, y) & \mapsto & \mbox{hor}\,(x,\, y)\\
  E\,(x,\, y) & \mapsto & \mbox{hor}\,(y,\, x)
\end{array}
\]

Any sound layout w.r.t. this guideline by definition contains all controls (i.e. nodes of the graph) and
only layout primitives ``\texttt{hor}'', each of which corresponds to exactly one edge of the graph. Since
``\texttt{hor}'' can not branch or loop, these edges form a Hamiltonian path.

On the other hand, it is rather easy to see that the number of all sound layouts can be worst-case
exponential on the size of a structure. As we are aimed at enumerating them all the solvers for
the problem would inevitably run in worst-case exponential time.


