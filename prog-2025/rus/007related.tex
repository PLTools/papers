% !TeX spellcheck = ru_RU
% !TEX root = mainrus.tex

\section{Обзор}
\label{related}

Дизайн и реализация GUI --- горячая тема уже много десятилетий, поэтому можно найти много инструментов, подходов, статей и отчётов по теме. Большая часть из них (если не все) предлагают декларативные и автоматические решения. Но если присмотреться, то их понятия \enquote{декларативности} и \enquote{автоматизации} отчаются от наших.

%GUI design and implementation has been a hot topic for decades. Thus, to no surprise there is a lot of frameworks, approaches, papers
%and reports on the subject. A fair share of them (if not all) present declarative and automatic solutions. A careful study, however,
%discovers that this ``declarativeness'' and ``automation'' is understood differently then in our case.

Во-первых, необходимо упомянуть некоторое количество инструментов для реализации GUI и визуализации данных, например,
\textsc{React}~\cite{react},
\textsc{Jetpack Compose}~\cite{Jetpack}, \textsc{SwiftUI}~\cite{SwiftUI}, \textsc{Streamlit}~\cite{Streamlit}, \textsc{D3}~\cite{D3} и другие.
Они предоставляют пользователю наборы примитивов, которые позволяют отображать данные и пользовательский интерфейс. Например, \textsc{Streamlit}
предоставляет набор встроенных примитивов отображения: \enquote{колонки}, \enquote{контейнеры}, \enquote{модальные диалоги} и т.п.~\cite{StreamlitLayout}, а также множество внешних компонент. Эти примитивы позволяют пользователям абстрагироваться от конкретного вычисления координат и относительного выравнивания. Также они предоставляют разумное поведение при изменении размера холста. Однако, выбор конкретного примитива остаётся на усмотрение разработчика, а не определяется системой.
И если расположение элементов должно по какой-то причине измениться, то эти изменения нужно реализовывать вручную. В нашем случае, разработчики не задают конкретное расположение, а только логическую структуру пользовательского интерфейса. Гайдлайн определяет конкретное расположение элементов управления,
в зависимости от внешних ограничений, таких как разрешение экрана или настройки региона (например, письмо справа-налево). Пока логическая структура не изменяется,
не потребуется никакого взаимодействия с программистом для отображения интерфейса по-другому. С другой стороны, данные инструменты могут использоваться совместно c нашим как способ визуализации интерфейса, так как они предоставляют сходные примитивы расположения. В этой работе мы таким образом задействовали Qt/QML.

%First of all, we need to mention some software frameworks and tools for design and implementation of GUI and visualization of data, for example, \textsc{React}~\cite{react},
%\textsc{Jetpack Compose}~\cite{Jetpack}, \textsc{SwiftUI}~\cite{SwiftUI}, \textsc{Streamlit}~\cite{Streamlit}, \textsc{D3}~\cite{D3} and others.
%These frameworks provide a number of layout primitives which end-users can employ in order to render their data or UI. For example, \textsc{Streamlit}
%provides a number of builtin layout primitives like ``columns'', ``container'', ``modal dialog'', etc.~\cite{StreamlitLayout} and an endless
%variety of third-party external components. These primitives allow end-users to abstract away of concrete controls coordinate calculation and their
%relative alignment; they also prescribe a reasonable behavior on enclosing pane resizing. However, which layout primitives to use is decided by
%end-users, not the system. If due to any reason the layout needs to be changed these changes have to be implemented manually. In our case
%end-users do not specify concrete layouts, only the logical structure of the UI. The guideline takes care of concrete layout, depending on
%external constraints such as enclosing pane size, screen resolution or even regional settings (for example, right-to-left writing system). As long
%as the logical structure remains unchanged no interference from end-users is required for laying out the UI in different settings. On the
%other hand these frameworks can be used as back-ends in our approach since they provide a similar set of layout primitives.

Программирование в ограничениях уже использовалось для размещения элементов управления GUI. Одним из примеров реактивного языка программирования в ограничениях является язык \textsc{Wallingford}~\cite{Wallingford2016} и система \textsc{Cassowary}~\cite{Cassowary2001}. \textsc{Wallingford} позволяет прикреплять ограничения различной силы к значениям в программе. Система реагирует на изменения времени и обновляет значения, не нарушая ограничения. Например, можно реализовать элемент управления с шириной равной синусу текущего времени. Система \textsc{Cassowary} и её потомки позволяют вычислять размеры и положения элементов динамически, например при изменении размера холста. Она поддерживает множество ограничений, в том числе глубину по оси Z; арифметические операции (например, ширина одного элемента может быть половиной высоты другого); элементы, накладывающиеся друг на друга и т.д. Эти система предназначена для задач автоматической изменения размеров элементов при изменении общего размера холста. Также, она не предоставляет поддержки правил общего вида для корректного позиционирования элементов.
Мы сомневаемся в выразимости неоднозначных расположений элементов в данных системах, например, если вертикально или горизонтальное взаимоположение определяется на основе размеров. С другой стороны, количество различных ограничений больше, чем у нас. Например, поддержка накладывающихся друг на друга элементов управления запрещена у нас с самого начала, и такие расположения не создаются вообще.

%Constraint programming has already been used for deciding the placement of GUI controls. One of the examples are constraint reactive programming
%language \textsc{Wallingford}~\cite{Wallingford2016} and the \textsc{Cassowary} system~\cite{Cassowary2001}. \textsc{Wallingford} allows to attach
%constraints of various strength to different values in the program. The system reacts to the time changes and updates these values without violation
%of the constraints. For example, one could calculate a width of a GUI control as the sine of current time. The \textsc{Cassowary} system and its
%descendants allow to calculate the sizes and positions of controls dynamically, for example at the moment of canvas resize.
%%The background theory is linear arithmetic.
%It supports many different constraints, for example, Z-ordering, arithmetic operators (for example, a control's width can be the half
%of another one's height), overlapping views, etc.  These systems are targeted for the tasks of dynamic adaptation the sizes of controls on resize.
%Also, they don't provide support of expressing general rules about correct control placing, the constraints are added for values of concrete structure.
%In our work we make a strict separation between GUI structure and rules of correct positioning of controls. We have doubts about expressing non-deterministic
%layouts in these systems, for example, if vertical or horizontal placement of controls depend on their sizes. On the other hand, the number of various
%constraint types is larger than in our approach. For example, they allow to express overlapping views, but based on our experience we initially ruled out
%this option, and our current implementation doesn't generate such layouts at all.

В последние годы появились методы на основе машинного обучения. Некоторые работы ставят гораздо более амбициозные цели, чем наши.

%Finally, in recent years methods and approaches from AI in ML/data-science sense start to percolate into the area. Some of the works address much more
%ambitious objectives, than ours.

Существует направление исследования, занимающееся порождением кода UI на основе картинок~\cite{Cai2023}. По полученному от дизайнера рисунку интерфейса,  инструмент распознает элементы управления и их относительное местоположение, и создает код реализации для инструмента создания GUI.
Подход ортогонален и не совместим с предлагаемым нами. Он требует взаимодействия с дизайнером во время получения каждой части интерфейса, у нас же дизайнер задействован только при описании гайдлайна. Также наш подход позволяет создавать много интерфейсов автоматически. В инструменте~\cite{Robust} была поставлена несколько другая задача: по картинке интерфейса синтезируется его реализация в терминах примитивов Android GUI, масштабируемая между различными устройствами и без типичных ошибок. Для выполнения этой задачи из картинки извлекается элементы управления, их местоположения, а также некоторые ограничения.
Эти ограничения напоминают наши примитивы расположения, но специализированы для семантики виджета \texttt{ConstraintLayout}~\cite{ConstraintLayout} из Android.
На основе этих ограничений и набору \emph{свойств надёжности}, разработанному авторами, вероятностная модель, обученная на большем наборе существующих интерфейсов, порождает код. Эти реализации, более устойчивы к измерению размера и  разрешения экрана, чем полученные только путём распознавания картинок.
Любопытно, что авторы мотивируют свою работу, заявляя, что \enquote{одно и то же расположение должно отображаться более чем на 15000 устройствах Android с $\approx$100 различных плотностей пикселей; требовать от программистов разработки и поддержки всех сочетаний входа чрезвычайно не желательно}.
Именно   это делает наша система за несколько минут со 100\% точностью, поэтому мы считаем свой подход более общим.

%First of all, there is a direction of research on UI code generation from images~\cite{Cai2023}. Given a designer-drawn form, a code
%generator recognizes UI controls and their relative placements and generates implementation code for one of GUI frameworks. This approach
%is completely orthogonal and incompatible with what we suggest. Indeed, it requires an interaction with a designer when implementing every piece of
%an interface while in our case a designer is only involved when guidelines are developed; then our system produces guideline-compatible
%interfaces \emph{en masse} automatically.
%In~\cite{Robust} a slightly different task is addressed: given a picture of an interface synthesize its
%implementable layout in terms of Android GUI primitives which would be scalable across various devices while avoiding a typical layout errors.
%To achieve this goal, after recognizing UI controls and their locations a certain set of relational layout constraints is extracted. This
%set of constraints resembles our layout primitives but is aligned with Android's \texttt{ConstraintLayout} widget~\cite{ConstraintLayout} semantics.
%Given these constraints and a set of \emph{robustness properties} developed by the authors an implementation code is generated with
%the aid of a probabilistic model trained on a large set of existing Android interfaces. This implementation is more stable under screen size
%or resolution change, than that provided by image recognition only.
%Interestingly, one of the motivations for the work, as the authors
%specify explicitly, is that ``the same layout needs to be rendered potentially on more than 15 000 Android devices with $\approx$100 different density
%independent screen sizes. Requiring the user to provide and maintain input specifications for all of them is infeasible yet highly desirable.''
%That is exactly what our system is capable of doing within a few minutes and with 100\% accuracy, so we consider our approach
%much more general.

Также решалась задача согласованного во всём приложении оформления GUI~\cite{LearningGUI}. Подход основан на идее заполнения:
в предположении, что уже есть набор согласованных расположений, решается проблема добавления ещё одного элемента так, чтобы новое расположение было согласовано со старым. Эта задача связана с нашей, так как мы можем рассматривать уже существующее расположение как неявно заданный гайдлайн, но мы можем указать несколько потенциальных недостатков. Во-первых,  учитывается только добавление элементов, без удаления. Во-вторых, добавление/удаление компонент не обязательно приводит к \enquote{монотонному} изменению расположения: добавление ещё одного поля ввода может существенно поменять расположение. Наконец, начальный согласованный дизайн редко появляется из воздуха, наверняка это результат следования уже существующему гайдлайну, который был описан явно.

%Automatic design of consistent (uniform throughout an application) GUI is addressed in~\cite{LearningGUI}. The approach is based on the idea of completion:
%assuming there is already a set of consistent layouts a problem of adding yet another element is addressed; the addition should be consistent with the previous
%designs. While this problem is, indeed, related to that we address (indeed, consider an existing set of designs as an implicitly specified guideline),
%we can identify a number of potential weaknesses. First, only addition of a component is considered, but not removal; second, the addition/subtraction
%of components does not necessarily lead to a ``monotonic'' change of the layout (adding yet another text field may result is a drastically different placement);
%finally, the initial set of designs to be consistent with rarely comes out of a thin air; most likely it is a result of following some
%existing (and perhaps implicit) guidelines, which should be specified and followed explicitly.

В системе~\cite{Grid} поставлена гораздо более широкая задача синтеза без гайдлайна, а только на основе эстетических, эргономических и других метрик.
Для синтеза используется генетический алгоритм, а качество вычисляется на основе отзывов пользователей. Сам синтез работает часами. Хотя подход потенциально позволяет создавать эстетически приятные интерфейсы без участия дизайнера, задача получения корректного расположения для существенно разных структур обходится стороной.

%In~\cite{Grid} a much more ambitious problem of synthesizing a layout with no guidelines, based only on aesthetic, ergonomic, etc., metrics is considered.
%Genetic algorithm is employed for synthesis, and users' feedback is used as a way to measure the quality of the synthesis; the synthesis itself can
%sometimes take hours. While the approach presumably allows to synthesize aesthetically convincing layouts with no designer input, the problem of
%layout consistency for sufficiently different structures is left unaddressed.

Встречается интересная задача~\cite{Scout} исследования различных расположений элементов. Она ставит своей целью помочь дизайнерам разрабатывать убедительные и разнообразные расположения элементов управления. Вводится набор ограничений, с помощью которого дизайнеры составляют требования к результату.
Любопытно, что в наших терминах этот набор ограничений является смесью структурных ограничений и ограничений на расположение: например, можно указывать и порядок элементов, и выравнивание. По набору ограничений система порождает множество расположений, удовлетворяющих им. Изменяя ограничения дизайнер исследует возможные интерфейсы.
Для решения ограничений используется метод ветвей и границ. С ростом числа возможных решений применяются эвристики для отсечения эстетически неподходящих ответов. Не смотря на то, что эта работа похожа на нашу, она нацелена на дизайнеров и может рассматриваться скорее как средство разработки гайдлайна, чем как средство получения расположения элементов, удовлетворяющего гайдлайну.

%An interesting problem of layout exploration is considered in~\cite{Scout}. The objective is to help the \emph{designers} to develop convincing and
%diverse layouts. A set of constraints is introduced which designers can use in order to specify some requirements for the design. Interestingly, in our terms the
%set of constraints is a mixture of layout and structural ones: for example, both order and alignment constraints are present. Given
%a number of constraints the system generates a set of layouts consistent with this constraints; by updating the constraints a designer can
%continue the exploration. As a constraint solver modifier branch-and-bound algorithm is used; as the number of feasible solutions
%turned out to be enormous a set of heuristic metrics was developed to rule aesthetically unfeasible designs out. While this work
%shares some similarities with ours, being targeted on designers it can be considered as a mean to \emph{develop guidelines}, not to synthesize
%guideline-consistent designs.
