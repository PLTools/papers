% !TeX spellcheck = ru_RU
% !TEX root = main.tex

%\begin{figure}[t]
%\centering
%\includegraphics{graph2.pdf}
%\caption{The Second Set of Benchmarks}
%\label{eval:second}
%\end{figure}

\section{Performance Evaluation}
\label{sec:evaluation}

One of our initial goals was to evaluate what performance impact would choosing OCaml as a host language makes. In addition we spent some
effort in order to implement \miniKanren in an efficient, tagless manner, and, of course, the outcome of this decision also has to be
measured. For comparison we took faster-miniKanren\footnote{\url{https://github.com/webyrd/faster-miniKanren}}~--- a full-fledged
\miniKanren implementation for Scheme/Racket. It turned out that faster-miniKanren implements a number of optimizations~\cite{WillThesis, Optimizations}
to speed up the search; moreover, the search order in our implementation initially was a little bit different. In order to make the comparison
fair, we additionally implemented all these optimizations and adjusted the search order to exactly coincide with
what faster-miniKanren does.

\begin{figure}[t]
\centering
\includegraphics[scale=0.4]{graph.png}
\caption{The Results of the Performance Evaluation}
\label{eval}
\end{figure}

\FloatBarrier

Для набора бенчмарков были выбраны следующие задачи:
%For the set of benchmarks we took the following problems:

\begin{itemize}
\item \textbf{pow, logo}~--- реляционное возведение в степень и логарифмирование~\cite{KiselyovArithm} чисел в двоичном представлении.
Конкретно выбраны $3^5=243$ и $log_3 243=5$.
%and logarithm for integers in binary form. The concrete tests relationally computed
%$3^5$ (which is 243) and $log_3 243$ (which is, conversely, 5). The implementaion was adopted from~\cite{KiselyovArithm}.
\item \textbf{quines, twines, trines}~--- программы, вычисляющиеся в себя, использованные для тестирования реляционных интерпретаторов~\cite{Untagged}.
Конкретно тесты вычисляю первые 100, 15 и 2 ответа(ов) для соответствующих задач.
%\item \textbf{quines, twines, trines}~--- self/co-evaluating program synthesis problems from~\cite{Untagged}. The
%concrete tests took the first 100, 15 and 2 answers for these problems respectively.
\end{itemize}

%Since the last bundle of benchmarks uses disequality constraints (and, hence, $\mu$Kanren is ruled out) we
%split all benchmarks into two sets.

Вычисления производились на компьютере с процессором Intel Core i7-4790K CPU @ 4.00GHz и 16GB памяти.
For OCanren \texttt{ocaml-4.04.0+frame\_pointer+flambda} was used,
for faster-miniKanren~--- Chez~Scheme~9.4.1.
All benchmarks were executed in the natively compiled mode ten times, then average user time was taken. The results of the evaluation
are shown on Figure~\ref{eval}. The whole evaluation repository with all scripts and detailed description is accessible
from \lstinline|GitHub|\footnote{\url{https://github.com/Kakadu/ocanren-perf/tree/ispras-paper2025}}.

The first conclusion, which is rather easy to derive from the results, is that the tagless approach indeed matters. Our initial
implementation did not show essential speedup in comparison even with $\mu$Kanren (and was even \emph{slower} on the logarithm
and permutations benchmarks). The situation was improved drastically, however, when we switched to the tagless version.

Yet, in comparison with faster-miniKanren, our implementation is still lagging behind. We can conclude that the optimizations
used in the Scheme/Racket version, have a different impact in the OCaml case; we save this problem for future research.

