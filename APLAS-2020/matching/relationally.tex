\section{Pattern Matching Synthesis, Relationally}
\label{sec:relationally}

In this section we describe a relational formulation for the pattern matching synthesis problem. Practically,
this amounts to constructing a goal with a free variable corresponding to the switch program to synthesize
for (arbitrary) list of patterns. In order to come up with a tractable goal certain steps have to be performed.
We first describe the general idea, and then consider these steps in detail.

Our idea of using relational programming for pattern matching synthesis is based on the following observations:

\begin{itemize}
\item For the switch language we can implement a relational interpreter\footnote{Conventionally for \textsc{miniKanren},
  the names of relations are superscripted by ``$^o$''.} $eval^o_\ir$ with the following property: for
  arbitrary $v\in\mathcal V$, $\pi\in\ir$ and $i\in\mathbb N$
 
  \[
  \setarrow{\xrightarrow}
  \setsubarrow{_\ir}
   eval^o_\ir\, v\, \pi\, i \Longleftrightarrow \withenv{v}{\trans{\pi}{}{i}}
  \]

  In other words, $eval^o_\ir$ interprets a program $\pi$ for a scrutinee $v$ and returns exactly the same branch (if any)
  which is prescribed by the semantics of the switch language. 
  
\item On the other hand, we can directly encode the declarative semantics of pattern matching as a relational
  program $match^o$ such that for arbitrary $v\in\mathcal V$, $p_i\in\mathcal P$ and $i\in\mathbb N$

  \[
  \setarrow{\xrightarrow}
  match^o\,v\,p_1,\dots,p_k\,i \Longleftrightarrow \trans{\inbr{v;\,p_1,\dots,p_k}}{}{i}
  \]

  Again, $match^o$ succeeds with $1\le i\le k$ iff $p_i$ is the leftmost pattern, matching $v$; otherwise it
  succeeds with $i=k+1$.
\end{itemize}

We address the construction of relational interpreters for both semantics in Section~\ref{sec:relints}.

Being relational, both $eval^o_\ir$ and $match^o$ do not just succeed or fail for ground arguments, but also can be \emph{queried} for
arguments with free logical variables, thus performing a search for all substitutions for these variables which make the
relation hold. This observation leads us to the idea of utilizing the definition of the pattern matching
synthesis problem, replacing ``$\xrightarrow{}{}\!\!$'' with $match^o$, ``$\xrightarrow{}{}_{\!\!\!\mathcal S}$`` with $eval^o$,
and $\pi$ with a free logical variable $\circled{?}$, which gives us the goal

\[
\forall v\in \mathcal V,\; \forall 1\le i\le n+1 : match^o\,v\,p_1,\dots,p_n\,i\Longleftrightarrow eval^o\,v\,\circled{?}\,i
\]

\noindent This goal, however, is problematic from relational point of view for a number of reasons.

First, \textsc{miniKanren} provides rather a limited support for universal quantification. Apart from being inefficient from
a performance standpoint, existing implementations either do not coexist with disequality constraints~\cite{eigen}
or do not support quantified goals with an infinite number of answers~\cite{moiseenko}. As we will see below, both restrictions are
violated in our case. Second, there is no direct support for the equivalence of goals (``$\Leftrightarrow$''). Thus,
reducing the original synthesis problem to a viable relational goal involves some ``massaging''.

We eliminate the universal quantification over the infinite set of scrutinees, replacing it by a \emph{finite}
conjunction over a \emph{complete set of samples}. For a sequence of patterns $p_1,\dots,p_k$ a
complete set of samples is a finite set of values $\mathcal{E}(p_1,\dots,p_k)\subseteq\mathcal{V}$ with the following
property:

\[
\setarrow{\xrightarrow}
\begin{array}{rcl}
  \forall\pi\in\mathcal S\! &\!: & [\forall v\in\mathcal{E}(p_1,\dots,p_k),\,\forall i\in\mathbb{N} : \trans{\inbr{v;\,p_1,\dots,p_k}}{}{i}\!\Longleftrightarrow \!{\setsubarrow{_{\mathcal S}}\withenv{v}{\trans{\pi\!}{}{i}}}]\!\Rightarrow\\
                          &   & [\forall v\in\mathcal V,\,\forall i\in\mathbb{N} : \trans{\inbr{v;\,p_1,\dots,p_k}}{}{i} \Longleftrightarrow  {\setsubarrow{_{\mathcal S}}\withenv{v}{\trans{\pi}{}{i}}}]
\end{array}
\]

In other words, if a program implements a correct and complete pattern matching for all values in a complete set of samples, then this
program implements a correct and complete pattern matching for all values. The idea of using a complete set of samples originates from the following observation: each pattern
describes a (potentially infinite) set of values, and pattern matching splits the set of all values into equivalence classes, each corresponding to a certain matching pattern.
Moreover, the values of different classes can be distinguished only by looking down to a \emph{finite} depth (as different patterns can be distinguished in this way).
The generation of a complete sample set will be addressed below (see Section~\ref{sec:samples}). Example-based program synthesis is not a completely new technique in relational
programming~\cite{unified}; in our case, however, we can ensure the correctness of the synthesis result, while in previous reports it had to be established externally.

\setarrow{\xrightarrow}

To eliminate the universal quantification over the set of answers we rely on the functionality of declarative pattern matching semantics. Indeed, given a fixed sequence $p_1,\dots,p_k$
of patterns, for every value $v$ there is exactly one answer value $i$, such that $\trans{\inbr{v;\,p_1,\dots,p_k}}{}{i}$. We can reformulate this property as

\[
\exists i:\, \trans{\inbr{v;\,p_1,\dots,p_k}}{}{i} \Longrightarrow  
\Big(\forall j : \trans{\inbr{v;\,p_1,\dots,p_k}}{}{j} \Longrightarrow  j=i\Big)
\]

Thus, we can replace universal quantification over the sets of answers by existential one, for which we have an efficient relational counterpart~--- the ``\lstinline|fresh|''
construct.

Following the same argument, we may replace the equivalence with conjunction: indeed, if

\[
\setarrow{\xrightarrow}
\trans{\inbr{v;\,p_1,\dots,p_k}}{}{i}
\]

for some $i$, then (by functionality), for any other $j\ne i$

\[
\setarrow{\xrightarrow}
\neg\;(\trans{\inbr{v;\,p_1,\dots,p_k}}{}{j})
\]

A correct pattern matching implementation $\pi$ should satisfy the condition

\[
\setarrow{\xrightarrow}
\setsubarrow{_{\mathcal S}}
\withenv{v}{\trans{\pi}{}{i}}
\]

But, by the determinism of the switch language semantics, it immediately follows, that for arbitrary $j\ne i$

\[
\setarrow{\xrightarrow}
\setsubarrow{_{\mathcal S}}
\neg\;(\withenv{v}{\trans{\pi}{}{j}})
\]

\begin{comment}
Alternatively\footnote{\color{red} Reviewer N1 said that passage about bool argument is unclear and may be omitted (or described with more details)}, we could switch to a more explicit relational representation of both semantics, adding an extra boolean argument to
both $eval^o_{\mathcal S}$ and $match^o$ and using the same fresh variable $b$ in the query of interest:

\[
match^o\,v\,p_1,\dots,p_k\,i\,b \wedge eval^o_{\mathcal S}\,v\,\pi\,i\,b
\]
\end{comment}
Thus, the goal we eventually came up with is

\[
\bigwedge_{v\in\mathcal{E}\,(p_1,\dots,p_k)}\mbox{\lstinline|fresh ($i$)|}\; \{match^o\ v\,\ p_1,\dots,p_k\ i \ \wedge \ eval^o_{\mathcal S}\,v\,\circled{?}\ i \}
\eqno{(\star\star)}
\]

From a relational point of view this is a pretty conventional goal which can be solved by virtually any decent \textsc{miniKanren} implementation in
which the relations $eval^o_{\mathcal S}$ and $match^o$ can be encoded.

Finally, we can make the following important observation. Obviously, any pattern matching synthesis problem has at least one trivial solution.
This, due to the completeness of relational interleaving search~\cite{search,certifiedSemantics}, means that the goal above \emph{can not diverge} with
no results. Actually it is rather easy to see that any pattern matching synthesis problem has \emph{infinitely many} solutions: indeed, having just
one it is always possible to ``pump'' it with superfluous ``$\primi{otherwise}$'' clauses; thus, the goal above is \emph{refutationally
complete}~\cite{WillThesis,DivergenceTest}. These observations justify the totality of our synthesis approach. In Section~\ref{sec:optimization} we show
how we can make it provide optimal solution.

\subsection{Constructing Relational Interpreters}
\label{sec:relints}

In this section we address the implementation of relations $eval^o_{\mathcal S}$ and $match^o$. In principle, it amounts to accurate encoding of
relations 
``$\xRightarrow{}{}\!\!$'' and ``$\xRightarrow{}{}_{\!\!\mathcal S}$'' 
in \textsc{miniKanren} (in our case, \textsc{OCanren}). We, however,
make use of a relational conversion~\cite{conversion} tool, called \textsc{noCanren}, which automatically converts a subset of \textsc{OCaml} into
\textsc{OCanren}. Thus, both interpreters are in fact implemented in \textsc{OCaml} and repeat corresponding inference rules almost
literally in a familiar functional style. For example, functional implementation of a declarative semantics looks like follows:

\begin{lstlisting}
   let rec $\inbr{v;\,p}$ =
     match ($v$, $p$) with
     | (_, Wildcard) -> true
     | ($C^k\;v^*$, $C^k\;p^*$) -> list_all $\inbr{;}$ (list_combine $v^*$ $p^*$)
     | _             -> false

  let $match^o$ $v$ $p^*$ =
    let rec inner $i$ $p^*$ =
      match $p^*$ with
      | []      -> $i$
      | $p$ :: $p^*$ -> if $\inbr{v;\,p}$ then $i$ else inner S($i$) $p^*$
    in inner O $p^*$
\end{lstlisting}

We mixed here the concrete syntax of \textsc{OCaml} and mathematical notation, used in the definition of the relation in question, to underline their similarity;
the actual implementation only a few lines of code longer. Note, we use here natural numbers in Peano form and custom list processing functions in order
to apply relational conversion later.

Using relational conversion saves a lot of efforts as \textsc{OCanren} specifications tend to be much more verbose; in addition
relational conversion implements some ``best practices'' in relational programming (for example, moves unifications forward in
conjunctions and puts recursive calls last). Finally, it has to be taken into account that relational conversion of pattern matching introduces
disequality constraints.

\subsection{Dealing with a Complete Set of Samples}
\label{sec:samples}

As we mentioned above, a complete set of samples plays an important role in our approach: it allows us to eliminate universal quantification over the
set of all values. As we replace the universal quantifier with a finite conjunction with one conjunct per sample value reducing the size of
the set is an important task. At the present time, however, we build an excessively large (worst case exponential of depth) number of samples. We discuss
the issues with this choice in Section~\ref{sec:eval} and consider developing a more advanced approach as the main direction for
improvement.

Our construction of a complete set of samples is based upon the following simple observations. We simultaneously define the \emph{depth} measure
for patterns and sequences of patterns as follows:

\[
\begin{array}{rcl}
   d\,(p_1,\dots,p_k)     & = & max\, \{ d\,(p_i)\}\\
   d\,(\_)                 & = & 0 \\
   d\,(C^k\,p_1,\dots p_k) & = & 1 + d\,(p_1,\dots,p_k)
\end{array}
\]

\noindent As a sequence of patterns is the single input in our synthesis approach we will call its depth \emph{synthesis depth}.

Similarly, we define the depth of matching expressions

\[
\begin{array}{rcl}
  d_{\mathcal M}\,(\bullet) & = & 1 \\
  d_{\mathcal M}\,(m\,[i]) & = & 1 + d_{\mathcal M}\,(m)\\
\end{array}
\]

and switch programs:

\[
\begin{array}{rcl}
  d_{\mathcal S}\,(\primi{return}\;i)&=&0\\
  d_{\mathcal S}\,(\primi{switch}\;m\;\primi{of}\;C_1\to s_1,\dots,C_k\to s_k\;\primi{otherwise}\;s)&=&\\
  \multicolumn{3}{c}{\qquad\qquad\qquad\qquad\qquad\qquad\qquad\qquad\qquad\qquad max\,\{d_{\mathcal M}\,(m),\,d_{\mathcal S}\,(s_i),\,d_{\mathcal S}\,(s)\}}
\end{array}
\]

Informally, the depth of a switch program tells us how deep the program can look into a value. 

From the definition of $\inbr{;}$ it immediately follows that a pattern $p$ can only discriminate values up to its depth $d\,(p)$: changing a value at the depth greater
or equal than $d\,(p)$ cannot affect the fact of matching/non matching. This means that we need only consider switch programs of depth no greater than the synthesis depth.
But for these programs the set of all values with height no greater than the synthesis depth forms a complete set of samples. Indeed, if the height of a value less or
equal to the synthesis depth, then this value is a member of complete set of samples and by definition the behavior of the synthesized program on this value is
correct. Otherwise there exists some value $s$ from the complete set of samples, such that given value can be obtained as an ``extension'' of $s$ at the
depth greater than the synthesis depth. Since neither declarative semantics nor switch programs can discriminate values at this depth, the behavior for a given value
will coincide with the correct-by-definition behavior for  $s$.

The implementation of complete set generation, again, is done using relational conversion. The enumeration of all terms up to a certain depth
can be acquired from a function which calculates the depth of a term: indeed, converting it into a relation and then running with \emph{fixed} depth
and \emph{free} term arguments delivers what we need. Thus, we add an extra conjunct which performs the enumeration of all values to the
relational goal $(\star\star)$, arriving at

\[
depth^o\,v\,n\wedge\mbox{\lstinline|fresh ($i$)|}\; \{match^o\,v\,p_1,\dots,p_k\,i \wedge eval^o_{\mathcal S}\,v\,\circled{?}\,i\}
\eqno{(\star\star\star)}
\]

Here $n$ is a precomputed synthesis depth in Peano form.

\begin{comment}
\begin{figure}[ht]
\begin{subfigure}[t]{0.2\linewidth}
  \[
  \{A^1,\,B^0,\,C^1,\,D^0\}
  \]
\vskip6mm
\caption{Constructors}
\label{fig:constructors}
\end{subfigure}
\hspace{0.5cm}
\begin{subfigure}[t]{0.26\linewidth}
  \[
  \begin{array}{c}
    C^1\,(A^1\,(B^0))\\
    C^1\,(\_)\\
    \_
  \end{array}
\]
\caption{Patterns}
\label{fig:patterns}
\end{subfigure}
\hspace{0.5cm}
\begin{subfigure}[t]{0.33\linewidth}
  \[
  \begin{array}{lcl}
     B             & \mapsto & 2 \\
     D             & \mapsto & 2 \\
     A\, (B)       & \mapsto & 2 \\
     A\, (D)       & \mapsto & 2 \\
     C\, (B)       & \mapsto & 1 \\
     C\, (D)       & \mapsto & 1 \\
     A\, (A\, (B)) & \mapsto & 2 \\
     A\, (A\, (D)) & \mapsto & 2 \\
     A\, (C\, (B)) & \mapsto & 2 \\
     A\, (C\, (D)) & \mapsto & 2 \\
     C\, (A\, (B)) & \mapsto & 0 \\
     C\, (A\, (D)) & \mapsto & 1 \\
     C\, (C\, (B)) & \mapsto & 1 \\
     C\, (C\, (D)) & \mapsto & 1 
  \end{array}
  \]
\caption{Generated samples}
\label{fig:samples}
\end{subfigure}
\caption{Complete set of samples example} 
\label{fig:complete-set-example}
\end{figure}
\end{comment}
