% !TEX TS-program = pdflatex
% !TeX spellcheck = en_US
% !TEX root = main.tex
\documentclass[acmsmall,screen,review,anonymous]{acmart}

\AtBeginDocument{%
  \providecommand\BibTeX{{%
    Bib\TeX}}}

%% Rights management information.  This information is sent to you
%% when you complete the rights form.  These commands have SAMPLE
%% values in them; it is your responsibility as an author to replace
%% the commands and values with those provided to you when you
%% complete the rights form.
\setcopyright{acmlicensed}
\copyrightyear{2024}
\acmYear{2024}
\acmDOI{XXXXXXX.XXXXXXX}

%% These commands are for a PROCEEDINGS abstract or paper.
\acmConference[miniKanren'24]{Make sure to enter the correct
  conference title from your rights confirmation emai}{September 02--07,
  2024}{Politecnico di Milano, Italy}
%%
%%  Uncomment \acmBooktitle if the title of the proceedings is different
%%  from ``Proceedings of ...''!
%%
%%\acmBooktitle{Woodstock '18: ACM Symposium on Neural Gaze Detection,
%%  June 03--05, 2018, Woodstock, NY}
%\acmISBN{978-1-4503-XXXX-X/18/06}
\acmISBN{XXXXXXXXXXXXXXXXXXXXXXX}


%%
%% Submission ID.
%% Use this when submitting an article to a sponsored event. You'll
%% receive a unique submission ID from the organizers
%% of the event, and this ID should be used as the parameter to this command.
%%\acmSubmissionID{123-A56-BU3}

%%
%% For managing citations, it is recommended to use bibliography
%% files in BibTeX format.
%%
%% You can then either use BibTeX with the ACM-Reference-Format style,
%% or BibLaTeX with the acmnumeric or acmauthoryear sytles, that include
%% support for advanced citation of software artefact from the
%% biblatex-software package, also separately available on CTAN.
%%
%% Look at the sample-*-biblatex.tex files for templates showcasing
%% the biblatex styles.
%%

%%
%% The majority of ACM publications use numbered citations and
%% references.  The command \citestyle{authoryear} switches to the
%% "author year" style.
%%
%% If you are preparing content for an event
%% sponsored by ACM SIGGRAPH, you must use the "author year" style of
%% citations and references.
%% Uncommenting
%% the next command will enable that style.
%%\citestyle{acmauthoryear}


%%
%% end of the preamble, start of the body of the document source.

%
%\usepackage[T1]{fontenc}

% The preceding line is only needed to identify funding in the first footnote. If that is unneeded, please comment it out.
%\usepackage{cite,easyReview}
\usepackage{amsmath}%,amssymb}
\usepackage{amsfonts}
\usepackage{algorithmic}
\usepackage{graphicx}
\usepackage{textcomp}
\usepackage{xcolor}
\usepackage{hyperref}
\usepackage{csquotes}
%\usepackage{lmodern}     % Fonts will be nicer 'sudo apt install lmodern'
%\def\BibTeX{{\rm B\kern-.05em{\sc i\kern-.025em b}\kern-.08em
%    T\kern-.1667em\lower.7ex\hbox{E}\kern-.125emX}}

% ======== Custom definitions =============
\usepackage[normalem]{ulem} % \sout
\usepackage{graphicx}
\usepackage{tikz}
\usetikzlibrary{shapes.geometric,backgrounds,fit,
  %positioning-plus,
  %node-families,
  calc}
\usepackage{xcolor}
\usepackage{mathtools}
%\usepackage[draft]{hyperref}
\usepackage{marginnote}
\usepackage{verbatim}
\usepackage{listings}
\usepackage{multirow}
\usepackage{array}
\usepackage{url}
\usepackage{caption}
\usepackage{subcaption}
\usepackage{xspace}
\usepackage{flushend} % Equalizes columns on the last page
\usepackage{placeins}
\usepackage{verbatimbox}
\usepackage[title]{appendix}

\newcommand{\term}[1]{\mbox{\texttt{\textbf{#1}}}}
\newcommand{\run}[2]{\term{run}^{#1}\,\left[#2\right]}
\newcommand{\todo}[1]{{\bf\color{red}#1}}
\newcommand{\rel}[3]{{#1}\xrightarrow{#2}{#3}}
\newcommand{\prg}[1]{\mbox{\lstinline|#1|}}
\renewcommand{\leq}{\ensuremath{\leqslant}}
\renewcommand{\geq}{\ensuremath{\geqslant}}

% !TEX TS-program = pdflatex
% !TeX spellcheck = en_US
% !TEX root = main.tex
\newcommand{\OCanren}{\textsc{OCanren}}
\newcommand{\OCaml}{\textsc{OCaml}}
\newcommand{\Scheme}{\textsc{Scheme}}
\newcommand{\Kotlin}{\textsc{Kotlin}}
\newcommand{\Klogic}{\textsc{Klogic}}

\lstdefinelanguage{guideline}{
keywords={},
sensitive=true,
commentstyle=\small\itshape\ttfamily,
keywordstyle=\ttfamily\underbar,
identifierstyle=\ttfamily,
basewidth={0.5em,0.5em},
columns=fixed,
fontadjust=true,
literate={->}{{$\to$}}3 {=>}{{$\Rightarrow$}}3,
morecomment=[s]{(*}{*)}
}

\lstdefinelanguage{ocanren}{
keywords={run, conde, fresh, let, in, match, with, when, class, type,
object, method, of, rec, repeat, until, while, not, do, done, as, val, inherit,
new, module, sig, deriving, datatype, struct, if, then, else, open, private, virtual, include, success, failure, switch,
true, false, ocanren},
sensitive=true,
commentstyle=\small\itshape\ttfamily,
keywordstyle=\ttfamily\textbf,
identifierstyle=\ttfamily,
basewidth={0.5em,0.5em},
columns=fixed,
mathescape=true,
fontadjust=true,
literate={fun}{{$\lambda$}}1 {->}{{$\to$}}3 {===}{{$\equiv$}}1 {=/=}{{$\not\equiv$}}1 {|>}{{$\triangleright$}}3 {\\/}{{$\vee$}}2 {/\\}{{$\wedge$}}2 {^}{{$\uparrow$}}1
%{[]}{{\texttt|[]|}}1
,
morecomment=[s]{(*}{*)}
}

\lstdefinelanguage{ocaml}{
keywords={type, struct},
sensitive=true,
commentstyle=\small\itshape\ttfamily,
keywordstyle=\ttfamily\textbf,
%keywordstyle=\ttfamily\underbar,
identifierstyle=\ttfamily,
basewidth={0.5em,0.5em},
columns=fixed,
fontadjust=true,
literate={->}{{$\to$}}3 {=>}{{$\Rightarrow$}}3,
morecomment=[s]{(*}{*)}
}

\lstdefinelanguage{algo}{
basicstyle=\small,
keywords={if, for, return},
sensitive=true,
commentstyle=\small\itshape\rmfamily,
keywordstyle=\rmfamily\textbf,
%keywordstyle=\ttfamily\underbar,
identifierstyle=\rmfamily,
basewidth={0.5em,0.5em},
columns=fixed,
fontadjust=true,
literate={->}{{$\to$}}3 {=>}{{$\Rightarrow$}}3,
morecomment=[l]{//},
morecomment=[s]{(*}{*)}
}

\lstset{
mathescape=true,
%basicstyle=\small,
identifierstyle=\ttfamily,
keywordstyle=\bfseries,
commentstyle=\scriptsize\rmfamily,
basewidth={0.5em,0.5em},
fontadjust=true,
language=guideline
}

%\setlength{\belowcaptionskip}{-10pt}

% =========================================

\sloppy
\begin{document}

\title{TODO}

%%
%% The "author" command and its associated commands are used to define
%% the authors and their affiliations.
%% Of note is the shared affiliation of the first two authors, and the
%% "authornote" and "authornotemark" commands
%% used to denote shared contribution to the research.

\author{Peter Lozov}
\email{lozov.peter@gmail.com}
\author{Dmitrii Kosarev}
\email{Dmitrii.Kosarev@pm.me}
\affiliation{%
  \city{St.~Petersburg}
  \country{Russia}}

%\author{Dmitrii Kosarev}
%\email{Dmitrii.Kosarev@pm.me}
%\author{Dmitry Boulytchev}
%\email{dboulytchev@math.spbu.ru}
%\affiliation{%
%  \institution{St.~Petersburg State University}
%  \city{St.~Petersburg}
%  \country{Russia}
%}

%%
%% By default, the full list of authors will be used in the page
%% headers. Often, this list is too long, and will overlap
%% other information printed in the page headers. This command allows
%% the author to define a more concise list
%% of authors' names for this purpose.
\renewcommand{\shortauthors}{Kosarev and Lozov}

\begin{abstract}
The myriad of various miniKanren implementation raised an interest in comparison between them, usually by expressive power and performance.
The straightforward solution to rewrite typical benchmarks for another host language
could be cumbersome: you need to have a familiarity with new host language.
The increasing size and number of benchmarks makes this task error prone: it's easy to occasionally make a spelling mistake when rewriting low-level constructions like unification. In this paper we discuss work-in-progress tool which translates programs written with miniKanren in OCaml (\OCanren{}) to \Scheme{} and miniKanren in \Kotlin{} (\Klogic{}).

\end{abstract}

%\begin{CCSXML}
%<ccs2012>
%   <concept>
%       <concept_id>10003752.10010070</concept_id>
%       <concept_desc>Theory of computation~Theory and algorithms for application domains</concept_desc>
%       <concept_significance>300</concept_significance>
%       </concept>
%   <concept>
%       <concept_id>10003752.10003790.10003795</concept_id>
%       <concept_desc>Theory of computation~Constraint and logic programming</concept_desc>
%       <concept_significance>500</concept_significance>
%       </concept>
%   <concept>
%       <concept_id>10002950.10003705.10003707</concept_id>
%       <concept_desc>Mathematics of computing~Solvers</concept_desc>
%       <concept_significance>500</concept_significance>
%       </concept>
%   <concept>
%       <concept_id>10003120.10003121.10003129.10011757</concept_id>
%       <concept_desc>Human-centered computing~User interface toolkits</concept_desc>
%       <concept_significance>500</concept_significance>
%       </concept>
% </ccs2012>
%\end{CCSXML}
%
%\ccsdesc[300]{Theory of computation~Theory and algorithms for application domains}
%\ccsdesc[500]{Theory of computation~Constraint and logic programming}
%\ccsdesc[500]{Mathematics of computing~Solvers}
%\ccsdesc[500]{Human-centered computing~User interface toolkits}
%
%\keywords{graphic user interface, relational programming, constraint solving}


\maketitle

%\begin{IEEEkeywords}
%component, formatting, style, GUI, constraint satisfaction, relational programming, synthesis
%\end{IEEEkeywords}

% !TEX TS-program = pdflatex
% !TeX spellcheck = en_US
% !TEX root = main.tex


\section{Introduction}
\label{sec:intro}


One of distinguishable features of \miniKanren{} is the fact that it is a family of languages:
many languages may host different \miniKanren{} implementations.
For example, \faster{}\footnote{\url{https://github.com/michaelballantyne/faster-minikanren} (access date: \DTMdate{2024-06-06})} for \Scheme{} and \Racket{}, \CoreLogic{} for \textsc{Closure}, \OCanren{}~\cite{OCanren} for \OCaml{}, \Klogic{}~\cite{Klogic2023} for \Kotlin{} and others.
The users of these DSLs may want to compare expressive power of various flavor of miniKanren, specifics due to host language, and performance implications of choosing a different host language.


The straightforward solution is to rewrite a number of significant benchmarks for many implementation, as it done for other languages\footnote{\url{https://benchmarksgame-team.pages.debian.net/benchmarksgame/index.html} (access date: \DTMdate{2024-06-06})}.
Doing it manually is time consuming and error prone.
Due to low-level nature of relational programs, it is easy to make spelling mistakes, for example accidentally write wrong identifier in unification arguments.
(We did many of mistakes of this kind while porting programs from \OCanren{} to \Klogic{}.) Moreover, \miniKanren{} doesn't pardon relational programs that solve the same task: it was reported, that the order of conjuncts significantly affects~\cite{scheduling2022} performance even if the search does the same unifications.

The differences between host languages also complicate porting relational code.
For example, \Kotlin{} doesn't support currying and partial applications comparatively to \OCaml{}, and sometimes full $\eta$-expansion is needed.
Also, porting from dynamically typed languages like \Scheme{} to statically typed ones like \OCaml{} could be uneasy for newcomers to statically typed languages.
This porting could be not straightforward:
basic data representations in \OCaml{}/\Klogic{} (algebraic data types and classes with subclasses~--- sum types) is different from \Scheme{} (lists, i.e. arbitrary length tuples~--- product types).
This fact in some cases requires special constraints~\cite{Wildcards2023} to level the expressivity, and in other cases (like relational interpreters) allows to get rid of \emph{absento/symbolo} constraints.

Things could get even more complicated where we want to port larger projects which are using functional/relational approach where relational parts are intermixed with straightforward programming.
The developer is obliged to know relational approach, the original host general purpose language and to have experience  with a new host language.

In this paper we describe current status of our converter from relational \OCanren{} to \Klogic{} and \miniKanren{} in \Scheme{}.
At the moment only relational subset of \OCanren{} is supported, we don't support whole \OCaml{} language.
In next section we describe technical aspects of our approach and currently supported features.
In section \ref{sec:interpreter} we discuss transformation in relational interpreter~\cite{Untagged} from \OCanren{} to \Scheme{} and peculiarities of autogenerated implementation.



% !TEX TS-program = pdflatex
% !TeX spellcheck = en_US
% !TEX root = main.tex

\section{Implementation}
\label{sec:implementation}

In this work we translate a subset of \OCanren{} (miniKanren in \OCaml{})  to \miniKanren in \Scheme{} and to \Klogic{} in \Kotlin{}.
\OCanren{} was picked as input implementation for various reasons, and the main is that we are the most familiar with \OCaml{} language and ecosystem.
Also we can observe an interest of porting \OCanren{} programs to JVM around us.
If we would use \Scheme{} implementation for fast prototyping, we expect more issue with automated translation of \Scheme{} to \Kotlin{} because of absence of types in relational programs in \Scheme{}.
There is an option to translate dynamically typed implementation to a weakly typed implementation in a typed language, but we believe that it is bad idea. (TODO Why?)

The translator takes typed syntax tree of \OCaml{} file, detects relational code by checking the types of declared values, reconstructs a representation for relational parts and prints them to \Kotlin{} and \Scheme{}.
All macro expansion are performed before typing phase.
As a result, typed relational AST in \OCaml{} uses low-level constructs, like \verb=call/fresh= and binary conjunction \verb=&&&=.
In this aspect, \OCanren{} implementation stands in between \Klogic{} and \miniKanren in \Scheme{}: the former doesn't use macro at all; the latter uses macro for everything: even \verb=conde= is expanded to \verb=mplus*=.
The task of translator is to collect the enriched AST with high-level relational constructs, reverse applications of macros, and print back to \Klogic{} and \Scheme{} a readable relational code, ideally without changes to a search order.

In some cases the typed representation of a relation is not enough, and we need to add user information that is skipped while type checking phase.
This is implemented using \OCaml{}'s attributes, for example \verb=[@@@name <OCaml expression>]=, placed in the top level of OCaml's modules.

\subsection{Kotlin-related features}

Code in \Kotlin{} and \Java{} usually requires adding many names to the namespace, usually using the \verb=import= keyword.
\Scheme{} is affected of this problem by a significantly lesser extent, because we can \verb=(include "filename")= to textually substitute relational implemenetation, where we require it.
Because of that we have special attributes \verb=[@@@klogic.preamble ...]= and \verb=[@@@klogic.epilogue ...]= that contain string literal with \Kotlin{} code that should be placed in the begging and in the end of generated file.

\subsection{Scheme-related features}
% !TEX TS-program = pdflatex
% !TeX spellcheck = en_US
% !TEX root = main.tex

\section{Showcase: Relational Interpreter}
\label{sec:interpreter}

Current largest application of the translator is a solver for \Java{} subtyping problems~\cite{JavaGenericsSolver2023} which has both \OCaml{} and (partially synthesized, halfway hand-written) \Kotlin{} implementation.
For \Scheme{} we tested relational arithmetic~\cite{Kiselyov2008PureDA} and interpreters~\cite{Untagged}.
The synthesized code for the former is almost identical to hand-written one, due to special treatment of lists in the translator.
On the contrary, relational interpreter is different in it's value representation.

\OCanren{} representation use algebraic data types to describe terms and result of interpretation.
This level of indirection (i.e. tagging) is necessary for \OCaml{}: you can't store either a closure with its contents or arbitrary value in a single data type.
After translation this tags appear in the synthesized relational interpreter (appendix~\ref{appendix:synQuines}) and play the role of second layer tagging (if we consider built-in \Scheme{} lists as a first layer).
And more tagging means higher unification cost and lower performance.

On the other hand explicit tagging (\texttt{val\_} and \verb=closure= for interpretation results, \verb=seq= and \verb=symb= for sequences (lists) and symbols) allow to avoid \emph{absento} constraint.
With tagging we always are able to distinguish closure returned from interpreter and application of function named ``closure'' to its arguments.
This feature was previously reported~\cite{OCanren}, but as far as we know nobody tried to benchmark\footnote{\url{https://github.com/fp2022-helper/miniKanren2024/blob/master/demo_scheme/bench/quines_bench.rkt}} the default \Scheme{} implementation and rewritten in \OCanren{} like manner.
It is difficult to say a priory which one performs better: the original has an extra constraint that does deep traversal of the values, the another one has more tagging.
The results (Fig.~\ref{fig:benchmarking1}) were collected  on the desktop machine Intel\copyright~Core\texttrademark~i7-4790K (16Gb RAM).
\Racket{} 8.7 was taken from the official Ubuntu 23.04 repository (compiled to native code using \Chez{} backend).
%\OCaml{} compiler 4.14.2+flambda was installed using \textsc{OPAM}\footnote{\url{https://opam.ocaml.org/}  (access date: \DTMdate{2024-06-02})} package manager.


\begin{figure}
\begin{tabular}{c|c|c|c }
Name &     Min &    Mean &     Max \\\hline
100 quines &   554&   577&   747\\
15 twines5 &   403&   426&   566\\
2 thrines &   723&   767&   939\\
100 quines (translated) &   291&   325&   364\\
15 twines (translated) &   267&   287&   330\\
2 thrines (translated) &   489&   527&   679\\
\end{tabular}

\caption{Benchmarking relational original relational interpreter (with \emph{absento}) and the one translated from \OCanren{}.
The table contains minimum, maximum and mean synthesis time (in ms, less is better).
The numbers show that on these tests the implementation without absento is faster.}
\label{fig:benchmarking1}
\end{figure}


%
%\begin{figure}
%\begin{lstlisting}[language=ocaml]
%fresh (t)
%    (term === seq (symb !!"quote" % (t % nil ())))
%    (r === val_ t)
%    (not_in_envo2 !!"quote" env)
%\end{lstlisting}
%\end{figure}
% !TEX TS-program = pdflatex
% !TeX spellcheck = en_US
% !TEX root = main.tex

\section{Discussion and Future Work}

We presented the results of our work on guideline-based synthesis of GUI layouts. The approach we take makes it possible to
automatically build GUI layouts which by construction comply with a designer-specified set of rules. The prototype implementation
we developed allows to synthesize the layouts for the real-word industrial GUI components w.r.t. the real-world guidelines in
appropriate time.

One interesting question which may arise is if the application of relational programming is essential for this problem to be solved. Indeed, the set of guidelines
describes a matching (or rewriting) system which, in principle, can be directly implemented without any use of relational techniques. We argue, however,
that in this case a whole piece of work on justification of the correctness of the solution would have been repeated anew. In our
case, the justification trivially follows from the completeness of the \textsc{miniKanren} search and refutational completeness of our solution. We also speculate
that such a solution would require reinventing of some implementation techniques to support nondeterminism and backtracking, which are already native to relational
programming. Finally, the duality between patterns over structure and relational goals (initially unexpected for us), to our opinion, witnesses, that relational
programming is a truly relevant technique for this problem. %, and we are not stretching an owl over a globe.

%We can also consider the task of getting rid of extra solver (\textsc{Z3}) as relevant; this would not only simplify the infrastructure of the system, but also allow to
%integrate the constraint resolution phase into the constraint synthesis, improving the performance of the whole system.
%, and finally provide the completeness
%of the synthesizer, which we strictly speaking do not yet have. To do this, we plan to develop a relational integer inequality solver with a binary representation
%of fixed-size numbers. Note that for satisfactory performance it will also be necessary to extend \OCanren{} with inequality operations for such numbers in the
%form of a new type of constraint.



\bibliographystyle{ACM-Reference-Format}
\bibliography{main}

% !TEX TS-program = pdflatex
% !TeX spellcheck = en_US
% !TEX root = main.tex
\appendix
\section{Relational Interpreter for Quines, Translated from OCanren}
\label{appendix:synQuines}

An implementation of relational interpreter, translated from \OCanren{}. Extra tagging is represented with \textbf{\underbar{underlined bold}} style.
\todo{TODO Link}

\lstinputlisting[language=miniKanren, numbers=left, ndkeywords={symb,seq,val}]{interpreter_syn.scm}

\section{Default Quines Implementation}
\label{appendix:defaultQuines}
\todo{TODO Link}

\lstinputlisting[language=miniKanren,numbers=left]{default_quines.scm}
%\end{subappendices}
%\appendix

\end{document}
