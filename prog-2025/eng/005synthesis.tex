% !TEX TS-program = pdflatex
% !TeX spellcheck = en_US
% !TEX root = main.tex

In this section we describe the construction of guideline-driven layout synthesizer. We do it in several steps starting
from the implementation of the simple functional verifier and performing a few optimizations.

The construction of initial naive verifier can be done
by straightforward encoding of the coverage, compatibility and confirmation properties (Section~\ref{sec:guidelines}) as a relation between structure and layout primitives.
% is rather straightforward.
%We use conventional functional data structures like lists and algebraic data types to encode structures and layout primitives.
%Given guideline rules we
%\emph{generate} a verifier which takes as arguments a structure and a set of layout instances and performs a routine check of
%the coverage, compatibility and confirmation properties following the definitions given in Section~\ref{sec:guidelines}.
The coverage can be assessed by traversing all layout primitives, collecting all non-virtual controls and
checking that the set of all collected controls coincides with the set of all non-virtual controls in the structure.
Similarly, the compatibility is checked according to the definition.

The confirmation procedure traverses all layout primitives and tries to confirm each of them. This requires each
rule of the guidelines to be \emph{inverted}. For example, consider the following guideline rule:

\begin{lstlisting}[language=ocanren,basicstyle=\small]
   $\term{describes}\,(X,\,Y)\,\mapsto\,\term{vert}\,(X,\,Y),\,\term{halign}\,(X,\,Y)$
\end{lstlisting}

\noindent It determines the following (sub)cases for the confirmation check procedure for a structure $\Sigma$ and
a set of layout instances $S$:

\begin{lstlisting}[language=ocanren,basicstyle=\small]
   if $\term{vert}\,(X,\,Y)\in S$
   then $\term{describes}\,(X,\,Y)\in\Sigma$ /\ $\,\term{halign}\,(X,\,Y)\in S$
   else if $\term{halign}\,(X,\,Y)\in S$
   then $\term{describes}\,(X,\,Y)\in\Sigma$ /\ $\,\term{vert}\,(X,\,Y)\in S$
\end{lstlisting}

We do not perform \emph{maximality} check here because it will diminish value of relational programming.
%The reason is specific to our approach of using functional-relational programming.
%Indeed, to perform maximality check we would need to
%find other compatible sets of instances which are confirmed and covering. But this is exactly the
%task which we are trying \emph{avoid} to perform by writing functional verifier.
%Thus, implementing it in a functional language would compromise the whole idea of using relational
%programming. We emphasize that this is rather a common case for the approach we advocate: in order
%to be efficient we need to carefully set the border between functional and relational programming
%in order to avoid doing explicitly what we intend to have later for free.
%
The maximality problem is solved in another manner.
We run verifier for a given concrete structure and a \emph{free variable} in
the position of the set of layout instances.
This gives us all sets which are compatible, covering and confirmed at
the same time by the completeness of \textsc{miniKanren} search procedure.
Then, having them at our hands already, we filter out non-maximal ones.

The first step gave us a complete, but not efficient synthesizer to play with.
We can identified two issues with this na\"ive implementation.

First, the efficiency on synthesizer depends on data structure that represents layout primitives.
Na\"ive implementation based on lists will generate many equivalent answers, which will be permutations of each another.
We devised a specific representation for the layouts (\emph{binary hypercube}) --- a data structure that strongly reminds a bitmap --- where all answers have unique representation.
For our task it will have finite size, because all elements in the structure are known beforehand.
%which any layout can be
%represented uniquely. This representation in essence is a bitscale representation of sets and relations, and it has to
%be specifically defined for each structure.

%\begin{itemize}
%\item The synthesizer produced an (exponentially) large collection of equivalent answers. Indeed, as
%  we initially used list representation for sets of layout instances, the synthesizer
%  considered all permutations of lists comprised of the same controls as different answers.
%\item We realized that encoding of structures using functional data structures was
%  excessive as they could be directly generated as \textsc{miniKanren} relations.
%\item The synthesizer generated a lot of partial (non-maximal) layouts which slowed
%  done the synthesis and made it impractical.
%\end{itemize}
%
%In the following subsections we address these issues.
%
%\subsection{Binary hypercube representation}
%
%For the first problem we devised a specific representation for the layouts (\emph{binary hypercube}) in which any layout can be
%represented uniquely. This representation in essence is a bitscale representation of sets and relations, and it has to
%be specifically defined for each structure. For example, let us have two GUI controls \lstinline|a| and \lstinline|b| and, for
%simplicity, only two layout primitives $\term{vert}$ and $\term{hor}$. Then the following \textsc{OCaml} definitions introduce
%the representation for layouts:
%
%\begin{lstlisting}[basicstyle=\small,language=ocaml]
%   type rels    =$\,$ {vert : bool; hor : bool}
%   type $\alpha$ roles = {a : $\alpha$; b : $\alpha$}
%   type layout  = rels roles roles
%\end{lstlisting}
%
%This approach eliminates an extra overhead of the search for equivalent answers with different representations. However,
%it requires the specialization of functional verifier for a concrete structure. This is not a great price; in addition, the
%technique described in the next subsection requires similar specialization to be performed.
%
%\subsection{Representing structure by relational definitions}

%As we know, a structure is just a set of relations over the set of controls. In a relational language we can directly
%represent relations as relations. Assume we have a (fragment) of structure shown in Fig.~\ref{rel_rel}\subref{rel_abs}.
%In original implementation it was represented as a data structure using dedicated types for representing
%controls and relations. However, the same elements of the structure can be directly encoded in
%\textsc{OCanren} as it shown in Fig.~\ref{rel_rel}\subref{rel_con} (the ellipses indicate that there can be other
%parts of these definitions corresponding to other controls bound with said relations).
%
%\begin{figure}[t]
%  \begin{subfigure}[t]{0.5\textwidth}
%    \begin{lstlisting}[basicstyle=\small,language=ocanren]
%  $\term{type}\,(A,\, \term{checkbox})$
%  $\term{type}\,(B,\, \term{label})$
%  $\term{describes}\,(B,\, A)$
%    \end{lstlisting}
%    %\vskip16mm
%    \caption{Relations in abstract form}
%    \label{rel_abs}
%  \end{subfigure}
%  \begin{subfigure}[t]{0.5\textwidth}
%    \begin{lstlisting}[basicstyle=\small,language=ocanren]
%  let $\mbox{\texttt{type}}^o$ x y = ocanren {
%    y === Checkbox /\ x === A \/
%    y === Label    /\ x === B \/ ...
%  }
%  let $\mbox{\texttt{describes}}^o$ x y = ocanren {
%    x === B /\ y === A \/ ...
%  }
%    \end{lstlisting}
%    \caption{Encoding in \textsc{OCanren}}
%    \label{rel_con}
%  \end{subfigure}
%  \caption{A relational representation of relations}
%  \label{rel_rel}
%\end{figure}
%
%After the conversion, a pattern to match over a structure can be turned into a conventional relational goal. This conversion
%completely eliminates the interpretation overhead for matching over structures. The drawback is that we need to regenerate a
%part of the system for each change of the structure; this is rather a natural trade-off when specialization techniques are used.
%
%\subsection{Greedy algorithm with conflict resolution}

Second, our initial solution generate all compatible sets of instances.
However their number can be huge even if only confirmed and covered sets are considered.
In order to make our synthesizer applicable we have to eliminate this overhead.
We do this using greedy approach: instead of enumerating all compatible layouts and
filtering out non-maximal ones we first build one maximal but not compatible layout, then identify all conflicts and,
finally, eliminate the conflicts by \emph{nondeterministically} canceling minimal numbers of rules which introduce these conflicts.
This approach gives us a set of all maximal compatible layouts.
This optimization works because a number of conflicting primitives is lesser than non-conflicting ones, and it's easier to remove them in the end.

All three steps of construction of maximal non-conflicting layout is implemented as an independent relational component.

On the first step we non-deterministically apply all guideline rules and combine them into the maximal
(but probably non-compatible) set of layout instances represented as a hypercube.

For the second step we use a specific relational verifier to find all conflicts.
For each kind of conflict we implement
a dedicated verifier which, given a hypercube and a set of layout instances, checks that

\begin{itemize}
\item the set, indeed, occurs in the hypercube;
\item the set constitutes the conflict in question.
\end{itemize}

\noindent Being run in the reverse direction for a ground hypercube and free variable this verifier returns all
sets of layout instances which occur in the hypercube and constitute a conflict of given kind.
Running a disjunction of all such verifiers in the reverse direction gives us the set of all sets of conflicting layout instances.

On the third step we take each set of conflicting layout instances and resolve all conflicts which gives us (for each set) the set of maximal compatible layouts.
We do this in the following manner.
Let $S$ be a set of conflicting instances.
The property of all identifiable conflicts is that each of them can be resolved by removing just one layout instance.
So, we nondeterministically one-by-one remove layout instances from $S$ and see what conflicts disappear due to this removal.
Note, the removal of just one instance can resolve multiple conflicts. When no conflicts remain we return the set as a maximal compatible layout.
There is, however, a subtlety which makes the implementation trickier.
By simply removing layout instances we may arrive at a layout which is not confirmed by a guideline since some layout instances can appear (and,
conversely, disappear) only simultaneously.
So we need to implement a certain ``canceling'' procedure to correctly
undo guideline rules application.
The whole step is, again, implemented relationally due to extensive use of
nondeterminism.

Note that this solution uses three independent relational programs which thread the answers from one to another.
Moreover, it is impossible to combine them into a single relational program since each step requires all the answers
of the previous one to be found.

\He{Absolute coordinates calculation}

In order to give a layout its final form all coordinates have to be calculated for all controls.
When a sound set of layout instances is calculated this task reduces to solving a system of \emph{integer linear constraints}.
Each layout primitive contributes a few inequalities to this system.
For example, a primitive $\term{hor}\,(C_1,\,C_2)$ (horizontal placement of controls $C_1$ and $C_2$ next to each other) contributes the following constraints
\begin{gather*}
  C_2\prg{.x} - C_1\prg{.x} \leq i_H + C_1\prg{.width}\\
  C_2\prg{.y} - C_1\prg{.y} = a_V
\end{gather*}

\noindent where \emph{variables} $C_i\prg{.x}$ and $C_i\prg{.y}$ designate $x$ and $y$ coordinates of given control respectively, integer
\emph{constants} $C_1\prg{.width}$~--- the width of $C_1$, $i_H$~--- horizontal inset, $a_V$~--- vertical alignment.

Also, for each virtual GUI control $C$ containing GUI controls $C_1, \ldots, C_n$ the following constraints are needed that define
its actual coordinates, width and height:
  \begin{gather*}
    C\prg{.x} = \min\,\{C_1\prg{.x}, \ldots, C_n\prg{.x}\}\\
    C\prg{.y} = \min\,\{C_1\prg{.y}, \ldots, C_n\prg{.y}\}\\
    C\prg{.x} + C\prg{.width} = \max\,\{C_i\prg{.x} + C_i\prg{.width}\}_{i=1}^n\\
    C\prg{.y} + C\prg{.height} = \max\,\{C_i\prg{.y} + C_i\prg{.height}\}_{i=1}^n\\
  \end{gather*}
Additionally, a number of inequality constraints is added which restrict the maximal possible values for
coordinates taking into account the size of enclosing panel.

There exists a number of ways to solve the set of integer linear inequalities (for example, using SMT solvers over the linear
integer arithmetic theory). It is, however, very appealing to try to employ relational verifiers yet again.
This relational solver would have the following benefit: it could be seamlessly integrated into the layout synthesis procedure, thus allowing some
constraint selection to be rejected at earlier stages.

However, the current relational implementation of this solver is underperforming in the presence of virtual GUI controls.
As mentioned above, the width and height of virtual GUI controls are not constants, which greatly increases the
search space. So, for the time being, we, indeed, use \textsc{Z3} theorem prover~\cite{Zthree} to determine the
absolute coordinates of the GUI controls (as well as the width and height of all virtual GUI controls).

This approach, however, can potentially compromise the completeness of the overall solver. Indeed, the conflict
detection and elimination procedures described in the previous section do not take into account the size of the
enclosing panel and the sizes of virtual controls (which solely depends on the manner their nested controls are
layed out). Thus, the system of integer inequalities can be inconsistent even for sound layouts. However,
by cancelling application of some guideline rules some of the integer constraints can be removed thus restoring
the consistency of the system. In other words, by ignoring size constraints we can ponentially overlook some
proper layouts. Note, the whole idea of conflict identification comes from the observation that we know
in advance that some sets of layout instances will inevitably introduce inconsistent constraints on
some controls' coordinates; however at the conflict identification and elimination stage we do not have
enough information to do this job in utterly precise manner.

\begin{figure}
\begin{lstlisting}[language=algo,mathescape=true,basicstyle=\ttfamily]
solve(layPart, sizePart) {
  // Try to solve the full system
  if (Z3.solve (layPart $\cup$ sizePart)) {
    // Return a model
    return [ Z3.model ]
  }

  // If unsat we need to find and solve
  // layout conflict

  // MUC is one of conflicts
  muc = Z3.minimalUnsatCore

  // A list of refined models
  answers = []

  for (mucElement $\in$ muc) {
    // To solve the conflict we need
    // to remove any layout element
    // of MUC from the system
    if (mucElement $\in$ layPart) {
      reducedLayPart =
        layPart $\setminus$ {mucElement}
      answers ++= solve(
        reducedLayPart, sizePart)
    }
  }

  return answers
}
\end{lstlisting}
\caption{Solving coordinate conflicts}
\label{Z3op}
\end{figure}

Restoring the completeness requires more accurate interoperation with \textsc{Z3}.
When the inconsistency of the system $S$ is discovered, a \emph{minimal unsatisfiable core}~\cite{minUnsatCore} is returned.
Minimal unsatisfiable core is an inconsistent subset $UC\subseteq S$ such that removing any constraint from $UC$ makes the remaining set consistent.
The first idea would be to take one constraint from $UC$ (nondeterministically) and see what layout instance contributed it.
By canceling guideline rule application which introduced this
layout instance we remove this constraint thus making the whole system consistent.
Alas, not all constraints can be treated this way.
We split all the constraints of $UC$ into \emph{layout} constraints $L$ (those
introduced by layout instances) and \emph{size}-constraints $Z$ (the inequalities for the sizes of virtual controls and
inequalities for control coordinates introduced to respect the size of the enclosing pane).
As we noticed earlier,
all constrains in $Z$ are determined by layout instances which contributed to $L$.
We can explicitly cancel only constraints from $L$; alas, they alone do not constitute minimal unsatisfiable core, so we have to act more accurately.

We build an auxiliary set $M$ by taking the constraints $c$ from $L$ one by one and asking \textsc{Z3} to solve the system $S\setminus\{c\}$ on each step.
There can be two outcomes:

\begin{enumerate}
\item $S\setminus\{c\}$ remains inconsistent. This means that $c$ does not contribute to the inconsistency, so
  we continue.
\item $S\setminus\{c\}$ becomes consistent. This means that $c$, indeed, in a conflict with some other
  constraint. We add $c$ to $M$ and continue.
\end{enumerate}

By the end of this procedure $M$ will by construction contain all constraints such that removing any of them potentially
restores the consistency. Each of them was contributed by some layout instances. We nondeterministically
cancel the application of guideline rules which introduced these instances which gives us $n$ subsystems of $S$ where $n=|M|$.
Now we try to solve these subsystems with \textsc{Z3} and for each inconsistent one repeat the whole procedure. In the
end we either cancel enough rules to enumerate all sound layouts or arrive at the state when the system of
constraints remain inconsistent with the minimal unsatisfiable core containing only size constraints, which means that
there are some controls which do not fit into the enclosing pane regardless their layout. This justifies the completeness
of our synthesis algorithm. The pseudocode for the whole procedure is shown in Fig.~\ref{Z3op}.

\begin{comment}
Как было сказано в секции \ref{sec:synthesizing:conflicts} до вычисления фактических размеров и координат контролов, обнаружить и решить все конфликты невозможно.
Поэтому Z3 солвер, может как решить данную ему систему (если эта система не содержит конфликтов), так и определить, что данная система несовместна.
Для гарантии полноты в этом случае нам необходимо выявить и разрешить кофликты, содержащиеся в системе.

В случае несовместной системы кофликтом является подсистема, содержащая минимальное количество констрейнтов, полученных из примитивов лайаута.
Действительно, из такой подсистемы мы можем однозначно восстановить конфликт --- минимальный набор несовместных примитивов лайаута.

Таким образом, исходную систему можно разделить на две части: \textbf{layout constraints} --- констрейнты, полученные из примитивов лайаута, и \textbf{size constraints} --- констрейнты размеры контролов.
Нам необходимо минимизировать количество layout constraints в контексте всех size constraints.

Стандартные методы Z3 солвера по получению UNSAT ядра (несовместной подсистемы исходной системы) и минимального UNSAT ядра~\cite{minUnsatCore} (несовместной подсистемы, любая подсистема которой является совместной) не позволяют выявить кофликт, так как минимизация происходит по всей системе, а не по нужной нам части системы.

На основе алгоритма минимизации UNSAT ядра мы разработали алгоритм минимизации одной части системы в контексте другой части.
\begin{lstlisting}[basicstyle=\small]
minimize(lay_part, size_part) {
    i = 0
    while (i < length(lay_part)) {
        subsystem = concat(
          lay_part[0 .. i],
          lay_part[i + 1 .. length(lay_part)])
        to_test = concat(subsystem, size_part)

        if Z3.solve(to_test) then
            i = i + 1
        else
            lay_part = to_test
    }

    return lay_part
}
\end{lstlisting}

Данный алгоритм за линейное количество запросов к Z3 солверу позволяет минимизировать layout constraints.
Далее для каждого констрейнта $C$ из layout constraints мы создаем новый лайаут, в котором отменяем все инстансы, подтверждающие констрейнт $C$. Для каждого из лайаутов мы снова пытаемся вычислить размеры и координаты.

Так исходный лайаут мог содержать несколько конфликтов, данная процедура повторяется, пока мы не вычислим кофликты. Ситуация, в которой мы удалили все layout constraints, но система по-прежнему несовместна, возможно только в случае, когда какой-либо контрол не помещается на enclosing
panel. Следовательно, разместить контролы исходного лайаута на enclosing
panel невозможно.
\end{comment}

\begin{comment}
%=========
\subsection{Completeness}
\label{sec:synthesizing:completeness}
Completeness of a synthesizer (the capability of producing all sound sets of layout instances) is a very desirable property from
practical standpoint since it allows to explicitly enumerate all possible layouts to choose from.

%=========

\subsection{Completeness}

Completeness of a synthesizer (the capability of producing all sound sets of layout instances) is a very desirable property from
practical standpoint since it allows to explicitly enumerate all possible layouts to choose from. Unfortunately, generally
speaking out current synthesizer is not complete: there are some rare cases when integer linear constraints for
compatible layouts can not be solved (for example, when the layout does not fit into the pane). This alone would not compromise
the completeness. What is worse, sometimes such layouts can be reduced even more to provide a solvable set of integer linear constraints.
This is the single case when we can miss a layout.
\end{comment}

\begin{comment}
The synthesizer we described is complete under assumption that

This solution turns out to be many times efficient that the previous ones. However, it is still incomplete because there are conflicts that cannot
be found before the coordinates are computed. In particular, the set of primitives

\[
\term{hor}\,(X,\,Y_1), \term{vert}\,(Y_1,\,Z), \term{vert}\,(X,\,Y_2), \,  \term{hor}\,(Y_2,\,Z)
\]

requires the knowledge of the actual sizes of the GUI controls to determine if there is a conflict. In this case, the sizes of virtual GUI controls
are not known in advance. At the moment, we have developed a rich enough conflict verifier to calculate conflict-free layouts using practical
examples, but this task is still being worked on.

To summarize, we implement the layout synthesizer in the following way:

\begin{itemize}
\item the synthesizer is specialized for a given set of guidelines and a given structure;
\item the structure is converted into a set of relational definitions;
\item a corresponding ``binary hypercube'' definition for layout representation is
  generated for the structure;
\item instead of searching for all layouts, our solution first builds one maximum conflict layout, and then finds conflicts and solves them non-deterministically;
\item the set of guideline rules is ``inverted'', and all matching on the
  structure is turned into appropriate goals using the definitions
  generated for this structure.
\end{itemize}

This, coupled with the usual coverage check, delivers us the layout synthesizer.
\end{comment}


\begin{comment}
The first task in guideline-based layout synthesis is generation of layout primitives for a given structure which respect
the guidelines in question. Note, for given guidelines and structure there can be potentially multiple
admissible layouts. We solve this problem by utilizing \emph{relational verifier}.

A verifier is a procedure which for a given guidelines $\mathcal G$, GUI structure $\mathcal S$ and a set of layout primitives $\mathcal L$
checks if this layout for this structure is admissible w.r.t. these guidelines:

\[
\term{verify}\,(\mathcal{G},\,\mathcal{S},\,\mathcal{L})=\left\{\begin{array}{rcl}
                                                                  \term{true}&,&\mbox{$\mathcal L$ respects $\mathcal G$}\\
                                                                  \term{false}&,&\mbox{otherwise}
                                                                \end{array}
                                                         \right.
\]


\noindent This verifier, being converted into a relational form $\term{verify}^o$, can be used as a layout \emph{synthesizer}~\cite{lozov2019relational}:

\[
\term{synth}\,(\mathcal{G},\,\mathcal{S})=\run{\alpha}{\term{verify}^o\,(\mathcal{G},\,\mathcal{S},\,\alpha,\,\term{true})}
\]

Here $\run{\alpha}{g}$ is a conventional \textsc{miniKanren} primitive for searching for all values of a variable $\alpha$ which
make the goal $g$ to succeed; it returns a (potentially infinite) stream of these values. Thus, the problem of layout constraints
synthesis can be reduced to the problem of (relational) verifier construction.

In our approach we provide a specialized version $\term{verify}^o_{\mathcal G}$ for every set of guidelines of interest. The motivation
is very simple: first, as a rule the main use case for us is the synthesis of layouts for \emph{multiple} structures with regard to a \emph{single}
fixed set of guidelines. Then, as we will see in a little while, providing a specialized version is simpler, than the generic one, and
this version runs faster.

First we describe the construction of a non-relational, \emph{functional} verifier. The guidelines description system matches the controls of the
structure to a set of layout primitives. Thus, in order to verify that some layout respects the guidelines we have to \emph{justify} each
control of the layout by some guideline rules. More concretely, we identify two important notions:

\begin{itemize}
\item \emph{Coverage}. We say, that layout \emph{covers} a system if for any non-virtual control there is a layout primitive
  addressing this control. Coverage informally witnesses that no ``real'' control of the structure is left out
  unattended by the layout. The lack of coverage means that there are some controls in the structure for
  which no placement constraints were derived according to the guidelines. We consider such layouts ill-formed.

\item \emph{Confirmation}. We say that a layout primitive occurrence $p$ is \emph{confirmed} by guidelines, if,
  first, there is an applicable rule in the guidelines which derives $p$ and, second, all other layout
  primitives, derived by this rule, occur in the layout.
\end{itemize}

The coverage can be easily assessed by traversing all layout primitives, collecting all non-virtual controls and
checking that set of all collected controls coincides with the set of all non-virtual controls in the structure.

The confirmation procedure traverses all layout primitives and tries to confirm each of them. This requires each
rule of the guidelines to be \emph{inverted}. For example, consider the following rule from JetBrains guidelines set:

\begin{lstlisting}[basicstyle=\small]
  $\rel{X}{descr}{Y}$, type ($Y$, $T$) [width ($Y$) > $K\,\wedge$
                           $T\ne$ checkbox]
     => vert ($X$, $Y$), halign ($X$, $Y$);
\end{lstlisting}

\noindent  It determines the following (sub)cases for the confirmation procedure \lstinline|confirm$_\mathcal G$|:

\begin{lstlisting}
  confirm$_\mathcal{G}$ ($\mathcal S$, $\mathcal L$) =
    vert   ($X$, $Y$) $\in\mathcal L$ ->
      ...
      $\rel{X}{descr}{Y}\in\mathcal{S}\;\wedge$
      type ($Y$, $T$)$\in\mathcal S\;\wedge$
      width ($Y$) > $K\;\wedge$
      $T\ne$ checkbox$\;\wedge$
      halign ($X$, $Y$)$\,\in\mathcal L$
      ...
    halign ($X$, $Y$) $\,\in\mathcal L$ ->
      ...
      $\rel{X}{descr}{Y}\in\mathcal{S}\;\wedge$
      type ($Y$, $T$)$\in\mathcal S\;\wedge$
      width ($Y$) > $K\;\wedge$
      $T\ne$ checkbox$\;\wedge$
      vert ($X$, $Y$)$\,\in\mathcal L$
      ...
\end{lstlisting}

Thus, building functional verifier is rather a simple routine procedure.

There are multiple ways to convert this verifier into relational form. One way is to use a relational interpreter
for the functional language the verifier is implemented in~\cite{Untagged}; another
one is to apply \emph{relational conversion}~\cite{lozov2017typed} which syntactically
transforms functional programs into relational ones.

We tried the latter and, indeed, acquired a relational layout synthesizer with a reasonable behavior. There were, however,
three problems:

\begin{enumerate}
\item Functional verifier used lists of layout primitives to represent layouts. Thus, there were multiple answers which, in
  fact, represented the same layouts due to permutations or repetitions of the controls of lists.
\item Incomplete answers: it turned our that often even a small number of layout primitives is enough to cover the structure and
  be confirmed by the guidelines. While according to our criteria this is not an invalid answer, it somewhat contradicts the
  informal semantics of the guidelines as it is expected that, on the contrary, as much rules as possible have to be used.
  On the other hand, too rich layouts can contain conflicting primitives, making it impossible to determine the coordinates of GUI controls.
\item Performance: even for a simple problem the synthesis took tens of seconds to complete.
\end{enumerate}

For the first problem we devised a specific representation for the layouts (``binary hypercube'') in which any layout can be
represented uniquely. This representation in essence is a bitscale representation of sets and relations, and it has to
be specifically defined for each structure. For example, let us have two GUI controls \lstinline|a| and \lstinline|b| and, for
simplicity, only two layout primitives \lstinline|vert| and \lstinline|hor|. Then the following definitions introduce
the representation for layouts:\footnote{\textcolor{red}{NOTE(Kakadu):} let's remove \textbf{struct} keyword to make the code be real OCaml}

\begin{lstlisting}[basicstyle=\small,language=ocaml]
  type rels    = struct { vert : bool; hor : bool}
  type $\alpha$ roles = struct { a : $\alpha$; b : $\alpha$}
  type layout  = rels roles roles
\end{lstlisting}

This approach eliminates the extra overhead of the search for equivalent answers with different representations.

The last two problems actually closely related. The relational verifier allows us to find all possible combinations of the guideline rules
applications, among which there are both incomplete (we can apply more rules while maintaining the correctness of the layout) and redundant
(some application of the rules introduce conflicting primitives).

The problem of incomplete answers can be solved using the following simple observation: as each guideline rule can only
\emph{add} some layout primitives, it is sufficient to filter out the answers which are \emph{subsumed} by other answers.
Thus, first we call the synthesizer for all answers (this is feasible as, due to the finiteness of the search space, the
synthesis is refutationally complete), and then in a post-processing throw away all incomplete answers.

However, firstly, this will not improve performance, and secondly, this way we can lose correct layouts subsumed in redundant ones.

Therefore, instead of such a solution, we divided the search for layouts into several steps.
\begin{itemize}
    \item Finding a single maximum layout containing all applications of the guideline rules.
    \item Detects all conflicts in a this layout.
    \item Non-deterministic solving of these conflicts to obtain all maximum non-conflict layouts.
\end{itemize}
Each of these steps is an independent relational solution.
In the first step, we non-deterministically get all applications of the guideline rules, which we then combine into the maximum
(may be conflicted) layout. In the second step, we use a relational verifier to find all conflicts. We have implemented a verifier
of various types of conflicts, which, when executed in the opposite direction, allows us to find all these conflicts in the layout.

For example, consider the conflict of cyclic primitives. If the layout contains a set of primitives of the form

\[
\term{prim}\,(X,\, X_1), \, \term{prim}\,(X_1,\, X_2), \, \ldots, \, \term{prim}\,(X_n,\, X),
\]

where $\term{prim}$ is any kind of layout primitives except horizontal alignment.

This set of primitives does not allow us to determine the coordinates of the control X, since all primitives require that the right
control is to the right and/or below the left. Thus control $X$ must be to the right and/or below itself.

This (and other) conflict can be solved quite simply: it is enough to remove any primitive from this set. And, if it is non-deterministic
to remove any of the primitives, we will get all the maximum layouts in which the this conflict is solved.

This is exactly what the third step is responsible for, which sequentially solves each of the conflicts, thereby obtaining all the
maximum non-conflict layouts.

Note that this solution uses three independent relational programs, the answers of each of which are combined into the input data for
the next one. Moreover, it is impossible to combine them into a single relational program, since each step requires all the answers
of the previous one.

This solution turns out to be many times higher performance. However, it is still incomplete because there are conflicts that cannot
be found before the coordinates are computed. In particular, the set of primitives

\[
\term{hor}\,(X,\,Y_1), \term{vert}\,(Y_1,\,Z), \term{vert}\,(X,\,Y_2), \,  \term{hor}\,(Y_2,\,Z)
\]

requires knowledge of the actual sizes of the GUI controls to determine if there is a conflict. In this case, the sizes of virtual UI controls
are not known in advance. At the moment, we have developed a rich enough conflict verifier to calculate conflict-free layouts using practical
examples, but this task is still being worked on.

To summarize, we implement the layout synthesizer in the following way:

\begin{itemize}
\item the synthesizer is specialized for a given set of guidelines and a given structure;
\item the structure is converted into a set of relational definitions;
\item a corresponding ``binary hypercube'' definition for layout representation is
  generated for the structure;
\item instead of searching for all layouts, our solution first builds one maximum conflict layout, and then finds conflicts and solves them non-deterministically;
\item the set of guideline rules is ``inverted'', and all matching on the
  structure is turned into appropriate goals using the definitions
  generated for this structure.
\end{itemize}

This, coupled with the usual coverage check, delivers us the layout synthesizer.

\section{Solving Layout Constraints}

Layout synthesizer for a given structure produces a set of layout primitives. This set in fact specifies a number of
\emph{integer linear constraints} which have to be solved in order to obtain the final, absolute coordinates of the
controls. Indeed, let us introduce the notations for integer constants and \emph{variables} which values have to be defined as
  a result of constraint solving:

\begin{itemize}
\item $C\prg{.width}$,  $C\prg{.height}$~--- the width and height of UI control $C$ (integer constant for real UI controls and a \emph{variable} for virtual UI controls);

%\item $C\prg{.width}$, ~--- the width of UI control $C$ (integer constant for real UI controls and a \emph{variable} for virtual UI controls);
%\item $C\prg{.height}$~--- the height of UI control $C$ (integer constant for real UI controls and a \emph{variable} for virtual UI controls);
\item $C\prg{.x}$,  $C\prg{.y}$~--- the $X$- and $Y$-coordinates of UI control $C$ (a \emph{variable} for each UI control);
%\item $C\prg{.x}$~--- the $X$-coordinate of UI control $C$ (a \emph{variable} for each UI control);
%\item $C\prg{.y}$~--- the $Y$-coordinate of UI control $C$ (a \emph{variable} for each UI control);
\item $a_H$, $a_V$~--- the horizontal/vertical alignments between two UI controls (integer constant);
\item $i_H$, $i_V$~--- the vertical/horizontal insets between two UI controls (integer constants);
\item $ind$~--- indent between two UI controls (integer constant);
\item $W$, $H$~--- the width and the height of enclosing panel (integer constants);
%\item $W$~--- the width of enclosing panel (integer constant);
%\item $H$~--- the height of enclosing panel (integer constant).
\end{itemize}

The set of layout primitives defines a set of integer constraints as follows.

\begin{itemize}
\item A primitive $\term{hor}\,(C_1,\,C_2)$ introduces constraints

\vspace{-1em}
  \begin{gather*}
    C_2\prg{.x} - C_1\prg{.x} \leq i_H + C_1{.width}\\
    C_2\prg{.y} - C_1\prg{.y} = a_V
  \end{gather*}

\item A primitive $\term{vert}\,(C_1,\,C_2)$ introduces constraints

\vspace{-1em}
  \begin{gather*}
    C_2\prg{.x} - C_1\prg{.x} = a_H\\
    C_2\prg{.y} - C_1\prg{.y} \leq i_V + C_1\prg{.height}
  \end{gather*}

\item A primitive $\term{halign}\,(C_1,\,C_2)$ introduces constraint
  \[
  C_1\prg{.x} - C_2\prg{.x} = a_H
  \]

\item A primitive  $\term{valign}\,(C_1,\,C_2)$ introduces constraint
  \[
  C_1\prg{.y} - C_2\prg{.y} = a_V
  \]

\item A primitive $\term{indent}\,(C_1,\,C_2)$ introduces constraints
  \begin{gather*}
    C_1\prg{.x} - C_2\prg{.x} = ind\\
    C_1\prg{.y} - C_2\prg{.y} \leq i_V + C_1\prg{.height}
  \end{gather*}
\end{itemize}

%Note, the last two layout primitives $\term{valign}$ and $\term{indent}$ are not used in the current implementation, but we provide constraints they introduce for completeness.

Also, for each virtual UI control $C$ containing UI controls $C_1, \ldots, C_n$ the following constraints are needed that define
its actual coordinates, width and height:
  \begin{gather*}
    C\prg{.x} = \min\,\{C_1\prg{.x}, \ldots, C_n\prg{.x}\}\\
    C\prg{.y} = \min\,\{C_1\prg{.y}, \ldots, C_n\prg{.y}\}\\
    C\prg{.x} + C\prg{.width} = \max\,\{C_i\prg{.x} + C_i\prg{.width}\}_{i=1}^n\\
    C\prg{.y} + C\prg{.height} = \max\,\{C_i\prg{.y} + C_i\prg{.height}\}_{i=1}^n\\
  \end{gather*}

Additionally, a number of inequality constraints is added which restrict the maximal possible values for
coordinates taking into account the size of enclosing panel. For each UI control $C$ we introduce the following constraints:

  \begin{gather*}
    C\prg{.x} \geq 0\\
    C\prg{.y} \geq 0\\
    C\prg{.x} + C\prg{.width} \leq W\\
    C\prg{.y} + C\prg{.height} \leq H\\
  \end{gather*}
\vspace{-2em}

There exists a number of ways to solve the set of integer linear inequalities (for example, using SMT solvers over the linear
integer arithmetic theory). It is, however, very appealing to try to employ relational verifiers yet again. This relational
solver would have the following benefit: it could be seamlessly integrated into the layout synthesis procedure, thus allowing some
constraint selection to be rejected at earlier stages.

However, the current relational implementation of this solver is underperforming in the presence of virtual UI controls.
As mentioned above, the width and height of virtual UI controls are not constants, which greatly increases the
search space. So, for the time being, we, indeed, used Z3 Theorem Prover~\cite{Zthree} to determine the absolute coordinates of the UI
controls (as well as the width and height of all virtual UI controls).


\end{comment}
