% !TeX spellcheck = ru_RU
% !TEX root = main.tex

\section{Представление термов и их погружение в логический домен}
\label{sec:injection}

Полиморфная унификация из предыдущего раздела работает для значений произвольных типов, при условии, что можно отличать логические переменные от других термов.
Это предположение зависит  от вида представления термов.
В оригинальной реализации на языке Scheme все термы были обычными s-выражениями, а логические переменные (в простом случае) --- одноэлементными векторами.
Пользователь обязан выполнять это соглашение и воздерживать от работы с пользовательскими значениями, состоящими из векторов.

%Polymorphic unification, considered in the previous section, works for the values of any type under the assumption that we
%are capable of identifying logical variables. The latter depends on the term representation. In the original
%implementation all terms are represented as a conventional S-expressions, while logical variables (in a simplest case)~--- as one-element vectors; it's an end user responsibility to respect this convention and refrain from operating
%with vectors as a user data.

В нашем случае мы хотим сохранить и строгую статическую типизацию и автоматический вывод типов.
Так как была выбрана полиморфная унификация, нежелательно представлять логические значения различных типов по-разному (технически это возможно, но будет противоречить выбранному легковесному подходу). Из этого следует, что термы с логическими переменными должны быть типизированы не так, как типизируется термы, объявленные пользователем, иначе будет возможно использовать термы в контекстах, где логические переменные не отображаются должным образом.
В то же время нежелательно, чтобы типы логических значений были полностью отделены от пользовательских типов: могут пригодиться конструкторы, объявленные пользователем и т.п.
Эти рассуждения подводят к идее \emph{логического представления} для пользовательских типов данных.
Неформально, логическим представлением типа $t$ является тип $\rho_t$, для которого определена пара функций:

%In our case we want to preserve both strong typing and type inference. Since we have chosen to use polymorphic
%unification, it is undesirable to represent logical variables of different types differently (while technically
%possible, it would compromise the lightweight approach we used so far). This means that terms with logical
%variables have to be typed differently from user-defined data~--- otherwise it would be possible to use
%terms in contexts where logical variables are not handled properly. At the same time we do not want term types
%to be completely different from user-defined types --- for example, we would like to reuse user-defined constructors, etc.
%This considerations boil down to the idea of \emph{logical representation} for a user-defined type. Informally,
%a logical representation for the type $t$ is a type $\rho_t$ with a couple of conversion functions:

\[
\begin{array}{rcl}
   \uparrow  \;: t \to \rho_t & - & \mbox{инъекция}\\
   \downarrow\;: \rho_t \to t & - & \mbox{проекция.}
\end{array}
\]

\noindent Тип $\rho_t$ структурно повторяет $t$, но может содержать логические переменные. Также инъекция --- это тотальная функция, а проекция --- частичная.

%\noindent  The type $\rho_t$ repeats the structure of $t$, but can contain logic variables. So, the injection is total, while the projection is partial.

Важно спроектировать представление значений следуя некоторой обобщённой схеме, иначе комбинаторы miniKanren будет невозможно правильно протипизировать.
Также желательно представить обобщённый способ получения пар инъекции/проекции (а ещё лучше, автоматизировать), чтобы снять с программиста груз ответственности о правильной их реализации. Наконец, представление должно представлять надежный способ отличать логические переменные.

%It is important to design representations as instances of some generic scheme (otherwise, miniKanren combinators
%could not be properly typed). In addition it is desirable to provide a generic way to build
%injection/projection pair in a uniform manner (and, even better, automatically) to lift the burden of
%their implementation off the end user shoulders and improve the reliability of the solution. Finally,
%the representation must provide a way to detect logic variable occurrences.

В этом разделе рассмотрен два способа реализовать логические представленя.
Первый достаточно просто реализовать, но к сожалению реализация будет показывать низкую производительность на практически значимых примерах.
Чтобы исправить этот недостаток, был разработан другой способ представления значений, который переиспользует некоторые компоненты первого.
\textcolor{red}{В разделе~\ref{sec:evaluation} представлены результаты экспериментального сравнения производительности.}

%In this part we consider two approaches to implementing logical representations.  The first is rather easy to
%develop and implement; unfortunately, the implementation demonstrates a poor performance for a number of
%important applications. In order to fix this deficiency, we develop a more elaborate technique which
%nevertheless reuses some components from the previous one. In Section~\ref{sec:evaluation}
%we present the results of performance evaluation for both approaches.

\subsection{Теггированные логические значения}

Наиболее естественным решением будет тегированное логическое представление.
Вводится полиморфный тип $[\alpha]$\footnote{В конкретном синтаксисе ``$\alpha\;$\lstinline|logic|''}, который соответствует логическому представлению типа $\alpha$.

%The first natural solution is to use tagging for representing logical representations.
%We introduce the following polymorphic type $[\alpha]$\footnote{In concrete syntax called ``$\alpha\;$\lstinline|logic|''}, which
%corresponds to a logical representation of the type $\alpha$:

\begin{lstlisting}
   type $[\alpha]$ = Var of int | Value of $\alpha$
\end{lstlisting}

\noindent Неформально выражаясь, любое значение типа $[\alpha]$ является либо значением типа $\alpha$, либо свободной логической переменной.
Обращаем внимание, что конструкторы этого типа не используются пользователем напрямую, так как единственным способом создания переменных является конструкция \enquote{\lstinline=fresh=}, т.е. тип логических значений является абстрактным.
Используя этот тип можно определить сигнатуры примитивов унификации, неравенства и создания переменных следующим образом:

% speaking, any value of type $[\alpha]$ is either a value of type $\alpha$, or a free
%logic variable. Note, the constructors of this type cannot be disclosed to an end user, since the only possible way to create a logic variable
%should still be by using the \enquote{\lstinline=fresh=} construct; thus the logic type is abstract in the interface.
%Now, we may redefine the signature of abstraction, unification and disequality primitives in the
%following manner

\begin{lstlisting}
   val (===)      : $[\alpha]$ -> $[\alpha]$ -> goal
   val (=/=)      : $[\alpha]$ -> $[\alpha]$ -> goal
   val call_fresh : ($[\alpha]$ -> goal) -> goal
\end{lstlisting}

\noindent И ограничение унификации, и неравенства использует одну и ту же полиморфную унификацию, но их тип ограничен только логическими значениями.
%Both unification and disequality constraint would still use the same polymorphic unification; their external, visible type, however, is restricted to logical types only.

Кроме переменных, другие логические значения также могут быть получены с помощью инъекции.
И наоборот, иногда логические значения могут быть спроецированы в обычные.
Для этого предоставляются два базовых примитива\footnote{В конкретном синтаксисе это \enquote{\lstinline|inj|} and \enquote{\lstinline|prj|}.},
которые могут быть использованы для \emph{поверхностной} инъекции/проекции. Как и ожидается, инъекция тотальна, а проекция --- частичная функция.

%Apart from variables, other logical values can be obtained by injection; conversely, sometimes a logical value can be projected to
%a regular one. We supply two basic functions\footnote{In concrete syntax called ``\lstinline|inj|'' and ``\lstinline|prj|''.}
%for these purposes

\begin{lstlisting}[mathescape=true]
   val ($\uparrow_\forall$) : $\alpha$ -> $[\alpha]$
   val ($\downarrow_\forall$) : $[\alpha]$ -> $\alpha$

   let ($\uparrow_\forall$) x = Value x
   let ($\downarrow_\forall$) = function Value x -> x | _ -> failwith "not a value"
\end{lstlisting}

%\noindent which can be used to perform a \emph{shallow} injection/projection. As expected, the injection is total, while the projection is partial.

Пара поверхностных функций хорошо работает для примитивных типов, но чтобы реализовать инъекцию/проекцию для произвольных, необходимо воспользоваться
идеей представления типов в виде неподвижных точек функторов~\cite{ALaCarte}.
Для наших нужд желательно сделать функторы полностью полиморфными: такими типами, где логические значения можно писать в произвольных позициях.
К тому же данный подход позволит переиспользовать имеющийся код с меньшим количеством изменений.

%The shallow pair works well for primitive types; to implement injection/projection for arbitrary types we exploit the
%idea of representing regular types as fixed points of functors~\cite{ALaCarte}. For our purposes it is desirable to make
%the functors fully polymorphic~--- thus a type, in which we can place a logical variable into arbitrary position,
%can be easily manufactured. In addition this approach makes it possible to refactor the existing code to use relational
%programming with only minor changes.

Продемонстрируем этот подход на каноническом примере связных списков. Рассмотрим обычный тип данных OCaml для полиморфных списков:

%To illustrate this approach, we consider an iconic example~--- the list type. Let us have a conventional definition for a regular polymorphic list in OCaml:

\begin{lstlisting}
   type $\alpha$ list = Nil | Cons of $\alpha$ * $\alpha$ list
\end{lstlisting}

В этом типе можно использовать логические переменные только на месте элементов списка, но не вместо хвоста, так как там ожидается значение типа \lstinline|$\alpha$ list|, зафиксированного в конструкторе  \lstinline|Cons|.
Чтобы получить подходящее логическое представление, вначале абстрагируем тип дл полностью полиморфного функтора:

% For this type we can only place a logical variable in the position of a list element, but not of the tail, since the tail always has the type \lstinline|$\alpha$ list|, fixed in the definition of constructor \lstinline|Cons|.
%In order to create a full-fledged logical representation, we first have to abstract the type into a fully-polymorphic functor:

\begin{lstlisting}
   type ($\alpha$, $\beta$) $\mathcal L$ = Nil | Cons of $\alpha$ * $\beta$
\end{lstlisting}

\noindent Таким образом, оригинальный тип может быть выражен как

%Now, the original type can be expressed as

\begin{lstlisting}
   type $\alpha$ list = ($\alpha$, $\alpha$ list) $\mathcal L$,
\end{lstlisting}

\noindent а его логическое представление как
%\noindent and its logical representation~--- as

\begin{lstlisting}
   type $\alpha$ list$^o$ = $[$($[\alpha]$, $\alpha$ list$^o$) $\mathcal L]$
\end{lstlisting}

Более того, с помощью специфичной для функтора функции \lstinline=fmap=
%Moreover, with the aid of conventional functor-specific mapping function

\begin{lstlisting}
   val fmap$_{\mathcal L}$ : ($\alpha$ -> $\alpha^\prime$) -> ($\beta$ -> $\beta^\prime$) -> ($\alpha$, $\beta$) $\mathcal L$ -> ($\alpha^\prime$, $\beta^\prime$) $\mathcal L$
\end{lstlisting}

\noindent можно реализовать инъекцию и проекцию:
%noindent both the injection and the projection functions can be implemented:

\begin{lstlisting}[mathescape=true]
   let rec $\uparrow_{\mbox{\texttt{list}}}$ l = $\uparrow_{\forall}$(fmap$_{\mathcal L}$ ($\uparrow_\forall$) $\uparrow_{\mbox{\texttt{list}}}$ l)
   let rec $\downarrow_{\mbox{\texttt{list}}}$ l = fmap$_{\mathcal L}$ ($\downarrow_\forall$) $\downarrow_{\mbox{\texttt{list}}}$ ($\downarrow_\forall$ l)
\end{lstlisting}

Специфичные для функторов функции \lstinline=fmap= могут быть легко написаны, и даже получены автоматически с помощью различных методов обобщённого программирования для OCaml. А с помощью \lstinline=fmap= уже будут написаны инъекции и проекции пользовательских типов данных.

%As functor-specific mapping functions can be easily written or, better, derived automatically using a number of existing frameworks for generic programming for OCaml, one can easily provide injection/projection pair for user-defined data types.

Теперь вернемся к вопросу определения переменных во время полиморфной унификации.
Так как типы не известны, мы не можем отличать переменные с помощью простого сопоставления с образцом.
В реализации переменные проверяются следующим способом:

%We now can address the problem of variable identification during polymorphic unification. As we do not know the types, we cannot discriminate logical
%variables by their tags only and, thus, cannot simply use pattern matching. In our implementation we perform a variable test as follows:

\begin{itemize}
\item в окружении хранится дополнительное значение-якорь, создаваемое вместе с начальным состоянием; этот якорь не изменяется для всех производных состояний во время запуска \lstinline=run=;
%\item in an environment, we additionally keep some unique boxed value~--- the \emph{anchor}~--- created by \lstinline|run| at the moment of initial
%state generation; the anchor is inherited unchanged in all derived environments during the search session;
\item  в тип логических переменных добавлен якорь, который при берется из окружения и при создании переменных кладется в них.
%we change the logic type definition into

\begin{lstlisting}
   type $[\alpha]$ = Var of int * anchor | Value of $\alpha$
\end{lstlisting}

%\noindent making it possible to save in each variable the anchor, inherited from the environment, in which the variable was created;

\item во время унификации, для проверки переменных проверяется конъюнкция следующих свойств:
%inside the unification, in order to check if we are dealing with a variable, we test the conjunction of the following properties:

  \begin{enumerate}
    \item рассматриваемое значение является boxed;
    \item тек и раскладка в памяти соответствует переменным, т.е. значениям, построенным с помощью конструктора \lstinline|Var| из типа \lstinline{[$\alpha$]};
    %the scrutinee's tag and layout correspond to those for variables (i.e. the values, created with the constructor \lstinline|Var| of type \lstinline{[$\alpha$]});
    \item адрес якоря соответствует адресу якоря в текущем окружении.
    %the scrutinee's anchor and the current environment's anchor have equal addresses.
  \end{enumerate}
\end{itemize}

Учитывая, что состояние абстрактно для пользователя, можно гарантировать, что только переменные, созданные в текущем запуске примитива \lstinline=run=, пройдут тест на переменные, так как указатель на якорь уникален среди всех указателей и никак может попасть во вне из-за использования примитива создания переменных.

%Taking into account that the state type is abstract at the user level, we guarantee that only those variables which were
%created during the current run session would pass the test, since the pointer to the anchor is unique among all pointers to a boxed value
%and could not be disclosed anywhere but in the variable-creation primitive.

Осталось описать фазу реификации, которая может быть представлена с помощью следующей функции:
%The only thing to describe now is the implementation of the reification stage. The reification is represented by the following function:

\begin{lstlisting}
   val reify : state -> $[\alpha]$ -> $[\alpha]$
\end{lstlisting}

Функция принимает состояние и логическое значение, и рекурсивно совершает подстановку всех переменных в логическом значении до тех пор, пока остаются связанные переменные. Так как наша реализация подстановки не является полиморфной, то функция \lstinline|reify| также реализована в небезопасном стиле.
Но легко показать, что \lstinline|reify| не создает неправильно типизированные термов. Все оригинальные типы переменных не нарушаются при подстановке, унификация также не изменяет унифицированных термов, а все термы в подстановке корректно типизированы.
Следовательно, \lstinline|reify| всегда подставляет подтермы в правильно типизированном терме, но термы таких же типов, что обуславливает безопасность системы типов.

%This function takes a state and a logic value and recursively substitutes all logic variables in that value w.r.t.
%the substitution in the state until no occurrences of bound variables are left. Since in our implementation the type of a substitution is
%not polymorphic, \lstinline|reify| is also implemented in an unsafe manner. However, it is easy to see that \lstinline|reify|
%does not produce ill-typed terms. Indeed, all original types of variables are preserved in a substitution; unification does not
%change unified terms, so all terms bound in a substitution are well-typed. Hence, \lstinline|reify| always substitutes
%some subterms in a well-typed term with other terms of the corresponding types, which preserves the well-typedness.

Кроме осуществления подстановки функция  \lstinline|reify| также реифицирует ограничения-неравенства.
При этом к каждой свободной переменной приписывается список реифицированных термов, представляющий ограничения-неравенства для этой свободной переменной.
Заметим, что ограничения неравенства могут быть созданы только для одинаково типизированных термов, что обосновывает типовую безопасность реификации.
Обратим также внимание, что нужны дополнительные проверки для избежания зацикливания реификации, так реификация ответов и ограничений взаимно рекурсивны, а реификация ответа может вызвать саму себя по цепочке ограничений-неравенств. В примере ниже

%In addition to performing substitutions, \lstinline|reify| also reifies disequality constraints. Constraint reification
%attaches to each free variable in a reified term a list of reified terms, describing the disequality constraints for that
%free variable. Note, disequality can be established only for equally-typed terms, which justifies the type-safety of reification.
%Note also, additional care has to be taken to avoid infinite looping, since reification of answers and constraints are
%mutually recursive, and the reification of a variable can be potentially invoked from itself due to a chain of disequality
%constraints. In the following example

\begin{lstlisting}
   let foo q =
      fresh (r s)
        (q === $\uparrow_{\forall}$ (Some r)) &&&
        (r =/= s) &&&
        (s =/= r)
\end{lstlisting}

\noindent ответ для переменной  $q$ будет содержать неравенства для переменной $r$; реификация $r$ приведет ограничения с переменной $s$, которая снова запустит реификацию переменной $r$, и т.д.

%\noindent the answer for the variable $q$ will contain a disequality constraint for the variable $r$; the reification of $r$ will in turn lead to the reification of its own constraint, this time the variable $s$; finally, the reification of $s$ will again invoke the reification of $r$, etc.

%After the reification, the content of a logical value can be inspected via the following function:
%
%\begin{lstlisting}
%   val destruct : $[\alpha]$ -> [`Var of int * $[\alpha]$ list | `Value of $\alpha$]
%\end{lstlisting}
%
%Constructor \lstinline|`Var| corresponds to a free variable with unique integer identifier and a list of terms,
%representing all disequality constraints for this variable.\textcolor{red}{выкинуть?}

\subsection{Логические переменные без тегирования. Отслеживание типов}

Решение, предложенное выше, страдает существенным недостатком: чтобы произвести унификацию, необходимо привести термы в логический домен, что сделает их в два раза больше. В результате, эта реализация проиграет по производительности оригинальной.



%The solution presented in the previous subsection suffers from the following deficiency: in order to perform unification,
%we inject terms into the logic domain, making them as twice as large. As a result, this implementation loses to the original one in
%terms of performance in many important applications, which compromises the very idea of using OCaml as a host language.

Было разработано другое представление без этого недостатка.
Идея в том, чтобы хранить данные (не переменные) в памяти без использования тегирования вообще.
%Here we develop an advanced version, which eliminates this penalty. As a first step, let's try to eliminate the tagging with a drastic measure:

\begin{lstlisting}
   type $[\alpha]$ = $\alpha$
\end{lstlisting}

\noindent Как следствие станет невозможно конструировать данные с переменными внутри путём использования конструкторов,
а придется использовать специальные функции-конструк\-торы.

%What consequences would this have? Of course, we would not be able to create logical variables in a conventional way. However, we still could have a separate type of variables

\begin{lstlisting}
   type var = Var of int * anchor
\end{lstlisting}

\noindent Так как тип $[\alpha]$ абстрактный, эти модификации не меняют интерфейса взаимодействия с данными из реляционного кода.
Но при этом всё ещё возможно представлять переменные старым способом, переиспользовав алгоритм проверки на переменные,
а реализация полиморфной унификации остается \emph{почти} такой же.
Некоторой проблемой является то, что можно вводить переменные в нелогические и нетегированные данные.
Эти свободные переменные не помещают унификации (она сумеет их отличить), но их нельзя оставить для нелогического домена как они есть.

%\noindent and use \emph{the same} variable test procedure. As the type $[\alpha]$ is abstract, this modification does not change the interface.
%As we reuse the variable test, polymorphic unification can continue to work \emph{almost} correctly. The problem is that
%now it can introduce the occurrences of free logic variables in non-logical, tagless, data structures. These free logic variables
%do not get in the way of unification itself (since it can handle them properly, thanks to the variable test), but they can not
%be disclosed to the outer world as is.

Идея заключается в том, чтобы использовать в общем случае некорректное представление в памяти --- ему нельзя придать тип в языке OCaml --- для внутренних нужд, а применять тегирование только на фазе реификации. Но тогда необходимо решить каким будет тип функции \lstinline|reify|.
Прямолинейное решение \lstinline|val reify : state -> $[\alpha]$ -> $[\alpha]$|
%Our idea is to use this generally unsound representation for all internal actions, and perform tagging only during the reification
%stage. However, this scenario raises the following question: what would the type of \lstinline|reify| be? It can not be simply
%\begin{lstlisting}
%   val reify : state -> $[\alpha]$ -> $[\alpha]$,
%\end{lstlisting}
не годится, так как тип аргумента совпадает с типом результата $[\alpha]$.
Хотелось бы что-то похожее на \lstinline|val reify : state -> $[\alpha]$ -> $(\mbox{``tagged'' } [\alpha])$|.

%anymore since $[\alpha]$ now equals $\alpha$. We \emph{want}, however, it be something like
%
%\begin{lstlisting}
%   val reify : state -> $[\alpha]$ -> $(\mbox{``tagged'' } [\alpha])$
%\end{lstlisting}

Если $\alpha$ --- тип без параметров, то можно затегировать результат конструкторами \lstinline|Var| или \lstinline|Value|  в зависимости того переменная это или нет.
Всё несколько сложнее для типов с параметрами, например, для типа целочисленных списков \lstinline|$[[$int$]$ list$]$|.
Здесь надо не только тегировать не верхнем уровне, но и запускать реификацию с тегированием рекурсивно для подвыражений внутри.
Итого, необходим следующий (мета)тип для функции реификации: \lstinline|val reify : state -> $[\alpha]$ -> $\mbox{(``tagged''} [\beta])$|,
%
%If $\alpha$ is not a parametric type, we can simply test if the value is a variable, and if yes, tag it with the constructor \lstinline|Var|;
%we tag it with \lstinline|Value| otherwise, and we're done. This trick, however, would not work for parametric types. Consider, for example,
%the reification of a value of type \lstinline|$[[$int$]$ list$]$|.
%The (hypothetical) approach being described would return a value of type \lstinline|$(\mbox{``tagged'' }[[$int$]$ list$])$|, i.e. tagged only on the top level; we need to repeat the procedure
%recursively. In other words, we need the following (meta) type for the reification primitive:
%
%\begin{lstlisting}
%   val reify : state -> $[\alpha]$ -> $\mbox{(``tagged''} [\beta])$
%\end{lstlisting}
%
где $\beta$ --- тип затегированного $\alpha$.
%\noindent where $\beta$ is the result of tagging $\alpha$.

Эти наблюдения обосновывают выбранную конкретную реализацию.
%These considerations can be boiled down to the following concrete implementation.

Оригинальное определение типа $[\alpha]$ будет тегированным представлением.
Введем дополнительный однопараметрический тип \enquote{$\alpha\;$\lstinline|ilogic|}, %\footnote{В конкретном синтаксисе ``$(\alpha)\;$\lstinline|ilogic|''.}
представляющий логические значения типа $\alpha$.

%First, we roll back to the initial definition of $[\alpha]$~--- it will play the role of our ``tagged'' type.
%We introduce a new, two-parameter type\footnote{In concrete syntax called ``$(\alpha,\;\beta)\;$\lstinline|injected|''.}

\begin{lstlisting}
   type $\alpha$ ilogic = $\alpha$
   val inj  : $\alpha$ -> $\alpha$ ilogic
\end{lstlisting}

\noindent Для пользователя он будет абстрактным, а его значения будут появляться с помощью функции поверхностной инъекции \lstinline=inj=.

Для рекурсивных типов всё несколько сложнее: необходимо применять инъекцию рекурсивно для всех подвыражений. Рассмотрим эту инъекцию на примере списков.

\begin{lstlisting}
   type $\alpha$ list = ($\alpha$, $\alpha$ list) $\mathcal L$,
   type $\alpha$ injected = ($\alpha$, $\alpha$ injected) $\mathcal L$ ilogic,
   type $\alpha$ list$^o$ = $[$($[\alpha]$, $\alpha$ list$^o$) $\mathcal L]$
\end{lstlisting}

\noindent В типе нетегированного представления необходимо расставить тип \lstinline=ilogic= для всех подвыражений, так образом инъекция целочисленных списков с типом \lstinline=int list= будет порождать значения типа \lstinline=int ilogic injected=.
На практике для построения логических представлений списков стоит использовать специальные функции-конструкторы вместо конструкторов алгебраического типа.
\begin{lstlisting}
  let cons h tl : 'a -> 'a injected -> 'a injected = inj (Cons (h, tl))
  let nil ()    : unit -> 'a injected              = inj Nil
\end{lstlisting}

Также необходимо реализовать реификацию значений из логического домена.
На данный момент подход к реификации не ограничивает тип, куда реификация будет производится, но каноническим является использование упомянутого выше $[\alpha]$.
Все реификаторы являются двупараметрическими типами, инкапсулирущими преобразование из логического домена в некоторый кодомен.
Например, для списков необходимо получить реификатор с типом \lstinline=(int ilogic injected, [int] list$^o$) reifier=.

\begin{lstlisting}
  val chain:  ('a reifier -> 'b reifier) -> ('a -> 'b) reifier

  let fmapt f g xs = pure (fmap$_{list}$) <*> f <*> g <*> xs

  let reify : ($\alpha$, $\beta$) reifier -> ($\alpha$ injected, $\beta$ list$^o$) reifier =
   fun ra ->
     Reifier.fix (fun self ->
       OCanren.reify <..> chain
         (Reifier.zed
             (Reifier.rework ~fv:(fmapt ra self)) ) )
\end{lstlisting}

\noindent Выше пример кода с реификатором для списков, который можно построить, предъявив реификатор \lstinline=ra= для элементов списка.
Сами реификаторы по сути являются монадами Reader, которые хранят в себе информацию для определения переменных в терме.
Типы реификаторов абстрактные, чтобы пользователь не смог провести реификацию переменных, созданных не в том состоянии.
При реализации реификации используется интерфейс аппликативных функторов с операциями \lstinline=pure=  и \lstinline=<*>=.
Также используются комбинатороы неподвижных точек для реификаторов (\lstinline=Reifier.fix=) и для call-by-value языков (\lstinline=Reifier.zed=),
а ещё специальная функция \lstinline=chain=. Примитив \lstinline=reify= проводит поверхностную реификацию, а функция \lstinline=rework= рекурсивно запускает реификацию для подвыражений и ограничений-неравенств, если таковые присутствуют.

Функции \lstinline=chain=, \lstinline=OCanren.reify= и комбинаторы неподвижной точки определены один раз для всех типов, и пользователю для своего типа данных необходимо реализовать только \lstinline=fmapt=, \lstinline=reify= и функции конструкторы, упомянутые выше.
Для облегчения этого было разработано синтаксическое расширение, которое по объявлению типа порождает всё необходимое во время компиляции. Пример его использования можно найте в разделе~\ref{sec:examples}.

\begin{comment}


Of course, this type is kept abstract at the end-user level. Informally speaking, the type $\{\alpha,\;\beta\}$ designates the
injection of a tagless type $\alpha$ into a tagged type $\beta$; the value itself is kept in the tagless form, but
the tagged type can be used during the reify stage as a constraint, which would allow us to reify a tagless
representation only to a feasible tagged one. In other words, we record the injection steps using the second
type parameter of the type ``\{,\}'', performing the bookkeeping on the type level rather than on the value level.

We introduce the following primitives for the type $\{\alpha,\;\beta\}$:

\begin{lstlisting}
   val lift : $\alpha$ -> $\{\alpha,\;\alpha\}$
   val inj  : $\{\alpha,\;\beta\}$ -> $\{\alpha,\;[\beta]\}$

   let lift x = x
   let inj  x = x
\end{lstlisting}

The function \lstinline|lift| puts a value into the ``bookkeeping injection'' domain for the first time, while
\lstinline|inj| plays the role of the injection itself. Their composition is analogous to what was
called ``$\uparrow_\forall$'' in the previous implementation:

\begin{lstlisting}
   val $\uparrow_\forall$ : $\alpha$ -> $\{\alpha,\;[\alpha]\}$
   let $\uparrow_\forall$ x = inj (lift x)
\end{lstlisting}

In order to deal with parametric types, we can again utilize generic programming. To handle the types with
one parameter, we introduce the following functor:

\begin{lstlisting}
   module FMap (T : sig type $\alpha$ t val fmap : ($\alpha$ -> $\beta$) -> $\alpha$ t -> $\beta$ t end) :
     sig
       val distrib : $\{\alpha,\;\beta\}$ T.t -> $\{\alpha$ T.t, $\beta$ T.t$\}$
     end =
     struct
       let distrib x = x
     end
\end{lstlisting}

Note, that we do not use the function ``\lstinline{T.fmap}'' in the implementation; however, first, we need an inhabitant of the
corresponding type to make sure we are indeed dealing with a functor, and next, we actually will use it in the
implementation of type-specific reification, see below.

In order to handle two-, three-, etc. parameter types we need higher-kinded polymorphism, which is
not supported in a direct form in OCaml. So, unfortunately, we need to introduce separate
functors for the types with two-, three- etc. parameters; existing works on higher-kinded
polymorphism in OCaml~\cite{HKinded} require the similar scaffolding to be erected as a bootstrap step.

Given the functor(s) of the described shape, we can implement logic representations for
all type's constructors. For example, for standard type \lstinline|$\alpha$ option| with two constructors
\lstinline|None| and \lstinline|Some| the implementation looks like as follows:

\begin{lstlisting}
   module FOption = FMap (struct
     type $\alpha$ t = $\alpha$ option
     let fmap = fmap$_{\mbox{\texttt{option}}}$
   end)

   val some : $\{\alpha, \beta\}$ -> $\{\alpha\mbox{\texttt{ option}},\;\beta\mbox{\texttt{ option}}\}$
   val none : unit  -> $\{\alpha\mbox{\texttt{ option}},\;\beta\mbox{\texttt{ option}}\}$

   let some x  = inj (FOption.distrib (Some x))
   let none () = inj (FOption.distrib None)
\end{lstlisting}

In other words, we can in a very systematic manner define logic representations for all constructors
of types of interest. These representations can be used in the relational code, providing a well-bookkept
typing~--- for each logical type we would be able to reconstruct its original, tagless preimage.

With the new implementation, the types of basic goal constructors have to be adjusted:

\begin{lstlisting}
   val (===) : $\{\alpha,\;[\beta]\}$ -> $\{\alpha,\;[\beta]\}$ -> goal
   val (=/=) : $\{\alpha,\;[\beta]\}$ -> $\{\alpha,\;[\beta]\}$ -> goal
\end{lstlisting}

As always, we require both arguments of unification and disequality constraint to be of the same type; in addition
we require the injected part of the type to be logical.

During the reification stage the bindings for the top-level variables, reconstructed using the final
substitution, have to be properly tagged. This process is implemented in a datatype-generic manner as well:
first, we have reifiers for all primitive types:

\begin{lstlisting}
   val reify$_{\mbox{\texttt{int}}}$ : helper -> $\{$int,$[$int$]\}$ -> $[$int$]$
   val reify$_{\mbox{\texttt{string}}}$ : helper -> $\{$string,$[$string$]\}$ -> $[$string$]$
   ...
\end{lstlisting}

and, then, we add the reifier to the output signature in all \lstinline|FMap|-like functors:

\begin{lstlisting}
   val reify: (helper -> $\{\alpha,\;\beta\}$ -> $\beta$) -> helper -> $\{\alpha$ T.t, $[\beta$ T.t$]$ as $\gamma\}$ -> $\gamma$
\end{lstlisting}

Note, since now \lstinline|reify| is a type-specific and, hence, constructed at the user-level, we refrain from passing
it a state (which is inaccessible on the user level). Instead, we wrap all state-specific functionality in
an abstract \lstinline|helper| data type, which encapsulates all state-dependent functionality needed for \lstinline|reify|
to work properly.
\end{comment}