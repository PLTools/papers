\section{Applications}
\label{sec:applications}

In addition to verification of correctness of different implementations of disequality constraints we can use our framework to formally
state and prove some of its other important properties. Thanks to our completeness result, we can do it in the denotational context,
where the reasoning is much easier.

For example, we can specify contradictory answers with empty interpretation, which we pointed out for the trivial implementation from the previous section,
and prove that there are no such answers in the realistic implementation if and only if there are infinitely many constructors in the language. So, for the realistic implementation the following holds iff the set of constructors is infinite:

\begin{lemma}
For any goal $g$, if all free variables in it belong to the set $\{\alpha_1,\dots,\alpha_n\}$, then

\[ \forall (\sigma, \cstore_\sigma, n_r) \in Tr_{\inbr{g, \epsilon, \cstoreinit, n}}, \quad \sembr{\sigma} \cap \sembr{\cstore_\sigma} \neq \emptyset. \]
\end{lemma}

The proof is based on the following lemma about combining constraints, which we can prove we can prove when there are infinitely many constructors (and otherwise it is not true).

\begin{lemma}
If for a finite constraint store $\cstore_\sigma$
\[ \forall \omega \in \cstore_\sigma,  \sembr{\sigma} \cap \sembr{\omega} \neq \emptyset, \]
then
\[ \sembr{\sigma} \cap \sembr{\cstore_\sigma} \neq \emptyset. \]
\end{lemma}

Another example of application is the justification of optimizations in constraint store implementation. For example, the following obvious (in denotational context) statement
allows deleting subsumed constraints in the realistic implementation.

\begin{lemma}
For any constraint store $\cstore_\sigma$ and two constraint substitutions $\omega$ and $\omega'$, if

\[ \exists \tau, \omega' = \omega \tau \]

then

\[ \sembr{\cstore_\sigma \cup \{\omega, \omega'\}} = \sembr{\cstore_\sigma \cup \{\omega\}}. \]
\end{lemma}
