% !TeX spellcheck = ru_RU
% !TEX root = main.tex

\section{Потоки, состояния и цели}
\label{sec:goals}

Этот раздел содержит некоторые детали  реализации.
Несмотря на то, что это почти повторение оригинальной реализации~\cite{MicroKanren, CKanren} для OCaml, он нужен, чтобы аккуратно ввести некоторые понятия.
%This section describes a top-level framework for our implementation. Even though it contains nothing more than a reiteration of the original implementation~\cite{MicroKanren, CKanren} in OCaml, we still need some notions to be properly established.

Процедура поиска реализована с использованием монады для ленивых потоков с возвратами~\cite{KiselyovBacktracking}:
%The search itself is implemented using a backtracking lazy stream monad~\cite{KiselyovBacktracking}:

\begin{lstlisting}
   type $\alpha$ stream

   val mplus : $\alpha$ stream -> $\alpha$ stream -> $\alpha$ stream
   val bind  : $\alpha$ stream -> ($\alpha$ -> $\beta$ stream) -> $\beta$ stream
\end{lstlisting}

\noindent Монадические примитивы используются для запуска поиска, низкоуровневые детали могут различаться в  различных вариантах \miniKanren.

%\noindent Monadic primitives describe the shape of the search, and their implementations may vary in concrete \miniKanren{} versions.
Важной компонентой является реализация следующих типов данных:
%An essential component of the implementation is a bundle of the following types:

\begin{lstlisting}
   type env         = $\dots$
   type subst       = $\dots$
   type constraints = $\dots$

   type state = env * subst * constraints
\end{lstlisting}
Тип состояния \lstinline|state| описывает позицию в лениво строящемся дереве поиска:
тип \lstinline|env| соответствует  \emph{окружению}, содержащему дополнительную информацию (в частности, для создания свежих переменных);
\lstinline|subst| хранить подстановку, которая содержит отображения некоторых логических переменных;
тип \lstinline|constraints| представляет ограничения-неравенства, которые надо соблюдать.
В самом простом случае \lstinline|env| содержит только номер последней введённой свежей переменной, \lstinline|subst| --- это конечное отображение, а \lstinline|constraints| --- просто список подстановок.

%Type \lstinline|state| describes a point in a lazily constructed search tree:
% type \lstinline|env| corresponds to an \emph{environment}, which contains some supplementary information (in particular, an environment is needed to
%correctly allocate fresh variables),
%type \lstinline|subst| describes a substitution, which keeps current bindings for some logical variables,
%and, finally, type \lstinline|constraints| represents disequality constraints, which have to be respected.
%In the simplest case \lstinline|env| is just a counter for the number of the next free variable, \lstinline|subst| is a map-like structure and \lstinline|constraints| is a list of substitutions.

%In our case, the environment contains some extra information to make it possible to identify variables
%in a constant time.

Тип целей \lstinline|goal| является преобразованием одного состояния в ленивый поток состояний:
%The next cornerstone element is the \emph{goal} type, which is considered as a transformer of a state into a lazy stream of states:

\begin{lstlisting}
   type goal = state -> state stream
\end{lstlisting}

\noindent В терминах поиска выполнение цели соответствует одному шагу поиска: для конкретного узла дерева поиска вычисляются его непосредственные потомки.
Со стороны пользователя тип \lstinline|goal| является абстрактным и все состояния полностью скрыты.
%\noindent In terms of the search, a goal nondeterministically performs one step of the search: for a given node in a search tree it produces its immediate descendants. On the user level the type \lstinline|goal| is abstract, and states are completely hidden.

Также присутствует набор заранее реализованных комбинаторов:
%Next to last, there are a number of predefined combinators:

\begin{lstlisting}
   val (&&&)      : goal -> goal -> goal
   val (|||)      : goal -> goal -> goal
   val call_fresh : ($v$ -> goal) -> goal
   ....
\end{lstlisting}

Конъюнкция \enquote{\lstinline=&&&=} объединяет результаты своих целей-аргументов с помощью \lstinline|bind|,
дизъюнкция \enquote{\lstinline=|||=} конкатенирует результаты с помощью \lstinline|mplus|,
примитив \lstinline|call_fresh| принимает абстрагированную цель и применяет её к только что созданной свежей переменной.
Тип $v$ соответствует типу свежей логической переменной, детали которого мы обсудим позже (раздел~\ref{sec:injection}).
Эти комбинаторы являются базовыми для реализации более удобных конструкций реляционного программирования (\lstinline|conde|, \lstinline|fresh| и т.д.)

%Conjunction ``\lstinline|&&&|'' combines the results of its argument goals using \lstinline|bind|,
%disjunction ``\lstinline=|||='' concatenates the results using \lstinline|mplus|, abstraction
%primitive \lstinline|call_fresh| takes an abstracted goal and applies it to a freshly created
%variable. Type $v$ in the last case designates the type for a fresh variable, which we leave
%abstract for now. These combinators serve as the bricks for the implementation of conventional
%\miniKanren constructs and syntax extensions (\lstinline|conde|, \lstinline|fresh|, etc.)

Наконец, существуют два примитива для создания примитивных целей: ограничения унификации и ограничения-неравенства. Тип термов  $t$  также оставим абстрактным и вернёмся к нему позже.
%Finally, there are two primitive goal constructors:

\begin{lstlisting}
   val (===) : $t$ -> $t$ -> goal
   val (=/=) : $t$ -> $t$ -> goal
\end{lstlisting}

%The first one is a unification, while the other is a disequality constraint. Here, we again left the type of terms $t$ abstract; it will be instantiated later.

\noindent В различных реализациях miniKanren все конструкции можно реализовывать сходным образом.
Но в случае с OCaml реализация полиморфной унификации (раздел~\ref{sec:unification}) потребовала использования нетривиальных решений, которые отсутствуют в оригинальном miniKanren.

%In the implementation of \miniKanren both of these goals are implemented using unification~\cite{CKanren}; this
%is true for us as well. However, due to a drastic difference between the host languages, the implementation of
%efficient polymorphic unification itself leads to a number of tricks with typing and data representation, which are
%absent in the original version.

С использованием объявленных выше типов, сигнатура примитива \lstinline|run| выглядит так:

\begin{lstlisting}
   val run : goal -> state stream
\end{lstlisting}

\noindent Данная функция создает начальное состояние и применяет к нему цель. Состояния в получившемся потоке описывают различные решения данной цели.
Так как поток строится постепенно, для проведения поиска следуют получать состояния по одному.

%\noindent This function creates an initial state and applies a goal to it. The states in the return stream describe various solutions for the goal.
%As the stream is constructed lazily, taking elements one by one makes the search progress.

Конкретные ответы реконструируются из состояний. Как правило, несколько переменных реифицируются в этом состоянии, т.е. оттуда достаются их конкретные значения.
Также для свободных переменных извлекаются ограничения-неравенства, представленные как список \enquote{запрещённых} термов.
Так как эти запрещённые термы могут в свою очередь содержать другие свободные переменные, то реификация ограничений является реализована рекурсивно.

%To discover concrete answers, the state has to be queried for its contents. As a rule, a few variables
%are reified in a state, i.e. their bindings in the corresponding substitution are retrieved.
%Disequality constraints for free variables have to be reified additionally (e.g. represented as a list of
%``forbidden'' terms). As forbidden terms can contain free variables, the constraint reification
%process is recursive.

В случае OCaml реификация является нетривиальной частью реализации, так как, как мы увидим, она не может быть реализована на типобезопасном фрагменте языка.
%In our case, the reification is a subtle part, since, as we will see shortly, it can not be implemented in a type-safe fragment of the language.
